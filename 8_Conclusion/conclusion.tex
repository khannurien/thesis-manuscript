\chapter{Conclusion et perspectives}
%\addcontentsline{toc}{chapter}{Conclusion et perspectives}
%\chaptermark{Conclusion et perspectives}

\section{Résumé des contributions}
%\addcontentsline{toc}{section}{Résumé des contributions}

\section{Limites des contributions}
%\addcontentsline{toc}{section}{Limites des contributions}

Contention : \cite{vanbeekCPUContentionPredictor2019} \cite{jacquetSweetspotVMOversubscribingCPU}

Interférences : \cite{kohAnalysisPerformanceInterference2007} \cite{vardasImprovedParallelApplication}

Pannes : \cite{javadiFailureTraceArchive2013, galletModelSpaceCorrelatedFailures2010}

Modèle de coût étendu : exemple, coût pour la société (carbone, eau) \cite{rickeCountrylevelSocialCost2018}

Complexité, passage à l'échelle : les noeuds remontent leur niveau de charge et leurs décisions de scaling à l'orchestrateur \cite{straesserPowerApplicationsVision2023}

\section{Travaux futurs}
%\addcontentsline{toc}{section}{Travaux futurs}

% Our next contribution will leverage Q-Learning to explore the design of an autonomous agent which efficiently rightsizes resources allocations on the serverless platform. It will make use of time series prediction to allow timely, proactive autoscaling. This agent will be evaluated in the simulator, showcasing the variety of policies that can be implemented with our tool.

Prédiction sur séries temporelles : \cite{bauerTimeSeriesForecasting2020}

Allocation proactive : \cite{parkGraphNeuralNetworkBased2024}

\section{Remarques finales}
%\addcontentsline{toc}{section}{Remarques finales}
