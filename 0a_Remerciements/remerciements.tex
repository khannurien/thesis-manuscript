\chapter*{Remerciements}

À Anne-Cécile Orgerie, Thomas Ledoux et Laurent Lefèvre : vous avez toute ma gratitude pour avoir pris de votre temps afin de lire et critiquer ce manuscrit, ainsi que pour avoir accepté de constituer un jury de thèse qui a donné lieu à un beau moment d'échange.

À Romain Rouvoy et Shadi Ibrahim : merci à tous les deux pour le temps que vous m'avez accordé au sein du comité de suivi de thèse durant mes trois années de doctorat. Votre expertise a été précieuse.

À Laurent d'Orazio et Olivier Barais : vous m'avez aidé à comprendre ce qu'était le travail de chercheur, à trouver des directions dans lesquelles creuser, et vous m'avez convaincu de persévérer en début de thèse. Pour cela, vous avez toute ma considération.

À Stéphane Paquelet : je garderai de bons souvenirs de nos réunions de travail, pour ta rigueur scientifique, et ta bonne humeur sans laquelle mon passage à l'IRT n'aurait pas été le même.

À Jalil Boukhobza, bien sûr, car te rencontrer lorsque j'ai repris l'école a été déterminant : c'est une chance d'avoir travaillé avec toi ces dernières années, et un plaisir de poursuivre sur de nouveaux sujets aujourd'hui.

Aux amis que l'on rencontre sur la route : Camélia, Quentin, Sylvain, Louis, Hocine... Quelle joie d'avoir fait votre connaissance, et partagé cette expérience avec vous.

À ma sœur, Audrey, et à ma mère, Chantal, dont j'ai eu la chance de pouvoir croiser les regards lors de ma soutenance.

À ma compagne, Elizabeth : repenser à notre rencontre, un an avant que je retrouve les bancs de la fac, est vertigineux. Ton courage d'avoir décidé de m'accompagner dans cette aventure pendant sept ans m'impressionne. Je suis fier d'être à tes côtés.

Enfin, à toutes celles et tous ceux pour qui j'ai une pensée en écrivant cette page -- vous vous reconnaîtrez : merci infiniment pour votre amour et votre patience.

{\color{white} Benjamin : tu as perdu !}

\clearpage

\vspace*{\fill}

\boitesimple{
    \footnotesize \textit{The first trick of asking questions is to determine if your question is a good one. Just because a question has never been asked does not make it good. Smart people have been asking questions for quite a few centuries now, so many of the questions that \emph{haven't} been asked are bound to yield uninteresting answers.
    \\
    But if you can question something that people really care about and find an answer that may surprise them -- that is, if you can overturn the conventional wisdom -- then you may have some luck.}
    \\ \\
    \hspace*{\fill}
    Steven \textsc{Levitt} et Stephen \textsc{Dubner}, \textit{Freakonomics}, 2009~\cite{levittFreakonomicsRogueEconomist2009}
}

\boitesimple{
    \footnotesize \textit{Third thing about computers, they're really dumb. They're exceptionally simple, but they're really fast. The raw instructions that we have to feed these little microprocessors, even the raw instructions that we have to feed these giant Cray-1 supercomputers, are the most trivial of instructions. {\NoAutoSpacing They're:} Get some data from here, get a number from here, fetch a number, add two numbers together, test to see if it's bigger than zero. Go put it over there. It's the most mundane thing you could ever imagine.
    \\ \\
    But a key thing about it is that, let's say I could move 100 times faster than anyone in here. In the blink of your eye, I could run out there and I could grab a bouquet of fresh spring flowers or something. And I could run back in here and I could snap my fingers, and you would all think I was a magician or something. And yet I was basically doing a series of really simple {\NoAutoSpacing instructions:} moving, running out there, grabbing some flowers, running back, snapping my fingers. But I could just do them so fast that you would think that there was something magical going on.}
    \\ \\
    \hspace*{\fill}
    Steve \textsc{Jobs}, \textit{International Design Conference in Aspen}, 1983~\cite{ObjectsOurLife}
}

\vspace*{\fill}
