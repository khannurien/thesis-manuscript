% Ce fichier main.tex est le fichier principal \`{a} partir duquel tout est g\'{e}n\'{e}r\'{e}
% This file is the main file where the final document is generated
\documentclass{these-dbl}

% Remplir les metadonnees du pdf
% Fill the pdf metadata
\hypersetup{
%    pdfauthor   = {XYZ},
%    pdftitle    = {Th\`{e}se de doctorat de XYZ},
%    pdfsubject  = {Th\`{e}se de doctorat de XYZ},
%    pdfkeywords = {mots-cl\'{e}s},
}

\geometry{vmargin=4.0cm}

% Spécifier vos fichiers de bibliographie
% Specify you bibliography files here
\addbibresource{./Bibliographie/biblio.bib}
\addbibresource{./Bibliographie/sources.bib}
\addbibresource{./Bibliographie/herofake.bib}
\addbibresource{./Bibliographie/herocache.bib}
\addbibresource{./Bibliographie/herosim.bib}

% Maths
\usepackage{amsmath,amssymb,amsfonts}
% Valeurs numériques
% \usepackage[allow-quantity-breaks]{siunitx}
% Polices
\usepackage{pifont}% http://ctan.org/pkg/pifont
\newcommand{\cmark}{\color{YellowGreen}\ding{51}}%
\newcommand{\xmark}{\color{BrickRed}\ding{55}}%
% Figures
\usepackage{float}
\usepackage[caption=false]{subfig}
\usepackage[inkscapelatex=false]{svg}
% Marqueurs
\usepackage{circledsteps}
\pgfkeys{/csteps/inner color=white}
\pgfkeys{/csteps/fill color=black}
% Tableaux
\usepackage{array}
\usepackage{multirow}
\usepackage{tabularx}
\usepackage{booktabs}
\newcolumntype{Y}{>{\centering\arraybackslash}X}
% Centered tabular column with text wrapping (L)
% \newcolumntype{L}{>{\centering\arraybackslash}m{0.6\linewidth}}
\newcolumntype{L}{>{\arraybackslash}m{0.75\linewidth}}
\newcolumntype{S}{>{\centering\arraybackslash}m{0.1\linewidth}}
% for vertical centering text in X colum
\renewcommand\tabularxcolumn[1]{m{#1}}

% Tableaux multi-pages
% \usepackage{tabularray,adjustbox}
% \UseTblrLibrary{booktabs}
% % make tabularray use \caption instead of own caption style
% \makeatletter\def\captionofnoinc#1{\def\@captype{#1}\addtocounter{#1}{-1}\caption}\makeatother\DefTblrTemplate{firsthead}{default}{\captionofnoinc{table}{\InsertTblrText{caption}}}\DefTblrTemplate{middlehead}{default}{\captionofnoinc{table}{\InsertTblrText{caption} \InsertTblrText{middlehead-text} (Suite)}}
% \DefTblrTemplate{lasthead}{default}{\captionofnoinc{table}{\InsertTblrText{caption} \InsertTblrText{lasthead-text} (Suite)}}

%%%%%%%%%%%%%%%%%%%%%%%%%%%%%%%%%%%%%%%%%%%%%%%%%%%%%%%%%%%%%%%%%%%%%%%%%%%%%%%%%%%%%
\newboolean{showcomments}
\setboolean{showcomments}{true}
\ifthenelse{\boolean{showcomments}}
{ \newcommand{\mynote}[3]{
    \fbox{\bfseries\sffamily\scriptsize#1}
    {\small$\blacktriangleright$\textsf{\emph{\color{#3}{#2}}}$\blacktriangleleft$}}}
{ \newcommand{\mynote}[3]{}}
\newcommand{\shrink}[1]{}
\newcommand{\jb}[1]{\mynote{Jalil}{#1}{blue}}
\newcommand{\vl}[1]{\mynote{Vincent}{#1}{red}}
%%%%%%%%%%%%%%%%%%%%%%%%%%%%%%%%%%%%%%%%%%%%%%%%%%%%%%%%%%%%%%%%%%%%%%%%%%%%%%%%%%%%%

\begin{document}

% Page de garde avec commande \maketitle
% Front cover calling \maketitle
% La page de garde est en français
% The front cover is in French
\selectlanguage{french}

% Inclure les infos de chaque établissement
% Include each institution data
\input{./Couverture/liste-ecoles-etablissements.tex}

% Inclure infos de l'école doctorale
% Include doctoral school data
\ecoledoctorale{SPIN}

% Inclure infos de l'établissement
% Include institution data
\etablissement{ENSTA}

%Inscrivez ici votre sp\'{e}cialit\'{e} (voir liste des sp\'{e}cialit\'{e}s sur le site de votre \'{e}cole doctorale)
%Indicate the domain (see list of domains in your ecole doctorale)
\spec{STIC -- Informatique}

%Attention : le pr\'{e}nom doit être en minuscules (Jean) et le NOM en majuscules (BRITTEF) 
%Attention : the first name in small letters and the name in Capital letters 
\author{Vincent LANNURIEN}

% Donner le titre complet de la th\`{e}se, \'{e}ventuellement le sous titre, si n\'{e}cessaire sur plusieurs lignes 
%Give the complete title of the thesis, if necessary on several lines
\title{Allocation et placement dynamiques sur ressources hétérogènes pour le cloud serverless}
% \lesoustitre{Optimisation de l'allocation dynamique des ressources matérielles et du placement des requêtes utilisateur dans le modèle de service serverless}

%Indiquer la date et le lieu de soutenance de la th\`{e}se 
%indicates the date and the place of the defense 
\date{20 novembre 2024}
\lieu{Brest}

%Indiquer le nom du (ou des) laboratoire (s) dans le(s)quel(s) le travail de th\`{e}se a \'{e}t\'{e} effectu\'{e}, indiquer aussi si souhait\'{e} le nom de la (les) facult\'{e}(s) (UFR, \'{e}cole(s), Institut(s), Centre(s)...), son (leurs) adresse(s)... 
%Indicates the name (or names) of research laboratories where the work has been done as well as (if desired) the names of faculties (UFR, Schools, institution...
\uniterecherche{Lab-STICC, CNRS, UMR 6285, équipe SHAKER}

%Indiquer le Numero de th\`{e}se, si cela est opportun, ou laisser vide pour faire disparaitre cet ligne de la couverture
%Indicate the number of the thesis if there is one. otherwise leave empty so the line disappeurs on the cover
% \numthese{« si pertinent »} % \numthese{}

%Indiquer le Pr\'{e}nom en minuscules et le Nom en majuscules, le titre de la personne et l’\'{e}tablissement dans lequel il effectue sa recherche  
%Indicates the first name on small letters and the Names on capital letters, the person's title and the institution where he/she belongs to.
%Exemples :  Examples :
%%%- Professeur, Universit\'{e} d’Angers 
%%%- Chercheur, CNRS, \'{e}cole Centrale de Nantes 
%%%-  Professeur d’universit\'{e} – Praticien Hospitalier, Universit\'{e} Paris V  
%%%-  Maitre de conf\'{e}rences, Oniris 
%%%- Charg\'{e} de recherche, INSERM, HDR, Universit\'{e} de Tours  
 %S’il n’y a pas de co-direction, faire disparaitre cet item de la couverture  
 %In there is no co-director, remove the item from the cover
\jury{
{\normalTwelve \textbf{Rapporteurs avant soutenance :}}\\ \newline
\footnotesizeTwelve
\begin{tabular}{@{}ll}
Laurent LEFÈVRE     & Professeur, ENS Lyon \\
Thomas LE DOUX      & Professeur, IMT Atlantique \\
\end{tabular}

\vspace{\baselineskip}
{\normalTwelve \textbf{Composition du Jury :}}\\
% {\fontsize{9.5}{11}\selectfont {\textcolor{red}{\textit{Attention, en cas d’absence d’un des membres du Jury le jour de la soutenance, la composition du jury doit être revue pour s’assurer qu’elle est conforme et devra être répercutée sur la couverture de thèse}}}}\\ \newline
\footnotesizeTwelve
\begin{tabular}{@{}lll}
Pr\'{e}sident :        & Pr\'{e}nom NOM      & Fonction et \'{e}tablissement d'exercice \\
Examinateurs :         & Laurent LEFÈVRE     & Professeur, ENS Lyon \\
                       & Thomas LE DOUX      & Professeur, IMT Atlantique \\
                       & Anne-Cécile ORGERIE & Directrice de recherche, IRISA \\
Dir. de th\`{e}se :    & Jalil BOUKHOBZA     & Professeur, ENSTA Bretagne \\
Co-dir. de th\`{e}se : & Laurent D'ORAZIO    & Professeur, Université de Rennes \\
                       & Olivier BARAIS      & Professeur, Université de Rennes \\
\end{tabular}

\vspace{\baselineskip}
{\normalTwelve \textbf{Invit\'{e}(s) :}}\\ \newline
\footnotesizeTwelve
\begin{tabular}{@{}ll}
Stéphane PAQUELET & Direction Adjointe Performance de l'Innovation, IRT b{\textless\textgreater}com \\
\end{tabular}
}


\maketitle


% Sélectionner la langue du contenu suivant cette ligne
% Select the content language following this line
\selectlanguage{french}

% Inclusion du chapitre remerciement
% Input acknowledgement chapter
\clearemptydoublepage
\chapter*{Remerciements}

\boitesimple{
    \footnotesize \textit{Third thing about computers, they're really dumb. They're exceptionally simple, but they're really fast. The raw instructions that we have to feed these little microprocessors, even the raw instructions that we have to feed these giant Cray-1 supercomputers, are the most trivial of instructions. They're: Get some data from here, get a number from here, fetch a number, add two numbers together, test to see if it's bigger than zero. Go put it over there. It's the most mundane thing you could ever imagine.
    \\ \\
    But a key thing about it is that, let's say I could move 100 times faster than anyone in here. In the blink of your eye, I could run out there and I could grab a bouquet of fresh spring flowers or something. And I could run back in here and I could snap my fingers, and you would all think I was a magician or something. And yet I was basically doing a series of really simple instructions: moving, running out there, grabbing some flowers, running back, snapping my fingers. But I could just do them so fast that you would think that there was something magical going on.}
    \\ \\
    \hspace*{\fill}
    Steve Job's talk at the 1983 International Design Conference in Aspen~\cite{ObjectsOurLife}
}


% Ne pas oublier cette commande qui g\'{e}n\`{e}re la page de couverture avant
% This command will generate the front cover
\frontmatter
\clearemptydoublepage
\renewcommand{\contentsname}{Table des matières}
\tableofcontents %sommaire %table of content
%\shorttableofcontents{Sommaire}{0}

\clearemptydoublepage
\chapter*{Publications}

\section*{International conferences}

\begin{enumerate}
    \item \underline{Vincent Lannurien}, Laurent D’Orazio, Olivier Barais, Esther Bernard, Olivier Weppe, Laurent Beaulieu, Amine Kacete, Stéphane Paquelet, Jalil Boukhobza, \textit{{\NoAutoSpacing HeROfake:} Heterogeneous Resources Orchestration in a Serverless Cloud -- An Application to Deepfake Detection}, \textbf{IEEE/ACM 23rd International Symposium on Cluster, Cloud and Internet Computing} (CCGrid), 2023, Bangalore, India
    \item \underline{Vincent Lannurien}, Camélia Slimani, Laurent D’Orazio, Olivier Barais, Stéphane Paquelet, Jalil Boukhobza, \textit{{\NoAutoSpacing HeROcache:} Storage-Aware Scheduling in Heterogeneous Serverless Edge – The Case of IDS}, \textbf{IEEE/ACM 24rd International Symposium on Cluster, Cloud and Internet Computing} (CCGrid), 2024, Philadelphia, USA
\end{enumerate}

\section*{Book chapter}

\begin{enumerate}
    \item \underline{Vincent Lannurien}, Laurent D’Orazio, Olivier Barais, Jalil Boukhobza, \textit{{\NoAutoSpacing Serverless Computing:} State of the Art and Challenges}, Krishnamurthi, R., Kumar, A., Gill, S.S., Buyya, R. (éditeurs), \textbf{{\NoAutoSpacing Serverless Computing:} Principles and Paradigms}. Lecture Notes on Data Engineering and Communications Technologies, vol 162. Springer, Cham.
\end{enumerate}


\clearemptydoublepage
\chapter{Introduction}
\label{chapter:introduction}
%\addcontentsline{toc}{chapter}{Introduction}
%\chaptermark{Introduction}

\section{Contexte scientifique}
%\addcontentsline{toc}{section}{Motivation}

En 1961, John McCarthy donne un discours pour célébrer les cent ans du MIT~\cite{greenberger1962management}. Il imagine alors que le partage du temps de calcul des ordinateurs pourrait permettre de vendre leur puissance d'exécution comme un service, à l'image de l'eau ou de l'électricité. Le matériel serait organisé de manière à rendre possible sa location à des clients qui paieraient ce service en fonction du volume, ou du temps d'utilisation.

Grâce à la démocratisation de l'accès à Internet à haut débit, au milieu des années 2000, l'idée de McCarthy se trouve implémentée dans ce que l'on appelle \textit{cloud computing}~\cite{hayesCloudComputing2008} (ou \textit{cloud}) : entreprises et particuliers peuvent dorénavant réduire drastiquement les coûts associés à l'achat et à l'entretien du matériel nécessaire au fonctionnement de leurs applications, en déléguant la responsabilité de l'infrastructure à des fournisseurs de services, qui bénéficient d'effets d'économie d'échelle en concentrant ces ressources dans d'immenses centres de données.

Ce modèle est appelé \textit{Infrastructure as a Service} (IaaS) : les clients louent et réservent une sous-partie des ressources du fournisseur (calcul, stockage, réseau) dont ils deviennent responsables du fonctionnement~\cite{mellNISTDefinitionCloud}. Dans ce modèle, le client est intégralement en charge du dimensionnement des ressources dédiées à son application ; il pourra avoir tendance à surestimer ses besoins en ressources, de manière à s'assurer que l'application déployée soit capable d'absorber la montée en charge lors de pics d'activité~\cite{takMoveNotMove}. De nouvelles évolutions apparaissent au fil des années avec pour objectif de diminuer la surface des responsabilités du client. Par exemple, dans le modèle \textit{Platform as a Service} (PaaS), les clients n'ont pas directement accès aux machines qui supportent leurs applications, et effectuent la plupart des tâches de gestion via des interfaces spécialisées. Ici, lorsque l'application n'est pas utilisée, elle reste toutefois déployée et occupe donc des ressources matérielles. Enfin, dans une offre \textit{Software as a Service} (SaaS), le fournisseur de services héberge et administre des applications, pour lesquelles il facture au client le droit d'en être utilisateur. Cette facturation est souvent forfaitaire, quelle que soit l'utilisation faite de l'application par le client.

Dans ces trois modèles traditionnels, le client paie pour des ressources qui sont parfois dormantes. En effet, la mesure de l'usage du service s'effectue dans chacun des cas sur la quantité de ressources engagées pour sa mise en œuvre par le fournisseur. C'est un problème largement documenté dans la littérature, qui considère que le taux d'usage des ressources dans un centre de données cloud peut être inférieur à 15\% en moyenne~\cite{vasanWorthTheirWatts2010, vermaLargescaleClusterManagement2015a}.

Cette valeur faible est expliquée par un ensemble de facteurs. D'abord, une part du matériel disponible dans un centre de données est déployée pour palier les éventuelles pannes -- celles-ci peuvent affecter 3\% des machines chaque semaine~\cite{BareMetal70B}. Ensuite, un centre de données est toujours dimensionné par rapport aux pics d'utilisation, qui peuvent être liés à des événements récurrents comme le Super Bowl aux États-Unis~\cite{wangTouchdownCloudImpact2019}, ou imprévisibles comme la crise sanitaire de 2020~\cite{alashhabImpactCoronavirusPandemic2021} : c'est une marge de sécurité, afin d'assurer la continuité du service. Enfin, le cloud est un secteur économique qui connaît une croissance soutenue, autour de 17\% par an~[TODO], ce qui explique des investissements réguliers dans du nouveau matériel pour tenir compte de la demande future.

Malgré tout, lorsque l'on met ce faible usage des ressources en regard de la consommation d'énergie estimée pour ces même centres de données, soit environ 1,5\% de la consommation d'électricité mondiale en 2010~\cite{masanetRecalibratingGlobalData2020} et peut-être jusqu'à 3\% en 2029~[TODO], il semble urgent de s'intéresser à l'optimisation des opérations cloud.

En effet, l'accroissement de l'efficacité des architectures matérielles et logicielles n'a pas permis de contrer un certain effet rebond : si le rapport, pour un serveur typique du cloud, entre énergie consommée et calculs effectués a été divisé par quatre entre 2010 et 2020~\cite{masanetRecalibratingGlobalData2020}, la demande en matériel ne cesse de croître, poussée notamment par les immenses besoins en apprentissage machine~\cite{commentMetaOperate6002024}. Les investissements cumulés des leaders du secteur en serveurs dédiés à l'intelligence artificielle (IA) pourraient quintupler entre 2022 et 2025~\cite{DerriereIADeferlante2024, elderNextWaveAI2024}. L'Agence Internationale de l'Énergie (AIE) estime que la consommation des centres de données pourrait doubler d'ici à 2026~\cite{Electricity2024Analysis2024}.

Une métrique couramment utilisée pour mesurer l'efficacité énergétique d'un centre de données est le PUE (\textit{Power Usage Effectiveness}). Cet indicateur mesure le rapport entre l'énergie totale engagée par le centre de données sur l'énergie engagée pour les opérations (calcul, stockage, etc.), et devrait donc tendre vers $1$ dans un monde idéal. En 2022, le PUE moyen se situe autour de 1,55~\cite{davisUptimeInstituteGlobal2022}, principalement gonflé par les besoins en refroidissement des machines.

Pourtant, un serveur éteint n'a pas besoin d'être refroidi. Comment expliquer alors que les ressources matérielles restent dans un état de stase, augmentant la consommation des centres de données sans générer de revenus ? Ne serait-il pas possible d'envisager un modèle dans lequel les ressources sont allouées au plus proche des besoins, laissant la possibilité aux fournisseurs de services de mettre en œuvre des politiques d'extinction pour les serveurs inutilisés ?

Au cours des années 2010, des acteurs du cloud public proposent de nouvelles déclinaisons de leurs offres commerciales, le modèle \textit{Function as a Service} (FaaS). En particulier, Amazon propose Lambda~\footnote{\href{https://aws.amazon.com/fr/lambda/}{https://aws.amazon.com/fr/lambda/}} en 2014. Avec Lambda, la granularité de réservation des ressources se déplace de la \textit{quantité} vers la \textit{durée} : les clients fournissent à Lambda le code qu'ils souhaitent exécuter sur la plateforme cloud, et les ressources nécessaires à cette exécution sont allouées \textit{lorsque} nécessaire, et \textit{autant} que nécessaire, c'est-à-dire à la hauteur des requêtes utilisateur reçues par l'application. Tant que l'application est inutilisée, les ressources sont libérées. Dès que la charge augmente, les ressources allouées augmentent pour absorber le trafic.

Le déplacement de la responsabilité du dimensionnement des ressources allouées aux applications du client vers le fournisseur de services ouvre de nouvelles possibilités d'optimisation pour ce dernier, en appliquant des stratégies intelligentes de gestion des ressources.

Le fournisseur de services a tout intérêt à optimiser l'orchestration des charges de travail dans ses centres de données. D'une part, cela permet de garantir un niveau satisfaisant de qualité de service aux clients, et donc d'éviter le paiement de pénalités lorsque les accords de niveau de service sont violés. D'autre part, sur les aspects réglementaires, de nouvelles exigences émergent en matière de mesure de la consommation d'énergie et de l'empreinte carbone pour les opérateurs de centres de données~\cite{davisUptimeInstituteGlobal2022}.

Si l'orchestration dynamique apparaît comme étant une solution toute trouvée aux problèmes d'usage des ressources dans le cloud, ce mécanisme amène avec lui un certain nombres de défis à relever~\cite{Lannurien2023}. Pour optimiser, il faut prendre en compte la diversité des ressources à disposition : le cloud est un environnement hautement hétérogène, dans lequel le matériel présente des niveaux de performances et de coût très divers~\cite{reissHeterogeneityDynamicityClouds}. Par ailleurs, il ne s'agit pas uniquement de proposer les performances les plus hautes de manière indiscriminée. Par exemple, de nombreux utilisateurs de services cloud n'ont pas besoin de garanties fortes en matière de latence~\cite{tirmaziBorgNextGeneration2020}. Ainsi, une connaissance fine des utilisateurs et des charges de travail est requise pour mettre en œuvre des politiques pertinentes de répartition de charge ou de consolidation des tâches.

\section{Problématiques de recherche}
%\addcontentsline{toc}{section}{Problématiques de recherche}

Les travaux menés au cours de cette thèse s'inscrivent dans le cadre du modèle de service serverless pour le cloud privé. Nous avons considéré un environnement dans lequel un fournisseur de services dispose de ressources matérielles que des développeurs peuvent exploiter pour déployer leurs applications à destination d'une variété d'utilisateurs. Nous avons cherché à concevoir une plateforme d'orchestration qui soit en mesure de garantir la qualité de service pour les utilisateurs tout en minimisant les coûts pour le fournisseur de services. Une progression s'est opérée, en partant de l'ordonnancement de fonctions simples (sans dépendances et sans état), pour arriver au déploiement d'applications complexes (dépendances temporelles et de données entre les fonctions des applications). Il nous a également fallu développer un environnement d'évaluation des différentes politiques que nous avons proposées.

Cette section résume les problèmes que nous avons identifiés et les questions de recherche que nous avons posées dans le but de lever ces verrous.

\begin{center}
    \rule{4cm}{0.4pt}
\end{center}

\textbf{Problème 1} -- L'allocation de ressources au sein d'une plateforme serverless se fait sur un mode réactif, c'est-à-dire en réponse à une variation du trafic. Lorsque de nouvelles instances des fonctions doivent être déployées, cela provoque une latence additionnelle qui peut dégrader la qualité de service pour les utilisateurs. Le problème est particulièrement saillant dans l'environnement hautement hétérogène qu'est celui du cloud : les ressources matérielles ont différents niveaux de performances et de coût ; les applications interactives présentent des motifs d'usage difficilement prédictibles ; et les utilisateurs ont des besoins irréguliers et variés en qualité de service.

\boitemagique{Question 1 (\textbf{QR1})}{
    Comment dimensionner les allocations dynamiques de ressources hétérogènes pour une application interactive, et comment ordonnancer efficacement les requêtes utilisateur, lorsque ces derniers ont des besoins variés en matière de qualité de service ?
}

\begin{center}
    \rule{4cm}{0.4pt}
\end{center}

\textbf{Problème 2} -- Ordonnancer des applications complexes, constituées d'un ensemble de fonctions qui présentent des relations de dépendances et communiquent entre elles des résultats intermédiaires, soulève le problème du stockage dans le modèle serverless. Les solutions commerciales s'appuient sur un stockage lent, accédé par le réseau, qui peut provoquer des délais à chaque étage de l'application. Ces délais s'accumulent alors et provoquent un effet boule de neige, dégradant drastiquement la qualité de service. Une possibilité existe d'exploiter le stockage local aux nœuds de calcul, mais induit un risque de contention sur les ressources, qui peuvent être limitées en capacité et en performances dans le cadre de l'informatique en périphérie (ou \textit{edge}).

\boitemagique{Question 2 (\textbf{QR2})}{
    Comment prendre en compte les délais d'initialisation et de communications lorsque l'on déploie des chaînes de fonctions de courte durée, et comment tirer parti de l'hétérogénéité des nœuds à disposition, pour respecter la qualité de service requise par les utilisateurs et contenir la consommation d'énergie de l'infrastructure ?
}

\begin{center}
    \rule{4cm}{0.4pt}
\end{center}

\textbf{Problème 3} -- Évaluer des politiques d'orchestration pour un environnement cloud privé peut se faire de deux manières différentes : par expérimentation, c'est-à-dire avec du matériel et des utilisateurs réels, ou en simulation. La méthode expérimentale se heurte à des contraintes de coût d'une part, car il faut réserver une partie de l'infrastructure pour mener les campagnes de mesure. D'autre part, il est complexe de mener une analyse \textit{what-if} en production, c'est-à-dire de se laisser la liberté de changer un certain nombre de paramètres en entrée pour évaluer leur impact (fréquence des requêtes, caractéristiques du matériel, etc.). Les simulateurs à l'état de l'art pour le cloud, quant à eux, présentent des limites qui rendent délicate la modélisation d'un environnement serverless permettant de comparer les performances de différentes stratégies de gestion des ressources.

\boitemagique{Question 3 (\textbf{QR3})}{
    Du point de vue d'un fournisseur de services pour le cloud privé, comment évaluer l'impact sur la qualité de service de différentes politiques d'allocation de ressources et d'ordonnancement de tâches dans le modèle serverless ?
}

\section{Contributions}
%\addcontentsline{toc}{section}{Contributions}

\subsection{Caractérisation et orchestration de tâches simples pour la détection de deepfake}

La \textbf{QR1} (\textit{dimensionnement et ordonnancement sur ressources hétérogènes}) a émergé d'une réflexion autour du déploiement d'une application interactive sur des ressources non-réservées au sein de l'IRT b{\textless\textgreater}com~\footnote{\href{https://b-com.com/}{https://b-com.com/}}. Notre cas d'étude concerne une application de détection de deepfake -- des images générées dans un but malicieux par une machine, avec l'objectif de tromper leur destinataire. Cette application est réveillée par l'envoi, par un utilisateur, d'une image à classifier. Elle s'appuie sur des algorithmes d'apprentissage machine pour déterminer si l'image est en effet synthétique ou bien légitime.

Pour répondre à la \textbf{QR1}, nous avons proposé HeROfake (chapitre~\ref{chapter:herofake}), une plateforme d'orchestration serverless qui vise à réaliser de manière intelligente l'allocation des ressources et l'ordonnancement des requêtes, dans le but d'optimiser l'orchestration pour la qualité de service tout en minimisant la consommation d'énergie du système. Nous avons mené une campagne de mesures afin de caractériser, d'une part, un ensemble de ressources matérielles hétérogènes (CPU, GPU, FPGA), et d'autre part, un ensemble de fonctions d'inférence utilisées par l'application et basées sur des réseaux de neurones. Cette phase de caractérisation permet d'obtenir des métadonnées sur le niveau de performances et de consommation de ressources pour l'application.

Ces métadonnées sont ensuite utilisées par l'orchestrateur lors d'une phase en ligne, afin d'ajuster ses décisions d'allocation et d'ordonnancement tout au long d'un scénario, rejoué en simulation dans un environnement ad-hoc. Nous avons établi un modèle de coût pour cet orchestrateur, puis conçu une politique d'optimisation multi-objectifs visant à minimiser ces fonctions de coût, nourries par les métadonnées issues de la phase de mesures.

Cette contribution a fait l'objet d'une publication dans la conférence CCGrid'23~\cite{herofake}.

\subsection{Modèle de coût pour l'orchestration serverless d'applications complexes à l'\textit{edge}}

La \textbf{QR2} (\textit{prise en compte des coûts associés au stockage dans le serverless}) a émergé pour deux raisons principales. D'une part, comme la suite logique des travaux menés sur la première contribution : si notre précédente contribution permet d'orchestrer des fonctions sans état, nous savons que la plupart des applications ont besoin de stocker des données intermédiaires pendant leur exécution. D'autre part, une étude de la littérature en matière d'orchestration serverless nous a permis de constater que de nombreuses contributions prenaient souvent comme point de départ une situation très favorable, considérant que les images servant à déployer les fonctions sont toujours disponibles sur les nœuds de calcul, et que les communications entre les fonctions sont transparentes du point de vue des performances.

Pour répondre à la \textbf{QR2}, nous avons cherché à établir un modèle de coût pour l'allocation de ressources et l'ordonnancement des tâches qui prenne en compte ces deux mécanismes liés au stockage. Notre cas d'étude pour cette contribution est une application de détection d'instrusions, déployée à l'edge, sur des nœuds hétérogènes aux ressources contraintes et limités en nombre par l'énergie disponible. Nous avons proposé HeROcache (chapitre~\ref{chapter:herocache}), une stratégie d'orchestration qui cherche à maximiser la consolidation des fonctions d'une même application sur les mêmes nœuds, afin de minimiser la consommation d'énergie de l'infrastructure. Grâce à un mécanisme de mise en cache des données sur le stockage local, cette stratégie permet de réduire les délais de communication des applications et limite les violations de qualité de service.

Cette contribution a fait l'objet d'une publication dans la conférence CCGrid'24~\cite{herocache}.

\subsection{Simulation à événements discrets pour l'évaluation de politiques d'orchestration}

% données de production (traces d'exécution), analyse (graphe des appels, données intermédiaires), caractérisation (métriques de performances)

La \textbf{QR3} (\textit{évaluer et comparer différentes politiques d'orchestration serverless}) a été transversale tout au long des travaux de thèse. Concevoir des stratégies d'allocation et d'ordonnancement pour le cloud demande d'avoir à notre disposition un environnement dans lequel les évaluer.

Pour répondre à la \textbf{QR3}, nous avons développé et proposé HeROsim (chapitre~\ref{chapter:herosim}), un simulateur \textit{open source} à événements discrets pour le cloud privé serverless, qui présente les caractéristiques nécessaires à la modélisation fine d'un environnement serverless et l'évaluation de politiques d'orchestration. La simulation progresse à une granularité au niveau d'une requête utilisateur, et permet de modéliser des applications complexes (dépendances entre les fonctions) et des environnements hétérogènes (nœuds à différents niveaux de performances, requêtes à différents niveaux de qualité de service). HeROsim permet de rejouer des scénarios (sur la base de traces d'exécution) et de comparer les résultats de différentes politiques au regard de métriques de qualité de service et de consommation d'énergie.

Cette contribution a fait l'objet d'une soumission au journal IEEE Internet Computing~\cite{herosim} en mai 2024.

\clearpage

\section{Organisation de la thèse}
%\addcontentsline{toc}{section}{Organisation du mémoire}

Ce manuscrit est composé de sept chapitres (dont cette introduction), organisés comme suit :

\begin{center}
    \rule{4cm}{0.4pt}
\end{center}

\textbf{Partie~\ref{part:one} : Contexte et état de l'art}

Le chapitre~\ref{chapter:context} décrit l'environnement général des travaux de la thèse, et donne en particulier les caractéristiques du cloud et du modèle serverless. Le chapitre~\ref{chapter:sota} présente des contributions de l'état de l'art dans le domaine de l'orchestration dynamique pour le cloud.

\begin{center}
    \rule{4cm}{0.4pt}
\end{center}

\textbf{Partie~\ref{part:two} : Contributions}

Cette partie présente les trois contributions de la thèse. Le chapitre~\ref{chapter:herofake} décrit notre solution d'allocation et d'ordonnancement dynamiques sous contrainte de qualité de service pour des fonctions simples, ainsi que notre méthodologie de caractérisation des plateformes et des tâches. Le chapitre~\ref{chapter:herocache} présente notre orchestrateur s'appuyant sur un modèle de coût intégrant les problématiques de stockage pour le serverless. Le chapitre~\ref{chapter:herocache} détaille l'environnement de simulation développé dans le cadre de la thèse pour évaluer des politiques d'orchestration serverless pour le cloud privé.

\begin{center}
    \rule{4cm}{0.4pt}
\end{center}

\textbf{Partie~\ref{part:three} : Conclusion et perspectives}

Enfin, le chapitre~\ref{chapter:conclusion} résume les contributions de la thèse et présente les limites identifiées dans les solutions proposées, pour ouvrir la discussion sur des perspectives de futurs travaux.


\part{Contexte et état de l'art}
\label{part:one}

\clearemptydoublepage
\mainmatter
\clearemptydoublepage
\chapter{Contexte}
\label{chapter:context}

\section{Cloud computing}

TODO: Intro générale sur le cloud (définition en un paragraphe)

TODO: Version propre de la figure du tech talk

TODO: Historique, quelques développements, situation contemporaine (acteurs commerciaux, solutions open source ?)

TODO: Ouverture sur les enjeux ?

\subsection{Caractéristiques}

Le NIST donne une définition formelle du cloud~\cite{mellNISTDefinitionCloud} en listant les caractéristiques essentielles d'une telle plateforme :

\begin{itemize}
    \item \textbf{Service à la demande} -- Les clients réservent des ressources matérielles de manière autonome, par exemple au travers d'une interface web, sans interagir avec un opérateur. En retour, ils n'ont généralement pas de contrôle fin sur la localité précise des ressources réservées ;
    \item \textbf{Accessible par le réseau} -- Ces ressources sont immédiatement mises à disposition des clients et accessibles par Internet ;
    \item \textbf{Partage des ressources} -- La puissance de calcul, les capacités de stockage et la bande passante sont partagées entre les clients du fournisseur de services. Des techniques de virtualisation sont mises en œuvre pour isoler les tâches déployées ;
    \item \textbf{Élasticité rapide} -- Les clients peuvent à tout moment décider d'augmenter ou de diminuer la quantité et les caractéristiques des ressources qu'ils réservent, de manière à garantir les performances de leurs applications ou maîtriser leurs coûts ;
    \item \textbf{Service mesuré} -- Les infrastructures cloud sont instrumentées de manière à fournir aux clients une information précise sur leur consommation de ressources, et les coûts monétaires associés.
\end{itemize}

\subsection{Modèles de service}

\begin{figure}[htbp]
    \centering
	\includegraphics[width=\textwidth]{2_Chapitre2/figures/service-models.png}
	\caption[Comparaison entre différents modèles de service pour le cloud en termes de responsabilités pour le client et le fournisseur de service.]{Comparaison entre différents modèles de service pour le cloud en termes de responsabilités pour le client et le fournisseur de service (inspiré de la documentation Red Hat \protect \footnotemark).}
	\label{fig:service-model}
\end{figure}

\footnotetext{\url{https://www.redhat.com/en/topics/cloud-computing/iaas-vs-paas-vs-saas}}

Ces caractéristiques sont déclinées dans trois modèles de service, comme illustré par la figure~\ref{fig:service-model}, qui constituent des déclinaisons de l'offre commerciale des fournisseurs de service cloud :

\begin{itemize}
    \item \textbf{Software as a Service} (SaaS) -- Cible l'utilisateur final en offrant l'accès à une application entièrement administrée par le fournisseur de services ;
    \item \textbf{Platform as a Service} (PaaS) -- Cible les clients qui souhaitent déployer leurs applications sans avoir la responsabilité d'administration des serveurs ;
    \item \textbf{Infrastructure as a Service} (IaaS) -- Cible les clients qui souhaitent un contrôle à grain fin sur leurs infrastructures. Les clients d'une offre IaaS sont responsables de l'administration de leurs serveurs, souvent virtuels. 
\end{itemize}

\subsection{Modèles de déploiement}

Il existe plusieurs topologies dans le cloud, qui correspondent à différentes contraintes métier pour les utilisateurs :

\begin{itemize}
    \item \textbf{Cloud privé} -- L'infrastructure est dédiée à une organisation qui regroupe plusieurs utilisateurs finaux. Ce modèle de déploiement est privilégié pour la sécurité et la confidentialité des données ;
    \item \textbf{Cloud public} -- L'infrastructure est partagée entre de nombreux clients hétérogènes, professionnels comme particuliers ;
    \item \textbf{Cloud communautaire} -- L'infrastructure est partagée entre différents acteurs ayant souvent des problématiques métier similaires (secteurs bancaire ou hospitalier par exemple) ;
    \item \textbf{Cloud hybride} -- Solution de répartition des tâches entre cloud privé et cloud public, en fonction de leur niveau de criticité.
\end{itemize}

\subsection{La promesse du cloud computing}

Dans les faits, l'élasticité rapide, considérée par le NIST comme une caractéristique essentielle du cloud, au sens de mise à l'échelle dynamique des ressources en adéquation avec les besoins des applications déployées, n'est pas une caractéristique des offres traditionnelles dans le cloud~\cite{herbstElasticityCloudComputing}. Il incombe aux développeurs de planifier à l'avance et de spécifier leurs besoins, c'est-à-dire de réserver une quantité adéquate de ressources. Ces ressources sont généralement appelées "instances" par les fournisseurs de services cloud. Les instances cloud se distinguent généralement en fonction de leurs spécifications en termes de type de ressources et de capacité : par exemple, on peut trouver des instances avec de nombreux cœurs CPU, tandis que d'autres donnent accès à un GPU ou à un autre accélérateur matériel.

Le choix du ou des types d'instances et de leur quantité pour une application dépend a) de la nature des calculs effectués, et b) de la latence acceptable et du débit souhaité~\cite{yallesRISCLESSReinforcementLearning}. Il est de la responsabilité du client de ne pas surprovisionner au-delà de ses besoins réels.

Cette conception de l'offre a d'autres conséquences. Tout d'abord, cela signifie que la facturation est faite à gros grain : par instances réservées, plutôt que par ressources réellement utilisées. En outre, le coût des ressources inactives incombe au client. Lorsque l'application déployée ne traite aucune requête, elle reste dans un état dormant, en attente d'une nouvelle demande entrante.

Les plateformes cloud doivent prendre en charge un nombre important de traitements, ce qui entraîne une situation massivement multi-tenant qui nécessite des techniques d'isolation et de virtualisation adéquates. Celles-ci sont présentées dans la section suivante.

\section{Ordonnancement dans le cloud}

\subsection{Virtualisation dans un cadre multi-tenant}

Le cadre multi-tenant est une caractéristique déterminante du cloud. Il s'agit de la capacité d'un fournisseur de services cloud à partager des ressources entre plusieurs clients afin de réaliser des économies~\cite{weissmanDesignForceCom2009}. Comme les ressources sont mises en commun et que différentes applications les utilisent, cela ouvre des canaux, auxiliaires ou non, entre les processus dans l'espace utilisateur~\cite{pedersen2017trash, wu2018side}. Bénéficier du cadre multi-tenant s'accompagne donc de la responsabilité, pour le fournisseur, de garantir la confidentialité et la sécurité des données et des différentes charges de travail des clients~\cite{vaqueroLockingSkySurvey2011}.

Pour respecter ces garanties, les fournisseurs doivent recourir à des mesures de protection pour assurer le cloisonnement entre différents processus appartenant à des applications et/ou des utilisateurs différents. Ce mécanisme consistant à présenter, de manière transparente, un environnement d'exécution distinct avec un espace d'adressage, un système de fichiers et des autorisations propres à chaque processus est appelé isolation~\cite{fehlingCloudComputingPatterns2014}. À cette fin, les fournisseurs peuvent s'appuyer sur des technologies de virtualisation.

La virtualisation est une technique d'isolation qui permet d'exécuter une application dans les limites d'un environnement d'exécution sécurisé, appelé \textit{sandbox}, en introduisant une couche d'indirection entre la plateforme hôte et l'application elle-même~\cite{singhviAtollScalableLowLatency2021}.

La virtualisation des ressources de l'hôte peut se faire à l'aide de machines virtuelles (VM) ou de conteneurs. Ces environnements d'exécution donnent aux processus sous-jacents l'illusion d'avoir une machine entière à leur disposition. Alors que les VM virtualisent les ressources physiques de l'hôte, en s'appuyant sur l'architecture du processeur pour réaliser l'isolation, les conteneurs exploitent des directives du système d'exploitation hôte pour isoler les charges de travail~\cite{mancoMyVMLighter2017}.

Ces techniques sont bénéfiques à la fois du côté du fournisseur et du côté du développeur. Le premier tire parti de la virtualisation pour isoler les charges de travail des clients, et exploite la flexibilité des VM pour gérer leur mise à l'échelle compte tenu d'une quantité finie de ressources matérielles. Les seconds organisent leur infrastructure de manière à reproduire un environnement de type production pendant les phases de développement, ainsi que pour livrer et déployer leurs produits.

La virtualisation est devenue une telle pierre angulaire des services cloud que Kubernetes~\cite{kubernetes}, un système d'orchestration qui s'appuie sur des conteneurs\footnote{On note que Kubevirt~\cite{kubevirt} vise à adapter Kubernetes au déploiement de VM.} pour gérer le cycle de vie des applications, de leur déploiement à leur mise à l'échelle, est de plus en plus souvent appelé "le système d'exploitation pour le cloud"~\cite{jonreeve2018kubernetes}.

Lorsqu'ils choisissent le modèle d'isolation sur lequel ils souhaitent s'appuyer pour exploiter le partage des ressources, les fournisseurs cloud doivent faire un compromis entre performances et sécurité. Les conteneurs sont fréquemment la cible d'attaques par élévation de privilèges~\cite{zomer2022containers, redhat2019containers}, mais leurs performances sont de plusieurs ordres de grandeur supérieures à celles des machines virtuelles : le temps de démarrage des conteneurs se compte en centaines de millisecondes, tandis que les VM démarrent en quelques secondes~\cite{mancoMyVMLighter2017}. La conception de machines virtuelles légères offrant des performances comparables à celles des conteneurs est un sujet de recherche essentiel~\cite{agacheFirecrackerLightweightVirtualization, Anjali2020BlendingCA}).

\begin{figure}[htbp]
    \centering
	\includegraphics[width=\textwidth]{2_Chapitre2/figures/virtualization.png}
	\caption{Aperçu des différents modèles d'isolation : virtualisation et conteneurisation.}
	\label{fig:virtualization}
\end{figure}

\subsubsection{Machines virtuelles}

Les machines virtuelles virtualisent les ressources physiques de l'hôte : la virtualisation assistée par le matériel permet à plusieurs systèmes d'exploitation \textit{invités} complets de fonctionner indépendamment sur des ressources physiques partagées, quelle que soit la nature du système d'exploitation \textit{hôte}~\cite{kivityKvmLinuxVirtual}.

Du point de vue d'une VM, son environnement d'exécution isolé est perçu comme une plateforme complète, alors qu'il s'agit en réalité d'un sous-ensemble des ressources de la plateforme hôte, déterminé par l'hyperviseur (ou VMM pour Virtual Machine Manager), un logiciel de bas niveau qui peut s'exécuter sur du \textit{bare-metal} ou en tant que processus du système d'exploitation hôte.

L'hyperviseur a la responsabilité de gérer le cycle de vie des machines virtuelles : la création, l'exécution, la destruction et parfois la migration des machines virtuelles sont gérées par l'hyperviseur.

Les hyperviseurs existent sous deux formes différentes, comme le montre la figure~\ref{fig:virtualization} :

\begin{itemize}
    \item Les hyperviseurs de type 1 (bare-metal) s'exécutent directement sur le matériel de la machine hôte. Étant donné qu'ils ne dépendent pas d'un système d'exploitation sous-jacent, ils sont considérés comme plus sûrs et plus efficaces que leurs homologues hébergés. Parmi les exemples courants d'hyperviseurs de type 1, citons VMware ESXi~\cite{esxi}, KVM~\cite{kvm}, Xen~\cite{xen} et Hyper-V~\cite{hyper-v} ;
    \item Les hyperviseurs de type 2 (hébergés) s'exécutent au-dessus d'un système d'exploitation. Ces hyperviseurs sont des produits grand public qui offrent aux utilisateurs finaux un moyen pratique d'exécuter des systèmes ou des programmes qui ne seraient pas pris en charge par leur matériel ou leur système d'exploitation. Parmi les hyperviseurs de type 2, citons QEMU~\cite{qemu} et Oracle VirtualBox~\cite{virtualbox}.
\end{itemize}

\subsubsection{Conteneurs}

La conteneurisation est une technique de virtualisation au niveau du système d'exploitation. Le noyau du système d'exploitation hôte est responsable de l'allocation des ressources. Les conteneurs virtualisent le système d'exploitation : ils donnent au processus conteneurisé l'impression d'avoir toute la machine à leur disposition, tout en étant en réalité contraint et limité en ce qui concerne l'utilisation des ressources par le noyau hôte~\cite{bentalebContainerizationTechnologiesTaxonomies2022}.

Du point de vue de l'application en cours d'exécution, la plateforme d'exécution se comporte comme s'il s'agissait de bare-metal. Toutefois, les ressources qui lui sont allouées sont en fait un sous-ensemble virtualisé des ressources matérielles de l'hôte.

Les conteneurs constituent un mécanisme d'isolation léger qui repose sur les capacités d'isolation du noyau du système hôte, comme le montre la figure~\ref{fig:virtualization}. En l'occurrence, sous Linux :

\begin{itemize}
    \item \texttt{chroot} : modifie le répertoire racine apparent pour une arborescence de processus donnée. Il permet à un conteneur d'opérer sur un répertoire virtuel \texttt{/} qui pourrait être situé n'importe où sur le système de fichiers de l'hôte ;
    \item \texttt{cgroups} : permettent de créer des groupes hiérarchiques de processus et allouer, limiter et surveiller les ressources matérielles pour ces groupes : E/S vers et depuis les périphériques de bloc, accès à l'unité centrale, à la mémoire et aux interfaces réseau ;
    \item \texttt{namespaces} : une couche d'abstraction autour des ressources globales du système, telles que le réseau ou les communications entre processus (IPC). Les processus au sein d'un espace de noms ont leurs propres instances isolées de ces ressources système.
\end{itemize}

L'ambition derrière les conteneurs est de contenir l'exécution d'une application dans un processus isolé du reste du système, de sorte à ce qu'il ignore les autres processus exécutés par le système hôte. Le conteneur est amorcé à partir d'une image qui contient toutes les dépendances nécessaires à la construction et/ou à l'exécution de l'application.

Dans l'écosystème des conteneurs, Docker~\cite{docker} en particulier a connu une forte progression depuis sa création en 2013. Docker a joué un rôle déterminant dans la spécification des normes industrielles pour les formats de conteneurs et les temps d'exécution par le biais de l'Open Container Initiative (OCI)~\cite{oci}.

La spécification OCI est une initiative de la Fondation Linux~\cite{linuxfoundation} visant à concevoir une norme ouverte pour les conteneurs. Elle définit les spécifications des images de conteneurs - des lignes directrices sur la manière de créer une image OCI avec son manifeste, ses couches de système de fichiers et sa configuration - et les spécifications d'exécution concernant la manière d'exécuter les paquets d'applications au fur et à mesure qu'ils sont décompressés sur le système d'exploitation hôte.

\subsection{Dimensionnement et passage à l'échelle}

Lorsque la charge sur une application augmente, il existe deux façons de faire de la place pour les nouvelles requêtes en procédant à une mise à l'échelle (ou \textit{scaling}) :

\begin{itemize}
    \item \textbf{Verticalement} : en attachant plus de ressources matérielles aux serveurs qui supportent l'application (par exemple, augmenter le nombre de CPU ou la quantité de mémoire alloués). Il peut s'agir de déplacer des données vers de nouveaux serveurs plus puissants, ce qui a un impact sur la disponibilité de l'application ;
    \item \textbf{Horizontalement} : en augmentant le nombre de serveurs pour faire fonctionner l'application. Cela peut signifier l'introduction d'un mécanisme d'équilibrage de la charge pour acheminer les requêtes et les réponses entre les utilisateurs et les multiples instances d'une application, ce qui a un impact sur la complexité du déploiement.
\end{itemize}

L'opération contraire peut être symétrique, ou non, lorsque le nombre ou la complexité des traitements diminue.

\subsection{Qualité de service et métriques de performances}

Problème : overbooking, overcommitting... conso d'énergie...

%%%%%%%%%%%%%%%%%%%%%%%%%%%%%%%%%%%%%%%%%%%%%%%%%%%%%%%%%%%%%%%
%%%%%%%%%%%%%%%%%%%%%%%%%%%%%%%%%%%%%%%%%%%%%%%%%%%%%%%%%%%%%%%
% Cf. https://jacquetpi.github.io/articles/240615_share_fr.html
%%%%%%%%%%%%%%%%%%%%%%%%%%%%%%%%%%%%%%%%%%%%%%%%%%%%%%%%%%%%%%%
%%%%%%%%%%%%%%%%%%%%%%%%%%%%%%%%%%%%%%%%%%%%%%%%%%%%%%%%%%%%%%%

TODO: Consommation d'énergie, niveaux d'usage des ressources < 15\% assez classiques~\cite{vasanWorthTheirWatts2010, vermaLargescaleClusterManagement2015a} pour absorber les pannes et les interférences entre workloads. PUE moyen en 2022 autour de 1.55~\cite{davisUptimeInstituteGlobal2022}

TODO: QoS, SLO, SLA

TODO: Caractérisation, grand nombre de tâches à basse priorité~\cite{tirmaziBorgNextGeneration2020}

\section{Applications dans le cloud}

\subsection{Du monolithe aux micro-services}

Le cloud computing a vu naître de nouvelles techniques de développement. Le développement "cloud-native" consiste à construire des applications pour le cloud dès le départ, en prenant en compte dès leur conception les besoins futurs de mise à l'échelle~\cite{dragoniMicroservicesHowMake2018, martinfowler2014microservices}.

Une application monolithique est construite comme une unité unique. Il n'y a pas de découplage entre les services qu'elle expose, car ils font tous partie de la même base de code~\cite{villamizarEvaluatingMonolithicMicroservice2015}. Lors de la mise à l'échelle d'un monolithe, l'ajout de ressources supplémentaires (mise à l'échelle verticale) ne résout pas le problème des priorités concurrentes au sein de l'application : lorsque la popularité d'une application monolithique augmente, certaines parties de la base de code seront sollicitées plus que d'autres, mais la pression ne sera pas répartie sur l'ensemble de l'application. D'autre part, l'augmentation du nombre d'instances du monolithe (mise à l'échelle horizontale) peut s'avérer inefficace en termes de coûts, car toutes les parties d'une application ne subissent pas des pics de charge en même temps : dans la figure \ref{fig:scaling-monolith}, une augmentation des requêtes d'authentification implique une mise à l'échelle de l'infrastructure pour l'ensemble de l'application.

\begin{figure}[ht]
    \centering
	\includegraphics[width=\textwidth]{2_Chapitre2/figures/scaling-monolith.png}
	\caption{La mise à l'échelle d'une application à architecture monolithique nécessite de répliquer le monolithe sur plusieurs serveurs.}
	\label{fig:scaling-monolith}
\end{figure}

La méthodologie Twelve-Factor App, un ensemble de lignes directrices pour la création d'applications cloud-natives, recommande "[d'exécuter] l'application sous la forme d'un ou plusieurs processus sans état"~\cite{12factor}. C'est ce qu'on appelle une architecture de microservices, qui consiste à organiser une application sous la forme d'une collection de services faiblement couplés. Chacun de ces services s'exécute dans son propre processus, communique avec les autres par le biais du passage de messages et peut être déployé de manière indépendante sur des serveurs hétérogènes afin d'atteindre les objectifs de niveau de service : dans la figure \ref{fig:scaling-microservices}, une augmentation des requêtes d'authentification peut être absorbée par la mise à l'échelle de l'infrastructure pour le seul microservice d'authentification.

\begin{figure}[ht]
    \centering
	\includegraphics[width=\textwidth]{2_Chapitre2/figures/scaling-microservices.png}
	\caption{La mise à l'échelle d'une application architecturée en microservices permet de distribuer et de répliquer chaque microservice de manière indépendante.}
	\label{fig:scaling-microservices}
\end{figure}

Toutefois, une infrastructure de microservices implique une couche complexe de gestion centralisée, qui entraîne des coûts dans les opérations ou dans les équipes DevOps. Elle repose sur des services dorsaux de longue durée (bases de données, bus de messages, etc.) qui doivent également être surveillés et gérés. Dans les environnements IaaS et PaaS, l'expérience des développeurs en particulier n'est pas satisfaisante : le déploiement du cloud s'accompagne d'un fardeau d'administration des systèmes~\cite{jonasCloudProgrammingSimplified2019}. Les microservices ne résolvent pas à eux seuls le problème du déploiement : l'écriture de recettes d'images de conteneurs n'ajoute pas de valeur au produit, car elle relève de la logique opérationnelle plutôt que de la logique commerciale.

TODO: Conclusion avec ouverture sur le serverless

\subsection{Vers un nouveau modèle de service}

Dans cette section, nous présentons le modèle serverless de programmation et de déploiement d'applications dans le cloud. Nous passerons en revue les caractéristiques essentielles des plateformes serverless et mettrons en évidence les compromis que les fournisseurs de services et les développeurs d'applications doivent prendre en compte lorsqu'ils ciblent des infrastructures serverless. Cette section propose également une description des offres serverless dans les solutions de cloud public, et des plateformes open source pour les architectes de cloud privé.

\subsubsection{Caractéristiques des plateformes serverless}

Le serverless désigne à la fois un modèle de programmation et un modèle de service pour le cloud computing. Dans une architecture serverless, les développeurs conçoivent leurs applications comme une composition de fonctions sans état. Sans état (ou "pur", sans effet de bord) signifie que le résultat du calcul dépend exclusivement des entrées~\cite{burckhardtNetheriteEfficientExecution}. Ces fonctions prennent en entrée une valeur et un contexte d'invocation, et produisent un résultat qui est stocké dans un niveau de stockage persistant. Leur exécution est déclenchée par un événement qui peut être décrit comme la notification d'un message entrant, qu'il s'agisse d'une requête HTTP, d'une tâche planifiée, d'un téléchargement de fichier, etc. Ainsi, le serverless est un modèle dirigé par les événements~\cite{SchleierSmith2021WhatSC}.

La conception susmentionnée est illustrée par un exemple d'application serverless dans la figure~\ref{fig:web-app} : lorsque la requête de l'utilisateur atteint la passerelle (ou API) de la plateforme serverless, celle-ci déclenche l'exécution de différentes fonctions selon la route HTTP demandée -- ces fonctions ne sont pas des \textit{daemons} à l'écoute d'événements (par exemple, sur un socket ouvert), elles sont exécutées à la demande.

\begin{figure}[htbp]
    \centering
	\includegraphics[width=\textwidth]{2_Chapitre2/figures/faas-web-app.png}
	\caption{Fictional reference architecture for a serverless e-commerce web application deployed in the Amazon Web Services ecosystem.}
	\label{fig:web-app}
\end{figure}

Dans les offres commerciales, serverless est généralement appelé \textit{Function as a Service} (FaaS). On a cru que les fonctions en tant qu'abstraction sur le cloud computing font partie d'une première génération d'offres serverless, et pourraient changer par la suite~\cite{hellersteinServerlessComputingOne2019}.

Serverless ne signifie pas que les serveurs ne sont plus utilisés pour héberger et exécuter des applications -- d'une manière similaire au PaaS, du point de vue du client, "serverless" fait référence à une abstraction sur les ressources matérielles qui permet aux ingénieurs de ne plus penser aux serveurs qui supportent leurs applications. Grâce aux mécanismes de mise à l'échelle automatique, ils n'ont pas à tenir compte du nombre optimal d'instances nécessaires pour exécuter leurs charges de travail dans le cadre d'une planification de la capacité. Les plateformes serverless sont conçues pour gérer les besoins de mise à l'échelle et répondre aux fluctuations de la demande, libérant ainsi les clients du fardeau d'avoir à définir des stratégies de mise à l'échelle explicites. Dans une vision stratifiée des déploiements cloud, le SPEC Research Group~\cite{spec-rg} présente une architecture de référence FaaS qui montre que le développement serverless permet aux développeurs d'être au plus près de la logique métier~\cite{vaneykSPECRGCloud2018, vaneykSPECRGReferenceArchitecture2019}.

Développer pour les plateformes serverless nécessite de repenser l'architecture d'une application. En effet, les serveurs dorsaux à longue durée de vie sont relégués aux solutions \textit{serverful} qui fournissent des serveurs toujours actifs~\cite{mateiTransitionServerfullServerless2020}, telles que les offres IaaS.

Comme l'ont souligné Shafiei et al.~\cite{shafieiServerlessComputingSurvey2021} dans leur enquête de 2022 sur l'informatique sans serveur, il n'existe pas de définition formelle du concept d'informatique sans serveur, bien que nous puissions identifier diverses différences essentielles entre les modèles serverful et serverless (résumées dans le tableau~\ref{table:serverful-serverless}).

Une différence majeure entre PaaS et FaaS est que FaaS permet une mise à l'échelle à zéro : les fournisseurs ne facturent les clients que lorsque leur application utilise effectivement des ressources matérielles, c'est-à-dire lorsque des fonctions sont exécutées sur la plateforme. Cela est possible parce que, dans le paradigme FaaS, les applications sont conçues comme une collection de microservices à courte durée d'exécution.

Les solutions Backend as a Service (BaaS) sont des offres commerciales et gérées de services backend, mises à la disposition des développeurs d'applications par le biais d'une API unifiée~\cite{mikeroberts2018serverless}. Les logiciels dorsaux consistent généralement en des services à état et à longue durée d'exécution qui ne peuvent pas être réduits à zéro. Afin de maintenir un modèle de tarification cohérent, les fournisseurs doivent proposer ces services de la même manière payante que les fonctions serverless. Ces services tiers constituent l'infrastructure de base des applications serverless en gérant l'état des fonctions déployées par les développeurs, par le biais, par exemple, de magasins de valeurs clés ou de stockage de fichiers ; en fournissant une authentification aux points d'extrémité de l'application ; en permettant les communications entre les fonctions à l'aide de bus de messages ; etc. La figure~\ref{fig:web-app} montre les dépendances possibles entre les fonctions serverless et les logiciels BaaS : l'application d'exemple s'appuie sur une authentification gérée par le fournisseur, un bus de messages, une base de données relationnelle, un moteur de recherche et un stockage d'objets, et l'utilisateur y accède par la passerelle API du fournisseur. Cette situation introduit un degré élevé de couplage entre l'application et les services spécifiques du fournisseur, ce qui risque de lier les développeurs à leur choix initial de fournisseur de services.

Serverless permet de réduire les frais généraux des développeurs en faisant abstraction de la gestion des serveurs, tout en permettant aux fournisseurs de partager les ressources physiques à un niveau de granularité très fin, ce qui permet d'obtenir une meilleure efficacité. Le niveau fin de granularité présenté dans le modèle FaaS permet au fournisseur d'offrir une élasticité parfaite : la mise à l'échelle et l'extension sont basées sur des événements, dans un modèle de tarification typique \textit{pay as you go}.

Cette abstraction permet aux fournisseurs de déployer du code dans plusieurs zones géographiques. Ce mécanisme de basculement garantit la disponibilité en cas de panne dans une zone de déploiement et réduit le risque de défaillance d'une fonction en cascade dans l'application~\cite{taibiPatternsServerlessFunctions2020}.

En outre, comme les instances de fonction sont créées à la demande par le fournisseur, le modèle de concurrence offert par les plateformes FaaS signifie que les performances d'une application peuvent évoluer linéairement avec le nombre de requêtes~\cite{mcgrathServerlessComputingDesign2017}.

\begin{table}[H]
    \caption{Comparison of key characteristics in serverless and serverful service models}
    \centering
    \begin{tabularx}{\textwidth} { 
      | >{\centering\arraybackslash}X 
      | >{\centering\arraybackslash}X 
      | >{\centering\arraybackslash}X  | }
         \hline
        \textit{Characteristic}  & \textit{Serverful} (IaaS, PaaS)           & \textit{Serverless} (FaaS)                                             \\ \hline
        Provisioning    & Customer responsibility          & Fully \textbf{managed} (\textit{i.e.} by the provider)    \\ \hline
        Billing         & Pay for \textbf{provisioned} resources    & Pay for \textbf{consumed} resources                                    \\ \hline
        Scaling         & Customer responsibility          & \textbf{autoscalinging} built in                                         \\ \hline
        Availability    & Depends on \textbf{provisioned} resources & Code runs in multiple high availability zones                 \\ \hline
        Fault tolerance & Depends on deployment strategy   & Backend services are fully \textbf{managed} and retries are guaranteed \\ \hline
        Concurrency     & Depends on \textbf{provisioned} resources & Virtually \textbf{infinite} \\
        \hline
    \end{tabularx}
    \label{table:serverful-serverless}
\end{table}

Divers auteurs (\cite{hellersteinServerlessComputingOne2019},~\cite{vaneykSPECRGCloud2018},~\cite{shafieiServerlessComputingSurvey2021},~\cite{khandelwalTaureauDeconstructingServerless2020}) considèrent déjà le serverless comme l'avenir du déploiement du cloud. Cependant, l'adoption du FaaS semble stagner parmi les développeurs de cloud~\cite{oreilly2020adoption}, et la Cloud Native Computing Foundation (CNCF) fait même état de chiffres en baisse~\cite{cncf2021report}.

\subsubsection{Charges de travail adaptées}

Dans son livre blanc de 2018~\cite{cncf2018whitepaper}, la Cloud Native Computing Foundation (CNCF) -- une initiative de la Fondation Linux soutenue par plus de 800 membres industriels impliqués dans les services cloud -- identifie des caractéristiques pour les cas d'utilisation serverless, notamment :

\begin{itemize}
    \item Charges de travail "étonnamment parallèles" : asynchrones et concurrentes, avec peu ou pas de communication et pas de synchronisation entre les processus ;
    \item Peu fréquentes avec des variations imprévisibles dans les exigences de mise à l'échelle, c'est-à-dire des tâches interactives ou événementielles plutôt que des tâches par lots ;
    \item Processus sans état et éphémères, sans besoin majeur d'un temps de démarrage instantané.
\end{itemize}

Nous pouvons affirmer que ces conditions sont trop restrictives pour l'informatique générale : par exemple, elles impliquent que les travaux de longue durée ne peuvent pas être mis en correspondance avec les déploiements FaaS. Les fournisseurs ont introduit des mécanismes tels que les fonctions d'étape ou les fonctions durables (\cite{aws-step-functions},~\cite{azure-durable-functions},~\cite{google-workflows}) pour mettre en œuvre des flux de travail serverless : une fonction d'orchestration maintient l'état à travers les fonctions sans état de l'application pour créer des flux de travail avec état~\cite{burckhardtNetheriteEfficientExecution}.

Les développeurs déploient déjà une partie de la logique de leurs applications vers des fonctions serverless. Selon une enquête~\cite{serverless2018survey} menée en 2018 par l'éditeur du framework Serverless auprès d'un panel de leurs utilisateurs, les exemples de cette logique incluent les pipelines de transformation de données, les plateformes d'alerte à haute disponibilité, les outils ETL (\textit{Extraire, Transformer, Charger}, la manipulation de données par lots), le transcodage de médias, etc. Ces applications sont un sous-ensemble de programmes informatiques qui produisent des résultats qui dépendent uniquement de l'entrée du programme : elles appliquent des transformations purement fonctionnelles aux données.

Les problèmes qui sont commodément divisés en lots de sous-tâches bénéficieraient également du niveau de concurrence virtuellement infini offert par le modèle serverless~\cite{golecIFaaSBusSecurityPrivacyBased2022}. Dans~\cite{agacheFirecrackerLightweightVirtualization}, les auteurs identifient des cas d'utilisation pour le serverless dans l'encodage vidéo à grande échelle, l'algèbre linéaire et la compilation parallèle.

Comme il existe des similitudes fondamentales entre une application architecturée en microservices et une application conçue pour un déploiement FaaS~\cite{jiaNightcoreEfficientScalable2021}, des applications à part entière peuvent être conçues avec FaaS à l'esprit. La figure~\ref{fig:web-app} donne un exemple d'architecture pour une application web de commerce électronique. La logique métier de l'application comprend trois fonctions serverless (catalogue, recherche et paiement) qui sont déclenchées par la navigation de l'utilisateur, et une fonction (notification) qui est ordonnancée pour s'exécuter périodiquement. Comme ces fonctions sont activées et désactivées en fonction de la charge de l'application, l'état doit être stocké dans un stockage persistant, c'est-à-dire une base de données relationnelle et un magasin d'objets qui sont tous deux gérés par le fournisseur. L'application s'appuie en outre sur des services gérés par le fournisseur : son moteur de recherche est alimenté par une solution BaaS, tout comme la fonction de notification et le mécanisme d'authentification.

En utilisant AWS Lambda en 2022, ce type d'application pourrait passer de zéro ressource utilisée à 150 To de RAM et 90 000 vCPU en moins de deux secondes~\cite{ionescu2022scaling}, ce qui permettrait de réagir rapidement aux variations de charge.

\subsubsection{Compromis dans les déploiements serverless}

Lorsque nous considérons le serverless comme un modèle de programmation, l'avantage immédiat est un coût de développement réduit pour les équipes qui s'appuient sur des offres BaaS : au lieu de mettre en œuvre des services backend internes tels que l'authentification ou les notifications, les développeurs se contentent d'introduire un boilerplate dans leur base de code de manière à connecter les applications frontales aux API BaaS de leur fournisseur de cloud. Toutefois, ce degré de couplage signifie que les développeurs peuvent se retrouver enfermés dans un environnement spécifique à un fournisseur, perdant en fin de compte le contrôle des coûts de déploiement~\cite{baarziMeritsViabilityMultiCloud2021}.

Du point de vue du client, le FaaS associé au BaaS géré permet une mise à l'échelle parfaite. Les clients sont facturés au plus juste, uniquement lorsque les ressources sont effectivement utilisées et pour la durée exacte de l'exécution. Du point de vue du fournisseur, l'augmentation du nombre de locataires permet un meilleur multiplexage des ressources, ce qui se traduit par une efficacité accrue et donc des bénéfices plus importants. Ce mécanisme d'autoscalinging a un coût en termes de latence : faire tourner de nouveaux bacs à sable pour les requêtes entrantes peut créer des situations de \textit{démarrages à froid} où les temps d'initialisation dominent les temps d'exécution des fonctions~\cite{Jiang2021TowardsDS}. L'autoscalinging serverless a d'autres implications en termes de débit : étant donné que l'état des fonctions doit être persisté dans un stockage désagrégé, les applications qui affichent des modèles de communications étendues entre les fonctions peuvent souffrir du temps d'expédition des données vers les nœuds de calcul~\cite{mullerLambadaInteractiveData2020}.

Un effet secondaire du modèle FaaS est l'augmentation du nombre de tâches par client. Bien que l'augmentation de la concurrence, de la densité et de l'utilisation des ressources soit un argument de vente pour les fournisseurs de FaaS et les clients, ces travaux de courte durée doivent être isolés les uns des autres pour éviter la fuite de secrets entre les clients, ou à l'échelle d'une application unique comprenant plusieurs fonctions individuelles~\cite{vaqueroLockingSkySurvey2011}.

\begin{table}[H]
    \caption{Considerations regarding the FaaS service model}
    \centering
    \begin{tabularx}{\textwidth} { 
      | >{\centering\arraybackslash}X 
      | >{\centering\arraybackslash}X  | }
     \hline
        \textit{Pros} & \textit{Cons} \\ \hline
        Reduced \textbf{development costs} for teams leaning on BaaS offerings & Can we map \textbf{any application} to the FaaS architecture? \\ \hline
        Reduced \textbf{costs in operations} thanks to fully managed infrastructure & Risks of \textbf{vendor lock-in} due to high degree of coupling with BaaS offerings \\ \hline
        \textbf{Perfect scaling} allows billing granularity close to actual use of resources & Increase in \textbf{latency} due to cold starts, and decreased \textbf{throughput} from communications through slow storage to handle statefulness \\ \hline
        Providers can achieve \textbf{better efficiency} in resources multiplexing leading to increased profits & Massive multitenancy might involve \textbf{security} threats \\ \hline
    \end{tabularx}
    \label{table:serverless-tradeoffs}
\end{table}

Pour tirer parti des atouts du modèle, les clients et les fournisseurs doivent tenir compte des compromis associés aux déploiements serverless. Le tableau~\ref{table:serverless-tradeoffs} résume les principaux points à prendre en compte lors de l'utilisation de serverless pour déployer des applications dans le cloud.

\section{Conclusion}


\clearemptydoublepage
\chapter{Serverless : allocation et placement dynamiques dans le cloud -- État de l'art}
\label{chapter:sota}

TOOD: Le modèle serverless n'est pas une idée nouvelle...
TODO: Première mention de "serverless" dans la littérature~\cite{andersonServerlessNetworkFile}

TODO: Pourtant, de nombreux défis restent encore ouverts...
TODO: Autoscaling : problème non résolu~\cite{straesserWhyItNot2022}

TODO: Dans ce troisième chapitre, ...

\section{Serverless : caractéristiques et écosystème}

Le modèle serverless constitue un changement de paradigme dans le cloud public : par opposition aux modèles traditionnels, les clients serverless ne réservent pas de ressources matérielles. L'exécution de leur code est dirigée par des événements (requêtes HTTP, tâches programmées, etc.) et la facturation s'effectue sur la base de l'usage réel des ressources. En contrepartie, la responsabilité de l'allocation des ressources et du placement des tâches incombe au fournisseur.

Malgré une tarification attractive avec des plans gratuits étendus dans les offres commerciales, et un panel varié de solutions open source ciblant les principaux orchestrateurs de cloud, le FaaS n'est pas encore devenu le modèle d'abonnement au cloud de référence : certains défis sont encore ouverts et doivent être relevés avant que le serverless ne devienne omniprésent.

Un véritable défi pour remédier à ces lacunes est d'éviter les solutions "serveurful" au problème de l'allocation dynamique des ressources, c'est-à-dire l'allocation de ressources stables supplémentaires qui ne s'étendent volontairement pas jusqu'à zéro~\cite{hellersteinServerlessComputingOne2019}.

Le serverless est un sujet animé dans le domaine du cloud computing et de nombreux auteurs contribuent à atténuer ces revers : le nombre d'articles publiés autour du serverless a presque doublé entre 2019 et 2020~\cite{hassanSurveyServerlessComputing2021}.

\subsection{Description de l'offre serverless}

Dans cette première sous-section, nous proposons un tour d'horizon des différentes offres serverless chez les fournisseurs de cloud public, ainsi que des solutions open source disponibles pour le cloud privé.

Le tableau~\ref{table:commercial-faas} présente un sommaire des offres commerciales FaaS dans le cloud public, leurs modèles de tarifications ainsi que certaines de leurs caractéristiques. Ce sommaire inclut Alibaba Function Compute~\cite{alibaba-function-compute}, Amazon Web Services Lambda~\cite{aws-lambda}, Microsoft Azure Functions~\cite{azure-functions}, Google Cloud Functions~\cite{google-cloud-functions}, IBM Cloud Functions~\cite{ibm-cloud-functions} et Oracle Cloud Functions~\cite{oracle-cloud-functions}.

Le tableau~\ref{table:foss-faas} présente ensuite quelques solutions majeures disponibles dans la communauté open source, et donne un aperçu de l'état du projet en fonction de son adoption par les utilisateurs et la quantité de contributions, ainsi que les partenaires industriels qui accompagnent le projet. On y trouve Apache OpenWhisk~\cite{openwhisk}, Fission~\cite{fission}, Fn~\cite{fn}, Knative~\cite{knative} et OpenFaaS~\cite{openfaas}.

\subsubsection{Solutions commerciales dans le cloud public}

Pour positionner chacune des offres commerciales, nous avons choisi de les comparer sur la base des critères suivants :

\begin{itemize}
    \item Modèle de tarification : les modalités selon lesquelles les clients peuvent anticiper le coût de leurs déploiements dans le cadre d'une offre serverless dans le cloud public ;
    \item Caractéristiques : les limites imposées par le fournisseur de services en matière d'utilisation et de disponibilité des ressources.
\end{itemize}

TODO: refaire le tableau

% \begin{longtblr}[
%     caption = {Cloud customers are faced with a diversity of FaaS offerings},
%     label = {table:commercial-faas},
%     note{a} = {billed per \num{10000} requests (for USD 0.02)}
% ]{
%     rowhead = 3,
%     colspec = { X[l,1.8] X[2,l]X[2,l] X[l]X[l]X[l]X[1.8,l] },
%     row{1-2} = { l, font = {\bfseries} },
%     row{3} = { b, font = \footnotesize%, cmd={\rotatebox{60}}
%                 },
%     row{4-Z} = { m, font = \footnotesize, rowsep = 1ex },
%     column{1} = { font = {\bfseries}}
% }

% \toprule
% \SetCell[r=3]{m} Service & \SetCell[c=2]{c} Pricing model && \SetCell[c=4]{c} Properties \\
% \cmidrule[lr=-0.4]{2-3}
% \cmidrule[lr=-0.4]{4-7}
% &
% free quota per month &
% pay-as-you-go cost &
% \\
% &
% \# of invocations / compute resources [\unit{\giga\byte\second}]&
% \num{1}M requests / \qty{1}{\giga\byte\second} compute [USD]&
% code size &
% memory &
% execution time &
% payload size \\
% \midrule


% Alibaba Function Compute &
% \num{1000000} / \num{400000} &
% \num{20}\TblrNote{a} / \num{0.000016384} &
% \qty{500}{\mega\byte} &
% \qty{3}{\giga\byte}&
% \qty{24}{\hour} &
% \qty{128}{\kilo\byte} (request), \qty{6}{\mega\byte} (response)
% \\

% AWS Lambda &
% \num{1000000} / \num{400000} &
% \num{0.2} / \num{0.0000166667} &
% \qty{10}{\giga\byte}&
% \qty{10240}{\mega\byte} &
% \qty{15}{\minute} &
% \qty{6}{\mega\byte} (synchronous), \qty{256}{\kilo\byte} (asynchronous) for requests and responses
% \\

% Azure Functions &
% \num{1000000} / \num{400000} &
% \num{0.2} / \num{0.000016} &
% N/A &
% \qty{1.5}{\giga\byte} &
% \qty{10}{\minute}&
% \qty{100}{\mega\byte} (request)
% \\

% Google Cloud Functions   &
% \num{2000000} / \num{400000} &
% \num{0.4} / \num{0.0000025} &
% \qty{500}{\mega\byte}&
% \qty{8}{\giga\byte}&
% \qty{9}{\minute}&
% \qty{10}{\mega\byte} for requests and responses
% \\

% IBM Cloud Functions  &
% \num{5000000} / \num{400000} &
% N/A / \num{0.000017} &
% \qty{48}{\mega\byte} &
% \qty{2048}{\mega\byte} &
% \qty{60}{\second} &
% \qty{5}{\mega\byte} for requests and responses
% \\

% Oracle Cloud Functions &
% \num{2000000} / \num{400000} &
% \num{0.2} / \num{0.00001417} &
% N/A &
% \qty{2048}{\mega\byte} &
% \qty{5}{\minute}&
% \qty{6}{\mega\byte}
% \\

% \bottomrule
% \end{longtblr}

\subsubsection{Solution open source pour le cloud privé}

Comme mesure de l'état d'un projet ainsi que de son adoption, nous choisissons deux indicateurs disponibles publiquement sur les dépôts GitHub~\cite{github} :

\begin{itemize}
    \item Les "GitHub stars" (ou étoiles) indiquent combien d'utilisateurs de GitHub ont choisi de s'abonner aux mises à jour d'un projet ;
    \item Nous considérons comme contributeurs les utilisateurs ayant publié dix, ou plus de dix modifications au code source (\textit{git commits}) du projet.
\end{itemize}

\begin{table}[H]
\caption{Open source solutions allow cloud providers to devise their own FaaS offering}
\centering
\begin{tabularx}{\textwidth} { 
  | >{\centering\arraybackslash}X 
  | >{\centering\arraybackslash}X 
  | >{\centering\arraybackslash}X  | }
\hline
                       & \textbf{Project status and adoption} & \textbf{Corporate backer} \\ \hline
Apache OpenWhisk       & 5.8k GitHub stars, 34 contributors ($\geq$ 10 commits) & IBM (Apache Foundation)             \\ \hline
Fission                & 7.3k GitHub stars, 10 contributors ($\geq$ 10 commits) & Platform9             \\ \hline
Fn                     & 5.3k GitHub stars, 21 contributors ($\geq$ 10 commits) & Oracle             \\ \hline
Knative                & 4.5k GitHub stars, 55 contributors ($\geq$ 10 commits) & Google             \\ \hline
OpenFaaS               & 22.2k GitHub stars, 13 contributors ($\geq$ 10 commits) & VMware             \\ \hline
\end{tabularx}
\label{table:foss-faas}
\end{table}

\section{Le défi de l'allocation dynamique des ressources}

\subsection{Délais et fréquence du démarrage à froid}
\label{sota-cold-start}

Comme les conteneurs serverless doivent passer un minimum de temps dans un état d'inactivité, ils sont mis en marche et arrêtés très fréquemment par rapport aux conteneurs PaaS ou aux VM IaaS. Chaque fois qu'une fonction est appelée et doit être mise à l'échelle à partir de zéro, le conteneur ou la machine virtuelle hébergeant le code de la fonction doit passer par sa phase d'initialisation : c'est ce qu'on appelle un "cold start"~\cite{lloydImprovingApplicationMigration2018}, ou "démarrage à froid". Les démarrages à froid peuvent entraîner des pénalités de latence, aggravées par les retards qui font boule de neige lors de la composition des fonctions dans le contexte d'applications complexes~\cite{mohanAgileColdStartsa}.

Les fonctions sont généralement invoquées en rafales - le modèle d'exécution AWS Lambda peut maximiser la concurrence en instanciant une fonction dans des centaines, voire des milliers de bacs à sable répartis sur différents sites géographiques~\cite{aws-lambda-scaling}. Quelques minutes après avoir traité une requête, le sandbox d'une fonction est libéré du nœud d'exécution ; de plus, les futures nouvelles instances ne sont pas garanties d'être créées sur le même nœud. Cela conduit à des situations dans lesquelles l'environnement d'une fonction n'est pas mis en cache sur le nœud. Le code et les bibliothèques associées doivent être récupérés et copiés à nouveau sur le système de fichiers, ce qui entraîne une latence de démarrage à froid.

Une approche "naïve" consisterait à pré-allouer des ressources matérielles afin de maintenir un pool de conteneurs de fonctions prêt à recevoir de nouvelles requêtes. Cette approche n'est pas acceptable car elle s'éloigne de la possibilité d'une mise à l'échelle à zéro.

Vahidinia et al.~\cite{vahidiniaColdStartServerless2020} proposent une étude complète de la position et des stratégies de diverses offres commerciales FaaS concernant le démarrage à froid. Alors que l'informatique sans serveur suggère de faire tourner des instances jetables de fonctions pour traiter chaque requête entrante, les auteurs notent que les acteurs commerciaux tels qu'AWS, Google et Microsoft réutilisent tous dans une certaine mesure les bacs à sable d'exécution, en les maintenant en activité pendant une période de temporisation afin de contourner les coûts de latence induits par les démarrages à froid.

\subsubsection{Réduire les temps d'initialisation}

Différentes approches peuvent être mises en œuvre par les fournisseurs de services en nuage pour réduire le temps d'initialisation des bacs à sable de fonctions. Il s'agit d'un travail crucial car les invocations de fonctions suivent des modèles généralement imprévisibles~\cite{shahradServerlessWildCharacterizing}. Cette section explore les contributions de la littérature qui se concentrent sur la réduction de l'écart de latence entre les modèles serverless et serverful.

\textbf{Approche d'optimisation en bac à sable}

Dans~\cite{mullerLambadaInteractiveData2020}, les auteurs proposent Lambada pour résoudre le problème du démarrage à froid dans le contexte de l'analyse des données distribuées en mettant en lots l'invocation des travailleurs en parallèle. Ils identifient un goulot d'étranglement dans le processus d'invocation de nouveaux travailleurs : dans leur évaluation, ils montrent que l'invocation de 1000 travailleurs AWS Lambda prend entre 3,4 et 4,4 secondes. Dans leur contribution, chaque travailleur est responsable de l'invocation d'une deuxième génération de bacs à sable, qui invoqueront à leur tour une génération suivante de travailleurs jusqu'à ce que le processus de mise à l'échelle soit terminé. Cette technique permet de créer plusieurs milliers de travailleurs en moins de 4 secondes.

Dans~\cite{mancoMyVMLighter2017}, les auteurs proposent LightVM pour mettre le temps de démarrage des VM au même niveau que les conteneurs. Les auteurs montrent que les temps d'instanciation augmentent linéairement avec la taille de l'image : la création d'un environnement sandbox pour l'exécution d'une application est une opération liée aux entrées-sorties. En repensant le plan de contrôle de Xen et en utilisant des VM légères qui incluent un environnement minimal nécessaire à l'exécution de l'application en bac à sable, ils obtiennent des temps de démarrage comparables aux performances de la mise en œuvre de \texttt{fork}/\texttt{exec} dans Linux.

Dans~\cite{akkusSANDHighPerformanceServerless}, les auteurs proposent que des mécanismes d'isolation complets tels que les conteneurs soient nécessaires pour isoler les charges de travail entre les clients, alors qu'à la granularité d'une application unique, les processus suffisent à isoler les fonctions. Dans SAND, ils mettent en œuvre un mécanisme d'isolation au-dessus de Docker qui permet des interactions efficaces en termes de ressources, élastiques et à faible latence entre les fonctions.

Dans~\cite{agacheFirecrackerLightweightVirtualization}, les auteurs présentent Firecracker, qui s'est développé pour devenir la technologie de virtualisation \textit{de facto} pour serverless, utilisée chez AWS Lambda. Ils s'attaquent au compromis isolation versus performances en introduisant des VM légères (ou MicroVM) à la place des conteneurs comme mécanisme de sandboxing pour les charges de travail serverless. Firecracker atteint des temps de démarrage inférieurs à 125 ms en remplaçant QEMU par une implémentation personnalisée d'un moniteur de machine virtuelle qui s'exécute au-dessus de KVM et permet de créer jusqu'à 150 MicroVM par seconde et par hôte avec une surcharge de mémoire de 3\%.

\textbf{Approche "snaptshotting"} \label{sota-snapshotting}

Dans~\cite{duCatalyzerSubmillisecondStartup2020}, les auteurs affirment que la surcharge de démarrage dans les bacs à sable basés sur la virtualisation est due à leur nature agnostique vis-à-vis des applications. En effet, ils montrent que la latence d'initialisation de l'application domine la latence de démarrage totale. Dans Catalyzer, les auteurs montrent que les instances de bac à sable d'une même fonction possèdent des états d'initialisation très similaires, et présentent donc une solution d'instantané qui permet de restaurer une instance de fonction à partir d'une image de point de contrôle, en sautant effectivement la phase d'initialisation de l'application lors d'une mise à l'échelle à partir de zéro. Ils construisent une solution basée sur le gVisor de Google~\cite{gvisor} qui surpasse systématiquement les technologies de pointe telles que Firecracker\cite{agacheFirecrackerLightweightVirtualization}, HyperContainer et Docker d'un ordre de grandeur.

Dans~\cite{ustiugovBenchmarkingAnalysisOptimization2021}, les auteurs présentent vHive, un cadre d'analyse comparative pour l'expérimentation serverless qui leur permet de montrer qu'une latence élevée peut être attribuée à des défauts de page fréquents pendant l'initialisation des bacs à sable, avec des modèles très similaires entre les exécutions d'une même fonction -- 97\% des pages mémoire étant identiques entre les invocations des fonctions étudiées. Les auteurs proposent REAP pour créer des images de la configuration de la mémoire d'un bac à sable qui permettent de prélever des pages du disque vers la mémoire, de remplir rapidement la mémoire d'invité avant l'exécution de la fonction et d'éviter ainsi la majorité des erreurs de page au moment de l'initialisation. Cette technique permet d'accélérer le délai de démarrage à froid de 3,7 fois en moyenne.

Dans~\cite{shillakerFaasmLightweightIsolation2020}, les auteurs proposent Faaslets, un nouveau mécanisme d'isolation basé sur l'isolation des fautes logicielles fournie par WebAssembly. Les faaslets permettent de restaurer l'état d'une fonction à partir d'instantanés déjà initialisés. Ces snapshots sont pré-initialisés à l'avance et peuvent être restaurés en quelques centaines de millisecondes, même d'un hôte à l'autre. Les fasslets tirent parti du modèle de mémoire de WebAssembly : les tableaux d'octets linéaires peuvent être copiés sans une longue phase de (dé)sérialisation.

\textbf{Approche de la mise en cache}

Dans~\cite{oakesSOCKRapidTask}, les auteurs soutiennent que si le serverless permet de réaliser des économies grâce à une élasticité accrue et à la vélocité des développeurs, les longs temps d'initialisation des conteneurs nuisent aux performances de latence des applications déployées. Ils identifient des goulets d'étranglement dans les primitives Linux impliquées dans l'initialisation des conteneurs, les dépendances des paquets étant le principal coupable des opérations d'E/S pendant le sandboxing. Ils proposent SOCK comme un système de conteneurs optimisé pour les tâches serverless qui s'appuie sur OpenLambda et repose sur un système de mise en cache conscient des paquets, et montrent que leur solution offre des accélérations jusqu'à 21 fois par rapport à Docker.

Dans~\cite{fuerstFaasCacheKeepingServerless2021}, l'auteur montre des similitudes conceptuelles entre la mise en cache d'objets et la fonction keep-alive, ce qui leur permet de concevoir des politiques qui réduisent les délais de démarrage à froid. En s'appuyant sur cette analogie, ils proposent une politique de maintien en vie qui est essentiellement une politique de terminaison (ou d'éviction) de fonction. En gardant les fonctions chaudes aussi longtemps que possible (c'est-à-dire aussi longtemps que les ressources du serveur le permettent), FaasCache parvient à doubler le nombre de requêtes pouvant être traitées dans sa mise en œuvre basée sur OpenWhisk.
\\

La capacité des plateformes serverless à mettre à l'échelle une fonction jusqu'à zéro réplique afin d'éviter de facturer les clients pour les ressources inactives est une différence clé par rapport aux modèles de services cloud traditionnels. La recherche de techniques qui minimisent l'impact d'un démarrage à froid sur la latence de la fonction est un sujet de recherche essentiel, car les temps d'initialisation prohibitifs entravent le potentiel des plateformes FaaS à concurrencer les plateformes PaaS.

\subsection{Hétérogénéité du matériel}

Les clients de l'informatique en nuage sont censés réserver différentes ressources en fonction des besoins de leurs applications, qu'il s'agisse d'une architecture CPU spécifique, d'accélérateurs matériels, d'un stockage ad hoc, etc. Un exemple frappant est celui de l'apprentissage automatique distribué, dans lequel de nombreux GPU sont utilisés pour accélérer l'apprentissage des modèles - en outre, les fournisseurs de cloud commencent à généraliser l'accès au matériel spécialisé tel que les TPU dans les VM.

La sélection manuelle des ressources matérielles (" type d'instance "), attendue des clients dans les offres IaaS comme Amazon EC2, n'a pas de sens dans le paradigme serverless. L'accélération matérielle devrait être décidée par le fournisseur par application ou par requête. À ce jour, cette possibilité n'est pas disponible dans les offres FaaS telles qu'AWS Lambda.

Dans~\cite{Jiang2021TowardsDS}, les auteurs ont entrepris de comparer les configurations IaaS et FaaS pour la formation à l'apprentissage automatique sur les offres Amazon Web Services (resp. EC2 et Lambda). Ils proposent une implémentation de l'apprentissage basée sur le FaaS, LambdaML, et la comparent à des frameworks de l'état de l'art s'exécutant sur des instances EC2. Ils ont mesuré que l'apprentissage serverless peut être rentable tant que le modèle converge suffisamment rapidement pour que les communications inter-fonctions ne dominent pas le temps d'exécution total. Dans le cas contraire, une configuration IaaS utilisant des GPU sera plus performante que n'importe quelle configuration FaaS, offrant de meilleures performances tout en étant plus rentable.

Dans~\cite{bacisBlastFunctionFPGAasaServiceSystem2020}, les auteurs explorent le multitenancy dans les FPGA pour atteindre un taux d'utilisation plus élevé de la carte. Ils proposent BlastFunction, un système évolutif pour le partage de temps FPGA dans un contexte serverless. Leur mise en œuvre repose sur trois éléments de base : une bibliothèque qui permet un accès transparent aux dispositifs partagés à distance, un plan de contrôle distribué qui surveille les FPGA pour réaliser le partage du temps, et un registre central qui gère l'allocation des cartes à chaque nœud de calcul. Cette conception permet d'atteindre des taux d'utilisation plus élevés sur les cartes et donc de traiter un plus grand nombre de requêtes, en particulier en cas de charge élevée, bien qu'au prix d'une augmentation de 36% de la latence en raison de la concurrence supplémentaire.

Dans~\cite{diamantopoulosAccelerationasauServiceCloudnativeMonteCarlo2021}, les auteurs se concentrent sur un cas d'utilisation des services financiers et proposent des FPGA pour réduire le temps de réponse de bout en bout et augmenter l'évolutivité dans une architecture microservices. L'application étudiée est à forte intensité de calcul et présente des caractéristiques de temps réel. Ils proposent CloudiFi, un cadre cloud-native qui expose les accélérateurs matériels en tant que microservices par le biais d'une API HTTP RESTful. CloudiFi permet de décharger les charges de travail sur les accélérateurs matériels au niveau des fonctions. Une évaluation des performances de l'application sous CloudiFi montre des gains de temps de réponse de 485x lors de l'utilisation de FPGA connectés au réseau par rapport à une configuration vanille.

Les CPU ARM et RISC, les GPU et les FPGA sont de plus en plus utilisés dans les centres de données pour répondre à la demande de performance, d'efficacité énergétique et de facteur de forme réduit. Dans~\cite{hortaXartrekRuntimeExecution2021}, les auteurs soutiennent que, puisque ces plateformes d'exécution hétérogènes sont généralement colocalisées avec un processeur hôte général, la possibilité de tirer parti de leurs caractéristiques en migrant les charges de travail pourrait entraîner des gains de performance significatifs. Ils proposent Xar-Trek, un compilateur et un moniteur d'exécution pour permettre la migration de l'exécution à travers les CPU et FPGA hétérogènes-ISA selon une politique d'ordonnanceur. Xar-Trek implique un effort de programmation limité : l'application est écrite une seule fois et compilée pour différentes cibles grâce à la chaîne d'outils Xilinx, sans annotations de synthèse de haut niveau nécessaires pour guider le compilateur. Le système d'exécution de Xar-Trek, un ordonnanceur en ligne dans l'espace utilisateur, est capable de déterminer si une migration est efficace et de procéder à la migration des fonctions sélectionnées qui bénéficient le plus de l'accélération. L'évaluation des charges de travail de vision industrielle et de calcul intensif révèle que tant que les charges de travail sont dominées par des fonctions à forte intensité de calcul, Xar-Trek est toujours plus performant que les configurations vanille, avec des gains de performance compris entre 26 et 32 %.

Même lorsque du matériel hétérogène est installé sur le même nœud, il est généralement interconnecté par des bus PCI-Express gérés par l'unité centrale de l'hôte. Les communications sont réalisées à l'aide d'interfaces de passage de messages qui introduisent des coûts de bande passante et de latence. Dans~\cite{vilanovaSlashingDisaggregationTax2022}, les auteurs présentent FractOS, un système d'exploitation distribué pour les centres de données hétérogènes et désagrégés. FractOS permet de décentraliser l'exécution des applications : au lieu de s'appuyer sur l'unité centrale pour transmettre le contrôle et les données d'une plate-forme d'exécution à l'autre, FractOS fournit aux applications une bibliothèque qui permet des communications directes entre les appareils, grâce à un contrôleur sous-jacent qui capte les appels système et fournit des fonctionnalités directes d'appareil à appareil. Lorsqu'elle a été testée sur une application de vérification des visages qui exploite les GPU pour accélérer les calculs, leur solution a permis d'accélérer le temps d'exécution de 47 % et de diviser le trafic réseau global par 3.
\\

Avec la progression exponentielle et l'intérêt croissant dans le domaine de l'apprentissage automatique, la demande d'accélérateurs matériels dans le cloud n'a jamais été aussi importante. Les offres commerciales serverless sont à la traîne des IaaS traditionnels à cet égard, car aucune n'offre d'accès aux GPU, TPU ni FPGA. En outre, l'allocation dynamique de ce type de matériel pour accélérer des tâches sélectionnées offre aux fournisseurs la possibilité d'améliorer leur utilisation des ressources et leur consommation d'énergie.

\section{Ordonnancer et placer les requêtes utilisateur}

\subsection{Communication des données} \label{sota-communications}

Dans les offres FaaS, les fonctions ne sont pas adressables : la composition se fait par le stockage des résultats dans un niveau de stockage lent avec état qui n'est généralement pas colocalisé avec le niveau de calcul.

Comme les fonctions d'une même application ne peuvent pas partager la mémoire ou les descripteurs de fichiers pour réaliser l'IPC, elles doivent établir la communication par le biais d'interfaces de passage de messages, ce qui introduit une surcharge lorsque les données doivent circuler à travers l'application.

Ce problème est particulièrement préoccupant lorsque des applications gourmandes en données doivent travailler avec des données froides, c'est-à-dire des données auxquelles on accède peu et qui ne sont donc pas mises en cache, et qui sont stockées sur des supports moins performants tels que des disques durs situés sur des nœuds distants. Dans~\cite{Jiang2021TowardsDS}, les auteurs présentent LambdaML, une plateforme d'analyse comparative qui permet de comparer les performances de l'apprentissage distribué de modèles de machine entre les offres IaaS et FaaS. Ils constatent que l'utilisation de FaaS pour la formation à l'apprentissage machine peut être rentable tant que les modèles présentent des schémas de communication réduits.

Dans~\cite{mullerLambadaInteractiveData2020}, les auteurs montrent que FaaS peut être rentable lors de l'exécution de requêtes interactives sporadiques sur des gigaoctets à un téraoctet de données froides. En fournissant des opérateurs de données serverless avec Lambada, ils réalisent des requêtes interactives sur plus de 1 TB de données stockées sur Amazon S3 en approximativement 15 secondes, ce qui est dans le même parc à balles que les solutions commerciales Query-as-a-Service.

Dans~\cite{Romero2021FaaTAT}, les auteurs soutiennent que les offres serverless manquent d'une couche de mise en cache des données en mémoire et par application qui permettrait une mise à l'échelle automatique et fonctionnerait de manière transparente. Faa\$T peut former une couche de mise en cache distribuée fortement cohérente lorsque plusieurs instances d'une application sont lancées, le dernier nœud de mise en cache disparaissant lorsque l'application est réduite à zéro, ce qui permet de facturer sur la base de l'utilisation effective des ressources. Les expériences montrent que Faa\$T peut améliorer les performances de diverses applications de 57\% en moyenne tout en étant 99\% moins cher que les alternatives basées sur des serveurs.

SAND~\cite{akkusSANDHighPerformanceServerless} introduit une hiérarchie dans les bus de communication. Dans SAND, les fonctions d'une application sont toujours déployées sur le même nœud. Un bus local au nœud sert de raccourci pour les communications entre fonctions, ce qui permet une exécution séquentielle rapide. Un bus global, distribué entre les nœuds, assure la fiabilité grâce à la tolérance aux pannes. Dans SAND, le bus local multiplie par trois la vitesse de transmission des messages par rapport au bus global.
\\

Comme les fonctions serverless sont éphémères par nature, et compte tenu des mécanismes d'isolation déployés par les fournisseurs pour répondre aux objectifs de confidentialité et de sécurité, minimiser le surcoût de la communication inter-fonctions semble être un double problème : d'une part, les plateformes serverless ont besoin à la fois de solutions spécifiques au domaine qui prennent en compte les caractéristiques des données qui sont alimentées et renvoyées par les fonctions ; d'autre part, il y a de la place pour des améliorations générales dans le domaine des caches distribués.

\subsection{État durable et état dynamique} \label{sota-state}

L'"état", ou "état local", fait référence aux données habituellement lues et écrites depuis et vers des variables ou un disque par un processus au cours de son exécution. FaaS n'offre aucune garantie quant à la disponibilité d'un tel stockage entre plusieurs exécutions. C'est pourquoi les fonctions serverless sont dites "sans état" : les données qui doivent être persistées doivent être stockées à l'extérieur, et les fonctions doivent être idempotentes afin d'empêcher la corruption de l'état.

En outre, les offres FaaS présentent des limitations arbitraires, notamment en ce qui concerne le temps d'exécution d'une fonction, la taille de la charge utile et la mémoire allouée (cf. tableau \ref{table:commercial-faas}). Il s'agit d'un problème lors de la conception d'applications "réelles" qui consistent en des tâches de longue durée et/ou qui comprennent des fonctions qui doivent communiquer ou se synchroniser, par exemple pour transmettre des résultats intermédiaires en fonction d'un état transitoire.

Étant donné la nature éphémère des instances de fonction, il faut être conscient de la tolérance aux pannes et de la cohérence des données dans leur application lorsqu'elles sont déployées sur FaaS.

Dans~\cite{wuTransactionalCausalConsistency2020}, les auteurs abordent la latence des E/S dans le contexte de la composition de fonctions serverless, où une application est divisée en plusieurs fonctions qui peuvent s'exécuter simultanément sur différents nœuds tout en accédant au stockage distant. Ils proposent HydroCache, un système qui met en œuvre leur idée de cohérence causale transactionnelle multisite (cohérence causale dans le cadre d'une transaction unique distribuée sur plusieurs nœuds). Ils observent des améliorations allant jusqu'à un ordre de grandeur en termes de performances tout en assurant la cohérence. HydroCache surpasse les solutions de l'état de l'art telles que Anna~\cite{Wu2018AnnaAK} et ElastiCache~\cite{elasticache}.

Dans~\cite{Perron2020StarlingAS}, les auteurs soutiennent que les analyses de bases de données serverless permettraient aux analystes de données d'éviter les coûts initiaux en réalisant l'élasticité. Cependant, comme ces types de charges de travail sont par nature imprévisibles, les fournisseurs de cloud ont tendance à avoir des difficultés à fournir des ressources adéquates, ce qui conduit à des solutions qui sont élastiques mais qui souffrent parfois de minutes de latence pendant les phases de mise à l'échelle. Ils présentent Starling, un moteur d'exécution de requêtes construit sur FaaS : trois niveaux de fonctions (Producers, Combiners et Consumers) peuvent évoluer indépendamment pour traiter des ensembles de données stockés sur un stockage à froid distant. Leur évaluation montre que Starling est rentable sur des volumes de requêtes modérés (moins de 120 requêtes par heure sur un ensemble de données TPC-H de 10 TB), tout en montrant de bons résultats de latence pour des analyses ad-hoc sur des données froides dans Amazon S3 et en étant capable de passer à l'échelle sur une base par requête.

Dans le serverless, la mise à l'échelle à partir de zéro lorsque l'activité revient après une période d'inactivité est généralement pilotée par les événements. Cela pose un problème lorsqu'aucune ressource matérielle n'est immédiatement disponible pour reprendre les charges de travail, ce qui induit une latence élevée. Dans~\cite{poppe2022moneyball}, les auteurs étudient l'auto-scaling proactif pour leur offre de base de données Azure SQL serverless. La contribution se concentre sur la prédiction des modèles de pause et de reprise afin d'éviter le problème de latence lors de la reprise de l'activité, et de minimiser la récupération des ressources en premier lieu lorsque les périodes d'inactivité sont courtes. En utilisant des échantillons de milliers de bases de données de production, ils ont constaté que seulement 23\% des bases de données sont imprévisibles, et ont formé des modèles d'apprentissage automatique sur trois semaines de données historiques pour construire un système de prédiction. L'approche a été utilisée avec succès en production chez Azure, atteignant 80\% de reprises proactives et évitant jusqu'à 50\% de pauses en moins.

Dans~\cite{Sreekanti2020CloudburstSF}, les auteurs s'appuient sur le KVS d'Anna~\cite{Wu2018AnnaAK} pour proposer une plateforme FaaS avec état. Cloudburst réalise un état mutable à faible latence et une communication avec un effort de programmation minimal. En s'appuyant sur les capacités d'Anna, ils fournissent des blocs de construction essentiels pour permettre l'état dans un contexte FaaS : communication directe entre les fonctions, accès à faible latence à l'état mutable partagé avec la cohérence de session distribuée, et la programmabilité pour mettre en œuvre de manière transparente les protocoles de cohérence de Cloudburst. Dans leur évaluation par rapport à des applications réelles, Cloudburst surpasse les solutions commerciales et celles de l'état de l'art d'au moins un ordre de grandeur tout en conservant des capacités d'auto-scaling.

\subsubsection{Magasins de données distribués}

L'invocation événementielle implique que les fonctions d'une application unique ne sont pas toujours exécutées sur le même nœud, de sorte que ces fonctions ne peuvent pas utiliser la mémoire partagée ou les communications interprocessus. En outre, compte tenu de la nature des offres serverless qui permettent une mise à l'échelle à zéro, les fonctions ne sont pas toujours dans un état d'exécution et, en tant que telles, ne sont pas adressables par le réseau. Compte tenu de ces contraintes, les développeurs doivent s'appuyer sur des communications accrues par le biais d'un stockage lent tel que les buckets S3 pour gérer l'état de leurs applications.

La mise à l'échelle d'une base de données à zéro présente des défis difficiles à relever : parvenir à une conception de base de données qui permette le serverless est un effort continu d'ingénierie et de recherche. Microsoft a récemment proposé des capacités de mise à l'échelle automatique dans sa base de données Azure SQL. En 2022, Cloudflare a introduit D1~\cite{cloudflare-d1}, qui est basé sur SQLite.

En effet, les applications serverless sont souvent déployées aux côtés d'un magasin de valeurs clés qui évolue beaucoup plus naturellement qu'une base de données, car les magasins de valeurs clés (KVS) sont essentiellement sans état et peuvent donc être distribués sur les nœuds~\cite{Klimovic2018PocketEE}. Étant donné que les systèmes KVS sont au cœur de l'état sans serveur, la mise en œuvre de KVS cohérents, efficaces et élastiques est un sujet de recherche animé. Cependant, les systèmes de stockage de qualité industrielle n'ont pas été conçus avec les propriétés serverless à l'esprit, ce qui entraîne une élasticité altérée et donc des coûts qui augmentent plus rapidement que linéairement avec la taille de l'infrastructure, ainsi que des performances incohérentes en fonction de l'échelle.

Dans~\cite{Wu2018AnnaAK}, les auteurs ont entrepris de concevoir un KVS pour n'importe quelle échelle : le magasin doit être extrêmement efficace sur un seul nœud et doit être capable d'évoluer de manière élastique vers n'importe quel déploiement dans le cloud. Leurs exigences en matière de conception comprennent le partitionnement de l'espace des clés (en commençant par le niveau multicœur pour garantir les performances) avec une réplication multimaître pour obtenir la concurrence, une exécution sans attente pour minimiser la latence et des modèles de cohérence sans coordination pour éviter les goulets d'étranglement lors des communications entre les cœurs et les nœuds. En utilisant une structure de données de pointe, les treillis, Anna peut fusionner efficacement les états de manière asynchrone (ou sans attente). L'évaluation montre qu'Anna surpasse Cassandra d'un facteur 10 lorsqu'elle est utilisée dans un cadre distribué, sur quatre nœuds à 32 cœurs situés dans des lieux géographiques différents.

Dans~\cite{Klimovic2018PocketEE}, les auteurs soutiennent que les services de stockage existants ont des objectifs orthogonaux ou contradictoires avec ceux d'un KVS serverless : ils sacrifient la performance ou le coût pour la durabilité ou la haute disponibilité des données. En particulier, ils constatent que ces systèmes ne sont intrinsèquement pas adaptés aux données intermédiaires (ou "éphémères") dans le contexte des communications inter-fonctions, car ils nécessitent un agent de longue durée pour assurer la communication entre les tâches. Les auteurs présentent Pocket, un magasin de données distribué conçu pour le partage de données intermédiaires dans le contexte de l'analytique serverless, avec des temps de réponse inférieurs à la seconde, un redimensionnement automatique des ressources et un placement intelligent des données sur plusieurs niveaux de stockage (DRAM, Flash, disque). Pour ce faire, les responsabilités sont réparties entre trois plans qui évoluent indépendamment : un plan de contrôle qui met en œuvre les politiques de placement des données, un plan de métadonnées qui permet de distribuer les données entre les nœuds, et le plan de stockage des données. Lorsqu'il est comparé à Redis pour des opérations MapReduce sur un ensemble de données de 100 Go, Pocket affiche des performances comparables tout en économisant près de 60\% en termes de coûts. Il est également beaucoup plus rapide qu'Amazon S3, avec une accélération de 4,1 fois sur les E/S éphémères.

\subsubsection{Stockage éphémère}

Le stockage dans le nuage est conçu comme un service à plusieurs niveaux : les données sont réparties entre les supports rapides, mais coûteux, et les supports lents, mais bon marché, en fonction de la fréquence d'utilisation, de la taille, de l'âge, etc.

\begin{table}[H]
    \caption{A simplified overview of media choice in tiered infrastructures}
    \centering
    \begin{tabular}{|c|ccc|cc|}
        \hline
        Capacity       & \multicolumn{3}{c|}{TB}                                     & \multicolumn{2}{c|}{GB}                     \\ \hline
        Addressability & \multicolumn{3}{c|}{Block}                                  & \multicolumn{2}{c|}{Byte}                   \\ \hline
        Consideration  & \multicolumn{3}{c|}{Cost}                                   & \multicolumn{2}{c|}{Data}                   \\ \hline
        Latency        & \multicolumn{1}{c|}{s}    & \multicolumn{1}{c|}{ms}   & µs  & \multicolumn{1}{c|}{µs}  & ns               \\ \hline
        Data           & \multicolumn{1}{c|}{Cold} & \multicolumn{1}{c|}{Warm} & Hot & \multicolumn{1}{c|}{Hot}   & Mission critical \\ \hline
        Medium         & \multicolumn{1}{c|}{Tape} & \multicolumn{1}{c|}{HDD}  & SSD (Flash) & \multicolumn{1}{c|}{NVRAM} & DRAM             \\ \hline
    \end{tabular}
    \label{table:tiered-storage}
\end{table}

Intel Optane sont des modules de mémoire persistante (PM) qui visent un niveau intermédiaire entre les SSD Flash et la DRAM : leur latence et leur bande passante sont légèrement inférieures à celles de la DRAM, mais ils offrent des capacités de mémoire non volatile de niveau SSD à un prix abordable (\cite{boukhobzaEmergingNvm},~\cite{Izraelevitz2019BasicPM},~\cite{boukhobzaFlashMemory}).

Dans~\cite{Chen2020FlatStoreAE}, les auteurs visent à fournir un moteur de stockage clé-valeur qui tirerait parti de la technologie de la mémoire persistante (PM, ou NVM pour mémoire non volatile) pour atteindre des performances supérieures à celles des disques en rotation ou de la mémoire Flash. Ils se concentrent sur les charges de travail à forte intensité d'écriture et de petite taille : en effet, des études antérieures (\cite{atikoglu2012WorkloadAnalysis},~\cite{rajesh2013Memcache}) ont montré que les pools Memcached dans la nature sont principalement utilisés pour stocker des objets de petite taille, par exemple 70\% d'entre eux sont plus petits que 300 octets chez Facebook. De plus, les analyses serverless échangent des données à courte durée de vie et sont donc très gourmandes en écriture, alors que les magasins d'objets ont historiquement été utilisés comme une couche de mise en cache dominée par la lecture. S'appuyant sur ces observations et sur les caractéristiques propres aux dispositifs PM, ils présentent FlatStore, un moteur KVS avec une surcharge d'écriture minimale, une faible latence et une évolutivité multicœur. Comme les mémoires persistantes présentent une adressabilité fine et une faible latence par rapport aux disques durs et aux disques SSD, les auteurs ont conçu FlatStore pour une mise en lots minimale des opérations d'écriture afin d'éviter la contention. Lors de l'évaluation comparative des données de Facebook avec des éléments minuscules (1-13 octets) à grands ($>$ 300 octets), l'évaluation montre que FlatStore est de 2,5 à 6,3 fois plus rapide que les solutions de l'état de l'art.
\\

Le statefulness est un problème majeur pour les plateformes serverless. Les fournisseurs de services déploient une variété de logiciels BaaS pour combler le fossé entre les modèles de services traditionnels et FaaS et permettre aux développeurs de déployer leurs applications complètes sur leurs offres serverless. Les fonctions serverless présentent un stockage et un calcul intrinsèquement désagrégés, car elles sont déployées à la volée dans plusieurs zones géographiques, sur des ressources matérielles allouées dynamiquement par le fournisseur. Elles ont besoin d'un moyen d'opérer sur les données qui soit suffisamment rapide, qui offre des garanties de cohérence et qui évolue en cohérence avec le modèle de tarification "pay-as-you-go". Il y a de la place pour la recherche dans le domaine des magasins de données distribués et de l'utilisation de la mémoire non volatile émergente pour accélérer le débit.

\section{Des défis transversaux dans le cloud}

\subsection{Isolation et sécurité} \label{sota-isolation}

Pour réaliser la mise en commun des ressources, les fournisseurs de services en nuage s'appuient sur les technologies de virtualisation afin d'isoler les charges de travail des clients. En outre, ils proposent différents modèles de services, allant de IaaS à FaaS, qui nécessitent tous différentes techniques de sandboxing offrant un équilibre différent entre les performances et l'isolation.

Le compromis habituel se produit entre la robustesse de l'isolation basée sur l'hypervision (VM), où chaque bac à sable exécute un système d'exploitation distinct, et les performances de la virtualisation au niveau du système d'exploitation (conteneurs), où les bacs à sable partagent tous le noyau de l'hôte. Idéalement, les fournisseurs de services en nuage ne devraient pas avoir à sacrifier l'une de ces deux caractéristiques essentielles. Des efforts ont été faits pour réduire la surcharge de virtualisation afin de diminuer les temps de démarrage et de réduire l'écart de performance entre ces deux techniques~\cite{mancoMyVMLighter2017}.

\subsubsection{MicroVMs}

Dans~\cite{agacheFirecrackerLightweightVirtualization}, les auteurs identifient de nombreux défis pour concevoir une méthode d'isolation spécifiquement adaptée aux charges de travail serverless dans le contexte d'AWS Lambda -- Firecracker doit fournir une sécurité de niveau VM avec une densité de sandboxing de niveau conteneur sur un seul hôte, avec des performances proches de bare-metal pour toute application compatible avec Linux. L'overhead de Firecracker doit être suffisamment faible pour que la création et l'élimination des sandboxes soient suffisamment rapides pour AWS Lambda ($\leq 150 \, ms$), et le gestionnaire doit permettre de sur-engager les ressources matérielles avec des sandboxes ne consommant que les ressources dont ils ont besoin. Avec Firecracker, les auteurs présentent un nouveau moniteur de machine virtuelle (VMM) basé sur Linux KVM pour exécuter des machines virtuelles minimales (ou MicroVM) qui contiennent un noyau Linux et un espace utilisateur minimaux non modifiés. Grâce à la mise en commun des bacs à sable, Firecracker permet d'obtenir des temps de démarrage rapides et une densité élevée de bacs à sable sur un seul hôte, pour n'importe quelle application Linux donnée. Il est utilisé avec succès en production dans AWS Lambda depuis 2018.

Dans~\cite{Anjali2020BlendingCA}, les auteurs étudient les différences d'utilisation des fonctionnalités du noyau hôte entre les conteneurs Linux (LXC), les MicroVM Firecracker et les conteneurs sécurisés gVisor de Google. Les bacs à sable gVisor sont des conteneurs \texttt{seccomp} : ils sont limités à 4 appels système, à savoir \texttt{exit}, \texttt{sigreturn}, et \texttt{read} et \texttt{write} sur des descripteurs de fichiers déjà ouverts. Les fonctionnalités étendues reposent sur un noyau Go-written user space appelé Sentry qui intercepte et met en œuvre les appels système et gère les descripteurs de fichiers. Cela empêche toute interaction directe entre l'application en bac à sable et le système d'exploitation hôte. Bien qu'elle permette de réaliser une isolation sécurisée, la conception de gVisor est compliquée et ajoute des frais généraux : les auteurs ont constaté que gVisor a la plus grande empreinte en termes d'utilisation du processeur et de la mémoire, avec la bande passante la plus lente pour les opérations sur le réseau.

Dans~\cite{wanningerIsolatingFunctionsHardware2022a}, les auteurs soutiennent que l'écosystème de virtualisation manque d'une solution adaptée à l'isolation à la granularité d'une fonction unique. Ils présentent virtines, un mécanisme léger d'isolation des VM, et Wasp, un hyperviseur de type 2 à bibliothèque minimale qui fonctionne sous GNU/Linux et Windows. Les virtines sont guidées par le programmeur : les annotations aux limites des fonctions permettent au compilateur d'emballer automatiquement des sous-ensembles de l'application dans des VM légères avec un temps d'exécution compatible POSIX. Wasp fonctionne de manière client-serveur : le moteur d'exécution (client) émet des appels à l'hyperviseur (serveur) qui détermine si chaque requête individuelle est autorisée à être traitée conformément à une politique définie par l'administrateur. Lors de leur évaluation avec une application JavaScript, les auteurs ont constaté que cette conception introduit un surcoût limité de 125 µs dans le temps de démarrage par rapport à la ligne de base, tout en réalisant efficacement une isolation finement ajustable pour des fonctions sélectionnées, sans presque aucun effort de la part du programmeur.

\subsubsection{Unikernels}

L'idée derrière les unikernels est de fournir la fonctionnalité du système d'exploitation comme une bibliothèque qui peut être incorporée dans une application sandbox afin d'éviter d'emballer et de démarrer un système d'exploitation complet pour exécuter l'application, et d'éliminer les commutations de contexte coûteuses de l'espace utilisateur à l'espace noyau. Dans~\cite{kuenzerUnikraftFastSpecialized2021a}, les auteurs présentent Unikraft, une initiative de la Fondation Linux. Unikraft vise à rendre le processus de portage aussi indolore que possible pour les développeurs qui souhaitent exécuter leurs applications sur des unikernels. Les images résultantes pour différentes applications (nginx, SQLite, Redis) sont proches de la taille la plus petite possible, c'est-à-dire la taille binaire de l'espace utilisateur Linux, avec une surcharge de mémoire très limitée pendant l'exécution ($<$ 10 MB de RAM) et des temps de démarrage rapides de l'ordre de la milliseconde. Les applications emballées par Unikraft permettent d'améliorer les performances de 1,7 à 2,7 fois par rapport aux machines virtuelles invitées Linux traditionnelles.

Dans~\cite{caddenSEUSSSkipRedundant2020}, les auteurs présentent un mécanisme de mise en cache à haute densité qui exploite les unikernels et l'instantanéité (voir \ref{sota-snapshotting}) pour accélérer les déploiements. Ils soutiennent que les fonctions serverless sont de bons candidats pour la mise en cache : comme elles sont généralement écrites dans des langages de haut niveau qui s'exécutent dans des interpréteurs, leur chemin de démarrage consiste principalement à initialiser cet interpréteur et les dépendances associées, qui peuvent être partagées dans différents bacs à sable. Le mécanisme de snapshotting bénéficie de l'agencement de la mémoire du noyau unique, où toutes les fonctionnalités (du système de fichiers à la pile réseau, en passant par l'application utilisateur) sont combinées dans un seul espace d'adressage plat. Nous mettons en œuvre ce mécanisme dans SEUSS afin de mettre en cache plus de 16 fois plus de bacs à sable unikernel en mémoire que les conteneurs basés sur Linux. En outre, les temps de déploiement passent de centaines de millisecondes à moins de 10 ms, et la gestion par la plateforme des rafales de requêtes s'améliore considérablement dans le cadre de la mise en cache à haute densité, ce qui entraîne une réduction du nombre de requêtes échouées.

Dans~\cite{tanLightweightServerlessComputing2020}, les auteurs présentent Unikernel-as-a-Function (UaaF), un espace d'adressage unique, un OS de bibliothèque visant à déployer des fonctions serverless. UaaF s'appuie sur l'observation que les invocations de fonctions croisées sont lentes dans les déploiements serverless qui s'appuient sur des interfaces de passage de messages basées sur le réseau (voir \ref{sota-communications}) ; en outre, les invités Linux souffrent d'une surcharge d'utilisation de la mémoire dans les bacs à sable et leur latence de démarrage n'est pas satisfaisante (voir \ref{sota-cold-start}). Les auteurs étudient l'utilisation de VMFUNC, une technologie Intel pour les invocations de fonctions entre sandboxes qui ne subissent pas de latence lors de la sortie d'une VM vers l'hyperviseur. Cette technologie permet effectivement l'invocation de fonctions à distance, donnant ainsi des capacités IPC sécurisées et prises en charge par le matériel aux fonctions serverless. Ils proposent également un nouveau modèle de programmation pour les fonctions serverless : \textit{session} et \textit{bibliothèque}, les premières étant des fonctions de "flux de travail" (ou squelette) et les secondes étant du code réel, téléchargé par les clients et éventuellement partagé entre les applications. Dans leur évaluation, les auteurs mettent en œuvre l'UaaF avec trois unikernels (Solo5, MirageOS et IncludeOS) et montrent que la communication inter-fonctions dans l'UaaF est inférieure de trois ordres de grandeur à la CIP native de Linux. Leur modèle de programmation permet de réduire la surcharge de mémoire et les temps d'initialisation à plusieurs millisecondes grâce aux fonctions partagées.
\\

Les charges de travail FaaS ont une durée de vie beaucoup plus courte que les charges de travail des offres traditionnelles. En tant que tel, s'appuyer sur des techniques de virtualisation qui n'ont pas été construites pour serverless est sous-optimal : les temps d'initialisation peuvent ne pas répondre aux exigences de latence lors de la mise à l'échelle à partir de zéro ; les tailles des bacs à sable peuvent être trop élevées pour être mises en cache dans la mémoire compte tenu de l'augmentation du multitenancy ; l'isolation peut être trop faible pour colocaliser les travaux de différents clients. Cette évaluation a suscité un intérêt pour la recherche autour des unikernels et des MicroVM, tandis que les fournisseurs commerciaux ont développé leurs propres approches, telles que Firecracker pour AWS ou gVisor pour Google Cloud.

\subsection{Modèle de programmation et enfermement propriétaire}

Comme le montre la figure \ref{fig:web-app}, les applications FaaS ont tendance à s'appuyer fortement sur les offres BaaS pour bénéficier des économies de coûts associées à leur capacité à passer à l'échelle zéro. Ce lien introduit un risque d'enfermement dans des solutions spécifiques au fournisseur qui pourraient ne pas être disponibles dans les offres commerciales, ou disponibles sous forme de logiciels open source prêts à l'emploi.

En outre, certains fournisseurs pratiqueront une tarification prohibitive de la bande passante de sortie afin de dissuader leurs clients de transférer leurs données à un concurrent.

Un autre aspect de ce problème est la difficulté de développer, tester et déboguer les applications FaaS localement~\cite{thalheimVMSHHypervisoragnosticGuest2022}. Au minimum, les développeurs devront simuler la passerelle API afin d'exécuter des suites de tests ; si leur application utilise des solutions de stockage ou des bus de communication spécifiques à un fournisseur, les développeurs devront déployer des solutions similaires ou simuler les spécificités de ces blocs de construction BaaS, par exemple leur API et leurs caractéristiques de performance.

Cela représente des efforts d'ingénierie non négligeables et, de fait, le déploiement d'une infrastructure serverless à part entière pour le staging pourrait contrebalancer les avantages en termes de coûts d'exploitation du choix du serverless pour la production. Les ingénieurs débutants~\cite{jeffreyd2022aws} et chevronnés~\cite{mitchell2022serverless} déclarent avoir des difficultés avec l'outillage, les tests, l'écosystème général et la complexité accrue du développement d'applications FaaS.

Dans~\cite{Pons2019OnTF}, les auteurs observent que la désagrégation des ressources de stockage et de calcul dans FaaS limite le développement d'applications qui font un usage intensif de l'état mutable partagé et qui se synchronisent beaucoup entre les itérations. En effet, l'état ne persiste pas entre les invocations d'une même fonction (voir \ref{sota-state}), et le passage de messages pour les communications entre fonctions induit un surcoût élevé (voir \ref{sota-communications}). Ils se concentrent en particulier sur les algorithmes d'apprentissage automatique (\textit{k}-means clustering et régression logistique). Ils présentent Crucial, un framework visant à soutenir le développement d'applications serverless stateful. Crucial fournit aux applications une couche de mémoire partagée qui garantit la durabilité grâce à la réplication, avec de fortes garanties de cohérence. Le modèle de programmation Crucial est basé sur les annotations, ce qui permet aux programmeurs de porter une application multithread sur une plateforme FaaS avec un minimum d'implication. L'évaluation par rapport à un cluster Spark sur un ensemble de données de 100 Go montre que Crucial fonctionnant sur AWS Lambda introduit très peu de frais généraux, ce qui lui permet de surpasser Spark de 18 à 40 % en termes de performances pour un coût similaire.

Dans~\cite{zhangKappaProgrammingFramework2020}, les auteurs reconnaissent que le modèle de programmation serverless est un défi pour les développeurs. Ils ont la responsabilité de partitionner correctement leur code en unités de travail sans état, de gérer les mécanismes de coordination pour réaliser une architecture microservices et de mettre en œuvre des modèles de cohérence pour la conservation de l'état en cas de défaillance. Cette complexité pourrait dissuader les clients de déployer des applications à usage général qui bénéficieraient grandement du niveau de parallélisme offert par les fournisseurs serverless. Ils présentent Kappa, un framework Python pour les applications serverless. Kappa fournit une API familière qui permet de réaliser des points de contrôle (en stockant périodiquement l'état de l'application afin que le programme puisse reprendre en cas de dépassement de temps), la concurrence (en prenant en charge la création de tâches, l'attente sur les futurs et le passage de messages entre fonctions) et la tolérance aux pannes (en garantissant une restauration idempotente de l'état lors de la reprise à partir de points de contrôle). Les applications Kappa peuvent être déployées sur n'importe quelle plateforme serverless, car le framework ne nécessite aucun changement côté serveur. Dans leur évaluation, ils mettent en œuvre cinq applications avec Kappa et les résultats indiquent que le mécanisme de point de contrôle fonctionne bien lorsque les fonctions expirent souvent, avec moins de 9 % d'augmentation du temps de réponse avec une durée d'expiration importante (15 secondes), et un maximum de 3,2 % avec une période d'expiration plus raisonnable de 60 secondes.

Afin de limiter l'augmentation de la latence lors de la mise à l'échelle à partir de zéro, les images de conteneurs ou de VM qui prennent en charge les applications serverless sont généralement rendues aussi maigres et légères que possible. Cela dissuade les développeurs d'inclure des outils de surveillance ou de débogage, ce qui rend très difficile l'inspection d'une fonction serverless au moment de l'exécution. Dans~\cite{thalheimVMSHHypervisoragnosticGuest2022}, les auteurs présentent VMSH, un mécanisme qui permet d'attacher des images invitées arbitraires à des VM légères en cours d'exécution afin de les instrumenter à des fins de développement ou de débogage. L'évaluation effectuée sur KVM - bien que VMSH soit conçu comme une solution indépendante de l'hyperviseur - montre que le chargement latéral de l'invité n'ajoute aucune surcharge à la VM invitée d'origine, ce qui permet de réduire le compromis entre les VM légères et sans fioritures et la fonctionnalité.
\\

Il existe un compromis clair pour fournir des bacs à sable aussi petits que possible afin de minimiser les coûts de stockage et de mémoire dans les plateformes serverless, tout en fournissant des outils adéquats aux développeurs pour construire, tester, distribuer et déployer leurs fonctions. En outre, le modèle de programmation basé sur des fonctions sans état a mis en lumière un nouveau défi : un outillage côté fournisseur et côté développeur pour le FaaS avec état est nécessaire pour permettre le déploiement serverless des applications existantes et futures qui utilisent des services à longue durée d'exécution et la persistance des données.

\section{Perspectives et orientations futures}

La section précédente a donné un aperçu des contributions liées aux défis techniques de l'informatique serverless. Dans cette section, nous présentons quelques orientations futures pour la recherche dans ce domaine. Nous présentons les problèmes étudiés dans les travaux des communautés du cloud, des systèmes et des bases de données. Nous soutenons que les contributions s'appuyant sur ces perspectives auraient le potentiel de renforcer les plateformes serverless pour une reconnaissance plus large du paradigme serverless.

\subsection{Accords de niveau de service}

En 2011, Buyya et al.~\cite{buyyaSLAorientedResourceProvisioning2011} ont plaidé pour une allocation des ressources orientée SLA dans l'informatique en nuage à l'aube de l'ère des microservices. Ils ont identifié la fiabilité dans l'informatique utilitaire comme un défi majeur pour les prochaines décennies : même avec des ressources réservées dans les modèles de service traditionnels, la complexité croissante des applications des clients a fait du respect des accords de niveau de service (SLA) un problème difficile mais inévitable pour les fournisseurs d'informatique en nuage.

La latence, le débit et la continuité du service sont difficiles à garantir dans l'informatique en nuage lors de l'utilisation de ressources non réservées~\cite{dartoisCuckooOpportunisticMapReduce2019}. En raison de la nature transitoire des sandboxes de fonctions dans l'informatique serverless, les plateformes d'auto-scaling sont confrontées à un problème similaire d'allocation dynamique des ressources. Cependant, être en mesure d'offrir des accords de niveau de service aux clients et de respecter les engagements de qualité de service (QoS) en tant que fournisseur est nécessaire pour l'adoption à grande échelle du modèle de service serverless~\cite{elsakhawyFaaS2FFrameworkDefining2020}.

Dans~\cite{chahalSLAawareWorkloadScheduling2020}, les auteurs soutiennent que les plateformes d'auto-scaling serverless sont mises au défi par les charges de travail en rafale. Dans leur travail, ils soulignent l'importance de la caractérisation de la charge de travail pour adapter la quantité de VM réservées nécessaires pour respecter les accords de niveau de service. Lorsque le nombre de requêtes entrantes fait grimper le niveau de concurrence dans les VM réservées et fait passer la latence des tâches au-delà du seuil acceptable négocié via le SLA, ils s'appuient sur une plateforme serverless pour s'adapter aux tâches supplémentaires et maintenir les performances. Bien que ce cadre ait réussi à maintenir la plupart des temps de réponse sous le seuil cible, les auteurs constatent encore un nombre incompressible de violations causées par des retards de démarrage à froid sur la plateforme serverless.

Dans~\cite{choSLADrivenMLInference}, les auteurs soutiennent que le modèle de tâches dans l'informatique serverless et la vue de l'infrastructure dans les plateformes auto-scaling sont inadéquats pour répondre aux besoins des clients en termes de niveau de service. En effet, les auto-scalers basent leurs décisions d'allocation sur des métriques génériques telles que les requêtes par seconde (QPS) qui ne reflètent pas les caractéristiques spécifiques de l'application et ne prennent pas en compte l'hétérogénéité des ressources matérielles disponibles. Ils proposent un cadre dans lequel les mesures de l'application (telles que le temps d'exécution de la requête) sont transmises à l'auto-scaler afin qu'il alloue les ressources en fonction des objectifs de niveau de service spécifiés par l'utilisateur, tels que le temps de latence cible. Cependant, les temps de réponse observés ne sont pas déterministes en raison des délais de démarrage à froid, et les latences cibles définies par l'utilisateur sont susceptibles d'être violées dans un scénario de mise à l'échelle automatique.

Afin de répondre aux exigences de qualité de service par utilisateur, les plateformes de mise à l'échelle automatique devraient prendre en compte les caractéristiques des ressources matérielles hétérogènes, et les accords de niveau de service devraient être négociés sur la base d'une requête plutôt que sur la base d'une fonction. Nous pensons que des politiques d'auto-scaling basées sur la caractérisation de la charge de travail et de la plateforme pourraient être mises en œuvre pour minimiser l'impact de la latence de démarrage à froid et permettre aux plateformes serverless de respecter les SLA avec des ressources hétérogènes non réservées.

\subsection{Efficacité énergétique}

La consommation d'énergie dans l'informatique en nuage est un défi crucial : en 2010, les centres de données représentaient entre 1,1 et 1,5 % de la consommation mondiale d'électricité~\cite{koomey2011Growth}, et les projections pour 2030 montrent que ces chiffres pourraient passer de 3 à 13 % de la consommation mondiale d'électricité~\cite{andraeGlobalElectricityUsage2015}. Le serverless devenant un modèle de service de plus en plus populaire pour le cloud, et de nombreux auteurs considérant le serverless comme l'avenir du cloud computing, il existe une opportunité pour les fournisseurs de cloud de mettre en œuvre des politiques énergétiques à l'échelle.

Pour être efficace en termes de coût et de consommation d'énergie, une plateforme d'auto-scaling devrait être capable de redimensionner les ressources allouées dans une infrastructure cloud serverless, tout en étant suffisamment réactive pour s'adapter aux changements de charge de travail sans impacter les utilisateurs finaux avec des pics de latence. Cela met en évidence un compromis entre l'énergie et la performance : la sursouscription des ressources peut aider à garantir une faible latence lors de l'invocation des fonctions, mais entraînera une consommation d'énergie plus élevée.

La multilocation a permis de ralentir la croissance du nombre de serveurs dans les centres de données~\cite{masanetRecalibratingGlobalData2020}. Avec les promesses de colocalisation massive de travaux éphémères, le serverless semble être une direction prometteuse pour les infrastructures cloud qui cherchent à réduire leur empreinte énergétique.

La consolidation de la charge de travail est une technique qui consiste à maximiser le nombre de tâches sur le plus petit nombre de nœuds~\cite{chaurasiaComprehensiveSurveyEnergyaware2021}. Cela permet une gestion dynamique de l'énergie : les nœuds qui ne sont pas sollicités peuvent alors être mis hors tension, et les nœuds qui observent une charge modérée peuvent être ralentis, c'est-à-dire via le CPU throttling~\cite{liuHierarchicalFrameworkCloud2017}.

L'un des problèmes fondamentaux du paradigme serverless est l'architecture intrinsèque d'expédition des données \footnote{Déplacer les données de l'endroit où elles sont stockées vers l'endroit où elles doivent être traitées}.~\cite{chikhaouiMultiobjectiveOptimizationData2021a}. Étant donné que les sandboxes de fonctions sont déployées sur des nœuds dans diverses régions géographiques pour réaliser l'équilibrage de charge et la disponibilité, les plateformes serverless expédient jusqu'à des téraoctets de données depuis les nœuds de stockage vers le code dont la taille peut varier de kilooctets à mégaoctets au sein des nœuds de calcul.

Les fonctions de stockage permettent d'exécuter de petites unités de travail directement sur les nœuds de stockage~\cite{zhangNarrowingGapServerless2019}, réalisant 14\% à 78\% d'accélération par rapport au stockage à distance. Les fonctions de stockage ne remettent pas en question la désagrégation physique des ressources de stockage et de calcul qui est essentielle dans l'informatique en nuage, tout en limitant efficacement le mouvement des données entre les nœuds et en réduisant ainsi la consommation d'énergie dans un centre de données.

Le stockage informatique est un moyen de décharger les charges de travail de l'unité centrale vers le contrôleur de stockage~\cite{barbalaceComputationalStorageWhere}. Lorsqu'il s'agit de traiter de grandes quantités de données, ces techniques peuvent contribuer à diminuer les transferts de données, à améliorer les performances et à réduire la consommation d'énergie. Bien que ces technologies ne soient pas encore prêtes pour une utilisation en production, elles offrent des opportunités de recherche intéressantes pour la communauté serverless.

Ces techniques pourraient être mises en œuvre dans les plateformes serverless pour obtenir des gains supplémentaires en matière de consommation d'énergie dans le cloud. Cela implique de prendre en compte la diversité des applications des utilisateurs et l'hétérogénéité des requêtes et des ressources matérielles.

\subsection{Allocation de ressources assistée par l'IA}

Dans le paradigme serverless, il incombe au fournisseur de redresser l'allocation des ressources matérielles afin que les charges de travail de leurs clients soient exécutées à temps. L'allocation dynamique des ressources matérielles appropriées pour les tâches événementielles dans une infrastructure hétérogène est un problème difficile qui peut frapper une barrière de complexité de calcul à l'échelle, avec l'ordonnanceur en ligne produisant des solutions sous-optimales~\cite{lopesTaxonomyJobScheduling2016b}. Les techniques d'intelligence artificielle (IA) peuvent aider à surmonter ce défi.

Certains auteurs s'attendent à ce que l'informatique autonome pilotée par l'IA devienne la norme dans les futurs systèmes~\cite{gillAINextGeneration2022a}. L'idée de l'informatique autonome est de construire des systèmes autogérés et auto-adaptatifs qui résistent à un environnement extrêmement changeant à grande échelle~\cite{puvianiSelfManagementCloudComputing2013}. Ces systèmes peuvent être mis en œuvre à l'aide de l'apprentissage automatique de manière rentable, en utilisant des modèles qui ne nécessitent pas d'intervention humaine importante pour la supervision.

Dans~\cite{schulerAIbasedResourceAllocation2021}, les auteurs montrent que l'apprentissage par renforcement (RL) peut réaliser une mise à l'échelle appropriée sur une base par charge de travail, résultant en une performance améliorée par rapport à la configuration de base. Dans leur contribution, ils proposent un modèle d'apprentissage par renforcement qui détermine et ajuste efficacement le niveau de concurrence optimal pour une charge de travail donnée.

Les plateformes serverless nécessitent une allocation réactive des ressources et une ordonnance des tâches dans le cadre de SLA avec des exigences de qualité de service par requête~\cite{gujaratiSwayamDistributedAutoscaling2017}. Les techniques d'apprentissage automatique peuvent aider à répondre aux exigences de qualité de service dans les paradigmes traditionnels du cloud computing~\cite{soniMachineLearningTechniques2022}, et ont été utilisées pour renforcer la consolidation des machines virtuelles~\cite{shawApplyingReinforcementLearning2022}. La gestion et l'optimisation des ressources à l'aide de l'IA et de la ML pourraient permettre de tirer davantage parti de l'hétérogénéité des ressources matérielles dans une infrastructure en nuage.

\section{Conclusion}

En libérant les utilisateurs de la contrainte du dimensionnement de leur infrastructure, le modèle de service serverless pour le cloud promet de faciliter le passage à l'échelle des applications. Grâce au mécanisme d'allocation à la demande, les clients peuvent bénéficier d'économies considérables, en ne payant plus pour des ressources qui seraient essentiellement dormantes, en attente d'une requête.

Toutefois, les solutions serverless actuelles présentent des inconvénients non négligeables qui limitent l'utilisation du serverless à des cas d'usage spécifiques. Ce paradigme se réalise aujourd'hui sous la forme d'un contrat sur le modèle de programmation : les utilisateurs des offres serverless doivent concevoir leurs applications comme un ensemble de fonctions pures -- idempotentes, leur exécution n'entraîne pas d'effets de bord -- ce qui constitue un lourd effort d'ingénierie.

Le fonctionnement de cette architecture logicielle, qui présente des similitudes avec l'architecture en micro-services, repose sur la communication par passage de messages entre fonctions. Les fonctions n'étant pas directement adressables sur le réseau dans les solutions commerciales actuelles, cette communication s'effectue par le biais d'un stockage lent : cela induit un surcoût important sur les performances de l'application lors des phases de composition et de synchronisation, jusqu'à parfois contrebalancer les gains offerts par le parallélisme massif inhérent au paradigme serverless.

Par ailleurs, le passage à l'échelle depuis zéro est associé à un fréquent risque de latence lors du réveil de l'application, puisque le fournisseur de services doit alors dynamiquement allouer des ressources matérielles et instancier l'environnement d'exécution des fonctions pour répondre à l'événement déclencheur. Les fournisseurs de services ont tendance à pré-allouer des ressources de manière à éviter ces démarrages à froid, ce qui contraint leurs gains potentiels en rendant ces ressources indisponibles pour d'autres clients.

Enfin, les accélérateurs matériels sont les grands absents de l'offre serverless commerciale. À l'heure où la demande en GPU et FPGA est croissante pour répondre aux besoins en calcul massivement parallèle, notamment dans le cadre de l'apprentissage machine ou de l'analyse de données "big data", les clients doivent se tourner vers une offre cloud plus conventionnelle s'ils souhaitent bénéficier de plateformes d'exécution hétérogènes.


\part{Contributions}
\label{part:two}

\clearemptydoublepage
\chapter{HeROfake : Orchestration serverless sur ressources hétérogènes pour le cloud privé}
\label{chapter:herofake}

TODO: détails sur la normalisation

TODO: détails sur Weighted Sum Method pour la résolution MOO + pourquoi pas ILP ? etc.

TODO: scalarization~\cite{ehrgottMulticriteriaOptimization882005}

TODO: Front de Pareto : comment choisir une solution plutôt qu'une autre ?

TODO: phase hors-ligne : wattmètres logiciels comme PowerAPI~\cite{fieniPowerAPIPythonFramework2024}, EcoFloc~\cite{valeraEnergySavingPerspective}, Scaphandre et d'autres~\cite{jayExperimentalComparisonSoftwarebased2023}

\section{Introduction}
\label{section:herofake-introduction}

\begin{table*}[t]
    \centering
    \caption{State of the Art work on autoscaling platforms}
    \resizebox{\textwidth}{!}{
        \begin{tabular}{lccccccc}
            \toprule
            & Serverless & Target cloud platform     & SLA & Hardware heterogeneity & Resources usage & Energy consumption & Cost-aware \\
            \cmidrule(lr){2-2}\cmidrule(lr){3-3}\cmidrule(lr){4-4}\cmidrule(lr){5-5}\cmidrule(lr){6-6}\cmidrule(lr){7-7}\cmidrule(lr){8-8}
            Swayam~\cite{gujaratiSwayamDistributedAutoscaling2017}        & \xmark         & Private (Azure, in-house) & \cmark& \xmark                     & \cmark            & \xmark                 & \xmark         \\
            Pigeon~\cite{lingPigeonDynamicEfficient2019}                  & \cmark       & Private                   & \xmark  & \cmark                   & \cmark            & \xmark                 & \xmark         \\
            MArk~\cite{zhangMArkExploitingCloud}                          & \xmark         & Public (AWS)              & \cmark& \cmark                   & \cmark            & \xmark                 & \cmark       \\
            ENSURE~\cite{sureshENSUREEfficientScheduling2020}             & \cmark       & Private                   & \xmark  & \xmark                     & \cmark            & \xmark                 & \cmark       \\
            Mampage et al.~\cite{mampageDeadlineawareDynamicResource2021} & \cmark       & Private                   & \cmark& \xmark                     & \cmark            & \xmark                 & \cmark       \\
            Atoll~\cite{singhviAtollScalableLowLatency2021}               & \cmark       & Private                   & \cmark& \xmark                     & \xmark              & \xmark                 & \xmark         \\
            INFless~\cite{yangINFlessNativeServerless2022}                & \cmark       & Private                   & \cmark& \xmark                     & \cmark            & \xmark                 & \cmark       \\
            SMIF~\cite{choSLADrivenMLInference}                           & \cmark       & Private                   & \cmark& \cmark                   & \cmark            & \xmark                 & \xmark         \\
            Target solution                                                & \cmark       & Private                   & \cmark& \cmark                   & \cmark            & \cmark               & \cmark       \\ \bottomrule
        \end{tabular}
    }
    \label{table:herofake-sota}
\end{table*}

\textbf{Modèle serverless}. Le serverless peut être compris à la fois comme un modèle de programmation, appelé Function as a Service (FaaS), et comme un modèle de déploiement pour le cloud. Dans un tel modèle, les développeurs conçoivent leurs applications comme une composition de fonctions sans état dont l'exécution est pilotée par des événements~\cite{SchleierSmith2021WhatSC}. 
Les services serverless libèrent les locataires d'une réservation complexe des ressources, car ils sont conçus pour gérer les exigences de mise à l'échelle à la demande.

Dans le modèle FaaS, les fournisseurs ne facturent les clients qu'en fonction de leur utilisation réelle des ressources~\cite{jonasCloudProgrammingSimplified2019}. Ils sont entièrement responsables du déploiement d'une gestion intelligente des ressources et du multiplexage à une granularité plus fine afin d'optimiser les mesures de qualité de service (QoS) telles que le temps de réponse, la consommation d'énergie, etc.

\textbf{Détection de deepfake et serverless}. Le travail présenté dans cet article faisait partie d'un projet (à l'institut de recherche b{\textless\textgreater}com \footnote{\href{https://b-com.com/en}{https://b-com.com/en}}) visant à déployer un service de détection de deepfake économe en énergie dans un cloud hétérogène. Les deepfakes sont des images, des vidéos ou des discours synthétiques, créés numériquement pour imiter une personne existante de manière à tromper les spectateurs. La détection de deepfake consiste à entraîner un réseau neuronal convolutif (CNN) pour détecter des modèles d'incohérences introduits dans le processus de création.

Les fonctions utilisées par notre application deepfake répondent à trois caractéristiques principales pour des charges de travail serverless adaptées~\cite{cncf2018whitepaper} : leur exécution peut être rendue parallèle (plusieurs images indépendantes), elles sont stateless (pure transformation sur les données d'entrée) et event-driven (lancées après l'upload des données).

D'une part, la détection de deepfake à l'aide de réseaux de neurones convolutifs (CNN) est une tâche qui peut tirer parti de la concurrence en exécutant plusieurs convolutions en parallèle et/ou en traitant différentes images sur plusieurs threads. Ces tâches sont essentiellement sans état, car elles appliquent une transformation pure sur les données d'entrée - en prenant une image en entrée et en renvoyant une valeur booléenne en sortie. Une telle application est pilotée par les événements, le calcul commençant après le téléchargement d'une image d'entrée. Ce sont là trois caractéristiques principales des charges de travail serverless appropriées~\cite{cncf2018whitepaper}.

D'autre part, les ressources matérielles nécessaires pour exécuter cette application à l'échelle seraient nombreuses et coûteuses : être en mesure de faire évoluer dynamiquement les ressources en fonction de la demande permettrait au client de réaliser d'importantes économies et au fournisseur d'accepter plus de clients sur le même nombre de nœuds.

Par conséquent, nous soutenons que le modèle de service serverless est parfaitement adapté à l'inférence à la demande rentable utilisant des CNN.

\textbf{Hétérogénéité matérielle dans le cloud}. Les infrastructures cloud sont de plus en plus hétérogènes pour répondre aux besoins des applications à forte intensité de données telles que l'apprentissage automatique de modèles ou l'analyse de données massives~\cite{reissHeterogeneityDynamicityClouds}. Cependant, les processeurs spécialisés et les GPU doivent encore être mis à la disposition des clients dans les offres serverless~\cite{khandelwalTaureauDeconstructingServerless2020}. L'accélération matérielle devrait être décidée par le fournisseur sur la base d'une application ou d'une demande.

Les travaux de l'état de l'art montrent que l'utilisation de ce matériel dans un cadre cloud permet des gains substantiels en termes de temps d'exécution et de consommation d'énergie~\cite{10.1145/3369583.3392679, 9195730}. Cependant, les orchestrateurs de référence tels que Kubernetes avec Knative ou OpenWhisk ne prennent pas en charge l'allocation dynamique de ce type de matériel.

\textbf{Défi de performance pour le déploiement serverless}. En raison de la nature transitoire des ressources FaaS non réservées, la latence, le débit et la continuité du service sont difficiles à garantir~\cite{vaneykSPECRGCloud2018, dartoisCuckooOpportunisticMapReduce2019}. Lorsque les applications ne reçoivent pas de demandes entrantes, les bacs à sable des fonctions sont détruits au lieu d'être maintenus dans un état d'inactivité. Ensuite, lorsqu'une nouvelle demande arrive, le fournisseur doit (ré)allouer des ressources et initialiser des fonctions pour déployer de nouveaux bacs à sable : c'est ce qu'on appelle un démarrage à froid. Les temps de démarrage à froid sont très pénalisants pour les performances de l'application, ils peuvent même dominer les temps d'exécution totaux~\cite{mullerLambadaInteractiveData2020}.

De plus, dans les offres commerciales serverless actuelles, les accords de niveau de service (SLA) sont généralement limités à des tentatives automatisées (redémarrages) en cas d'échec, et les fournisseurs de FaaS limitent généralement le temps d'exécution des fonctions serverless à quelques minutes. L'absence de garanties de qualité de service dans les offres commerciales serverless les empêche d'être plus largement utilisées~\cite{buyyaSLAorientedResourceProvisioning2011}.

\textbf{Énoncé du problème -- mettre tout ensemble}. Le problème que nous tentons de résoudre dans cet article est de déterminer comment dimensionner automatiquement et de manière réactive des ressources matérielles hétérogènes dans le cloud en fonction de la charge sur l'application et des exigences de qualité de service des utilisateurs, tout en maintenant le coût des ressources et de l'énergie au niveau le plus bas possible pour le fournisseur. Nous considérons une application de détection de deepfake comme cas d'étude pour² notre travail.

\textbf{État de l'art}. Des études antérieures ont exploré le besoin d'une plateforme de mise à l'échelle automatique qui prend en charge les tâches de courte durée comprises dans des applications telles que l'apprentissage automatique en tant que service. Le tableau~\ref{table:herofake-sota} résume les différences entre ces solutions et la plateforme cible que nous essayons d'atteindre, et la section~\ref{section:herofake-sota} fournit des détails supplémentaires. Bien que de nombreuses études aient établi la nécessité d'une accélération à la demande comme solution pour garantir le temps de réponse des fonctions, aucune n'a mesuré l'impact de l'exploitation des ressources hétérogènes sur la consommation d'énergie dynamique. En outre, les études précédentes considèrent la consolidation des tâches comme un moyen de libérer des ressources pour d'autres calculs - nous soutenons que ces techniques ouvrent des possibilités pour le fournisseur de services d'appliquer des politiques d'économie d'énergie dans le cloud privé. Enfin, comme les plateformes serverless sont à usage général et conçues pour être hautement configurables, notre solution cible devrait être consciente des coûts pour permettre au fournisseur de faire des choix de configuration se rapportant à leurs propres objectifs.

\textbf{Notre contribution}. Nous soutenons que l'utilisation opportuniste des accélérateurs matériels (GPU et FPGA) pour planifier les tâches de détection de deepfake peut permettre aux fournisseurs de cloud de garantir le temps de réponse des tâches serverless et d'atteindre le SLA tout en réduisant l'utilisation des ressources et la consommation d'énergie.

Dans cet article, nous proposons un cadre complet pour déployer une application de détection de deepfake sur un cloud serverless. Ce cadre comprend une phase hors-ligne et une phase en ligne. Le \textbf{phase hors-ligne} est utilisé pour caractériser la performance et le comportement énergétique des plates-formes matérielles hétérogènes déployées. La \textbf{phase en ligne} consiste en une plateforme de mise à l'échelle automatique et une stratégie d'ordonnancement qui utilisent efficacement les ressources matérielles hétérogènes (caractérisées) pour atteindre les accords de niveau de service par demande tout en réduisant la consommation d'énergie de la plateforme. 

Pour cette étude de cas, nous avons conçu un environnement de simulation qui modélise l'infrastructure d'une application de détection de deepfake, exécutée par le fournisseur en tant que Software as a Service à l'aide d'une infrastructure serverless.

\textbf{Certains chiffres de performance}. Avec notre politique d'allocation et d'ordonnancement, nous avons été en mesure de traiter 50000 tâches dans le même makespan que Knative avec moins de 36\% de pénalités de QoS. Notre cadre réduit la consommation d'énergie pour l'exécution des tâches de près de 35\% et permet au fournisseur de réduire davantage la consommation d'énergie statique en consolidant les tâches sur moins de 29\% des nœuds disponibles.

Le document est organisé comme suit : dans une première section, nous décrivons le modèle de plateforme global pour le projet. Ensuite, nous décrivons la plateforme d'exécution et la phase de caractérisation de la charge de travail. Dans la section III, nous décrivons les défis de l'orchestration de ressources serverless, notre modèle de tâches et les politiques d'allocation et d'ordonnancement de l'orchestrateur. La section IV présente notre méthodologie d'évaluation et une discussion des résultats expérimentaux. La section V donne des détails concernant l'état de l'art sur les plateformes d'autoscaling. Enfin, nous concluons par quelques perspectives pour les travaux futurs.

\section{Déployer des tâches de détection de deepfake dans un cloud serverless}
\label{section:herofake-deepfake}

\begin{figure*}[t]
\centering
\includegraphics[width=0.8\textwidth]{4_Chapitre4/figures/placement.png}
\caption{Plateforme de détection de deepfake serverless, vue d'ensemble du système.}
\label{figure:herofake-placement}
\end{figure*}

Cette section présente le modèle de plateforme serverless utilisé et le projet global.

\subsection{Modèle de la plateforme}

Nous considérons un système de détection de deepfake qui est déployé comme une application serverless composée de trois fonctions sans état qui réalisent des tâches d'inférence sur des images d'entrée. Ces images sont toutes RVB et de taille $224 * 224$ pixels~\footnote{Notez que les vidéos ne sont pas encore prises en compte dans notre projet.}.

La figure~\ref{figure:herofake-placement} présente la plateforme utilisée, nous distinguons une phase \textit{hors-ligne} (boîte bleue dans la figure) et une phase \textit{en ligne} (boîte verte dans la figure). Pendant la phase hors-ligne, nous collectons les métadonnées relatives à l'exécution des tâches sur les accélérateurs hétérogènes ; pendant la phase en ligne, nous allouons les ressources et programmons les tâches.

Les demandes d'invocation de fonctions émanant des utilisateurs sont reçues par le fournisseur et traitées par l'orchestrateur. Dans notre modèle, une invocation de fonction correspond à un \textit{tâche}. L'utilisateur sélectionne l'un des trois modèles fournis (ResNet50, VGG16 et VGG19, voir Section~\ref{section:herofake-offline:workload}) et l'utilise pour détecter un éventuel deepfake sur une image.

L'infrastructure du fournisseur de cloud est modélisée comme un ensemble de \textit{nœuds} hétérogènes (Section~\ref{model:nodes}) comprenant diverses combinaisons de \textit{plateformes} (Section~\ref{model:platforms}) qui peuvent exécuter des \textit{tâches} entrantes (Section~\ref{model:tasks}). 

\subsubsection{Nœuds}
\label{model:nodes}
Un nœud est un serveur disponible dans l'infrastructure du fournisseur de services. Dans ce travail, nous ne tenons pas compte de la localité du stockage et des données. Les données d'entrée sont toujours fournies \textit{via} le téléchargement de fichiers par l'utilisateur au moment de sa demande.
Ainsi, la seule caractéristique qui définit un nœud dans notre modèle d'infrastructure est la taille de la mémoire dédiée. Un nœud est constitué d'un ensemble de plateformes d'exécution définies ci-après.

\subsubsection{Plateformes d'exécution}
\label{model:platforms}

Une plateforme d'exécution est une unité de traitement matérielle disponible sur un nœud. Chaque plateforme consomme une quantité d'énergie à l'état "inactif" exprimée en kilowattheures (kWh). Lorsqu'elle commence à exécuter une tâche, elle consomme une énergie supplémentaire caractérisée par les propriétés/le type de la tâche : elle est alors dans un état "actif". Nous distinguons le temps "inactif" et le temps "actif" pour chaque plateforme, afin de mesurer l'utilisation des ressources.
Les plateformes sont caractérisées par un \textit{type de plateforme} qui englobe les paramètres suivants :

\begin{itemize}
    \item \textit{Type de matériel} -- CPU, GPU ou FPGA ;
    \item \textit{Prix} -- le coût d'acquisition d'une telle plateforme par le fournisseur de services cloud ;
    \item \textit{Énergie au repos} -- la consommation d'énergie de base de la plateforme lorsqu'elle n'exécute aucune tâche.
\end{itemize}

\textbf{Mise en cache des tâches et modèle de démarrage à froid}. Nous considérons un mécanisme simple de mise en cache des tâches au niveau de la plateforme, qui s'apparente à un mécanisme de maintien en vie~\cite{7279063}. Dans notre système, si une plateforme a déjà exécuté une tâche de type $t$ et qu'une nouvelle tâche du même type $t$ est programmée sur cette même plateforme, le délai de démarrage à froid n'est pas appliqué. Toutefois, si cette même plateforme devait exécuter une tâche de type différent $tt$, la tâche subira un démarrage à froid avant d'entrer dans sa phase d'exécution. Enfin, si la plateforme n'a pas été allouée précédemment, la tâche subira également un délai de démarrage à froid.

\subsection{Description générale du système}

L'institut de recherche b{\textless\textgreater}com travaille sur un projet qui vise à déployer une application de détection de deepfake sur un cloud privé. Les utilisateurs soumettent une image au système et lorsque leur demande est satisfaite, ils obtiennent une valeur booléenne en guise de réponse. L'application vise différentes catégories d'utilisateurs : certains d'entre eux peuvent être des médias ou des autorités ayant des exigences élevées en matière de qualité de service, tandis que d'autres peuvent être des utilisateurs occasionnels tolérant une latence plus élevée.

Pour différencier ces catégories d'utilisateurs, nous proposons différents niveaux d'accords de niveau de service par demande. Les utilisateurs ayant des exigences plus élevées accepteront de payer un prix plus élevé par demande, mais si nous ne parvenons pas à satisfaire leur demande dans le temps de réponse imparti, nous consentirons à une remise - plus le niveau de qualité de service est élevé, plus la remise est importante. Le fournisseur est donc fortement incité, sur le plan pécuniaire, à assurer la qualité de service.

\textbf{Phase hors-ligne}. Dans notre plateforme, le cycle de vie de l'application commence par une phase hors-ligne au cours de laquelle le développeur fournit le code de ses fonctions pour différentes architectures matérielles \Circled{1}. Ce code est stocké dans un référentiel de fonctions. Les fonctions sont ensuite déployées sur un nœud de mesure \Circled{2} où elles sont exécutées afin de générer des métadonnées relatives aux fonctions : les besoins en mémoire, le temps d'exécution, le temps de démarrage à froid et la consommation d'énergie pour chaque fonction sont écrits dans un magasin de métadonnées \Circled{3}. La phase hors-ligne doit être exécutée une fois pour une fonction donnée sur une plateforme donnée, elle est décrite dans la section~\ref{section:herofake-offline}.

\textbf{Phase en ligne}. Lorsqu'un utilisateur envoie une demande à l'application \Circled{4}, il fournit une image d'entrée et spécifie le niveau de qualité de service souhaité. La demande est ajoutée à une file d'attente \Circled{5} au niveau de l'orchestrateur. Lorsque le planificateur extrait la demande de la file d'attente, le magasin de métadonnées est interrogé pour récupérer les métadonnées de fonction appropriées \Circled{6}.

L'ordonnanceur tente ensuite de planifier une tâche (c'est-à-dire l'invocation d'une fonction) pour répondre à la demande. Les tâches sont placées sur des \textit{répliques} de fonctions \Circled{7} déjà déployées. Ces répliques peuvent être des conteneurs ou des machines virtuelles, c'est-à-dire des environnements d'exécution dédiés pour la fonction donnée.
Simultanément, l'autoscaler surveille les files d'attente de requêtes dans toutes les répliques de fonctions \Circled{8}. Le rôle de l'autoscaler est de dimensionner l'allocation des ressources en fonction des fluctuations de charge pour chaque fonction.
L'ordonnanceur et l'autoscaler sont décrits dans la section~\ref{section:herofake-online}.

\section{Phase hors-ligne : mesures et extraction des métadonnées}
\label{section:herofake-offline}

\subsection{Caractérisation des plateformes d'exécution}

\begin{table}[t]
\caption{Execution platform characterization}
\begin{center}
\resizebox{\columnwidth}{!}{%
\begin{tabular}{|c|c|c|c|c|}
\hline
                             \textbf{Platform} & \textbf{Hardware type}& \textbf{Price (MSRP)} & \textbf{Idle energy} \\ \hline
Intel Xeon ES-1620 v4         & CPU           & 294          & 0.067       \\ \hline
Nvidia GeForce RTX 2070 Super & GPU           & 499          & 0.010       \\ \hline
Xilinx Alveo U250             & FPGA          & 7695         & 0.030       \\ \hline
\end{tabular}%
}
\end{center}
\label{table:herofake-platforms}
\end{table}

L'utilisation de l'inférence par apprentissage profond et les impacts énergétiques augmentant simultanément dans l'informatique, l'efficacité énergétique des dispositifs cibles devient une préoccupation majeure. Les cartes d'accélération basées sur les FPGA sont décrites comme un concurrent pertinent face à l'approche dominante des GPU. Notre étude propose un benchmark, utilisant des approches basées sur les réseaux neuronaux convolutifs (CNN) pour la détection de deepfake sur les technologies CPU, GPU et FPGA en ce qui concerne l'efficacité énergétique pendant le temps d'inférence. Notre comparaison porte sur la consommation d'énergie, la vitesse d'inférence et la précision en utilisant le traitement CPU et GPU traditionnel par rapport au FPGA. Ces mesures sont cruciales pour une orchestration efficace sur des plateformes hétérogènes.

Le CPU utilisé était un Intel Xeon CPU ES-1620 v4 (3,5 GHz) tandis que le GPU était un Nvidia GeForce RTX 2070 Super qui peut être utilisé avec les nouvelles versions des frameworks d'IA. Par conséquent, les deux étaient compatibles avec TensorFlow, c'est-à-dire la plateforme utilisée pour l'inférence. En ce qui concerne le FPGA, nous avons utilisé l'Alveo U250, une carte de cloud computing de Xilinx, qui est compatible avec Vitis-AI~\cite{vitis-ai}. Les processus de silicium utilisés pour les deux dispositifs sont similaires (12 nm pour le GPU et 16 nm pour le FPGA), mais le GPU peut obtenir un léger avantage dans ce benchmark grâce à sa technologie de silicium plus avancée.

Pour effectuer l'inférence sur le FPGA, nous avons utilisé Vitis-AI. Au moment de cette étude, la dernière version disponible (v. 2.0) a été utilisée. Vitis-AI propose deux méthodes pour l'optimisation des modèles. La première est l'élagage, qui consiste à réduire la complexité du modèle par une compression tout en supprimant certaines sections non critiques de l'arbre. La seconde est la quantification, où l'on convertit les poids flottants de 32 bits en entiers de 8 bits. Cette dernière méthode, qui est librement disponible, est celle que nous avons utilisée pour optimiser notre modèle avant la compilation, qui convertit notre modèle en instructions DPU (Deep Learning Processing Unit).

\subsection{Caractérisation des tâches logicielles}
\label{section:herofake-offline:workload}

\begin{table}[t]
\caption{Workload characterization }
\centering
\resizebox{\columnwidth}{!}{%
\begin{tabular}{|c|cc|ccc|ccc|ccc|}
\hline
Task     & \multicolumn{2}{c|}{Memory (GB)} & \multicolumn{3}{c|}{Cold start (s)}                              & \multicolumn{3}{c|}{Execution time (s)}                         & \multicolumn{3}{c|}{Energy (mWh)}                            \\ \hline
         & \multicolumn{1}{c|}{CPU}  & GPU  & \multicolumn{1}{c|}{CPU}   & \multicolumn{1}{c|}{GPU}   & FPGA   & \multicolumn{1}{c|}{CPU}   & \multicolumn{1}{c|}{GPU}   & FPGA  & \multicolumn{1}{c|}{CPU}  & \multicolumn{1}{c|}{GPU}  & FPGA \\ \hline
ResNet50 & \multicolumn{1}{c|}{1.3}  & 3.3  & \multicolumn{1}{c|}{1.232} & \multicolumn{1}{c|}{2.340} & 9.952  & \multicolumn{1}{c|}{0.124} & \multicolumn{1}{c|}{0.024} & 0.009 & \multicolumn{1}{c|}{3.11} & \multicolumn{1}{c|}{1.7}  & 0.5  \\ \hline
VGG16    & \multicolumn{1}{c|}{1.8}  & 3.3  & \multicolumn{1}{c|}{2.514} & \multicolumn{1}{c|}{4.641} & 14.528 & \multicolumn{1}{c|}{0.143} & \multicolumn{1}{c|}{0.046} & 0.010 & \multicolumn{1}{c|}{4.34} & \multicolumn{1}{c|}{3.43} & 0.55 \\ \hline
VGG19    & \multicolumn{1}{c|}{1.9}  & 3.4  & \multicolumn{1}{c|}{2.559} & \multicolumn{1}{c|}{4.641} & 14.758 & \multicolumn{1}{c|}{0.167} & \multicolumn{1}{c|}{0.048} & 0.012 & \multicolumn{1}{c|}{5.16} & \multicolumn{1}{c|}{3.58} & 0.65 \\ \hline
\end{tabular}
}%
\label{table:herofake-tasks}
\end{table}

Pour les besoins de cette étude, trois modèles populaires ont été formés. Le premier est basé sur les réseaux résiduels (ResNet50), qui utilise des blocs résiduels et peut être entraîné efficacement~\cite{NEURIPS2019_7716d0fc}. Le deuxième est VGG16 (VGG pour Visual Geometry Group), qui utilise uniquement des convolutions comme blocs~\cite{DBLP:journals/corr/SimonyanZ14a} et le troisième est VGG19, une variante de VGG16 avec trois couches supplémentaires~\cite{biom10070984}. Ces réseaux sont entraînés sur un GPU, l'entraînement n'étant pas le sujet de cette étude.

\subsection{Mesures de performances}

\begin{figure}[t]
\centering
\includegraphics[width=\columnwidth]{4_Chapitre4/figures/characterization/time_of_inference_1_image.png}
\caption{Inference time for one image with ResNet50, VGG16 and VGG19.}
\label{figure:herofake-time-inference}
\end{figure}

La carte d'accélération FPGA étant censée être plus efficace qu'un CPU ou un GPU~\cite{5272532}, la comparaison du temps d'inférence avec ces trois technologies est une première condition pour permettre la comparaison du coût énergétique par image. L'évaluation des performances en termes de temps d'exécution a été réalisée avec les mêmes 10 000 images pour les trois modèles différents. Nous avons construit un ensemble de données deepfake à deux classes, les vraies provenant de l'ensemble de données CelebA~\cite{https://doi.org/10.48550/arxiv.1411.7766}, et les fausses générées à l'aide d'un Generative Adversarial Network (GAN)~\cite{jimaging7080128}. La quantification et la compilation du graphe ont été effectuées avec Vitis-AI afin de l'exécuter sur le FPGA. En ne considérant que le temps d'inférence, il s'est avéré que sur les trois modèles testés (ResNet50, VGG16 et VGG19), le FPGA est de 13,08 à 13,79 fois plus rapide que le CPU mais aussi de 2,52 à 4,48 fois plus rapide que le GPU (voir Figure~\ref{figure:herofake-time-inference}).

\subsection{Mesures de consommation d'énergie}

\begin{figure}[t]
\centering
\includegraphics[width=\columnwidth]{4_Chapitre4/figures/characterization/consumption_per_image.png}
\caption{Energy consumption of inference per image (mWh).}
\label{figure:herofake-consumption-per-image}
\end{figure}

La consommation d'énergie instantanée mesurée pendant l'inférence est la consommation globale de la machine (y compris l'unité centrale, la mémoire, la carte mère et l'alimentation) pendant l'exécution de l'inférence. 
Les mesures ont été effectuées à l'aide d'une unité de distribution d'énergie (PDU) (Raritan PX3-5190R) capable de surveiller la puissance instantanée et la consommation d'énergie du serveur (Dell Precision T5810). Les résultats montrent que l'inférence sur le CPU produit la consommation d'énergie instantanée la plus faible. Ce résultat est assez attendu car l'inférence sur GPU ou FPGA inclut également la consommation d'énergie du CPU.


Cependant, la seule consommation d'énergie instantanée ne reflète pas correctement le coût total de chaque plateforme. Le temps d'exécution nécessaire pour traiter toutes les images doit être pris en compte. La mesure pertinente est le coût énergétique par image. La consommation d'énergie a été mesurée en kilowattheures (kWh) pour les 10 000 images, puis convertie en milliwattheures (mWh) par image. De ce point de vue, il est clair que le FPGA est le plus économe en énergie en ce qui concerne le temps d'exécution, consommant de 6,2 à 6,9 fois moins que le CPU et de 3,3 à 6,2 fois moins que le GPU (voir Figure~{figure:herofake-consumption-per-image}).

\subsection{Discussion}

\begin{figure}[t]
\centering
\includegraphics[scale=0.2]{4_Chapitre4/figures/characterization/cost_devices_time.png}
\caption{Total cost of inference on selected devices over time.}
\label{figure:herofake-cost-over-time}
\end{figure}

Les résultats de ce benchmark montrent un net avantage de l'inférence sur FPGA en termes de performance et d'efficacité énergétique. Les gains de performance sont significatifs, en particulier avec les réseaux d'apprentissage profond plus complexes~\cite{8782524}. Les ressources informatiques basées sur des serveurs équipés de cartes d'accélération FPGA, au lieu de cartes d'accélération GPU, bénéficieraient de ces avantages.

La consommation d'énergie brute du dispositif d'inférence ne reflète pas le coût total de la solution. En effet, il faut également inclure le coût de l'équipement lui-même. C'est un point important dans la comparaison entre GPU et FPGA, car il existe un écart de prix entre les deux technologies : le GPU (RTX 2070 Super) utilisé pour ce benchmark a été introduit aux alentours de 600€, alors que le FPGA (Alveo U250) est vendu aux alentours de 6000€. Le coût de l'énergie électrique pour effectuer l'inférence est très faible (nous avons utilisé la moyenne européenne de 0,1833 € par kWh proposée dans~\cite{energy-price}), comparé au coût initial du dispositif : la durée d'exécution nécessaire pour bénéficier de l'avantage de coût du FPGA est de l'ordre de plusieurs mois de fonctionnement continu. La figure~\ref{figure:herofake-cost-over-time} représente le coût cumulé (en euros) de l'utilisation d'un serveur avec accélération GPU ou FPGA en fonction du temps (en années). Notre estimation du coût comprend le nombre de GPU nécessaires et leur coût pour égaliser les performances des FPGA et utilise un facteur 2x~\cite{shehabiUnitedStatesData2016}, pour tenir compte de la consommation d'énergie totale de l'infrastructure (principalement le refroidissement et la mise en réseau). Le FPGA peut devenir une solution rentable après quelques mois pour les CNN complexes. Pour les réseaux moins complexes, l'avantage financier du FPGA est atteint après plus de deux ans.

\begin{figure}[t]
\centering
\includegraphics[width=\columnwidth]{4_Chapitre4/figures/characterization/power_usage.png}
\caption{Power usage breakdown for FPGA and GPU.}
\label{figure:herofake-power-usage}
\end{figure}

L'analyse précédente est valable dans le cas où l'inférence est toujours effectuée à pleine charge. En effet, lorsque l'on décompose la consommation d'énergie du GPU entre la consommation au repos et la consommation pour l'inférence, il est clair que le GPU est capable d'adapter dynamiquement sa consommation d'énergie à l'intensité du traitement. Le FPGA, quant à lui, semble avoir une gestion de l'énergie très limitée. Une fois que la conception du DPU est chargée dans le dispositif, sa consommation d'énergie au repos reste très élevée (voir la figure. \ref{figure:herofake-power-usage}). Si l'on ajoute les 38 W de la carte FPGA, il y a en effet une consommation résiduelle de 60 W lorsque le DPU est inactif. Même si l'évolution de l'implémentation du DPU sur le FPGA peut résoudre ce problème (par exemple en réduisant l'activité de l'arbre d'horloge lorsqu'il est inactif), cela a un impact sur le coût total et doit être pris en compte si le dispositif n'est pas toujours utilisé à pleine charge. Avec seulement 12 W d'énergie au repos, le GPU est un meilleur candidat lorsque l'utilisation à pleine charge du dispositif ne peut être garantie.

Comme la tendance vers des CNN plus complexes se poursuit~\cite{8807741}, l'utilisation des dispositifs les plus efficaces deviendra un défi majeur. La solution FPGA offre une nouvelle option pour effectuer l'inférence. Cependant, les FPGA ne remplacent pas encore les GPU : le flux de compilation reste complexe et prend du temps. Il faudra trouver un compromis entre la flexibilité des GPU et l'efficacité des FPGA. La section suivante traite d'un premier orchestrateur qui prend en compte la caractérisation mentionnée ci-dessus pour l'allocation et l'ordonnancement de ressources hétérogènes.

\section{Phase en ligne : allocation des ressources et placement des tâches}
\label{section:herofake-online}

Dans cette section, nous formulons le problème que notre contribution aborde, et nous donnons une description détaillée de notre modèle. Enfin, nous présentons une description formelle de notre stratégie pour la mise à l'échelle automatique des ressources et l'ordonnancement des tâches. 

\subsection{Défis pour l'orchestration dynamique}

L'ordonnancement des charges de travail dans le paradigme serverless est un problème à deux volets : les fournisseurs doivent gérer dynamiquement l'allocation des ressources (c'est-à-dire gérer les pools de ressources lors de la mise à l'échelle du nombre de répliques pour une application) et le placement des tâches (c'est-à-dire le mappage des tâches aux répliques existantes).

L'augmentation du nombre de répliques pose un problème de performance : lorsqu'une nouvelle réplique est créée, que ce soit sous la forme d'un conteneur ou d'une machine virtuelle, le bac à sable d'exécution doit passer par sa phase d'initialisation. C'est ce qu'on appelle un "démarrage à froid".

Les solutions commerciales telles que AWS Lambda évitent souvent le problème du démarrage à froid en maintenant des pools de bacs à sable préchauffés~\cite{vahidiniaColdStartServerless2020}. Ces machines virtuelles (VM) ou conteneurs sont démarrés par anticipation et mis en pause dans un état post-initialisation. Lorsque l'activité reprend, les demandes entrantes peuvent être servies sans souffrir d'un délai de démarrage à froid, au détriment du multiplexage des ressources du côté du fournisseur. Bien que cette solution permette de réduire, voire d'éliminer les délais de démarrage à froid, elle pèse sur la capacité de multiplexage des ressources du fournisseur et augmente le coût total de possession (TCO).

En outre, les applications de Machine Learning as a Service (MLaaS) présentent une charge très fluctuante~\cite{gujaratiSwayamDistributedAutoscaling2017}, ce qui renforce l'argument selon lequel une stratégie d'allocation des ressources réactive est nécessaire pour redimensionner l'infrastructure. Cependant, comme le temps d'exécution des tâches d'inférence est de l'ordre du centième ou du dixième de seconde, tandis que le temps d'initialisation des bacs à sable peut aller du centième de seconde à la seconde~\cite{mancoMyVMLighter2017}, nous avons besoin d'un mécanisme pour éviter d'encourir d'énormes coûts de latence pour l'exécution des fonctions.

Les tâches critiques nécessitent des garanties de niveau de service de la part du fournisseur. Les accords de niveau de service dans le cloud consistent généralement à convenir d'un taux de disponibilité des ressources dans le temps ; si le fournisseur ne respecte pas cet accord, une remise est proposée au client. Bien que cela puisse fonctionner pour des ressources réservées, nous pouvons voir que cela n'a pas de sens dans le paradigme serverless. La possibilité de garantir le temps de réponse des tâches permettrait à un fournisseur serverless d'atteindre des accords de niveau de service par invocation~\cite{zhangMArkExploitingCloud}.

L'utilisation d'accélérateurs matériels est une possibilité d'améliorer les rapports performance-coût. Bien qu'il s'agisse d'un investissement coûteux (voir Figure~\ref{figure:herofake-cost-over-time}), ces dispositifs permettent d'accélérer considérablement les tâches parallèles (voir Figure~\ref{figure:herofake-time-inference}), améliorant ainsi le temps de réponse des fonctions, avec un coût énergétique réduit (voir Figure~\ref{figure:herofake-consumption-per-image}).

\subsection{Modèle de tâche} \label{model:tasks}

\begin{table}[t]
    \caption{Notation dictionary}
    \begin{center}
    \begin{tabular}{|c|L|}
    \hline
    \textbf{Notation} & \textbf{Description} \\ \hline
    $f_{N, P}$ & A function $f$ scheduled to run on a platform $P$ available on node $N$ \\ \hline
    $QP$ & QoS penalty \\ \hline
    $QD$ & QoS deviation \\ \hline
    $WET$ & Worst execution time \\ \hline
    $TT$ & Task total time \\ \hline
    $WT$ & Wait time \\ \hline
    $CST$ & Cold start time \\ \hline
    $ET$ & Execution time \\ \hline
    $EC$ & Energy consumption \\ \hline
    $HP$ & Hardware price \\ \hline
    $TC$ & Task consolidation \\ \hline
    $Q$ & Task queue on a replica \\ \hline
    $replicaCount_{f}$ & Size of the replica pool in the system for a function $f$ \\ \hline
    $concurrency_{f}$ & Average number of in-flight requests for a function $f$ \\ \hline
    $threshold$ & Concurrency threshold for function replicas in vanilla Knative \\ \hline
    $replicaCount_{f, h}$ & Size of the replica pool for a function $f$ on hardware type $h$ \\ \hline
    $concurrency_{f, h}$ & Average number of in-flight requests for a function $f$ on replicas of hardware type $h$ \\ \hline
    $x_{f, h}$ & Concurrency threshold for a function $f$ on a replica of hardware type $h$ \\ \hline
    $scaleCost_{{f}_{N, P}}$ & Cost of creating a new replica for function $f$ on a platform $P$ available on node $N$ \\ \hline
    $schedCost_{{f}_{N, P}}$ & Cost of scheduling an execution of function $f$ on a platform $P$ available on node $N$ \\ \hline
    \end{tabular}
    \label{table:herofake-notation}
    \end{center}
\end{table}

Les applications sont composées de fonctions. L'exécution d'une fonction est appelée \textit{tâche}. Dans ce travail, il n'y a pas de dépendances entre ces tâches : l'application est composée de fonctions pures et sans état. Les événements qui déclenchent l'exécution d'une tâche arrivent dans le système à un intervalle aléatoire et borné. Nous formulons l'hypothèse qu'une requête aboutit toujours et conduit à l'exécution d'une \textit{task} (une instance de \textit{fonction}). Lorsqu'une tâche a commencé son exécution sur la plateforme qui lui a été attribuée, elle s'exécute pendant la totalité de son temps d'exécution. Nous ne prenons pas en compte la préemption ou les défaillances dans cette contribution : une tâche termine toujours son exécution avec succès, même si son temps de réponse peut dépasser ses exigences en matière de qualité de service. Nous ne tenons pas compte des interférences possibles entre les charges de travail sur le même nœud~\cite{dartoisInvestigatingMachineLearning2021}. 

Nous considérons les tâches qui peuvent être exécutées sans distinction sur des plateformes d'exécution hétérogènes. Dans le contexte de notre étude de cas spécifique, la mise en œuvre des différentes fonctions a été effectuée à la main pour chaque plateforme ; cependant, des travaux existent pour permettre une compilation croisée automatique pour les architectures hétérogènes~\cite{hortaXartrekRuntimeExecution2021, 10.1145/3445814.3446699}. Les métadonnées suivantes ont été mesurées pour chaque fonction, sur chaque plateforme d'exécution : 

\begin{itemize}
    \item \textit{Besoins mémoire en pic} -- la quantité de mémoire (en GB) allouée à la tâche ;
    \item \textit{Durée de démarrage à froid} -- la durée de l'initialisation du bac à sable lors de l'exécution de la tâche sur une plateforme qui n'a pas la fonction en cache ;
    \item \textit{Temps d'exécution} -- la durée prévue de l'exécution effective de la tâche, à l'exclusion de sa phase d'initialisation ;
    \item \textit{Consommation d'énergie} -- la différence entre l'énergie inactive et l'énergie active encourue par la plateforme d'exécution lorsqu'elle exécute la tâche.
\end{itemize}

L'équation~\ref{eq:herofake-HRO-total-time} décompose le temps de réponse attendu pour l'exécution d'une fonction $f$ sur une plateforme $P$ sur un nœud $N$.

\begin{equation}
    {TT}_{{f}_{N, P}} = {WT}_{{f}_{N, P}} + {CST}_{{f}_{N, P}} + {ET}_{{f}_{N, P}}
\label{eq:herofake-HRO-total-time}
\end{equation}

Où :

\begin{itemize}
    \item ${WT}_{{f}_{N, P}}$ correspond à la durée de la décision d'ordonnancement, y compris le temps passé par la requête en file d'attente ;
    \item ${CST}_{{f}_{N, P}}$ est la durée de l'initialisation de l'invocation de la fonction, y compris son temps potentiel de démarrage à froid ;
    \item ${ET}_{{f}_{N, P}}$ est le temps d'exécution de la fonction sur la plateforme.
\end{itemize}

Nous proposons différents niveaux de qualité de service en fonction des besoins des utilisateurs en termes de garanties sur le temps d'exécution. Chaque niveau de qualité de service présente un \textit{écart de durée} différent (noté $QD$ dans l'équation~\ref{eq:herofake-task-penalty}) -- un facteur par lequel le pire temps d'exécution d'une fonction est multiplié pour donner une limite supérieure au temps d'exécution de cette fonction pour ce niveau de qualité de service.

Le temps d'exécution prédit d'une fonction est toujours basé sur le pire temps d'exécution (noté $WET_{f}$), c'est-à-dire le temps d'exécution d'une tâche lorsqu'elle est programmée sur la plateforme d'exécution présentant le niveau de performance le plus faible pour cette fonction :


\begin{equation}
    \forall \, (N, P), \, WET_{f} = \max ET_{N, P}
\label{eq:herofake-task-wet}
\end{equation}

Une fois qu'une tâche est programmée sur une plateforme d'exécution, elle passe par sa durée totale d'exécution décrite dans l'équation \ref{eq:herofake-HRO-total-time}. L'échéance de la tâche est calculée en multipliant le pire temps de réponse de la fonction (tel qu'exprimé dans l'équation~\ref{eq:herofake-task-wet}) par l'écart de durée de la qualité de service associé au niveau de qualité de service de la demande de l'utilisateur. L'équation~\ref{eq:herofake-task-penalty} montre que nous fixons une valeur booléenne $QP_{f_{N, P}}$ pour chaque invocation de fonction si la tâche ne respecte pas son délai. 


\begin{equation}
    QP_{f_{N, P}} = TT_{f_{N, P}} \cdot QD_{f_{N, P}} > WET_{f}
\label{eq:herofake-task-penalty}
\end{equation}

\subsection{Stratégie d'allocation de ressources} \label{section:herofake-autoscaling-strategy}

Dans une plateforme serverless, l'autoscaler a la responsabilité d'allouer des ressources matérielles pour les exécutions de fonctions. Pour toute fonction, un autoscaler peut allouer $n$ \textit{répliques}. Le nombre de répliques pour une fonction donnée à un moment donné détermine le niveau de concurrence.

Dans Knative, le nombre de répliques pour une fonction donnée (équation~\ref{eq:herofake-kn-replica-count}) dépend de la charge moyenne mobile pour une fonction, c'est-à-dire le nombre moyen de requêtes en vol pour la fonction sur une fenêtre de 60 secondes (concurrence dans le système par fonction). Il est limité par un seuil de concurrence par réplique, c'est-à-dire le nombre maximum de demandes en file d'attente dans la réplique d'une fonction à tout moment. La valeur par défaut dans Knative est de 100 requêtes en vol dans chaque réplique~\cite{knative-autoscaling}.

\begin{equation}
    replicaCount_{f} = \frac{concurrency_{f}}{threshold}
\label{eq:herofake-kn-replica-count}
\end{equation}

Ce mécanisme de dimensionnement permet d'allouer des CPU sous Knative, en réaction aux changements de l'état actuel de la concurrence dans le système. La principale contribution de l'autoscaler que nous proposons est d'améliorer Knative afin de prendre en compte l'hétérogénéité des plateformes d'exécution.

Le mécanisme Knative simple ne fonctionne pas lorsque l'infrastructure est constituée d'une variété de plateformes d'exécution. En effet, ces plateformes présentent différents niveaux de performance, de consommation d'énergie et de coût. Cela a une conséquence sur le nombre de répliques que le fournisseur doit déployer sur ces plateformes : pour un niveau donné de charge d'application, les répliques hétérogènes seront capables de traiter différents nombres de tâches dans le même délai. Pour que notre plateforme puisse gérer l'hétérogénéité de l'infrastructure sous-jacente, nous proposons un nombre de répliques par fonction et \textbf{par type de matériel} comme dans l'équation~\ref{eq:herofake-HRO-replica-count}.

\begin{equation}
    replicaCount_{f, h} = \frac{concurrency_{f, h}}{x_{f, h}}
\label{eq:herofake-HRO-replica-count}
\end{equation}

Une décision d'autoscaling peut introduire des coûts d'opportunité dans le système : les accélérateurs matériels sont peu disponibles par rapport aux CPU, et le fait de les allouer à une fonction donnée à un moment donné les rendra indisponibles pour d'autres calculs. Pour que l'autoscaler puisse décider quand il est pertinent d'allouer de tels accélérateurs, il doit être conscient des coûts. 

Afin de déterminer le seuil de concurrence par réplique $x_{f, h}$ pour une fonction $f$ sur un type de matériel $h$ (par exemple, GPU et FPGA), nous avons fixé le seuil de concurrence par réplique sur les CPU à $x_{f, c} = 100$, comme c'est la valeur par défaut dans Knative~\cite{knative-concurrency}. Ensuite, nous avons utilisé les mesures de la phase hors-ligne (Table~\ref{table:herofake-tasks}) pour établir un ratio composite (incluant la performance, l'énergie, le prix de la plateforme) comme décrit dans l'équation~\ref{eq:herofake-HRO-concurrency-target}. Dans notre politique, nous avons choisi de favoriser le temps de réponse en fixant $k_{ET} = \frac{2}{3}$, $k_{EC} = \frac{1,5}{6}$ et $k_{HP} = \frac{0,5}{6}$. Par exemple, pour la fonction ResNet50 (décrite dans Table~\ref{model:tasks}), les files d'attente de tâches dans les réplicas sont dimensionnées à 100 pour les CPU, 489 pour les GPU et 1292 pour les FPGA.

\begin{equation}
    x_{f, h} = x_{f, c} \cdot (k_{ET} \cdot \frac{ET_{{f}_{c}}}{ET_{{f}_{h}}} + k_{EC} \cdot \frac{EC_{{f}_{c}}}{EC_{{f}_{h}}} + k_{HP} \cdot \frac{HP_{{f}_{c}}}{HP_{{f}_{h}}})
\label{eq:herofake-HRO-concurrency-target}
\end{equation}

Lorsque le seuil de simultanéité d'une fonction est dépassé dans les files d'attente des répliques sur un type de matériel donné, l'autoscaler procède à la \textit{scale out} de la fonction : une nouvelle réplique est mise en route pour traiter les demandes ultérieures des utilisateurs.

L'allocation commence par la liste complète des nœuds disponibles dans l'infrastructure. Nous commençons par constituer un sous-ensemble de nœuds disponibles, appelé \textit{nœuds appropriés}. Compte tenu des besoins en mémoire que nous avons mesurés pour chaque fonction, nous éliminons les nœuds qui ne disposent pas actuellement de suffisamment de mémoire pour exécuter une réplique de la fonction. Chaque réplique déployée sur la plateforme d'exécution d'un nœud consomme la quantité totale de mémoire requise par le type de fonction. Si le nœud n'a plus de mémoire, ses plateformes d'exécution ne peuvent plus être utilisées pour déployer d'autres répliques.

Afin de sélectionner le type de ressource à allouer à cette réplique, l'autoscaler minimise la fonction de coût donnée dans l'équation~\ref{eq:herofake-HRO-allocation-cost-function}. 
Dans notre politique, comme pour l'autoscaling, nous avons choisi de favoriser le temps total d'exécution des tâches en fixant $k_{TT} = \frac{2}{3}$, $k_{EC} = \frac{1.5}{6}$ et $k_{HP} = \frac{0.5}{6}$. 
En fonction du matériel disponible dans le pool au moment de la mise à l'échelle, l'autoscaler favorisera la création d'une nouvelle réplique de fonction sur la plateforme qui exécutera la tâche dans le temps total le plus court, y compris le démarrage à froid, avec la consommation d'énergie la plus faible et le prix le plus bas.

\begin{equation}
\begin{split}
    scaleCost_{{f}_{N, P}} = \, &k_{TT} \cdot {TT}_{{f}_{N, P}} \\
    + &k_{EC} \cdot {EC}_{{f}_{N, P}} \\
    + &k_{HP} \cdot {HP}_{{f}_{N, P}}
\end{split}
\label{eq:herofake-HRO-allocation-cost-function}
\end{equation}

Au contraire, lorsque la concurrence pour une fonction tombe en dessous du seuil sur un type de matériel donné, l'autoscaler emploiera une politique de meilleur effort et essaiera de désallouer tout réplica avec une file d'attente de tâches vide sur ce type de matériel. Si une réplique a une file d'attente de tâches vide, elle sera libérée dans le pool de plates-formes disponibles et la mémoire qui lui avait été allouée sur le nœud sera libérée.

Les différents poids ($k$) utilisés dans les équations~\ref{eq:herofake-HRO-concurrency-target} et~\ref{eq:herofake-HRO-allocation-cost-function} peuvent être modifiés par le fournisseur afin de personnaliser la politique d'allocation en fonction de différentes priorités.

\subsection{Stratégie d'ordonnancement} \label{section:herofake-scheduling-strategy}

La caractérisation de la charge de travail est essentielle à la prédiction des performances, car elle peut guider les décisions d'ordonnancement qui conduisent à la satisfaction des exigences de qualité de service~\cite{mampageHolisticViewResource2022}. Notre stratégie d'ordonnancement s'appuie sur les métadonnées des tâches décrites dans la section~\ref{model:tasks}. La construction de connaissances sur les tâches serverless est réalisée au cours d'une phase hors-ligne sur notre plateforme, car le code est poussé vers les registres du fournisseur avant l'exécution réelle~\cite{shahradServerlessWildCharacterizing}.

Dans Knative, le planificateur gère les tâches entrantes de manière FIFO. Pour gérer les différents niveaux d'exigences en matière de qualité de service, nous proposons que notre ordonnanceur retire les tâches de la file d'attente de l'orchestrateur par \textbf{échéance la plus proche}. Nous calculons l'échéance de la tâche en utilisant sa pire durée d'exécution sur la plateforme à l'aide de l'équation~\ref{eq:herofake-task-wet}, et en la multipliant par l'écart de durée autorisé fixé par le niveau de qualité de service. Après l'exécution de la tâche, nous vérifierons si nous avons dépassé son échéance et fixerons la pénalité associée en conséquence, comme décrit dans l'équation~\ref{eq:herofake-task-penalty}. 

Nous itérons sur les répliques de la fonction pour récupérer et prédire les métriques suivantes basées sur les métadonnées de la tâche :

\begin{itemize}
    \item \textbf{pénalité potentielle} : nous calculons la longueur de la file d'attente de la plateforme et vérifions si l'échéance de la tâche sera dépassée, comme décrit dans l'équation~\ref{eq:herofake-task-penalty} ;
    \item \textbf{consommation d'énergie} : nous récupérons les mesures hors-ligne pour établir la consommation d'énergie dynamique de cette tâche sur la plateforme ;
    \item \textbf{fusion des tâches} : nous calculons la longueur de la file d'attente des tâches de la plateforme $Q$ en additionnant les durées totales de toutes les tâches en file d'attente, comme décrit dans les équations~\ref{eq:herofake-HRO-scheduling-platform-queue} (longueur de la file d'attente) et~\ref{eq:herofake-HRO-total-time} (durée totale de la tâche). 
\end{itemize}

\begin{equation}
    len \, Q_{N, P} = \sum TT_{f_{N, P}}
\label{eq:herofake-HRO-scheduling-platform-queue}
\end{equation}

Ces valeurs sont normalisées pour s'adapter à une fonction de coût pondérée décrite dans l'équation~\ref{eq:herofake-HRO-scheduling-cost-function}. Nous avons utilisé $k_{QP} = \frac{2}{3}$, $k_{EC} = \frac{0.5}{6}$ et $k_{TC} = \frac{1.5}{6}$ (comme pour l'autoscaler). L'ordonnanceur minimise ensuite cette fonction de coût pour toutes les répliques $(N, P)$ (c'est-à-dire le nœud et la plateforme d'exécution).

\begin{equation}
\begin{split}
    schedCost_{{f}_{N, P}} = \, &k_{QP} \cdot QP_{{f}_{N, P}} \\
    + &k_{EC} \cdot {EC}_{{f}_{N, P}} \\
    + &k_{TC} \cdot TC_{{f}_{N, P}}
\end{split}
\label{eq:herofake-HRO-scheduling-cost-function}
\end{equation}

Si l'ordonnanceur ne trouve pas de réplique disponible pour exécuter la tâche, celle-ci sera repoussée dans la file d'attente de l'orchestrateur. Cela augmentera la concurrence dans le système pour la fonction, poussant l'autoscaler à allouer une autre réplique sur le matériel approprié.

\section{Évaluation}
\label{section:herofake-evaluation}

\begin{figure*}[t]
    \centering
    \subfloat[Task consolidation (based on the unused node count)\label{figure:herofake-evaluation-full-nodes}]{
        \includegraphics[width=0.28\linewidth]{4_Chapitre4/figures/evaluation/z-nodes-20221212-232143-169224.png}
    }\qquad
    \subfloat[QoS violations (based on tasks with missed deadline)\label{figure:herofake-evaluation-full-penalty}]{
        \includegraphics[width=0.28\linewidth]{4_Chapitre4/figures/evaluation/z-penalty-20221212-232143-169224.png}
    }\qquad
    \subfloat[Dynamic energy consumption (in kWh)\label{figure:herofake-evaluation-full-energy}]{
        \includegraphics[width=0.28\linewidth]{4_Chapitre4/figures/evaluation/z-energy-20221212-232143-169224.png}
    }
    \caption{Evaluation 1 -- Comparison against baselines}
    \label{figure:herofake-evaluation-hro-full}
\end{figure*}

\begin{figure*}[t]
    \centering
    \subfloat[Task consolidation (based on the unused node count)\label{figure:herofake-evaluation-mixed-nodes}]{
        \includegraphics[width=0.28\linewidth]{4_Chapitre4/figures/evaluation/x-nodes-20221212-185844-053283.png}
    }\qquad
    \subfloat[QoS violations (based on tasks with missed deadline)\label{figure:herofake-evaluation-mixed-penalty}]{
        \includegraphics[width=0.28\linewidth]{4_Chapitre4/figures/evaluation/x-penalty-20221212-185844-053283.png}
    }\qquad
    \subfloat[Dynamic energy consumption (in kWh)\label{figure:herofake-evaluation-mixed-energy}]{
        \includegraphics[width=0.28\linewidth]{4_Chapitre4/figures/evaluation/x-energy-20221212-185844-053283.png}
    }
    \caption{Evaluation 2 -- Impact of HeROfake components on the overall performance}
    \label{figure:herofake-evaluation-hro-mixed}
\end{figure*}

\subsection{Protocole expérimental}

Nous avons utilisé des mesures provenant de l'évaluation de trois modèles d'apprentissage automatique différents (voir Tableau~\ref{table:herofake-tasks}). Ces modèles ont été mis en œuvre sur trois plateformes d'exécution différentes (voir Tableau~\ref{table:herofake-platforms}) comme expliqué dans la Section~\ref{section:herofake-offline}.

Ces données ont servi d'entrée à un simulateur que nous avons construit en utilisant SimPy~\cite{simpy}. Le simulateur suit le modèle de système décrit dans les sections~\ref{model:nodes}, \ref{model:platforms}, \ref{model:tasks}.

Nous avons mesuré les délais de démarrage à froid pour les applications de notre étude de cas, voir le tableau~\ref{table:herofake-tasks}. Il apparaît que les délais d'exécution sont dominés par les délais de démarrage à froid, ce qui fait de l'allocation adéquate des ressources une exigence stricte pour respecter les accords de niveau de service.

Dans la partie consacrée à l'évaluation des performances, nous comparons deux autoscalers :

\begin{itemize}
    \item HeROfake (HRO) -- Notre allocateur de ressources basé sur les métadonnées et conscient de l'hétérogénéité ;
    \item Knative (KN) -- Nous avons modélisé le comportement de l'autoscaler Knative au mieux de nos connaissances.
\end{itemize}

Notre évaluation comporte quatre ordonnanceurs :

\begin{itemize}
    \item HeROfake (HRO) -- Notre ordonnanceur conscient des coûts qui minimise les violations de SLA, la consommation d'énergie et l'utilisation des ressources ;
    \item Knative (KN) -- Knative sélectionne une plateforme sur le nœud le plus disponible~\cite{sureshENSUREEfficientScheduling2020}. Les plates-formes d'exécution sont triées en fonction du nombre de demandes en cours d'exécution. La plateforme avec la file d'attente la plus courte est sélectionnée ;
    \item Random Placement (RP) -- Les tâches sont assignées à une plateforme d'exécution aléatoire sur un nœud aléatoire ;
    \item Bin Packing First-Fit (BPFF) -- Les tâches sont consolidées sur le nombre minimum de plateformes d'exécution. Tant qu'un nœud dispose de suffisamment de mémoire pour accueillir de nouvelles répliques, il est systématiquement choisi jusqu'à ce qu'il n'ait plus de mémoire ; un nouveau nœud est alors sélectionné. BPFF sera probablement la politique d'ordonnancement pour AWS Lambda~\cite{wangPeekingCurtainsServerlessb}.
\end{itemize}

Nous avons conçu une évaluation des performances en deux étapes basée sur des simulations :

\begin{itemize}
    \item \textbf{Comparaison avec les lignes de base} (Figure~\ref{figure:herofake-evaluation-hro-full}) : dans cette partie, nous avons comparé notre combinaison HeROfake d'autoscaler et d'ordonnanceur (HRO-HRO) à : (1) l'autoscaler et l'ordonnanceur Knative complet (KN-KN), (2) l'autoscaler Knative avec l'ordonnanceur BPFF (KN-BPFF), (3) l'autoscaler Knative avec l'ordonnanceur RP (KN-RP) ; 
    \item \textbf{Impact des composants HeROfake sur la performance globale} (Figure~\ref{figure:herofake-evaluation-hro-mixed}) : nous discutons ici de l'impact individuel de chacun des autoscalers et de l'ordonnanceur. Pour ce faire, nous avons conçu différentes stratégies : (1) en utilisant l'autoscaler HeROfake avec l'ordonnanceur Knative, et (2) en utilisant l'autoscaler Knative avec l'ordonnanceur HeROfake, et nous avons comparé ces stratégies avec les fonctionnalités complètes de HeROfake et Knative.
\end{itemize}

La dénomination de chaque scénario dans ces figures se compose de deux parties divisées par un symbole de tiret. La première partie correspond à la politique d'allocation, la seconde à la politique d'ordonnancement (par exemple, HRO-KN signifie que nous avons utilisé l'autoscaler HeROfake en conjonction avec l'ordonnanceur Knative). 

Pour chacune des combinaisons de politiques d'autoscaler et d'ordonnanceur, nous avons réalisé l'expérience sur un scénario de charge de travail synthétique composé de 50000 tâches (demandes d'utilisateurs). Les tâches se voient attribuer un type aléatoire (ResNet50, VGG16 ou VGG19) et un niveau de QoS aléatoire (élevé, moyen, faible) suivant une distribution uniforme, avec des écarts de durée de QoS respectivement fixés à 2, 3 et 4. L'infrastructure du scénario se compose de 10 nœuds (10 CPU, 6 GPU, 2 FPGA).

Les pondérations pour le niveau de concurrence (équation~\ref{eq:herofake-HRO-concurrency-target}) ont été fixées à $k_{ET} = \frac{2}{3}$, $k_{EC} = \frac{1,5}{6}$ et $k_{HP} = \frac{0,5}{6}$. Les pondérations pour la décision de réduction d'échelle (équation~\ref{eq:herofake-HRO-allocation-cost-function}) ont été fixées à $k_{TT} = \frac{2}{3}$, $k_{EC} = \frac{1,5}{6}$ et $k_{HP} = \frac{0,5}{6}$. Les pondérations pour la décision d'ordonnancement (équation~\ref{eq:herofake-HRO-scheduling-cost-function}) ont été fixées à $k_{QP} = \frac{2}{3}$, $k_{EC} = \frac{0,5}{6}$ et $k_{TC} = \frac{1,5}{6}$. 

\subsection{Analyse des résultats}

\subsubsection{Comparaison aux politiques de base}

\textbf{Consolidation des tâches}. La figure~\ref{figure:herofake-evaluation-full-nodes} montre que notre combinaison d'autoscaler et d'ordonnanceur permet d'obtenir le plus grand nombre de nœuds inutilisés. Avec l'autoscaler de Knative, l'ordonnanceur BPFF assure la meilleure consolidation, mais cette politique nécessite toujours plus de trois fois les nœuds dont nous avons besoin avec notre politique.

\textbf{Accords de niveau de service}. La figure~\ref{figure:herofake-evaluation-full-penalty} montre que HRO-HRO est le plus performant en termes de violations de la QoS, avec 35\% de tâches qui ne respectent pas les délais. Il s'agit d'une amélioration considérable par rapport aux résultats de Knative, où les tâches ne respectent pas les délais dans plus de 99\% des cas : le retard introduit par l'allocation réactive des ressources ne peut pas être compensé à temps en utilisant uniquement des unités centrales.

\textbf{Consommation d'énergie}. La figure~\ref{figure:herofake-evaluation-full-energy} montre que notre politique, avec l'autoscaler et l'ordonnanceur HRO fonctionnant conjointement, est toujours la plus performante en termes de consommation d'énergie dynamique. Cela s'explique évidemment par le fait que nous allouons des accélérateurs matériels ; cependant, au cours de notre évaluation, la durée d'exécution de notre scénario est similaire avec les politiques Knative et HRO (environ 13,5 minutes). La politique d'ordonnancement BPFF est également la moins performante en termes de temps d'exécution, car elle maximise les files d'attente des tâches dans les plateformes d'exécution, ce qui donne les pires résultats en termes de consommation d'énergie.

\subsubsection{Impact des composants individuels}

\textbf{Consolidation des tâches}. La figure~\ref{figure:herofake-evaluation-mixed-nodes} montre que HRO-HRO est le plus performant en matière de consolidation des tâches, laissant un peu moins de 70\% des nœuds disponibles inutilisés, alors que l'ordonnanceur de Knative, dans le cadre de notre politique d'autoscaling, n'atteint que 40\% de nœuds inutilisés. Ce résultat est attendu, car l'ordonnanceur de Knative utilise une politique de moindre connexion. Les résultats de consolidation de KN-HRO sont médiocres, mais pour une raison différente : notre ordonnanceur tente de minimiser les violations de la qualité de service et répartit la tâche sur tous les processeurs alloués.

\textbf{Accords de niveau de service}. La figure~\ref{figure:herofake-evaluation-mixed-penalty} montre que notre ordonnanceur ne fonctionne pas bien en conjonction avec l'autoscaler de Knative. En effet, notre ordonnanceur tente de minimiser les pénalités : lorsqu'il ne dispose que de CPU, il se comporte de la même manière que l'ordonnanceur de Knative et répartit les tâches sur ces CPU afin de limiter les violations de la qualité de service. Cependant, notre ordonnanceur sous l'autoscaler Knative parvient toujours à maintenir les violations de la qualité de service à environ 50\% des tâches, ce qui montre qu'il y a une marge d'amélioration même lorsque les tâches d'inférence sont déployées sur des CPU uniquement. Notez que lors de notre évaluation, l'autoscaler Knative a donné les pires résultats en ce qui concerne la fréquence des démarrages à froid (6,5 fois plus fréquents avec KN-HRO qu'avec HRO-KN).

\textbf{Consommation d'énergie}. La figure~\ref{figure:herofake-evaluation-mixed-energy} montre que la consommation d'énergie est toujours plus faible lorsque l'on utilise notre autoscaler, qui peut allouer des accélérateurs matériels. Cependant, notre planificateur utilisé avec l'autoscaler de Knative donne les pires résultats en termes de consommation d'énergie. Cela s'explique à nouveau par le fait que l'ordonnanceur tente de minimiser les pénalités et de répartir les tâches sur un nombre maximal de CPU.

\section{État de l'art}
\label{section:herofake-sota}

Les travaux antérieurs se sont concentrés sur les plates-formes de mise à l'échelle automatique pour le déploiement de tâches de courte durée, comprises dans des applications présentant des modèles de charge imprévisibles. Le tableau~\ref{table:herofake-sota} résume les différences entre ces contributions et notre plateforme cible.

Certaines de ces contributions ont tenté d'atteindre le SLA avec des ressources non réservées~\cite{gujaratiSwayamDistributedAutoscaling2017, zhangMArkExploitingCloud, mampageDeadlineawareDynamicResource2021, singhviAtollScalableLowLatency2021, handaoui2020releaser, handaoui2020salamander, yalles2022riscless}.
Parmi ces contributions, certaines se concentrent sur l'utilisation de ressources matérielles hétérogènes supplémentaires pour accélérer l'exécution de la charge de travail~\cite{zhangMArkExploitingCloud, lingPigeonDynamicEfficient2019, yangINFlessNativeServerless2022}.
Elles nécessitent souvent un surprovisionnement des ressources pour utiliser l'accélération matérielle, par exemple en s'appuyant sur des instances AWS réservées qui donnent accès aux GPU~\cite{zhangMArkExploitingCloud}, en utilisant un pool de conteneurs préchauffés~\cite{lingPigeonDynamicEfficient2019}, ou même en provisionnant de manière proactive les nœuds pour respecter les délais des fonctions définies par l'utilisateur~\cite{singhviAtollScalableLowLatency2021}. Ces solutions intéressantes peuvent toutefois s'avérer insuffisantes en termes d'utilisation des ressources et entraîneraient une consommation d'énergie supplémentaire dans un cloud privé.

En outre, certains auteurs se concentrent sur des infrastructures homogènes \cite{gujaratiSwayamDistributedAutoscaling2017, sureshENSUREEfficientScheduling2020, mampageDeadlineawareDynamicResource2021, singhviAtollScalableLowLatency2021, yangINFlessNativeServerless2022}. Ces études pourraient difficilement s'adapter au contexte du cloud privé que nous visons, où les ressources sont généralement transitoires et hétérogènes. En outre, certaines de ces contributions proposent des modèles de tâches qui ne couvrent pas les accords de niveau de service définis par l'utilisateur et par demande~\cite{sureshENSUREEfficientScheduling2020, lingPigeonDynamicEfficient2019}. Enfin, certaines de ces contributions sont axées sur les performances plutôt que sur les coûts, ce qui est crucial dans notre contexte de cloud privé~\cite{gujaratiSwayamDistributedAutoscaling2017, lingPigeonDynamicEfficient2019, singhviAtollScalableLowLatency2021, choSLADrivenMLInference}.

Bien que l'alimentation soit l'un des éléments les plus importants du coût total de possession (TCO) dans un centre de données - dépassant parfois le coût d'achat du matériel~\cite{7279063} -- à notre connaissance, aucune de ces contributions ne couvre l'impact de l'allocation et du placement dynamiques sur la consommation d'énergie, ni ne considère la consommation d'énergie comme une mesure de la qualité de service. Il s'agit d'une limitation sérieuse, car l'optimisation de la consolidation des tâches ouvre la voie à des politiques d'étranglement et de mise hors tension qui peuvent avoir un impact majeur sur l'efficacité énergétique d'un centre de données~\cite{chaurasiaComprehensiveSurveyEnergyaware2021}.

\section{Conclusion et perspectives}

Dans cet article, nous avons présenté HeROfake, notre cadre pour le déploiement de tâches de détection de deepfake interactives et de courte durée sur un cloud privé hétérogène serverless.

Nous avons présenté les deux phases qui composent ce cadre : une phase hors-ligne au cours de laquelle nous caractérisons les performances des plateformes d'exécution et les exigences des tâches ; et une phase en ligne au cours de laquelle nous allouons dynamiquement les ressources et planifions les tâches à exécuter sur ces plateformes.

Les résultats expérimentaux montrent que si le temps total d'exécution des tâches dans HeROfake est similaire à celui de la version vanille de Knative, nous obtenons une réduction de plus de 60\% des pénalités de qualité de service ; les tâches sont consolidées sur moins de 40\% des nœuds de l'infrastructure, 77\% des plates-formes d'exécution restant inutilisées ; enfin, la consommation d'énergie dynamique est réduite de 35\% par rapport à Knative.

L'inclusion du traitement vidéo dans le cadre est un défi intéressant, car il introduirait des dépendances entre les tâches. Les exécutions de fonctions ne seraient plus \textit{stateless}, ce qui entraînerait la nécessité de s'attaquer au problème du stockage des données intermédiaires dans une infrastructure serverless.

Nous avons également l'intention d'étendre le simulateur avec un analyseur syntaxique afin d'être en mesure d'utiliser des traces de centres de données réels comme scénarios d'entrée, au lieu d'utiliser uniquement des charges de travail synthétiques.


\clearemptydoublepage
\chapter{HeROcache : Applications serverless et coûts associés aux systèmes de stockage}
\label{chapter:herocache}

\section{Introduction}
\label{section:herocache-introduction}

% TODO: traduction française des figures

\textbf{IDS, des applications critiques et sensibles au temps.} Un large éventail de systèmes embarqués fonctionnant dans des environnements statiques et contrôlés (capteurs dans une usine) ou dynamiques et non contrôlés (essaims de drones en mouvement) peuvent être temporairement ou constamment exposés à des attaques critiques par l'intermédiaire de liaisons réseau. Comme ces attaques peuvent compromettre leur exécution et endommager gravement les infrastructures connexes, il est essentiel de les prendre en compte. Pour atténuer ces menaces, les systèmes de détection d'intrusion (IDS, pour \textit{Intrusion Detection Systems}) sont utilisés pour analyser le trafic réseau et détecter des motifs d'activités potentiellement malveillantes. Les modèles d'apprentissage machine sont particulièrement utiles pour classifier ce trafic, mais ils sont très gourmands en ressources de calcul, en mémoire et en stockage. Par conséquent, les exécuter directement sur la plateforme embarquée n'est pas une solution sûre, car cela peut affecter leur durée de vie s'ils fonctionnent sur une batterie~\cite{slimani:hal-04159551}, interférer avec d'autres tâches critiques, ou même être totalement impossible en raison d'un manque de ressources.

\textbf{IDS à l'edge.} Pour décharger les systèmes embarqués de l'exécution de ces algorithmes gourmands en ressources, tout en maintenant le système réactif aux attaques, une solution consiste à exécuter les IDS dans le cloud, et en particulier sur les dispositifs \textit{edge}~\cite{eskandari2020}. Les IDS doivent répondre à des exigences variables en matière de qualité de service (QoS) et peuvent n'être nécessaires que pendant des périodes critiques, identifiées à l'avance. En outre, différents types d'attaques peuvent présenter des risques différents sur l'infrastructure sous-jacente, et le risque d'attaque peut varier dans le temps et dans l'espace (en fonction du domaine d'application). Par conséquent, il pourrait être inutilement coûteux de déployer des systèmes de détection d'intrusion sur des dispositifs edge réservés. Nous soutenons que le déploiement d'IDS sur des ressources non réservées à faible consommation d'énergie à l'edge pourrait offrir l'avantage d'une solution rentable pour l'exécution de telles applications, tout en offrant une latence plus faible que lorsque l'on s'appuie sur le cloud.

\textbf{Serverless et IDS à l'edge.} L'un des principaux paradigmes du cloud qui permet d'exécuter des applications interactives sur des ressources non réservées, avec une granularité fine d'allocation des ressources, est le serverless~\cite{Lannurien2023}. Le déploiement serverless à l'edge pour l'IDS, et plus généralement pour les applications critiques et sensibles au temps, peut être intéressant du point de vue des coûts en ouvrant des possibilités d'optimisation pour le fournisseur de services, grâce à la mise à l'échelle dynamique des ressources suite à des pics de charge dans les applications interactives, ainsi qu'à une granularité d'allocation fine et mesurée pour les ressources limitées à l'edge.

\textbf{Défis pour le déploiement serverless d'applications critiques à l'edge.} Pour déployer des applications sensibles au temps composées de fonctions à courte durée de vie dans un contexte edge, hétérogène et serverless, trois défis doivent être relevés : (1) réduire les délais d'initialisation, (2) éviter des délais de communication élevés et (3) exploiter des ressources hétérogènes pour satisfaire une QoS variable.
\textbf{Délais d'initialisation.} Les fonctions d'IDS sont de courte durée, et le serverless s'appuyant sur des ressources non réservées, cela implique un taux plus élevé d'initialisations de fonctions, chacune nécessitant de récupérer l'image de la fonction depuis un support distant vers les nœuds edge~\cite{yanHermesEfficientCache2020}. Les dispositifs edge exposent des supports de stockage de faible capacité et de faible performance, derrière des liaisons réseau limitées en termes de fiabilité et de vitesse. Cette première problématique doit donc être adressée pour satisfaire la qualité de service des utilisateurs.
\textbf{Délais de communication.} Dans une infrastructure distribuée telle que le serverless edge, les fonctions d'une même application peuvent être déployées sur plusieurs nœuds éloignés les uns des autres, ce qui implique l'utilisation du réseau lorsque ces fonctions ont besoin de communiquer des résultats intermédiaires. Cela entraîne des retards qui peuvent conduire à des violations de la qualité de service~\cite{wawrzoniakBoxerDataAnalytics2021a}.
\textbf{Hétérogénéité matérielle}. La plateforme serverless ne peut pas considérer tous les placements comme égaux, car ils produiront divers niveaux de performance. Cependant, l'affinité d'une fonction avec une plateforme d'exécution spécifique ne peut pas guider à elle seule les décisions d'ordonnancement, car ces fonctions peuvent appartenir à différentes chaînes en fonction de l'application demandée.

\textbf{Énoncé du problème.} Le problème que nous abordons est de savoir comment prendre en compte \textbf{les délais d'initialisation et de communication} lors du déploiement de \textbf{chaînes de fonctions serverless à courte durée de vie} à \textbf{l'edge}, en tirant parti de \textbf{l'hétérogénéité matérielle} pour optimiser le déploiement d'applications sensibles à la latence qui nécessitent \textbf{une qualité de service variable}, tout en limitant le nombre de nœuds edge utilisés.

\begin{table*}[!ht]
    \centering
        \caption{State of the Art work on data-aware autoscaling platforms}
        \resizebox{\textwidth}{!}{
            \begin{tabular}{lSSSSSSS}
                \toprule
                & Function chains & QoS-aware & Hardware heterogeneity & Programming constraint & Energy consumption & Function cache & Function communications \\
                \cmidrule(lr){2-2}\cmidrule(lr){3-3}\cmidrule(lr){4-4}\cmidrule(lr){5-5}\cmidrule(lr){6-6}\cmidrule(lr){7-7}\cmidrule(lr){8-8}
                Cypress~\cite{bhasiCypressInputSizesensitive2022} & \cmark & \cmark & \xmark & \cmark & \cmark & \xmark & \cmark \\
                FaDO~\cite{smithFaDOFaaSFunctions2022} & \xmark & \xmark & \xmark & \cmark & \xmark & \xmark & \cmark \\
                FaasFlow~\cite{zijunFassflowEfficient2022} & \cmark & \xmark& \xmark & \xmark & \xmark & \xmark & \xmark \\
                FIRST~\cite{zhangFIRSTExploitingMultiDimensional2023} & \xmark & \xmark & \xmark & \cmark & \cmark & \xmark & \xmark \\
                HeROfake~\cite{herofake} & \xmark & \cmark & \cmark & \cmark & \cmark & \xmark & \xmark \\
                Netherite~\cite{burckhardtNetheriteEfficientExecution} & \cmark & \xmark & \xmark & \cmark & \xmark & \xmark & \cmark \\
                Palette~\cite{abdiPaletteLoadBalancing2023} & \cmark & \xmark & \xmark & \xmark & \xmark & \cmark & \cmark \\
                Target solution & \cmark & \cmark & \cmark & \cmark & \cmark & \cmark & \cmark \\
                \bottomrule
            \end{tabular}
        }
    \label{table:herocache-sota}
\end{table*}

\textbf{État de l'art.} Des études antérieures ont exploré le besoin de plateformes d'orchestration qui prennent en charge l'ordonnancement de chaînes de fonctions sur des ressources non réservées. Le tableau~\ref{table:herocache-sota} résume dans quelle mesure ces solutions ne sont pas applicables à notre étude de cas, et la section~\ref{section:herocache-sota} donne plus de détails. Ces contributions visent généralement les déploiements dans le cloud, où l'enjeu est de traiter autant de tâches que possible dans une infrastructure homogène de nœuds toujours en service, afin de maximiser l'usage des ressources. La portée de notre étude est de montrer qu'avec des politiques d'allocation et d'ordonnancement adéquates, nous pouvons déployer des applications bien définies sur un nombre limité de nœuds hétérogènes et réduire la consommation d'énergie globale de l'infrastructure en consolidant les tâches connexes.

\textbf{Contribution : HeROcache, une plateforme d'orchestration des ressources hétérogènes, optimisée pour la qualité de service, pour le serverless à l'edge et basée sur la mise en cache et la consolidation}. Dans cet article, nous présentons une solution qui répond aux trois défis mentionnés ci-dessus. HeROcache : (1) exploite un mécanisme de mise en cache sur les nœuds edge qui réduit \textbf{les délais d'initialisation} sans saturer leur capacité de stockage ; (2) consolide les tâches sur la base d'une application afin de limiter le nombre de \textbf{délais de communication} entre les nœuds ; (3) gère le respect des exigences de qualité de service pour les tâches critiques en utilisant des métadonnées collectées auprès des applications et des plateformes hétérogènes utilisées pour le déploiement. Ces données comprennent des mesures de performance et d'énergie qui guident l'orchestrateur dans ses prises de décision, lors de l'allocation et l'ordonnancement de ressources et de requêtes \textbf{hétérogènes}.

\textbf{Quelques résultats.} Nous avons évalué HeROcache dans le contexte d'une application IDS réelle, caractérisée sur différentes plateformes d'exécution. Cette évaluation a été réalisée à l'aide d'un simulateur \textit{ad hoc}. Nous avons également mis en œuvre le comportement d'un orchestrateur Knative~\cite{knative}. HeROcache parvient à surpasser Knative, en maintenant les violations de la qualité de service à moins de 28\% tout en consolidant les tâches sur 20\% des nœuds edge de l'infrastructure. La mise hors tension de ces nœuds entraînerait une réduction drastique de la consommation d'énergie statique.

Le chapitre est organisé comme suit : la section~\ref{section:herocache-background} détaille les défis pour le déploiement de l'IDS sur une plateforme serverless ; la section~\ref{section:herocache-before-contrib} présente le projet et son architecture globale ; la section~\ref{section:herocache-workload} détaille notre approche de collecte de métadonnées hors-ligne ; la section~\ref{section:herocache-contribution} décrit notre stratégie d'orchestration en ligne ; la section~\ref{section:herocache-evaluation} discute des résultats de l'évaluation ; la section~\ref{section:herocache-sota} examine l'état de l'art ; enfin, la section~\ref{section:herocache-conclusion} conclut le chapitre en donnant des limites à cettre contribution et des perspectives pour de futurs travaux.

\section{Contexte et motivation}
\label{section:herocache-background}

\subsection{Défis de l'orchestration dynamique}

Le serverless est un modèle de service en croissance pour le cloud~\cite{Lannurien2023} : en transférant la responsabilité de l'allocation des ressources des clients vers les fournisseurs de services, il allège une partie importante de la complexité pour les développeurs d'applications et ouvre de nouvelles possibilités d'optimisation et de contrôle des coûts pour le gestionnaire d'infrastructure. Dans une architecture serverless, les développeurs conçoivent leurs applications comme une composition de fonctions sans état. Sans état (ou "pur", sans effet secondaire) signifie que le résultat du calcul dépend exclusivement des entrées \cite{burckhardtNetheriteEfficientExecution}. Ces fonctions prennent en entrée une charge utile et un contexte d'invocation, et produisent un résultat qui est stocké dans un niveau de stockage persistant accessible par le réseau. Cela signifie que les dépendances de données entre les fonctions d'une chaîne doivent être gérées par la plateforme.

Lorsqu'un événement déclenche leur exécution, les fonctions sont déployées sur des nœuds de l'infrastructure, dans des environnements d'exécution appelés \textbf{répliques}. Comme les fonctions sont sans état, les requêtes peuvent être attribuées à n'importe quelle réplique disponible. La mise à l'échelle d'une application serverless, \textit{i.e.} pour maintenir un niveau de performance constant, consiste à faire croître ou décroître le nombre de répliques des fonctions en suivant les pics de charge. Les plateformes serverless basées sur Kubernetes, telles que Knative~\cite{knative} ou OpenWhisk~\cite{openwhisk}, ont proposé un modèle basé sur un seuil de concurrence pour le dimensionnement du pool de répliques. Pour toute fonction, un \textit{autoscaler} peut déployer plusieurs \textit{répliques} pour absorber la charge. Chaque réplique est allouée à une plateforme d'exécution (\textit{i.e.} un cœur de CPU, un GPU, etc.) et dispose d'une file d'attente pour les requêtes entrantes. Le nombre de répliques pour une fonction donnée à un instant donné détermine son niveau de concurrence. Un \textit{ordonnanceur} place les requêtes utilisateur dans la file d'attente d'une réplique de la fonction. Lorsqu'une réplique n'a plus de requêtes en attente, elle peut être détruite. Lorsqu'une fonction est demandée alors qu'aucune réplique n'existe, elle passe par un \textbf{démarrage à froid} qui entraîne un délai d'initialisation qui s'ajoute au temps de réponse de la fonction.

Dans le serverless, la fréquence des allocations de ressources augmente considérablement par rapport aux environnements à ressources réservées, toujours actives, tels que les offres IaaS ou PaaS (cf. chapitre~\ref{chapter:context}). La capacité des plateformes serverless à mettre à l'échelle une fonction jusqu'à zéro réplique afin d'éviter de facturer les clients pour des ressources inactives est une différence essentielle par rapport aux modèles de services cloud traditionnels.

Ce démarrage à froid présente un risque d'augmentation du temps de réponse d'une application, car la plateforme doit allouer des ressources matérielles pour instancier chacune des fonctions qui la composent avant de répondre à la requête. Plus l'application est complexe, plus le risque de retards cumulés est élevé~\cite{mohanAgileColdStartsa}. Les fournisseurs pré-allouent généralement certaines ressources pour éviter les démarrages à froid, ce qui a un coût en termes de provisionnement des ressources. Les acteurs commerciaux tels qu'AWS, Google et Microsoft réutilisent tous, dans une certaine mesure, des instances de fonctions, en les gardant en vie pendant une période de grâce afin d'éviter les coûts en latence induits par les démarrages à froid~\cite{vahidiniaColdStartServerless2020}.

Une étude récente a montré que 50\% des applications serverless déployées sur Microsoft Azure Durable Functions~\footnote{\href{https://learn.microsoft.com/en-US/azure/azure-functions/durable/durable-functions-overview}{https://learn.microsoft.com/en-US/azure/azure-functions/durable/durable-functions-overview}} sont constituées de 3 fonctions ou moins, 65\% des applications présentant un simple DAG (\textit{Directed Acyclic Graph}, graphe acyclique dirigé) de fonctions agencées sous forme de chaînes linéaires \cite{mahgoubORIONThreeRights}. Notre application d'IDS se compose de différentes chaînes de deux fonctions, comme décrit dans la section~\ref{section:herocache-characterization-workloads}. Des travaux de caractérisation des charges de travail serverless ont montré que 25\% des fonctions déployées sur Microsoft Azure Functions~\footnote{\href{https://azure.microsoft.com/en-us/products/functions/}{https://azure.microsoft.com/en-us/products/functions/}} s'exécutent en 100 ms ou moins \cite{shahradServerlessWildCharacterizing}. Les fonctions qui composent notre application d'IDS s'exécutent pendant quelques centièmes ou dixièmes de seconde, ce qui les rend particulièrement sujettes à des ralentissements critiques dans le contexte de l'allocation dynamique des ressources.

\subsection{Mise en cache des images de fonctions}
\label{section:herocache-background-cache}

\begin{figure}[!ht]
    \centering
    \includegraphics[width=0.8\columnwidth]{5_Chapitre5/figures/function-cache.png}
    \caption{Lifecycle of a user request in a serverless platform.}
    \label{figure:herocache-function-cache}
\end{figure}

Afin de répondre aux requêtes des utilisateurs sans dégrader les performances, l'autoscaler ajuste périodiquement le nombre de répliques pour chaque fonction déployée : le pool de répliques croît et décroît en fonction des variations sur la charge des fonctions d'une application. Lorsque la charge sur une fonction augmente au-delà du seuil de concurrence de la plateforme, l'autoscaler crée une nouvelle réplique qui traitera les requêtes supplémentaires des utilisateurs. Lorsque la charge diminue, les répliques inactives sont supprimées. S'il n'y a plus de requêtes pour une fonction donnée, celle-ci peut être \textit{mise à l'échelle jusqu'à zéro}, ce qui permet de réattribuer des ressources matérielles à des tâches plus pressantes.

Les répliques de fonctions sont initialisées à partir d'\textbf{images de fonctions} (\textit{e.g.} une image Docker ou de machine virtuelle). Celles-ci sont stockées dans un registre d'images. Ces registres peuvent être accessibles à distance par Internet, ou déployés dans l'infrastructure du fournisseur. Toutefois, de nombreuses études antérieures~\cite{bhasiCypressInputSizesensitive2022, zijunFassflowEfficient2022, smithFaDOFaaSFunctions2022, zhangFIRSTExploitingMultiDimensional2023} n'envisagent que des scénarios favorables, dans lesquels les images de fonctions sont déjà disponibles sur les nœuds edge. Cela ne reflète pas la réalité, puisque les images de fonctions sont stockées dans des registres, sur des nœuds dédiés, et téléchargées sur les nœuds edge lors du déploiement des fonctions. En fonction de la taille de l'image, cela peut avoir des conséquences négatives sur la latence des requêtes avec des déploiements où les démarrages à froid dominent le temps de réponse total d'une fonction \cite{yanHermesEfficientCache2020}.

Ce processus d'initialisation des répliques à partir d'images sur les nœuds edge peut représenter jusqu'à 80\% du temps de réponse d'une fonction \cite{yanHermesEfficientCache2020} dans les cas où la latence introduite par le démarrage à froid domine le temps de réponse total de la fonction. Cette situation n'est pas acceptable lorsque la plateforme doit répondre à des exigences strictes en matière de qualité de service, comme c'est le cas pour les tâches critiques telles que l'IDS.

\subsection{Communications entre les fonctions}
\label{section:herocache-background-communications}

\begin{figure}[!ht]
    \centering
    \includegraphics[width=0.8\columnwidth]{5_Chapitre5/figures/function-communications.png}
    \caption{Serverless functions communicate intermediate results through persistent storage that can be local to edge nodes or remotely accessible.}
    \label{figure:herocache-function-communications}
\end{figure}

Pour rendre possible la mise à l'échelle dynamique des fonctions, il est nécessaire que chaque invocation d'une fonction serverless soit indépendante, c'est-à-dire qu'elle ne porte pas les données ou le contexte des invocations précédentes. Cela permet aux répliques de mettre en file d'attente les requêtes utilisateur et de les traiter de manière séquentielle sans avoir besoin de procéder à un démarrage à froid entre les requêtes. Cela introduit une contrainte sur la plateforme serverless : si une application est composée de plusieurs fonctions qui forment une chaîne de traitement, la sortie de chaque fonction doit être sauvegardée dans un stockage persistant pour être passée en entrée de la fonction suivante dans la chaîne~\cite{mullerLambadaInteractiveData2020}.

Les travaux de l'état de l'art ont montré que les fonctions serverless qui communiquent, \textit{via} le réseau, par le biais d'un stockage distant, peuvent subir un ralentissement jusqu'à 11x par rapport aux fonctions utilisant des communications directes~\cite{wawrzoniakBoxerDataAnalytics2021a} (\textit{e.g.} au travers de mémoire partagée au sein d'un même nœud). Les fonctions de notre application d'IDS doivent communiquer des résultats intermédiaires à chaque étape du DAG de l'application. Lorsque les fonctions sont déployées sur différents nœuds edge, les communications inter-fonctions devront être réalisées par l'utilisation d'un stockage à distance. Cela introduit des ralentissements qui peuvent faire boule de neige tout au long de l'exécution des fonctions et détériorer la qualité de service de l'ensemble de l'application.

\section{Détection d'intrusion à l'edge dans le modèle serverless}
\label{section:herocache-before-contrib}

\begin{figure*}[!ht]
    \centering
    \includegraphics[width=0.65\textwidth]{5_Chapitre5/figures/serverless-platform-storage.png}
    \caption{Serverless IDS platform, system overview}
    \label{figure:herocache-serverless-platform}
\end{figure*}

Orchestrer des applications tout en respectant les accords de niveau de service (\textit{SLA}, pour \textit{Service Level Agreement}) nécessite de modéliser précisément les caractéristiques de l'application et d'en tenir compte lors de l'allocation des ressources et de l'ordonnancement des requêtes utilisateur sur la plateforme serverless. La figure~\ref{figure:herocache-serverless-platform} donne un aperçu du cycle de vie d'une application sur notre plateforme. Il est divisé en deux phases ; une \textbf{phase hors-ligne} qui consiste à caractériser les applications déployées par les développeurs sur les plateformes edge, et une \textbf{phase en ligne} où les requêtes des utilisateurs vers ces applications sont ordonnancées sur la plateforme.

\textbf{Phase hors-ligne}. Dans notre plateforme, le cycle de vie de l'application commence par une phase hors-ligne, au cours de laquelle le développeur fournit le code de ses fonctions pour différentes architectures matérielles (GPU, CPU, DLA, etc.)~\Circled{1}. Ce code est stocké par le fournisseur de services dans un registre de fonctions. Les fonctions sont ensuite déployées sur un nœud de mesure~\Circled{2} où elles sont exécutées afin de produire des métadonnées relatives à leur exécution sur des nœuds hétérogènes. Les besoins en mémoire, le temps d'exécution, le temps de démarrage à froid, la consommation d'énergie, la taille de la fonction et la taille des données communiquées pour chaque fonction sont enregistrés dans un registre de métadonnées~\Circled{3}. L'exécution de la phase hors-ligne est nécessaire une fois pour une fonction donnée sur une plateforme donnée, comme décrit dans la section~\ref{section:herocache-workload}.

\textbf{Phase en ligne}. Les requêtes utilisateur envoyées aux applications d'IDS comportent un extrait de trafic réseau (sous forme de paquets \textit{TCP}, pour \textit{Transmission Control Protocol}, sérialisés) à analyser~\Circled{4}, et sont associées à un niveau de qualité de service souhaité pour le temps de réponse de la requête. La requête est ajoutée à une file d'attente~\Circled{5} au niveau de l'orchestrateur. Lorsque l'ordonnanceur extrait la requête de la file d'attente, il interroge le registre pour récupérer les métadonnées de fonction appropriées~\Circled{6}.

L'ordonnanceur tente alors de trouver une réplique disponible de la première fonction de l'application pour traiter la requête~\Circled{7}. Si une telle réplique n'existe pas encore, il sera demandé à l'autoscaler d'initialiser une nouvelle instance de la fonction~\Circled{8}. Au cours du cycle de vie de l'application, l'autoscaler vérifie périodiquement la charge moyenne de chaque fonction pour ajuster le nombre de répliques déployées sur la plateforme, en fonction du seuil de concurrence fixé par le fournisseur de services.

Lorsque l'application termine son exécution, elle retourne à l'utilisateur un vecteur de classification qui indique les probabilités que le trafic soit malveillant, c'est-à-dire qu'il présente les caractéristiques d'une attaque potentielle.

\section{Phase hors-ligne : caractérisation} 
\label{section:herocache-workload}

\begin{figure}[!ht]
    \centering
    \includegraphics[width=0.8\columnwidth]{5_Chapitre5/figures/ids-application.png}
    \caption{Architecture of an IDS application that can make use of different preprocessing functions, and different inference functions to provide the user with a classification of TCP traffic.}
    \label{figure:herocache-ids-application}
\end{figure}

Pour parvenir à une allocation des ressources et à un placement des tâches adéquats lors de l'exécution des applications d'IDS, il est nécessaire de caractériser les plateformes d'exécution ainsi que les fonctions logicielles qui seront mobilisées~\cite{mampageHolisticViewResource2022}. À cette fin, nous avons évalué plusieurs modèles d'IDS en termes de performance et de consommation d'énergie sur des plateformes hétérogènes représentatives de dispositifs edge~\cite{kljucaric2020}. Cette section décrit notre méthodologie et nos résultats.

\subsection{Caractérisation des plateformes d'exécution} \label{section:herocache-characterization-platforms}

Nous avons utilisé des plateformes représentatives de ce que l'on peut trouver à l'edge~\cite{slimani:hal-04159551,kljucaric2020} :
\textbf{(1) Raspberry Pi 4B}, équipée d'un CPU quadricœur ARM Cortex-A72, de 4 GB LPDDR4 en mémoire principale et d'une carte SD de 16 GB. La carte fonctionne sous Linux 5.4, avec la distribution Raspbian.
\textbf{(2) Nvidia Jetson Xavier AGX}, composée de trois éléments de traitement : un CPU NVIDIA ARM Carmel à 8 cœurs, un GPU NVIDIA Volta avec 512 cœurs CUDA et un accélérateur d'apprentissage profond (\textit{DLA}, pour \textit{Deep Learning Accelerator}), qui est un accélérateur matériel à fonction fixe conçu pour les réseaux neuronaux convolutifs (\textit{CNN}, pour \textit{Convolutional Neural Network}). Il est supposé être plus économe en énergie que le GPU. La carte Xavier AGX est équipée de 16 GB de mémoire LPDDR4 et de 32 GB de mémoire flash eMMC 5.1. Elle fonctionne sous Linux Tegra 4.9.10. Le mode d'alimentation \textit{15 Watts Desktop} a été utilisé.
\textbf{(3) PYNQ-Z2 Development Board}, une carte basée sur le système sur puce (\textit{SoC}, pour \textit{System on Chip}) Xilinx Zynq XC7Z020. Elle est équipée d'un FPGA Artix-7, d'une mémoire DDR3 de 512 MB et d'une carte SD de 16 GB.

\subsection{Caractérisation des applications}
\label{section:herocache-characterization-workloads}

Notre application, schématisée en figure~\ref{figure:herocache-ids-application}, se compose de différents préprocesseurs et classificateurs. Le préprocesseur sélectionne un sous-ensemble de caractéristiques pertinentes des paquets TCP fournis par l'utilisateur. Trois approches de prétraitement différentes ont été retenues : (1) utilisation de toutes les caractéristiques des paquets sans aucune sélection (\textit{NoFS}, pour \textit{No Feature Selection}) ; (2) utilisation d'un auto-encodeur \textit{DNN} (\textit{Deep Neural Network}) pour projeter les caractéristiques dans un espace latent plus petit (\textit{AE}, pour \textit{Auto-Encoder}) ; et (3) sélection experte d'un sous-ensemble de caractéristiques (\textit{ES}, pour \textit{Expert Selection}). Pour la partie classification, nous avons utilisé un algorithme de \textit{Random Forest} (\textit{RF}), deux architectures différentes de réseaux neuronaux denses (\textit{DNN}) et un \textit{CNN}.

\begin{table}[t]
\caption{IDS models architecture and size}
\resizebox{\textwidth}{!}{
\begin{tabular}{|c|c|cc|}
\hline
Model     & Architecture                                                                                                                                         & \multicolumn{1}{c|}{Model Size on CPUs (MB)} & Model Size on GPU (MB) \\ \hline
NoFS-RF   & \begin{tabular}[c]{@{}c@{}}5 trees of 100\\ maximum depth\end{tabular}                                                                               & \multicolumn{1}{c|}{28}                      & 15.4                   \\ \hline
AE-RF     & \begin{tabular}[c]{@{}c@{}}5 trees of 50 \\ maximum depth\end{tabular}                                                                               & \multicolumn{1}{c|}{-}                       & 32.9                   \\ \hline
ES-RF     & \begin{tabular}[c]{@{}c@{}}10 trees of 10 \\ maximum depth\end{tabular}                                                                              & \multicolumn{1}{c|}{9.1}                     & 5.5                    \\ \hline
NoFS-DNN1 & \multirow{3}{*}{\begin{tabular}[c]{@{}c@{}}4 Dense Layers \\ (128x64x32x10)\end{tabular}}                                                            & \multicolumn{2}{c|}{0.144}                                            \\ \cline{1-1} \cline{3-4} 
AE-DNN1   &                                                                                                                                                      & \multicolumn{2}{c|}{0.321}                                            \\ \cline{1-1} \cline{3-4} 
ES-DNN1   &                                                                                                                                                      & \multicolumn{2}{c|}{0.053}                                            \\ \hline
NoFS-DNN2 & \multirow{3}{*}{\begin{tabular}[c]{@{}c@{}}5 Dense Layers \\ (7024x704x288x64x10)\end{tabular}}                                                      & \multicolumn{2}{c|}{3.33}                                             \\ \cline{1-1} \cline{3-4} 
AE-DNN2   &                                                                                                                                                      & \multicolumn{2}{c|}{2.96}                                             \\ \cline{1-1} \cline{3-4} 
ES-DNN2   &                                                                                                                                                      & \multicolumn{2}{c|}{2.61}                                             \\ \hline
NoFS-CNN  & \multirow{3}{*}{\begin{tabular}[c]{@{}c@{}}2 Conv1D (x64) - MaxPool \\ 3 Conv1D (x256) - MaxPool\\ 3 Dense Layers (100x20x10)\end{tabular}} & \multicolumn{2}{c|}{4.77}                                             \\ \cline{1-1} \cline{3-4} 
AE-CNN    &                                                                                                                                                      & \multicolumn{2}{c|}{2.9}                                              \\ \cline{1-1} \cline{3-4} 
ES-CNN    &                                                                                                                                                      & \multicolumn{2}{c|}{2.6}                                              \\ \hline
\end{tabular}
}
\label{table:herocache-workload}
\end{table}

Le tableau~\ref{table:herocache-workload} présente les modèles d'IDS pris en compte dans cette étude et certaines de leurs caractéristiques. Ces modèles ont été entraînés et caractérisés sur un jeu de données de l'état de l'art représentant des intrusions sur le réseau, UNSW-NB15\footnote{\href{https://research.unsw.edu.au/projects/unsw-nb15-dataset}{https://research.unsw.edu.au/projects/unsw-nb15-dataset}}. Dans ce jeu de données, chaque observation représente des caractéristiques statistiques, de contenu et de temps sur le trafic au cours d'une fenêtre temporelle, et est étiquetée comme "normale" ou "attaque". Le jeu de données comprend neuf catégories d'attaques. Les modèles de réseaux neuronaux ont été exportés et optimisés à l'aide de TensorFlow Lite et TensorRT lorsqu'ils étaient destinés à des plateformes CPU et GPU/DLA, respectivement. En ce qui concerne Random Forest, les modèles ont été exportés en utilisant les frameworks Emlearn et HummingBird.ml lorsqu'ils étaient destinés aux plateformes CPU et GPU, respectivement. hls4ml a été utilisé pour exporter les modèles de réseaux neuronaux pour la cible FPGA par un procédé de synthèse de haut niveau (\textit{HLS}, pour \textit{High-Level Synthesis}), qui transforme le code source C++ en une représentation \textit{RTL} (\textit{Register Transfer Level}) adaptée à la cible.

\subsection{Mesures de performances}

Pour mesurer leur latence lors de l'inférence, chacun des modèles d'IDS a été déployé sur les plateformes cibles pour être évalué sur un ensemble de $80 000$ paquets provenant du jeu de données \textit{UNSW-NB15}. Les résultats sont présentés dans la figure~\ref{figure:herocache-performance}. Seul un modèle (ES-DNN1) a été caractérisé sur la plateforme FPGA car les représentations RTL des autres modèles après HLS n'ont pas pu être acceptées par la cible. La conclusion qui a été tirée de ces résultats est que pour les réseaux neuronaux, le CPU Xavier atteint la meilleure performance dans la majorité des cas, à l'exception de NoFS-CNN qui profite des capacités du GPU en raison de son nombre élevé de paramètres et de l'efficacité du GPU pour les opérations de convolution. Pour les modèles Random Forest, l'élément de traitement le plus rapide est le GPU. En termes de coût et de disponibilité, la Xavier AGX est respectivement environ 20x et 10x plus chère que les plateformes RBPI4 et Pynq-Z2. Nous dimensionnons notre infrastructure en conséquence en fournissant plus de plateformes RBPI4 que de Xavier AGX pour être représentative des déploiements réels.

\begin{figure}
    \centering
    \includegraphics[width=0.9\columnwidth]{5_Chapitre5/figures/latency_bar.pdf}
    \caption{Latency characterization of IDS models.}
    \label{figure:herocache-performance}
\end{figure}

\subsection{Mesures de consommation d'énergie}

Nous avons exécuté des inférences sur les modèles d'IDS sur chaque élément de traitement et mesuré la consommation d'énergie de la plateforme à l'aide de l'analyseur de puissance N6705A DC. Les résultats sont présentés dans la figure~\ref{figure:herocache-energy}. Pour les mêmes raisons que mentionné ci-dessus, seul ES-DNN1 a été caractérisé sur FPGA. Nous observons que les plateformes CPU affichent une consommation d'énergie inférieure à celle du GPU dans la majorité des cas. Le seul cas où le GPU obtient de meilleurs résultats est lorsque la vitesse qu'il atteint par rapport aux CPU est élevée. C'est par exemple le cas pour NoFS-CNN, où le CPU RBPI4 est plus de 30 fois plus lent que le GPU. Même si la carte Pynq-Z2 présente la meilleure efficacité énergétique avec le modèle ES-DNN1, étant donné qu'elle est plus chère et présente une généricité limitée en matière de conception, nous supposons qu'elle est moins disponible que le RBPI4.

\begin{figure}
    \centering
    \includegraphics[width=0.9\columnwidth]{5_Chapitre5/figures/energy_bar.pdf}
    \caption{Energy characterization of IDS models.}
    \label{figure:herocache-energy}
\end{figure}

\section{Phase en ligne : orchestration avec HeROcache} \label{section:herocache-contribution}

\subsection{Présentation générale}

L'orchestrateur HeROcache est principalement composé de deux modules, un \textbf{autoscaler} et un \textbf{ordonnanceur} (voir figure~\ref{figure:herocache-serverless-platform}). L'autoscaler est chargé de l'allocation dynamique des ressources : il affecte les plateformes d'exécution aux répliques de fonctions. L'ordonnanceur s'occupe du placement des requêtes utilisateur sur les répliques.

HeROcache relève les trois défis susmentionnés en proposant des stratégies complémentaires de minimisation des coûts au niveau de l'autoscaler et de l'ordonnanceur. HeROcache minimise \textbf{les délais d'initialisation} en tenant compte des temps de latence induits par la récupération des images de fonctions au niveau de l'autoscaler. Une stratégie de \textit{prefetching} (ou \textit{prélecture}) est également mise en œuvre, qui consiste à essayer de mettre en cache des images de fonctions pertinentes en avance de phase. Les coûts de \textbf{communications inter-fonction} sont pris en compte principalement dans l'ordonnanceur, qui tend naturellement à consolider les fonctions d'une même application. L'autoscaler participe indirectement à cette consolidation par le \textit{prefetching}, en mettant en cache les fonctions suivantes du DAG de l'application sur le même nœud. Enfin, l'\textbf{hétérogénéité matérielle} est prise en compte car les différents facteurs de coût d'exécution extraits pendant la phase hors-ligne (voir section~\ref{section:herocache-workload}) sont exploités tout au long du processus d'autoscaling et d'ordonnancement. Les sections suivantes décrivent les stratégies de mise à l'échelle automatique et d'ordonnancement.

\begin{table}[t]
    \caption{Dictionnaire des notations}
    \begin{center}
    \scalebox{0.85}{\begin{tabularx}{\linewidth}{|c|Y|}
    \hline
    \textbf{Notation} & \textbf{Description} \\ \hline
    $x_a$ & Allocation des ressources pour une application $a$ \\ \hline
    $y_a$ & Invocation d'une application $a$ \\ \hline
    $z_i$ & Placement d'une tâche pour la fonction $i$ \\ \hline
    $f_{N, P}$ & Une fonction $f$ déployée pour exécution sur une plateforme $P$ disponible sur un nœud $N$ \\ \hline
    $f_{a}$ & Une fonction $f$ appartenant à une application $a$ \\ \hline
    $A$ & Nombre total d'applications disponibles pour déploiement sur la plateforme \\ \hline
    $F_{a}$ & Nombre total de fonctions appartenant à une application $a$ \\ \hline
    $RT_{{f}_{N, P}}$ & Temps de récupération de l'image de la fonction $f$ pour un déploiement sur une plateforme $P$ disponible sur un nœud $N$ \\ \hline
    $NB_{N}$ & Bande passante réseau disponible entre un nœud $N$ et l'infrastructure \\ \hline
    $SMT_{N}$ & Débit du support de stockage disponible sur un nœud $N$ \\ \hline
    $SML_{N}$ & Latence du support de stockage disponible sur un nœud $N$ \\ \hline
    $QP$ & Pénalité sur violation de qualité de service \\ \hline
    $QD$ & Facteur de ralentissement par niveau de qualité de service \\ \hline
    $WET$ & Pire temps d'exécution \\ \hline
    $TT$ & Temps total pour une tâche \\ \hline
    $CST$ & Temps de démarrage à froid \\ \hline
    $ST$ & Temps des opérations de stockage \\ \hline
    $ET$ & Temps d'exécution nominal d'une fonction \\ \hline
    $EC$ & Consommation d'énergie \\ \hline
    $IS$ & Taille de l'image d'une fonction \\ \hline
    $HP$ & Prix du matériel \\ \hline
    $TC$ & Consolidation des tâches \\ \hline
    $Q$ & File d'attente des requêtes dans une réplique \\ \hline
    $CP$ & Proportion de fonctions mises en cache sur un nœud pour une application donnée \\ \hline
    $SIS^{f}_{a}$, $SOS^{f}_{a}$ & Respectivement les tailles des données d'entrée et de sortie d'une fonction $f$ appartenant à une application $a$ \\ \hline
    $threshold_{f, h}$ & Seuil de concurrence pour une fonction $f$ sur un type de matériel $h$ \\ \hline
    $scaleCost^{{f}_{{i}_{N, P}}}_a$ & Coût de la création d'une nouvelle réplique pour une fonction $f_i$ d'une application $a$ sur une plateforme $P$ disponible sur un nœud $N$ \\ \hline
    $schedCost^{{f}_{{i}_{N, P}}}_a$ & Coût de l'ordonnancement d'une tâche pour une fonction $f_i$ d'une application $a$ sur une plateforme $P$ disponible sur un nœud $N$ \\ \hline
    \end{tabularx}}
    \label{table:herocache-notation}
    \end{center}
\end{table}

\subsection{Stratégie de minimisation des coûts d'allocation des ressources}

Nous formulons l'allocation des ressources comme un problème d'optimisation et nous le résolvons à l'aide d'un algorithme glouton simple. L'objectif de l'autoscaler est de minimiser la somme des coûts des allocations $scaleCost_{a}$ pour $y_a$ invocations de l'application $a$ (équation~\ref{eq:herocache-objective-allocation}), pour toutes les applications dans $A$, sous contrainte d'une infrastructure finie avec $x_a$ l'allocation de ressources pour l'application $a$ (équation~\ref{eq:herocache-constraint-allocation}).

\begin{equation}
    \forall A, \, \min \sum_{a = 0}^{A} y_a \cdot scaleCost_{a}
\label{eq:herocache-objective-allocation}
\end{equation}

\begin{equation}
    \text{s. t.} \, \sum_{a = 0}^{A} x_a \leq Total Resources
\label{eq:herocache-constraint-allocation}
\end{equation}

Le coût de l'allocation des ressources pour une application $a$ est la somme des coûts d'allocation pour les fonctions qui la composent (équation~\ref{eq:herocache-scale-cost-app}). Une réplique est allouée à une plateforme d'exécution.

\begin{equation}
    scaleCost_{a} = \, \sum_{i = 0}^{F_{a}} scaleCost^{{f}_{{i}_{N, P}}}_a
\label{eq:herocache-scale-cost-app}
\end{equation}

Chaque réplique de fonction a un coût d'allocation associé, car l'allocation dynamique des ressources matérielles introduit un temps de latence lors du traitement des requêtes utilisateur.

Nous avons conçu un modèle de coût (équation~\ref{eq:herocache-scale-cost-function}) pour l'allocation des ressources nécessaires au déploiement d'une fonction d'une application donnée. Il est composé de quatre éléments, dont nous devons minimiser la somme :

\begin{itemize}
    \item La \textit{proportion de cache} $CP$ traduit la dispersion des fonctions sur les différents nœuds de bordure. Plus le score est élevé, plus les fonctions sont dispersées sur les nœuds. La minimisation de ce terme permet de consolider les fonctions ;
    \item Le \textit{temps total} $TT$ représente le temps d'exécution total de la fonction. Il tient compte de la qualité de service de l'application, de l'hétérogénéité de la plateforme et du coût de déploiement (si l'image est mise en cache sur le nœud ou distante). Plus ce coût est élevé, plus la qualité de service est faible ;
    \item La \textit{consommation d'énergie} $EC$ traduit la consommation d'énergie du déploiement de la fonction. Plus $EC$ est élevé, plus le coût est important ;
    \item Le \textit{prix du matériel} $HP$ décrit le coût total de possession (\textit{TCO}, pour \textit{Total Cost of Ownership}) supporté par les fournisseurs de services en fonction du temps d'exécution. Il traduit le coût de déploiement sur une plateforme matérielle donnée. Plus $HP$ est élevé, plus le coût de la solution est important.
\end{itemize}

L'objectif global du modèle de coût est de déployer une fonction au coût le plus bas possible, c'est-à-dire avec une consolidation accrue, une réduction du \textit{makespan} (le temps total d'exécution), une réduction de la consommation d'énergie et une réduction du coût de possession. Nous détaillerons chaque partie de l'équation~\ref{eq:herocache-scale-cost-function} dans les paragraphes suivants. Chaque composante de l'équation est pondérée pour permettre un réglage souple ; les valeurs que nous avons choisies pour le déploiement de l'application d'IDS sont spécifiées dans la partie consacrée à l'évaluation (section~\ref{section:herocache-evaluation}).

\begin{equation}
\begin{split}
 \forall N, \forall P \in N, scaleCost^{{f}_{{i}_{N, P}}}_{a} = \,   &k_{CP} \cdot {CP}_{{a}_{N}}    \\
    + &k_{TT} \cdot {TT}_{{f}_{N, P}} \\
    + &k_{EC} \cdot {EC}_{{f}_{N, P}} \\
    + &k_{HP} \cdot {HP}_{{f}_{N, P}}
\end{split}
\label{eq:herocache-scale-cost-function}
\end{equation}

\textbf{Proportion de cache}. Comme nous l'avons vu précédemment, la mise en œuvre de la consolidation des tâches (donc des exécutions de fonctions) au sein des applications devrait permettre de minimiser les délais de communication qui pourraient s'accumuler au sein des chaînes de fonctions. HeROcache favorise le déploiement de répliques d'une fonction sur des nœuds où d'autres fonctions appartenant à la même application sont déjà déployées.

Pour ce faire, HeROcache surveille $CF_{a}^{{f}_{i_{N, P}}}$ le nombre d'images de fonctions ${f}_{i}$ de l'application $a$ déployée sur le nœud $N$ sur une plateforme d'exécution donnée $P$ (par exemple, GPU) disponibles en cache sur le stockage local du nœud. La proportion de fonctions mises en cache est calculée pour chaque application (équation~\ref{eq:herocache-cached-functions}), puis la moyenne est calculée sur toutes les applications s'exécutant sur un nœud donné. Enfin, on l'inverse pour obtenir une valeur élevée pour les applications peu consolidées (l'objectif étant de minimiser cette proportion), voir équation~\ref{eq:herocache-cache-proportion-app}.

\begin{equation}
    \forall a \in A, \, \forall f \in a, \, CF_{a}^{{f}_{i_{N, P}}} = \frac{\sum_{i = 0}^{Fa} isCached(f_{i}, N, P)}{F_{a}}
\label{eq:herocache-cached-functions}
\end{equation}

\begin{equation}
    \forall N, \forall P \in N, \, {CP}_{{a}_{N}} = \, \frac{A}{\sum_{i = 0}^{F_{a}} CF_{a}^{{f}_{i_{N, P}}}}
\label{eq:herocache-cache-proportion-app}
\end{equation}

En plus de la minimisation des coûts, afin de réduire les délais de déploiement, l'autoscaler essaie de mettre en cache les images des chaînes de fonctions en avance de phase, lors du déploiement d'une nouvelle réplique sur un nœud. Il inspecte les chaînes de fonctions et télécharge séquentiellement les images de fonctions manquantes du registre distant vers le stockage local du nœud, de manière asynchrone. Cette technique permet de réduire la latence des requêtes futures~\cite{leeSPESOptimizingPerformanceResource2024a}.

\textbf{Temps total}. La deuxième composante du coût de mise à l'échelle est le \textit{temps total}. La minimisation du temps total devrait empêcher les retards d'initialisation de faire boule de neige dans les chaînes de fonctions, évitant ainsi les violations des accords de niveau de service (SLA).

Grâce aux métadonnées collectées sur chaque fonction pendant la phase hors-ligne, l'autoscaler est en mesure de prédire le temps total ${TT}_{{f}_{N, P}}$ de la première requête qui sera ordonnancée sur une nouvelle réplique de fonction (équation~\ref{eq:herocache-total-time-function}).

\begin{equation}
    {TT}_{{f}_{N, P}} = \, {RT}_{{f}_{N, P}} + {WT}_{{f}_{N, P}} + {CST}_{{f}_{N, P}} + {ET}_{{f}_{N, P}}
\label{eq:herocache-total-time-function}
\end{equation}

\begin{itemize}
    \item ${RT}_{{f}_{N, P}}$ est la durée de récupération de l'image de la fonction. Si l'image de la fonction est déjà mise en cache sur le nœud de calcul, cette durée est nulle ; sinon, elle dépend de la taille de l'image $IS$ et est influencée par la bande passante $NB$ du lien réseau, car l'image sera lue à partir d'un registre d'images distant, et par le débit du support de stockage du nœud $SMT$ ainsi que sa latence $SML$, car l'image sera écrite et stockée localement en vue d'une utilisation ultérieure (équation~\ref{eq:herocache-retrieval-time}) ;

    \begin{equation}
        {RT}_{{f}_{N, P}} = \, \frac{IS_{{f}_{N, P}}}{\min (NB_{N}, SMT_{N})} + SML_{N}
        \label{eq:herocache-retrieval-time}
    \end{equation}

    \item ${WT}_{{f}_{N, P}}$ est le temps que la tâche passera à attendre dans la file d'attente de la plateforme. Au moment de la création de la réplique, ce temps sera égal à zéro car nous ne prévoyons que la latence encourue par la première requête sur la réplique ;
    \item ${CST}_{{f}_{N, P}}$ est le temps de démarrage à froid nécessaire pour initialiser l'instance de la fonction (\textit{i.e.} décompresser l'image, préparer le conteneur, initialiser les bibliothèques, etc.). Il est mesuré en fonction des métadonnées extraites lors de la phase hors-ligne ;
    \item ${ET}_{{f}_{N, P}}$ est la durée d'exécution de la fonction, y compris le temps de communication avec ses prédécesseurs et successeurs potentiels dans le DAG. Cette durée tient compte des métadonnées extraites pour la plateforme de déploiement (équation~\ref{eq:herocache-execution-time}).
\end{itemize}

\begin{equation}
    {ET}_{{f}_{N, P}} = \, {CT}_{{f}_{N, P}} + {ST}_{{f}_{N, P}}
\label{eq:herocache-execution-time}
\end{equation}

${CT}_{{f}_{N, P}}$ est le \textit{temps de calcul} de la fonction -- le temps attendu pour que la fonction termine son exécution une fois entièrement initialisée. Cette valeur dépend des performances et de la disponibilité de la plateforme d'exécution. ${ST}_{{f}_{N, P}}$ est le \textit{temps de stockage} de la fonction -- le temps attendu pour que la fonction récupère ses données d'entrée et stocke ses données de sortie. Cette valeur dépend des performances du lien réseau et des dispositifs de stockage, comme montré dans l'équation~\ref{eq:herocache-storage-time}.

\begin{equation}
    {ST}_{{f}_{N, P}} = \, \frac{SIS_{a}^{f_{i_{N, P}}} + SOS_{a}^{f_{i_{N, P}}}}{\min (NB_{N}, SMT_{N})} + SML_{N}
\label{eq:herocache-storage-time}
\end{equation}

\textbf{Consommation d'énergie et prix du matériel}. Enfin, la prise en compte de la consommation d'énergie et du prix du matériel devrait permettre de départager les candidats lorsque plusieurs allocations possibles semblent produire le même coût (en fournissant le même niveau de qualité de service).

${EC}_{{f}_{N, P}}$ et ${HP}_{{f}_{N, P}}$ correspondent respectivement (a) à la consommation d'énergie dynamique générée par cette allocation, obtenue grâce à la phase de caractérisation hors-ligne de la charge de travail et de la plateforme et (b) au prix catalogue suggéré par le fabricant (\textit{MSRP}, pour \textit{Manufacturer's Suggested Retail Price}) $Hardware Price_{P}$ du matériel mobilisé pour déployer la fonction sur ledit nœud et ladite plateforme, au regard du temps d'exécution de la fonction $ET_{{f}_{N, P}}$ (équation~\ref{eq:herocache-hardware-price}).

\begin{equation}
    {HP}_{{f}_{N, P}} = \frac{Hardware Price_{P}}{ET_{{f}_{N, P}}}
\label{eq:herocache-hardware-price}
\end{equation}

\subsection{Stratégie de minimisation des coûts d'ordonnancement et de placement des données}

Comme pour l'autoscaling, nous formulons un problème d'optimisation pour trouver la configuration d'ordonnancement optimale pour chaque requête utilisateur (puisque la qualité de service doit être garantie sur la base d'une requête utilisateur) et nous le résolvons à l'aide d'un simple algorithme glouton. L'objectif de l'ordonnanceur est de minimiser le coût du placement de $z_i$ tâches sur $R_i$ répliques de la fonction $i$ pour $y_a$ invocations de l'application $a$ (équation~\ref{eq:herocache-objective-scheduling}), sous contrainte d'un ensemble fini de répliques de fonction $R_{i}$ (équation~\ref{eq:herocache-constraint-scheduling}) précédemment déployée par l'autoscaler. Nous supposons que les applications sont toujours exécutées jusqu'au bout et que les nœuds ne tombent pas en panne ; il n'y a donc pas de coût associé aux migrations de tâches ou aux nouvelles tentatives.

\begin{equation}
    \min \sum_{a = 0}^{A} y_a \cdot schedCost_{a}
\label{eq:herocache-objective-scheduling}
\end{equation}

\begin{equation}
    \text{s. t.} \, \forall a \sum_{i = 0}^{F_a} z_i \leq \sum_{i = 0}^{F_a} R_{i}
\label{eq:herocache-constraint-scheduling}
\end{equation}

Comme la plateforme fonctionne à la granularité des fonctions, le coût d'ordonnancement d'une application $a$ est la somme du coût d'ordonnancement de sa chaîne de fonctions (équation~\ref{eq:herocache-scheduling-cost-app}).

\begin{equation}
    schedCost_{a} = \, \sum_{i = 0}^{A} schedCost^{{{f}_{i}}}_{a}
\label{eq:herocache-scheduling-cost-app}
\end{equation}

Chaque fonction ordonnancée dans la chaîne a un coût associé, calculé pour chaque placement possible sur une réplique existante. Nous avons conçu un modèle de coût (équation~\ref{eq:herocache-scheduling-cost-function}) pour le placement des tâches nécessaires au traitement d'une requête utilisateur pour une application.

\begin{equation}
    schedCost_{{f}_{{i}_{N, P}}} = \, k_{QP} \cdot QP_{{f}_{N, P}} + k_{EC} \cdot {EC}_{{f}_{N, P}} + k_{TC} \cdot TC_{{f}_{N, P}}
\label{eq:herocache-scheduling-cost-function}
\end{equation}

Ce modèle est composé de trois éléments, dont nous devons minimiser la somme :

\begin{itemize}
    \item La \textit{pénalité sur qualité de service} $QP$ est encourue par le fournisseur de services lorsqu'une requête utilisateur n'est pas traitée en temps voulu. Il s'agit d'une valeur booléenne qui détermine si, en raison d'un placement donné, l'application ne respectera pas son échéance ;
    \item La \textit{consommation d'énergie} $EC$ traduit la consommation d'énergie dynamique induite par l'exécution de la fonction. Plus $EC$ est élevée, plus le coût est important ;
    \item La \textit{consolidation des tâches} $TC$ décrit l'utilisation des ressources pour un placement donné. Plus $TC$ est faible, plus la file d'attente de la réplique est proche de son seuil de concurrence, ce qui maximise l'utilisation du matériel.
\end{itemize}

L'objectif global du modèle de coût est de placer les requêtes utilisateur dans les répliques de fonctions au coût le plus bas possible, c'est-à-dire en évitant les pénalités subies par le fournisseur de services en cas de dépassement de l'échéance de l'application au regard des besoins définis par l'utilisateur ; en utilisant les plateformes d'exécution les moins gourmandes en énergie possible ; et en appliquant un ratio d'utilisation élevé pour les ressources allouées à chaque fonction. Nous décrivons chaque partie de l'équation~\ref{eq:herocache-scale-cost-function} dans les paragraphes suivants. Chaque composante de l'équation est pondérée pour permettre un réglage flexible ; les valeurs que nous avons choisies pour le déploiement de l'application d'IDS sont spécifiées dans la partie consacrée à l'évaluation (section~\ref{section:herocache-evaluation}).

\textbf{Pénalité de qualité de service}. L'ordonnanceur sélectionne les tâches entrantes en fonction de leur \textbf{échéance la plus proche} (\textit{EDF}, pour \textit{Earliest Deadline First}), en s'appuyant sur les métadonnées de la fonction pour calculer un temps d'exécution dans le pire cas noté $WET$ (équation~\ref{eq:herocache-task-wet}). La requête utilisateur est associée à un niveau de qualité de service qui définit un facteur de ralentissement variable $QD$ appliqué au temps d'exécution de l'application. Ces deux éléments permettent d'identifier l'échéance pour la requête.

\begin{equation}
    \forall \, (N, P), \, WET_{f} = \, \max ET_{f_{N, P}}
\label{eq:herocache-task-wet}
\end{equation}

Nous pouvons prédire la pénalité de l'application en additionnant le temps total prévu pour ses tâches et en le comparant à l'échéance de l'application (somme des échéances des fonctions), comme montré dans l'équation~\ref{eq:herocache-scheduling-penalty}. Nous réutilisons l'équation~\ref{eq:herocache-total-time-function} pour calculer le temps total d'exécution d'une fonction ; cependant, ici, les composantes $RT$ et $CST$ seront nulles car la réplique a déjà été initialisée par l'autoscaler lors de l'allocation. $WT$ sera égal à la somme des temps d'exécution des tâches de priorité supérieure actuellement en file d'attente sur la réplique.

\begin{equation}
   QP_{a} = \, \sum_{i = 0}^{F_a} TT_{{f}_{{i}_{N, P}}} > \sum_{i = 0}^{F_a} WET_{f_{i}} \cdot QD_{a}
\label{eq:herocache-scheduling-penalty}
\end{equation}

En prenant en compte le temps de stockage dans le coût d'ordonnancement, nous cherchons à inciter l'ordonnanceur à placer les tâches aussi près que possible des données sur lesquelles elles opèrent. Pour éviter de saturer le stockage local des nœuds, la plateforme procède au nettoyage des données intermédiaires dès que l'application a terminé son exécution, \textit{i.e.} lorsque la dernière fonction de la chaîne retourne sa valeur.

\textbf{Consommation d'énergie}. ${EC}_{{f}_{N, P}}$ correspond à la consommation d'énergie dynamique générée par cette configuration d'ordonnancement. Elle est liée au temps d'exécution de la fonction. Les résultats des mesures hors-ligne sont utilisés pour déterminer ce terme.

\textbf{Consolidation des tâches}. Nous voulons que les files d'attente des répliques de fonctions atteignent leur seuil de concurrence : le pire cas est d'avoir une file d'attente vide, ce qui signifie que des ressources matérielles auraient été inutilement allouées. Nous voulons également empêcher les files d'attente de répliques de croître trop rapidement au-delà de ce seuil, car cela pourrait entraîner des violations de la qualité de service en raison de longs temps d'attente.

Nous commençons par établir le ratio d'\textit{utilisation de la plateforme} $PU$ de chaque réplique pour la fonction que nous essayons d'ordonnancer (équation~\ref{eq:herocache-platform-usage}) : plus la longueur de la file d'attente de la réplique $Q$ est proche de son seuil de concurrence ($threshold$ dans l'équation), plus le score est faible.

\begin{equation}
    PU_{f_{N, P}} = \frac{Q_{N, P}}{threshold_{f, P}}
\label{eq:herocache-platform-usage}
\end{equation}

Ensuite, nous calculons un score de consolidation des tâches $TC$ en appliquant une fonction exponentielle à $PU$ (équation~\ref{eq:herocache-task-consolidation}). Ainsi, $TC$ est le plus faible pour les placements dans les répliques inactives, et ce coût augmente fortement au fur et à mesure que les files d'attente se remplissent, ce qui conduit l'ordonnanceur à donner la priorité aux placements sur les répliques vides et à éviter les répliques où la file d'attente des requêtes est saturée.

\begin{equation}
    TC_{{f}_{N, P}} = \, exp(PU_{f_{N, P}})
\label{eq:herocache-task-consolidation}
\end{equation}

\section{Évaluation}
\label{section:herocache-evaluation}

Cette section présente notre méthodologie d'évaluation et les résultats obtenus dans un scénario de déploiement d'IDS sur des dispositifs edge. L'évaluation se fait en deux phases : nous comparons HeROcache à plusieurs stratégies de référence, puis nous évaluons l'impact de chacun de ses composants (autoscaler et ordonnanceur) pris séparément.

\begin{figure*}[t]
    \center
    \subfloat[Consolidation\label{figure:herocache-evaluation-full-unused-nodes}]{
        \includegraphics[width=0.155\linewidth]{5_Chapitre5/figures/eval/2-unused-nodes.png}
    }
    \subfloat[QoS\label{figure:herocache-evaluation-full-penalty}]{
        \includegraphics[width=0.155\linewidth]{5_Chapitre5/figures/eval/3-penalty-proportions.png}
    }
    \subfloat[Energy\label{figure:herocache-evaluation-full-energy-consumption}]{
        \includegraphics[width=0.155\linewidth]{5_Chapitre5/figures/eval/6-energy-consumption.png}
    }
    \subfloat[Consolidation\label{figure:herocache-evaluation-components-unused-nodes}]{
        \includegraphics[width=0.155\linewidth]{5_Chapitre5/figures/eval-components/2-unused-nodes.png}
    }
    \subfloat[QoS\label{figure:herocache-evaluation-components-penalty}]{
        \includegraphics[width=0.155\linewidth]{5_Chapitre5/figures/eval-components/3-penalty-proportions.png}
    }
    \subfloat[Energy\label{figure:herocache-evaluation-components-energy-consumption}]{
        \includegraphics[width=0.155\linewidth]{5_Chapitre5/figures/eval-components/6-energy-consumption.png}
    }
    \caption{Evaluation -- Comparison against baselines (a-c) and impact of individual components (d-f)}
    \label{figure:herocache-evaluation}
\end{figure*}

\subsection{Protocole expérimental}

\textbf{Métadonnées de caractérisation hors-ligne}. Pour évaluer notre contribution, nous avons effectué des mesures pour trois applications d'IDS (voir section~\ref{section:herocache-characterization-workloads}). Ces applications consistent en différentes combinaisons de fonctions de prétraitement et d'inférence, mises en œuvre sur du matériel hétérogène (voir section~\ref{section:herocache-characterization-platforms}). Ces métadonnées ont servi d'entrée à un simulateur à évènements discrets, HeROsim (voir chapitre~\ref{chapter:herosim}), utilisé pour estimer la pertinence de nos stratégies.

\textbf{Orchestration en ligne}. Nous avons généré des scénarios synthétiques en modélisant les requêtes utilisateur comme un processus de Poisson, suivant une distribution uniforme entre les invocations d'applications, comme couramment pratiqué dans l'état de l'art~\cite{9928755}. En modifiant le paramètre $\lambda$ du processus de Poisson, nous pouvons générer diverses traces avec différents taux de requêtes par seconde (\textit{RPS}). Nous avons envisagé un scénario avec 10 nœuds edge communiquant via un lien 4G (LTE). La bande passante pour la 4G LTE dépend de divers facteurs allant de la couverture de l'antenne à la qualité de service du fournisseur, en passant par la qualité du récepteur. Nous avons choisi d'utiliser une valeur représentative à 100 Mbps (12,5~MB/s). Les paquets TCP à analyser ont une taille de 1,5~KB et sont envoyés par lots de 100 aux applications d'IDS. Cela donne un taux de 83~RPS dans notre scénario, pour 10 minutes de requêtes utilisateur.

Les pondérations pour les décisions de mise à l'échelle automatique (équation~\ref{eq:herocache-scale-cost-function}) ont été fixées à $k_{CP} = \frac{3}{8}$, $k_{TT} = \frac{3}{8}$, $k_{EC} = \frac{1}{8}$ et $k_{HP} = \frac{1}{8}$ dans le but de privilégier une latence faible à basse consommation d'énergie. Les pondérations pour les décisions d'ordonnancement (équation~\ref{eq:herocache-scheduling-cost-function}) ont été fixées à $k_{QP} = \frac{2}{3}$, $k_{EC} = \frac{0.5}{6}$ et $k_{TC} = \frac{1.5}{6}$ pour favoriser un faible nombre d'échéances manquées. Nous utilisons des valeurs inspirées de~\cite{herofake} afin d'être comparables.

Pour éviter une forme d'"emballement" (on parle de \textit{thrashing}) dans lequel les répliques sont créées et détruites en boucle lorsque la concurrence dans le système est très proche du seuil de concurrence, l'autoscaler applique un temps de maintien en vie (\textit{keep-alive delay}) faible qui empêche la destruction d'une réplique récemment allouée. Nous avons fixé ce temps de maintien en vie à 30 secondes, ce qui est la valeur par défaut dans Knative.

Dans nos expériences, nous nous autorisons à évaluer autoscalers et ordonnanceurs séparément afin de mieux comprendre leur comportement. Nous avons évalué différentes combinaisons pour montrer quelle composante de chaque politique est pertinente pour relever les différents défis de notre problème. Nous avons mis en œuvre trois autoscalers dans notre simulateur :

\begin{itemize}
    \item HeROcache (HRC) -- Notre politique de mise à l'échelle automatique repose sur la mise en cache d'images de fonctions sur les nœuds edge et tente d'anticiper la mise en cache des images de fonctions pour satisfaire les dépendances du DAG localement lors du déploiement d'applications ;
    \item HeROfake (HRO)~\cite{herofake} -- Applique une politique similaire à HRC, mais ne tient pas compte des coûts de stockage : l'ordonnanceur n'utilise pas le cache d'image local du nœud lors de l'instanciation des répliques de fonctions, et n'effectue pas non plus la recherche préalable des images de fonctions en fonction du DAG de leur application ;
    \item Knative (KN)~\cite{sureshENSUREEfficientScheduling2020} -- Nous avons modélisé le comportement de l'autoscaler de Knative du mieux que nous avons pu, en nous appuyant sur la documentation disponible au moment de cette étude~\cite{knative-autoscaling}. Il déploie les répliques de fonctions sur le nœud le plus disponible, \textit{i.e.} il effectue un équilibrage de charge.
\end{itemize}

En plus de ces autoscalers, nous avons utilisé quatre ordonnanceurs :

\begin{itemize}
    \item HeROcache (HRC) -- Notre politique d'ordonnancement sélectionne les requêtes utilisateur entrantes en fonction de leur échéance la plus proche afin de maximiser la qualité de service. Notre ordonnanceur tient compte de la latence de communication prévue et sélectionne une réplique en fonction des exigences de temps de réponse dictées par la requête ;
    \item HeROfake (HRO)~\cite{herofake} -- Applique une politique similaire à HRC, mais ne tient pas compte des coûts de stockage : l'ordonnanceur ne prédit pas la latence des communications dans le DAG de l'application ;
    \item Knative (KN)~\cite{knative} -- Knative considère les plateformes d'exécution comme homogènes et ne tient pas compte de la QoS requise par les utilisateurs. Les répliques sont triées en fonction du nombre de requêtes en attente ; la réplique ayant la file d'attente la plus courte est sélectionnée ;
    \item Bin-Packing First Fit (BPFF)-- Les tâches sont consolidées sur un nombre minimum de répliques. L'ordonnanceur place les requêtes sur une première réplique jusqu'à ce que celle-ci atteigne son seuil de concurrence. BPFF est susceptible d'être la politique d'ordonnancement mise en œuvre pour AWS Lambda~\cite{wangPeekingCurtainsServerlessb} ;
    \item Random Placement (RP) -- Les tâches sont ordonnancées sur une réplique sélectionnée de manière aléatoire.
\end{itemize}

Le nom de chaque scénario se compose de deux parties séparées par un tiret. La première partie correspond à la politique d'autoscaler ; la deuxième partie correspond à la politique d'ordonnancement (par exemple, HRC-KN signifie que nous avons utilisé l'autoscaler HeROcache avec l'ordonnanceur Knative).

Nous avons conçu une évaluation des performances en deux étapes : \\
(1) \textbf{Comparaison avec les politiques de référence} : nous comparons HeROcache complet (HRC-HRC) à : (1) Knative complet (KN-KN), (2) HeROfake complet (HRO-HRO), (3) l'autoscaler Knative avec l'ordonnanceur BPFF (KN-BPFF), (4) l'autoscaler Knative avec l'ordonnanceur RP (KN-RP). \\
(2) \textbf{Impact des composants individuels sur les performances globales} : nous discutons de l'impact respectif de l'autoscaler et de l'ordonnanceur dans différentes stratégies : (1) l'autoscaler HeROcache avec l'ordonnanceur HeROfake (HRC-HRO), et (2) l'autoscaler HeROfake avec l'ordonnanceur HeROcache (HRO-HRC), en les comparant aux versions complètes de HeROcache et HeROfake.

Nous évaluons HeROcache sur la base de trois mesures : (1) le nombre de nœuds inutilisés dans l'infrastructure, qui mesure le niveau de consolidation atteint ; (2) les pénalités sur qualité de service, qui expriment la capacité de notre stratégie à répondre aux exigences des utilisateurs ; (3) la consommation d'énergie, qui est un défi important à l'edge avec des ressources limitées.

\subsection{Analyse des résultats}

\subsubsection{Comparaison aux politiques de base}

\textbf{Consolidation des tâches}. La figure~\ref{figure:herocache-evaluation-full-unused-nodes} montre que notre combinaison d'autoscaler et d'ordonnanceur réalise la meilleure consolidation des tâches, en utilisant seulement 20\% de l'infrastructure edge pour l'exécution du scénario. Knative se comporte comme prévu, en répartissant la charge sur l'ensemble de l'infrastructure. On note que BPFF sous Knative produit des résultats légèrement différents : comme les files d'attente des tâches sont maximisées, l'autoscaler n'a pas besoin d'allouer autant de répliques. Dans ce scénario, si les nœuds edge inutilisés étaient mis hors tension au lieu de rester inactifs, notre stratégie permettrait au fournisseur de services d'économiser près de 100 Wh (soit 80\% de l'énergie statique et plus de 83\% de l'énergie totale) en mettant hors tension 80\% de l'infrastructure, tout en garantissant le temps de réponse des applications pour 72\% des requêtes utilisateur.

\textbf{Qualité de service}. La figure~\ref{figure:herocache-evaluation-full-penalty} illustre la pertinence de la prise en compte de l'hétérogénéité des ressources. En effet, notre politique parvient à maintenir les violations de la qualité de service à 27,5\% des requêtes tout en laissant 80\% de l'infrastructure inutilisée. Knative viole un peu plus de 30\% de la QoS des requêtes utilisateur tout en répartissant la charge sur tous les nœuds edge disponibles, ce qui est contre-intuitif. Cela s'explique par les dépendances entre les tâches que Knative ne prend pas en compte. En conséquence, les tâches ont tendance à communiquer \textit{via} le réseau, sur un support de stockage lent. Ainsi, bien que les tâches dans Knative passent en moyenne moins de temps en file d'attente, elles présentent toujours une latence plus élevée que dans HeROcache. Lors de l'utilisation de la politique BPFF, les violations atteignent presque 70\% : dans cette situation, les files d'attente des répliques sont trop longues pour que les requêtes puissent être traitées dans les délais impartis. À titre de comparaison, Knative utilisant l'ordonnanceur RP maintient les violations de la QoS autour de 50\%. HeROfake génère 39\% de violations de la qualité de service.

Notre politique maintient la proportion de démarrages à froid en dessous de 0,011\% des requêtes utilisateur, alors que Knative souffre de 4 fois plus de démarrages à froid. Dans HeROcache, le cache d'images local au nœud est utilisé dans 33\% des initialisations de fonctions, ce qui réduit les délais d'initialisation de 17,6\%.
Avec HeROcache, 30\% des tâches parviennent à communiquer par le biais du stockage local au niveau du nœud, ce qui accélère l'exécution de l'application en réduisant la latence des communications de 88,4\%.

\textbf{Consommation d'énergie}. La figure~\ref{figure:herocache-evaluation-full-energy-consumption} montre que HeROcache parvient à réduire la consommation d'énergie dynamique d'un tiers : avec un \textit{makespan} de 1505 secondes pour le scénario, l'infrastructure consomme 0,0088 kWh, contre 0,0266 kWh pour 2193 secondes de temps d'exécution sous Knative. Non seulement la stratégie de consolidation de HeROcache permettrait d'appliquer des politiques de mise hors tension susceptibles de réduire considérablement les besoins en énergie statique pour l'exécution d'applications d'IDS à l'edge, mais en sélectionnant des plateformes d'exécution adéquates, elle réduit également la consommation globale de l'infrastructure. HeROfake consomme le plus d'énergie (0,31 kWh) en raison d'un temps d'exécution beaucoup plus long pour le scénario.

\subsubsection{Impact des composants individuels}

\textbf{Consolidation des tâches}. La figure~\ref{figure:herocache-evaluation-components-unused-nodes} montre que les stratégies qui ne tiennent pas compte des coûts de stockage ne parviennent pas à consolider les tâches aussi bien que HeROcache : HRO-HRC et HRO-HRO utilisent respectivement 80\% et 70\% de l'infrastructure. Nous expliquons ces résultats de la manière suivante : les dépendances n'étant pas satisfaites à temps, la charge mesurée pour les différentes fonctions déployées continue d'augmenter, ce qui conduit l'autoscaler à incrémenter le nombre de répliques, enrôlant ainsi plus de nœuds pour la durée du scénario.

\textbf{Qualité de service}. La figure~\ref{figure:herocache-evaluation-components-penalty} illustre la conséquence du point précédent : les pénalités sur qualité de service sont plus élevées avec un autoscaler qui ne tient pas compte des délais introduits par la récupération des images de fonctions et les communications entre les fonctions. Bien que HRO-HRC soit conscient de l'hétérogénéité du matériel et des requêtes, il termine tout de même avec 37,9\% des applications qui ne respectent pas leur échéance.

\textbf{Consommation d'énergie}. La figure~\ref{figure:herocache-evaluation-components-energy-consumption} montre que, bien que HRO-HRC alloue 70\% de l'infrastructure, il parvient à maintenir une consommation d'énergie presque aussi faible que HRC-HRC. Cela s'explique par le fait qu'il a choisi les nœuds les moins gourmands en énergie, au prix de pénalités qu'il ne pouvait pas prévoir puisqu'il n'est pas conscient des enjeux liés au stockage.

\textbf{Note sur la complexité.} HeROcache utilise une technique d'optimisation gloutonne comparable à HeROfake. Dans HeROcache, la complexité est bornée par le nombre d'applications $A$, leur taille $f_{a}$ et la taille de l'infrastructure $N$ (équation~\ref{eq:herocache-complexity-autoscaler}) : dans le pire des cas, où toutes les ressources sont disponibles, l'autoscaler parcourt l'ensemble de l'infrastructure $N$ pour évaluer chaque nœud en vue de la création de répliques.

\begin{equation}
    \mathcal{O}_{autoscaling}(A \cdot f_{a} \cdot N)
\label{eq:herocache-complexity-autoscaler}
\end{equation}

Comme l'ordonnanceur travaille avec les répliques de fonctions déjà créées $R_{f}$, sa complexité est moindre (équation~\ref{eq:herocache-complexity-scheduler}).

\begin{equation}
    \mathcal{O}_{scheduling}(A \cdot f_{a} \cdot R_{f})
\label{eq:herocache-complexity-scheduler}
\end{equation}

Comme notre étude de cas actuelle implique un sous-ensemble limité de fonctions d'IDS avec un nombre raisonnable de nœuds edge, la complexité de l'algorithme n'a pas posé de problème. Toutefois, ce coût devrait être pris en compte pour des déploiements plus larges, pour différentes études de cas. Ce coût n'a pas été mesuré en simulation. Toutefois, comme l'autoscaler fonctionne de manière périodique, sa fréquence pourrait être ajustée en fonction des besoins et des contraintes du fournisseur de services. L'ordonnanceur est appelé au moment des requêtes utilisateur et pourrait être réparti sur les nœuds si la charge est trop lourde à gérer, au prix d'une vue d'ensemble instantanée de l'infrastructure.

\section{Travaux connexes}
\label{section:herocache-sota}

Des travaux antérieurs se sont concentrés sur les plateformes de mise à l'échelle automatique pour le déploiement de tâches de courte durée, comprises dans des applications présentant des modèles de charge imprévisibles (voir tableau~\ref{table:herocache-sota}).

Parmi ces travaux, \cite{smithFaDOFaaSFunctions2022} propose un orchestrateur conscient des dépendances de données, mais ne tenant pas compte de l'effet boule de neige des retards dans les chaînes de fonctions. \cite{zhangFIRSTExploitingMultiDimensional2023} ne prend pas en charge l'ordonnancement de ces chaînes de fonctions.
Toutes ces contributions considèrent une infrastructure homogène~\cite{bhasiCypressInputSizesensitive2022, zijunFassflowEfficient2022, smithFaDOFaaSFunctions2022, zhangFIRSTExploitingMultiDimensional2023, abdiPaletteLoadBalancing2023}. Cela n'est pas représentatif de notre cas d'utilisation, dans lequel les dispositifs edge sont très hétérogènes. HeROfake~\cite{herofake} exploite l'hétérogénéité matérielle dans sa politique d'orchestration, mais n'intègre pas les dépendances inter-fonctions ni la mise en cache des images dans son modèle de coût. Elle a été choisie à des fins d'évaluation, pour souligner la nécessité de prendre en compte ces coûts.
Certaines de ces contributions optimisent la consommation d'énergie au niveau de l'autoscaler \cite{bhasiCypressInputSizesensitive2022, zhangFIRSTExploitingMultiDimensional2023}. Toutefois, elles se concentrent sur la partie dynamique de la consommation d'énergie : elles ne tiennent pas compte de l'impact possible de la consolidation sur la consommation d'énergie statique. Nous soutenons que les fournisseurs de services devraient chercher à consolider les tâches afin de mettre hors tension le plus grand nombre de nœuds possible, ce qui réduirait considérablement les besoins énergétiques globaux de leurs infrastructures.
Dans \cite{fuerstIluvatarFastControl2023}, les auteurs ont étudié les différents coûts induits par la complexité de l'orchestration serverless. Cet élément n'a pas été pris en compte dans notre étude, car nous visons une infrastructure edge de taille limitée pour le déploiement d'applications d'IDS bien identifiées.

\section{Conclusion et perspectives}
\label{section:herocache-conclusion}

Dans ce chapitre, nous avons présenté une politique d'allocation et d'ordonnancement pour le serverless à l'edge. Cette politique cherche à optimiser le déploiement d'applications sensibles au temps pour la qualité de service sur des dispositifs à énergie limitée. En tirant parti de la caractérisation de la charge de travail, de l'hétérogénéité du matériel et des dispositifs de stockage locaux sur les nœuds edge, HeROcache consolide les applications et parvient à réduire les délais d'initialisation moyens de 17,6\% et les délais de communication de 88,4\%. Cela permet de réduire la consommation d'énergie statique de la plateforme de 80\% tout en maintenant moins de 28\% de violations de la qualité de service. Nous prévoyons de généraliser l'approche HeROcache pour des études de cas incluant plusieurs types d'applications sur des infrastructures plus importantes, à l'edge ou dans le cloud. Dans ce cas, des stratégies d'apprentissage automatique ou des métaheuristiques pourraient être utilisées à des fins de passage à l'échelle.


\clearemptydoublepage
\chapter{HeROsim : Élaborer et évaluer des politiques d'orchestration serverless pour le cloud privé}
\label{chapter:herosim}

\section{Introduction}
\label{section:herosim-introduction}

Les deux chapitres précédents ont abordé un ensemble de défis qui doivent être relevés par les fournisseurs de services qui souhaitent déployer des applications sur une plateforme serverless optimisée pour la \gls{QoS} et économe en énergie. Le chapitre~\ref{chapter:herofake} traite en particulier de la surcharge de latence causée par l'allocation dynamique du matériel pour les nouvelles instances de fonction, appelée délai de démarrage à froid~\cite{vahidiniaColdStartServerless2020}. Un autre problème récurrent, introduit dans le chapitre~\ref{chapter:herocache}, est celui de la prise en compte de la localité des données~\cite{yuFollowingDataNot, chikhaouiMultiobjectiveOptimizationData2021a} par la plateforme pour éviter une dégradation considérable des performances, en particulier dans le cas d'applications conçues comme des chaînes de fonctions qui communiquent entre elles.

Pour surmonter de tels défis, nous avons conçu des politiques d'orchestration qui guident les décisions d'allocation de ressources et d'ordonnancement de requêtes, dans le but de minimiser les coûts, selon des objectifs d'optimisation définis par le fournisseur de services. Il est nécessaire d'évaluer la pertinence de ces politiques dans divers cas d'utilisation ; différentes mesures peuvent être utiles pour cette évaluation, par exemple la durée totale de complétion pour un scénario donné, le temps de réponse des fonctions, la consommation d'énergie dynamique et statique dans la plateforme, etc.

\boitemagique{Question 3 (\textbf{QR3})}{
    Du point de vue d'un fournisseur de services pour le cloud, comment évaluer et comparer l'impact sur la qualité de service de différentes politiques d'allocation de ressources et d'ordonnancement de tâches dans le modèle serverless ?
}

Il y a deux façons possibles d'évaluer si les politiques que nous concevons ont un impact positif sur ces mesures : la simulation (estimation) ou la mise en œuvre (mesure). Ces deux approches présentent un compromis entre temps, coût et précision ; choisir la simulation plutôt que la mise en œuvre permet de parcourir l'espace du problème en itérant rapidement sur des solutions possibles. En particulier, la modélisation à événements discrets nous permet d'explorer des solutions en représentant les différents composants de la plateforme avec des processus qui modélisent l'état du système et son évolution dans le temps.

L'orchestration est un processus dynamique : chaque décision crée un nouvel état pour le système, ce qui conduit à une explosion combinatoire. De plus, les plateformes serverless sont des logiciels génériques qui exposent de nombreux paramètres sur lesquels le fournisseur peut agir. Ces paramètres peuvent avoir un impact sur la latence des requêtes des utilisateurs, le débit de la plateforme, l'utilisation des ressources et la consommation d'énergie de l'infrastructure. Comme il est coûteux d'expérimenter en production, nous soutenons que les outils de simulation sont essentiels pour les fournisseurs de cloud serverless privé qui cherchent à optimiser ces mesures de qualité de service en fonction de leurs propres objectifs.

Les outils de simulation des politiques d'orchestration dans le cloud serverless privé doivent permettre de tracer les événements d'allocation des ressources et de placement des tâches à la \textbf{granularité la plus fine}, afin de comprendre l'impact des politiques sur les métriques de performance. Pour représenter des cas d'utilisation réalistes, le logiciel doit offrir la possibilité de modéliser des applications qui présentent des \textbf{dépendances de données et temporelles} entre les exécutions de tâches. En outre, le simulateur doit prendre en charge l'\textbf{hétérogénéité} globale dans le cloud : d'une part, les centres de données sont constitués de divers matériels présentant différents niveaux de coût et de performance ; d'autre part, il existe un large éventail d'utilisateurs ayant des exigences différentes en matière de qualité de service. Enfin, le simulateur devrait pouvoir \textbf{rejouer les traces d'exécution} pour différentes politiques d'orchestration et \textbf{comparer les résultats} en termes de métriques de qualité de service pour chaque politique évaluée.

\begin{table*}[!ht]
    \centering
    \caption{État de l'art des outils de simulation pour l'orchestration des ressources et des charges de travail dans le cloud.}
    \resizebox{\linewidth}{!}{
        \begin{tabular}{lYYYYYYYYY}
        \toprule
        & Serverless & Déploiements & Chaînes de fonctions & Hétérogénéité matérielle & \gls{QoS} par requête & Énergie & Visualisation \\
        \cmidrule(lr){2-2}\cmidrule(lr){3-3}\cmidrule(lr){4-4}\cmidrule(lr){5-5}\cmidrule(lr){6-6}\cmidrule(lr){7-7}\cmidrule(lr){8-8}
        CloudSim~\cite{calheiros_cloudsim_2011} & \xmark & Public, privé, hybride & \xmark & \cmark & \xmark & \cmark & \xmark \\
        CloudSimSC~\cite{mampage_cloudsimsc_2023} & \cmark & Public, privé, hybride & \xmark & \cmark & \xmark & \cmark & \xmark \\
        CloudAnalyst~\cite{wickremasinghe_cloudanalyst_2010} & \xmark & Public, privé, hybride & \xmark & \cmark & \xmark & \cmark & \cmark \\        DFaaSCloud~\cite{jeonCloudSimExtensionSimulatingDistributed2019} & \cmark & Hybride multi-strates & \xmark & \xmark & \cmark & \xmark & \cmark \\
        ElasticSim~\cite{cai_elasticsim_2017} & \xmark & Public & \cmark & \xmark & \xmark & \xmark & \cmark \\
        GridSim~\cite{buyyaGridSimToolkitModeling2002} & \xmark & Grille & \xmark & \cmark & \cmark & \xmark & \cmark \\
        iCanCloud~\cite{nunez_icancloud_2012} & \xmark & Public & \xmark & \xmark & \xmark & \xmark & \cmark \\
        iFogSim2~\cite{mahmudIFogSim2ExtendedIFogSim2021} & \xmark & Edge, Fog & \xmark & \cmark & \xmark & \cmark & \xmark \\
        OpenDC 2.0~\cite{mastenbroekOpenDCConvenientModeling2021} & \cmark & Public, privé, hybride & \cmark & \cmark & \xmark & \cmark & \cmark \\
        SimFaaS~\cite{mahmoudiSimFaaSPerformanceSimulator2021} & \cmark & Public & \xmark & \xmark & \xmark & \cmark & \cmark \\
        \textbf{HeROsim} & \cmark & Privé & \cmark & \cmark & \cmark & \cmark & \cmark \\
        \bottomrule
        \end{tabular}
    }
    \label{table:herosim-sota}
\end{table*}

Des travaux antérieurs ont proposé différents outils de simulation intéressants pour le cloud, que nous présentons au regard de leurs caractéristiques dans le tableau~\ref{table:herosim-sota} ; le chapitre~\ref{chapter:sota} donne plus de détails en section~\ref{section:sota-herosim}. Certains de ces travaux ne ciblent pas le paradigme serverless~\cite{calheiros_cloudsim_2011, wickremasinghe_cloudanalyst_2010, cai_elasticsim_2017, buyyaGridSimToolkitModeling2002, nunez_icancloud_2012, mahmudIFogSim2ExtendedIFogSim2021}. Parmi les simulateurs serverless, certains se concentrent sur les offres de cloud public~\cite{nunez_icancloud_2012, mahmoudiSimFaaSPerformanceSimulator2021}, principalement pour permettre de concevoir des stratégies hybrides où une partie des charges de travail est déléguée à un cloud public, moins coûteux. Pour le cloud privé, certains simulateurs ne prennent pas en compte la consommation d'énergie~\cite{jeonCloudSimExtensionSimulatingDistributed2019, cai_elasticsim_2017, buyyaGridSimToolkitModeling2002, nunez_icancloud_2012} ; d'autres considèrent une infrastructure homogène~\cite{jeonCloudSimExtensionSimulatingDistributed2019, nunez_icancloud_2012, mahmoudiSimFaaSPerformanceSimulator2021}. La plupart des simulateurs de plateformes serverless ne permettent pas de modéliser des applications composées de multiples fonctions avec des dépendances de données~\cite{calheiros_cloudsim_2011, mampage_cloudsimsc_2023, wickremasinghe_cloudanalyst_2010, jeonCloudSimExtensionSimulatingDistributed2019, buyyaGridSimToolkitModeling2002, nunez_icancloud_2012, mahmudIFogSim2ExtendedIFogSim2021}. Dans certaines études, la qualité de service ne peut pas être spécifiée à la granularité d'une requête utilisateur~\cite{calheiros_cloudsim_2011, mampage_cloudsimsc_2023, wickremasinghe_cloudanalyst_2010, cai_elasticsim_2017, nunez_icancloud_2012, mahmudIFogSim2ExtendedIFogSim2021, mastenbroekOpenDCConvenientModeling2021, mahmoudiSimFaaSPerformanceSimulator2021}.

Pour remédier à ces limites, nous proposons \textbf{HeROsim}~\footnote{\href{https://github.com/b-com/HeROsim}{https://github.com/b-com/HeROsim}} (pour \textbf{He}terogeneous \textbf{R}esources \textbf{O}rchestration \textbf{sim}ulator), un simulateur pour le cloud privé serverless, libre et open source (licence Apache 2.0~\footnote{\href{https://www.apache.org/licenses/LICENSE-2.0}{https://www.apache.org/licenses/LICENSE-2.0}}), présentant les caractéristiques suivantes :

\begin{itemize}
    \item Description fine des applications serverless sur du matériel hétérogène : les fonctions sont définies selon un ensemble de métadonnées sur leurs performances, flux de données, besoins en mémoire, consommation d'énergie ;
    \item Allocation dynamique des ressources matérielles et placement des requêtes utilisateur compatibles avec la qualité de service : HeROsim offre des points d'entrée aux utilisateurs pour mettre en œuvre leurs propres politiques de sélection des ressources, à la granularité d'une requête ;
    \item Évaluation des politiques d'orchestration en fonction de paramètres de qualité de service : HeROsim permet de comparer les résultats de différentes stratégies en termes de métriques couramment utilisées pour mesurer la qualité de service, tels que la latence et l'énergie.
\end{itemize}

La conception du logiciel suit l'architecture de référence des orchestrateurs de l'état de l'art tels que Google Knative~\footnote{\href{https://knative.dev}{https://knative.dev}} ou Apache OpenWhisk~\footnote{\href{https://openwhisk.apache.org}{https://openwhisk.apache.org}}. Ces orchestrateurs se composent de deux modules principaux, comme montré dans la figure~\ref{figure:herosim-platform} :

\begin{itemize}
    \item Un \textbf{autoscaler} qui détermine comment mettre à l'échelle automatiquement et de manière réactive les ressources matérielles disponibles dans l'infrastructure, en adéquation avec la charge sur les applications ;
    \item Un \textbf{ordonnanceur} qui détermine sur quelle réplique mettre en file d'attente les requêtes utilisateur pour une fonction donnée, étant données des exigences de qualité de service définies au niveau des requêtes.
\end{itemize}

Ce chapitre est organisé comme suit : la section~\ref{section:herosim-overview} donne un aperçu du fonctionnement d'une plateforme serverless ; la section~\ref{section:herosim-herosim} détaille la conception de HeROsim ; la section~\ref{section:herosim-case-study} montre comment HeROsim peut être exploité à travers deux cas d'utilisation tirés de nos publications ; enfin, nous concluons par les limites de la contribution et quelques perspectives pour de futurs travaux en section~\ref{section:herosim-conclusion}.

\section{Présentation générale de la plateforme simulée}
\label{section:herosim-overview}

\begin{figure*}[!ht]
    \centering
    \includegraphics[width=0.9\textwidth]{6_Chapitre6/figures/platform-complete.png}
    \caption{Vue de haut niveau d'une plateforme serverless, telle que modélisée dans HeROsim. L'autoscaler alloue des ressources matérielles pour les répliques de fonctions ; tandis que l'ordonnanceur place les requêtes utilisateur en file d'attente sur ces répliques.}
\label{figure:herosim-platform}
\end{figure*}

Dans les modèles de service traditionnels pour le cloud (voir chapitre~\ref{chapter:context}), les clients réservent généralement des ressources auprès d'un fournisseur de services. Il s'agit d'un sous-ensemble virtualisé de ressources matérielles hétérogènes disponibles sur des serveurs appelés nœuds. Une fois la réservation effectuée, la plateforme cloud offre un accès à distance aux clients, qui sont responsables du déploiement de leurs applications, et facturés en fonction de la quantité de ressources qu'ils ont réservée~\cite{Lannurien2023}.

Dans le paradigme serverless, les clients commencent par mettre en ligne le code de leur application dans un registre du côté du fournisseur, comme illustré par la figure~\ref{figure:herosim-platform}. Le fournisseur alloue des ressources qui sont automatiquement mises à l'échelle en fonction de la charge sur l'application. Pour que ce mécanisme fonctionne, les applications sont divisées en petites unités d'exécution appelées \textbf{fonctions}. Ces fonctions sont dites sans état : si elles produisent des données de sortie, celles-ci doivent être conservées dans un stockage persistant~\cite{yuFollowingDataNot}.

Dans les modèles de service traditionnels, les ressources sont mises à l'échelle sur deux dimensions : horizontalement (nouvelles instances d'application créées sur d'autres nœuds) et verticalement (ressources supplémentaires allouées aux instances existantes). Dans les plateformes serverless, les ressources sont mises à l'échelle horizontalement : les variations de charge sur les applications sont absorbées par l'ajout de nouvelles instances des fonctions, appelées \textbf{répliques}, et par leur suppression lorsqu'elles ne sont plus nécessaires. Une réplique peut être créée pour chaque requête utilisateur, ou réutilisée pour plusieurs requêtes. On peut considérer les répliques de fonctions comme des files d'attente de requêtes ayant des capacités différentes.

Ces répliques sont créées par l'\textbf{autoscaler}. Pour gérer le nombre de répliques déployées pour chaque fonction, il existe trois stratégies principales : une stratégie par requête, une stratégie par niveau de concurrence, et une stratégie par métriques~\cite{mahmoudiSimFaaSPerformanceSimulator2021}. Dans la première stratégie, chaque requête entrante est traitée par une réplique inactive. Si aucune réplique inactive n'est disponible, une nouvelle instance de la fonction est créée ; le nombre de répliques pour chaque fonction dépend donc directement du nombre de requêtes à traiter à un instant donné. Dans l'autoscaling basé sur la concurrence, chaque réplique peut mettre en file d'attente plusieurs requêtes utilisateur pour les traiter séquentiellement ; le nombre de répliques dépend du nombre de requêtes en attente pour chaque fonction, rapporté à un seuil de concurrence prédéfini~\cite{herofake}. Dans l'autoscaling basé sur les métriques, le nombre de répliques déployées dépend de divers objectifs, tels que le taux de requêtes par seconde (\gls{RPS}) à atteindre ; pour ce faire, l'autoscaler doit avoir une vue sur les performances et l'état global du système.

Les répliques peuvent se trouver dans trois états différents : initialisation, exécution et inactivité~\cite{SchleierSmith2021WhatSC}. Lorsqu'une réplique de fonction vient d'être créée, elle est en état d'initialisation : la plateforme instancie son environnement d'exécution, extrait le code de la fonction depuis un registre distant et le met éventuellement en cache sur le nœud de déploiement, puis commence à exécuter la fonction. Lorsque la réplique traite les requêtes utilisateur, elle est en état d'exécution. Dans le cas contraire, la réplique est inactive et peut être supprimée, ou gardée en vie dans un état prêt à traiter d'éventuelles futures requêtes. Lorsqu'une réplique est supprimée, les ressources matérielles qui lui avaient été allouées sont libérées : la création d'une nouvelle réplique pour traiter une requête utilisateur entrante entraîne un \textbf{démarrage à froid}. Les orchestrateurs adoptent diverses politiques pour atténuer ce problème, allant de l'application d'une période de maintien en vie sur les répliques de fonctions pour éviter de les détruire trop tôt, à l'allocation proactive de ressources pour initialiser des répliques en avance de phase. L'orchestrateur fait alors un compromis entre utilisation des ressources et performances.

Enfin, les requêtes utilisateur sont attribuées aux répliques de fonctions disponibles par l'\textbf{ordonnanceur} qui met en œuvre différentes stratégies : par exemple, \gls{AWS} Lambda\footnote{\href{https://aws.amazon.com/en/lambda/}{https://aws.amazon.com/en/lambda/}} utilise un algorithme de \textit{bin-packing}, tandis qu'une plateforme open source telle que Knative met en œuvre une politique d'équilibrage de la charge~\cite{Lannurien2023}. Ces stratégies ont des résultats différents en ce qui concerne l'utilisation des ressources et la qualité de service, mais il peut être difficile de prédire dans quelle mesure elles auront un impact sur les charges de travail avant de pouvoir l'observer en production.

\section{Choix de conception}
\label{section:herosim-herosim}

Cette section présente les choix de conception et les hypothèses formulées pour le développement de HeROsim.

\begin{figure}[!ht]
    \centering
    \includegraphics[width=0.9\textwidth]{6_Chapitre6/figures/software-architecture.png}
    \caption{Vue de haut niveau de l'architecture logicielle du simulateur.}
\label{figure:herosim-software-architecture}
\end{figure}

HeROsim utilise la bibliothèque SimPy~\footnote{\href{https://simpy.readthedocs.io}{https://simpy.readthedocs.io}} comme moteur de simulation à événements discrets. Cette bibliothèque est disponible sous licence libre (\textit{MIT}) ; elle est largement documentée et utilisée dans la recherche~\cite{matloffIntroductionDiscreteEventSimulation, zinovievDiscreteEventSimulation2024a}. Notre simulateur fournit trois classes de base, comme montré dans la figure~\ref{figure:herosim-software-architecture} -- \texttt{Orchestrator}, \texttt{Autoscaler} et \texttt{Scheduler} -- destinées à être sous-classées par les utilisateurs désireux d'implanter leurs propres algorithmes. Comme l'essentiel du comportement de la plateforme est hérité des classes de base, le coût de mise en œuvre d'une nouvelle politique est minime : le couple le plus simple d'autoscaler et d'ordonnanceur (pour une politique de placement aléatoire) peut être mis en œuvre en moins de 20 lignes de code (voir listings~\ref{code:herosim-random-autoscaler} et~\ref{code:herosim-random-scheduler} ci-dessous).

\begin{longlisting}
    \captionof{listing}{Code Python pour un autoscaler appliquant une politique d'allocation de ressources sélectionnées aléatoirement dans HeROsim.}
    \label{code:herosim-random-autoscaler}
    \begin{minted}[linenos,breaklines]{python}
class RandomAutoscaler(BaseAutoscaler):
  def create_first_replica(
    self, system_state: SystemState, task_type: TaskType
  ) -> Generator:
    # Create a new function instance on any available hardware platform
    # Return None if no replica can be created
    stop = yield self.env.process(
        self.scale_up(1, system_state, task_type["name"], "any")
    )

    return stop

  def create_replica(
    self, couples_suitable: Set[Tuple[Node, Platform]], task_type: TaskType
  ) -> Tuple[Node, Platform]:
    # Select random platform on random node
    found = random.randint(0, len(couples_suitable) - 1)
    random_couple = list(couples_suitable)[found]

    return random_couple

  def remove_replica(
    self,
    couples_suitable: Set[Tuple[Node, Platform]],
    task_type: TaskType,
    state: SystemState,
  ) -> Generator:
    # Array is shuffled in place...
    random.shuffle(list(couples_suitable))

    # Mark replica for removal if its task queue is empty
    # Return None if no replica can be removed
    removed_couple = next(
        (
            replica
            for replica in couples_suitable
            if not replica[1].queue.items and not replica[1].current_task
        ),
        None,
    )

    return removed_couple
    \end{minted}
\end{longlisting}

\begin{longlisting}
    \captionof{listing}{Code Python pour un ordonnanceur appliquant une politique de placement aléatoire des requêtes utilisateur dans HeROsim.}
    \label{code:herosim-random-scheduler}
    \begin{minted}[linenos,breaklines]{python}
class RandomScheduler(BaseScheduler):
  def placement(
    self, system_state: SystemState, task: Task
  ) -> Tuple[Node, Platform]:
    replicas: Set[Tuple[Node, Platform]] = system_state.replicas[task.type["name"]]

    # Select random platform
    found = random.randint(0, len(replicas) - 1)
    random_couple = list(replicas)[found]

    return random_couple
    \end{minted}
\end{longlisting}

\subsection{Données d'entrée}

HeROsim expose une interface déclarative permettant aux utilisateurs de définir leur infrastructure cloud, leurs charges de travail et leurs contraintes de qualité de service. Le simulateur reproduit un scénario d'allocation et d'ordonnancement selon différentes politiques d'orchestration. Une exécution du simulateur nécessite les entrées suivantes, sous la forme de fichiers \gls{JSON} (pour \textit{JavaScript Object Notation}, un format textuel structuré), pour définir un tel scénario :

\begin{enumerate}
    \item Une \textbf{description des applications} (voir listing~\ref{code:herosim-json-application}) comportant les mesures des \textbf{caractéristiques des fonctions} (voir listing~\ref{code:herosim-json-functions}) invoquées au cours du scénario, \textit{i.e.} leurs temps d'exécution et de démarrage à froid, leurs besoins mémoire, leur consommation d'énergie, la taille de l'image des fonctions, la taille des entrées/sorties lors des phases de communication. Ces métadonnées sont déclinées par plateforme d'exécution, et peuvent être définies en \textbf{mesurant} et \textbf{analysant} le comportement des applications sur une configuration de banc d'essai (voir figure~\ref{figure:herosim-characterization}) ;
    \item Une \textbf{description de l'infrastructure} (voir listing~\ref{code:herosim-json-infrastructure}) listant les différents \textbf{nœuds disponibles} et les \textbf{plateformes d'exécution} qu'ils comportent, leurs supports de stockage et la bande passante disponible sur le réseau. Les plateformes d'exécution sont définies en termes de consommation d'énergie au repos et de prix de détail ; les dispositifs de stockage sont caractérisés par leur capacité, leur bande passante et leur latence. Ces métadonnées sont spécifiques à une plateforme cible et peuvent être \textbf{mesurées} ou obtenues auprès des fabricants ;
    \item Une \textbf{trace d'exécution} (voir listing~\ref{code:herosim-json-trace}) faisant état des temps d'arrivée pour toutes les \textbf{requêtes utilisateur} vers les applications déployées, associées au niveau de qualité de service demandé par l'utilisateur, par ordre chronologique. Ces données peuvent être extraites par \textbf{observation} réelle des applications en production (voir figure~\ref{figure:herosim-characterization}), ou estimées statistiquement.
\end{enumerate}

Les données issues de nos contributions (voir chapitres~\ref{chapter:herofake} et~\ref{chapter:herocache}) sont mises à disposition pour référence, avec le code source du simulateur~\footnote{\href{https://github.com/b-com/HeROsim}{https://github.com/b-com/HeROsim}}. Alors que (1) et (2) peuvent être rédigés à la main en fonction des exigences du cas d'utilisation, HeROsim fournit un générateur synthétique pour créer diverses traces pour (3) en utilisant des processus de Poisson, avec une durée et un taux requêtes par seconde configurables, comme cela est couramment fait dans la littérature~\cite{9928755}.

\begin{longlisting}
    \captionof{listing}{Description \gls{JSON} d'une application \texttt{hello-world} composée de deux fonctions, \texttt{hello} et \texttt{world}, exécutées séquentiellement.}
    \label{code:herosim-json-application}
    \begin{minted}[linenos,breaklines]{json}
{
  "hello-world": {
    "name": "hello-world",
    "dag": {
      "hello": [],
      "world": ["hello"]
    }
  }
}
    \end{minted}
\end{longlisting}

\begin{figure}[!ht]
    \captionof{listing}{Description \gls{JSON} des caractéristiques des fonctions \texttt{hello} et \texttt{world}, pouvant être déployées sur des plateformes \texttt{exampleCpu}.}
    \label{code:herosim-json-functions}
    \begin{minipage}{0.49\textwidth}
        \centering
        \begin{minted}[linenos,breaklines]{json}
  "hello": {
    "name": "hello",
    "platforms": [
      "exampleCpu"
    ],
    "memoryRequirements": {
      "exampleCpu": 0.0015
    },
    "coldStartDuration": {
      "exampleCpu": 0.002
    },
    "executionTime": {
      "exampleCpu": 8.5e-3
    },
    "energy": {
      "exampleCpu": 5.5-9
    },
    "imageSize": {
      "exampleCpu": 0.003
    },
    "stateSize": {
      "hello-world": {
        "input": 0,
        "output": 6
      }
    }
  }
        \end{minted}
        % \captionof{listing}{Sub caption}
    \end{minipage}
    \begin{minipage}{0.49\textwidth}
        \centering
        \begin{minted}[linenos,breaklines]{json}
  "world": {
    "name": "world",
    "platforms": [
      "exampleCpu"
    ],
    "memoryRequirements": {
      "exampleCpu": 0.0030
    },
    "coldStartDuration": {
      "exampleCpu": 0.004
    },
    "executionTime": {
      "exampleCpu": 8.5e-3
    },
    "energy": {
      "exampleCpu": 5.5-9
    },
    "imageSize": {
      "exampleCpu": 0.003
    },
    "stateSize": {
      "hello-world": {
        "input": 6,
        "output": 12
      }
    }
  }
        \end{minted}
        % \captionof{listing}{Another sub caption}
    \end{minipage}
\end{figure}

\begin{figure}[!ht]
    \captionof{listing}{Description \gls{JSON} d'une infrastructure hypothétique comportant deux nœuds hétérogènes. La partie droite détaille les métadonnées de la plateforme d'exécution \texttt{exampleCpu} et du support de stockage \texttt{exampleFlash}.}
    \label{code:herosim-json-infrastructure}
    \begin{minipage}{0.49\textwidth}
        \centering
        \begin{minted}[linenos,breaklines]{json}
{
  "network": {
    "bandwidth": 100
  },
  "nodes": [
    {
      "memory": 8,
      "platforms": [
        "exampleCpu",
        "exampleCpu"
      ],
      "storage": [
        "flashCard"
      ]
    },
    {
      "memory": 16,
      "platforms": [
        "exampleCpu",
        "exampleCpu",
        "exampleCpu",
        "exampleCpu"
      ],
      "storage": [
      ]
    }
  ]
}
        \end{minted}
        % \captionof{listing}{Sub caption}
    \end{minipage}
    \begin{minipage}{0.49\textwidth}
        \centering
        \begin{minted}[linenos,breaklines]{json}
{
  "exampleCpu": {
    "shortName": "exampleCpu",
    "name": "Example CPU",
    "hardware": "cpu",
    "price": 60,
    "idleEnergy": 0.00065
  }
},
{
  "exampleFlash": {
    "name": "Example Flash Card",
    "hardware": "flash",
    "capacity": 64,
    "iops": {
      "read": 3000,
      "write": 1000
    },
    "throughput": {
      "read": 100,
      "write": 40
    },
    "latency": {
      "read": 0.001,
      "write": 0.001
    }
  }
}
        \end{minted}
    \end{minipage}
\end{figure}

\begin{listing}
    \caption{Description \gls{JSON} d'une trace d'exécution comportant deux appels à l'application \texttt{hello-world}.}
    \label{code:herosim-json-trace}
    \begin{minted}[linenos,breaklines]{json}
{
  "rps": 2,
  "duration": 1,
  "events": [
    {
      "timestamp": 0.47,
      "application": "hello-world",
      "qos": "low"
    },
    {
      "timestamp": 0.98,
      "application": "hello-world",
      "qos": "high"
    }
  ]
}
    \end{minted}
\end{listing}

\subsection{Flot d'une simulation}

L'utilisateur peut choisir les politiques d'orchestration souhaitées et exécuter le programme principal. Le simulateur :

\begin{itemize}
    \item initialise l'infrastructure comme décrit : le scénario commence avec tous les nœuds inactifs, en attente de nouvelles requêtes ;
    \item initialise l'orchestrateur avec l'autoscaler et l'ordonnanceur choisis ;
    \item suit les temps d'arrivée des événements à partir de la trace d'exécution et transmet les requêtes utilisateur à l'orchestrateur ;
    \item charge l'ordonnanceur de placer ces requêtes sur des répliques de fonctions ;
    \item charge l'autoscaler d'allouer et désallouer les ressources matérielles qui hébergent ces répliques pour traiter les requêtes utilisateur ;
    \item fait avancer l'horloge au fur et à mesure du traitement des requêtes utilisateur.
\end{itemize}

\begin{figure}[!ht]
    \centering
    \includegraphics[width=0.7\textwidth]{6_Chapitre6/figures/characterization.png}
    \caption{Une vue d'ensemble de notre méthodologie de caractérisation.}
\label{figure:herosim-characterization}
\end{figure}

Pendant la simulation, des journaux sont écrits sur le disque. Lorsque toutes les requêtes de la trace d'exécution ont été traitées, la simulation s'arrête et retourne les résultats ainsi que des graphiques résumant la simulation, respectivement dans les répertoires \texttt{result} et \texttt{chart}.

Le simulateur est monotâche, ce qui signifie que chaque politique est évaluée de manière séquentielle, au cours d'une exécution complète du scénario. Cependant, plusieurs instances de HeROsim peuvent être exécutées en parallèle, avec différentes configurations, sur plusieurs cœurs de \gls{CPU}, voire même distribuées sur plusieurs nœuds. Chaque instance de HeROsim exécutant une politique d'orchestration différente et toutes les instances utilisant un répertoire de sortie commun, il est possible d'accélérer la durée totale de la simulation lors de l'évaluation de nombreuses politiques d'orchestration. Lorsque toutes les exécutions du simulateur sont terminées, les résultats consolidés peuvent être passés au module d'analyse et de génération des graphiques.

\subsection{Module Orchestrateur}

Le rôle principal du module d'orchestrateur est de maintenir une vue sur l'état du système qui sera utile à l'autoscaler et à l'ordonnanceur. La classe de base \textbf{\texttt{Orchestrator}} fournit des méthodes d'initialisation abstraites qui doivent être redéfinies pour spécifier la structure de données à l'état du système que l'on souhaite modéliser et l'initialiser. Par exemple, un ordonnanceur \textit{Round Robin} devra savoir combien de fois chaque réplique de fonction est sélectionnée pour le placement des tâches, tandis qu'un ordonnanceur \textit{Least Connected} devra connaître la concurrence moyenne dans chaque réplique pour équilibrer la charge. Cette classe est le point d'entrée permettant aux utilisateurs de définir des méthodes qui seront appelées pendant la simulation, périodiquement ou à la suite d'un évènement, et permettront de maintenir à jour un ensemble de métriques qui représente de manière pertinente l'état de la plateforme à tout instant.

Les utilisateurs peuvent mettre en œuvre des processus périodiques de gestion de l'état du système en fonction de leurs besoins pour supporter une variété de politiques d'orchestration. L'orchestrateur de base est doté d'un système simple qui peut fonctionner tel quel ou être étendu. Dans notre implantation, un processus de surveillance est appelé périodiquement pour garder une trace de la concurrence moyenne dans chaque réplique de fonction. Ceci est utile pour les politiques basées sur des seuils, telles que présentées dans les chapitres~\ref{chapter:herofake} et~\ref{chapter:herocache}.

La méthode du point d'entrée de l'orchestrateur est appelée chaque fois qu'une requête utilisateur arrive sur la plateforme. Elle prend en entrée l'état du système et la requête utilisateur. Elle peut être redéfinie dans chaque implantation de politique d'orchestration pour permettre une mise à l'échelle automatique et un ordonnancement à grain fin, selon les besoins.

Enfin, cette classe est responsable de l'instanciation des implantations sélectionnées pour l'\texttt{Autoscaler} et le \texttt{Scheduler}. Toute combinaison de ces deux modules peut être instanciée par l'orchestrateur.

\subsection{Module Autoscaler}

La classe de base \textbf{\texttt{Autoscaler}} fournit le comportement commun de la plateforme d'autoscaling, qui correspond essentiellement au cycle de vie des répliques de fonctions, de leur création à leur suppression. Plusieurs méthodes abstraites doivent être redéfinies pour mettre en œuvre une nouvelle politique : stratégie de sélection des ressources pour la création de répliques, processus d'initialisation des répliques, sélection des répliques pour la suppression, etc. Ces méthodes opèrent à la granularité d'une seule fonction, en prenant l'état du système et la liste des ressources matérielles disponibles comme données d'entrée. Les utilisateurs sont libres de mettre en œuvre les algorithmes qu'ils souhaitent évaluer pour la gestion des ressources.

L'autoscaler de HeROsim a été principalement conçu pour une \textbf{mise à l'échelle horizontale}. Les répliques de fonctions sont créées en allouant une plateforme d'exécution et la quantité requise de mémoire sur un nœud. Une plateforme d'exécution ne peut pas héberger plus d'une réplique de fonction à la fois. Pour faire face à l'augmentation de la charge d'une application sans dégradation de la qualité de service, de nouvelles répliques de fonctions doivent être allouées par l'autoscaler, à condition qu'il y ait suffisamment de ressources matérielles disponibles. Les répliques nouvellement allouées passent par une phase d'initialisation au cours de laquelle les images des fonctions doivent être récupérées via le réseau. L'autoscaler peut gérer un cache d'images dans la mémoire du nœud et sur le stockage local du nœud afin d'accélérer les démarrages à froid.

La suppression des répliques inactives se fait sur le mode \textit{best effort} : l'autoscaler tente de supprimer les répliques dont les files d'attente de tâches sont vides. Par défaut, une réplique avec des tâches en attente ne peut pas être supprimée. Ce comportement pourrait être redéfini par une nouvelle politique, qui par exemple prendrait en charge les migrations de tâches.

L'autoscaler garde une trace de chaque événement d'allocation pour calculer l'utilisation des ressources à la fin de la simulation. HeROsim permet à l'utilisateur de savoir quels nœuds et quelles plateformes d'exécution ont été enrôlés pendant le scénario, à quel moment et pour quelle durée, et pour quel déploiement de fonction ils ont été choisis. Cela permet également à HeROsim de calculer la consommation d'énergie à différentes granularités : la consommation statique nécessaire au matériel alloué, et la consommation dynamique liée à l'exécution des applications.

% TODO: modèle énergétique, modèle temporel ?

HeROsim est livré avec les politiques de mise à l'échelle automatique suivantes, basées sur des seuils de concurrence et prêtes à l'emploi :

\begin{itemize}
    \item \textit{Random} -- Sélectionne un nœud aléatoire et une plateforme d'exécution pour les nouvelles répliques ;
    \item Knative -- Sélectionne le nœud le moins chargé pour allouer de nouvelles répliques~\cite{sureshENSUREEfficientScheduling2020}, \textit{i.e} équilibre les charges de travail sur un grand nombre de nœuds ;
    \item HeROfake (voir chapitre~\ref{chapter:herofake}) -- Exploite l'hétérogénéité du matériel pour minimiser les pénalités de qualité de service, la consommation d'énergie et le coût total de possession. Plus de détails sont donnés dans la section~\ref{section:herosim-case-study};
    \item HeROcache (voir chapitre~\ref{chapter:herocache}) -- Optimise les allocations pour les chaînes de fonctions ; maximise la consolidation des fonctions de chaque application. Plus de détails sont donnés dans la section~\ref{section:herosim-case-study}.
\end{itemize}

\subsection{Module Ordonnanceur}

La classe de base \textbf{\texttt{Scheduler}} met en œuvre la sélection des tâches dans la file d'attente de la plateforme. Une méthode abstraite doit être redéfinie pour mettre en œuvre une nouvelle politique : la sélection d'une réplique parmi le pool de répliques disponibles, pour placer chaque requête utilisateur en attente. Cette méthode opère à la granularité la plus fine : elle est appelée lors de la réception d'une requête, et prend en entrée l'état du système et la liste des répliques de fonctions disponibles.

Les requêtes utilisateur arrivent dans une file d'attente au niveau de l'ordonnanceur. Les utilisateurs peuvent mettre en œuvre leur propre politique de priorité pour la sélection des tâches ou choisir une politique déjà disponible dans le simulateur, \textit{e.g.} \textit{First In, First Out} (\gls{FIFO}) ou \textit{Earliest Deadline First} (\gls{EDF})~\cite{herofake}.

L'ordonnanceur de HeROsim a été conçu sans tenir compte des défaillances ou des migrations de tâches : le comportement par défaut considère des tâches qui s'exécutent toujours avec succès jusqu'à leur terme. Cependant, les tâches seront marquées comme "en pénalité" si l'ordonnanceur manque leur échéance. Cette valeur booléenne permet d'évaluer la qualité de politiques d'orchestration en matière de respect des exigences de qualité de service : elle indique la proportion de requêtes qui ne sont traitées en temps voulu.

S'il n'y a pas de réplique disponible au moment de l'ordonnancement d'une requête utilisateur, l'ordonnanceur fera un appel à l'autoscaler pour forcer la création d'une première réplique pour la fonction. Dans l'intervalle, la requête est remise dans la file d'attente et reportée. Les tâches reportées sont signalées comme telles, de sorte que, par exemple, elles puissent avoir une priorité plus élevée si toutefois l'utilisateur souhaite appliquer une telle politique~\cite{herocache}.

HeROsim est livré avec les politiques d'ordonnancement suivantes implantées et prêtes à l'emploi :

\begin{itemize}
    \item \textit{Random} -- Sélectionne une réplique aléatoire pour le placement des tâches ;
    \item Knative -- Sélectionne la réplique avec la file d'attente la plus courte pour le placement des tâches~\cite{sureshENSUREEfficientScheduling2020} ;
    \item \gls{BPFF} -- Sélectionne la réplique avec la plus longue file de requêtes en attente pour le placement des tâches ;
    \item HeROfake (voir chapitre~\ref{chapter:herofake}) -- Sélectionne la réplique qui minimise un score calculé en fonction de l'échéance de la tâche, de la consommation d'énergie de la fonction et de la dispersion des tâches sur les nœuds. Plus de détails sont donnés dans la section~\ref{section:herosim-case-study} ;
    \item HeROcache (voir chapitre~\ref{chapter:herocache}) -- Sélectionne la réplique d'une manière similaire à HeROfake, mais prend en compte les opérations de stockage et de communication pour calculer la latence de bout en bout de la requête, et prend en compte les chaînes de fonctions lors du calcul du score des répliques en ce qui concerne la consolidation des tâches. De plus amples détails sont donnés dans la section~\ref{section:herosim-case-study}.
\end{itemize}

\subsection{Interface utilisateur}

HeROsim s'appuie sur des journaux d'évènements pour fournir un aperçu du déroulement de la simulation, ce qui aide l'utilisateur à déboguer ses politiques. À la fin de la simulation, des fichiers de résultats sont enregistrés sur le disque. Ces fichiers contiennent un résumé des résultats de la simulation, c'est-à-dire les performances de la politique au regard des métriques d'évaluation (qualité de service, consommation d'énergie, etc.). Ces résultats sont représentés sur différents graphiques qui peuvent être utilisés dans des publications ultérieures (voir chapitres~\ref{chapter:herofake} et~\ref{chapter:herocache}).

HeROsim peut également générer des graphiques supplémentaires, utiles pour tracer le comportement de la plateforme lors de l'élaboration d'une politique d'orchestration. On peut ainsi visualiser la proportion de démarrages à froid et d'accès au cache parmi les invocations de fonctions. HeROsim peut également tracer la latence par quantiles pour les requêtes des utilisateurs, ce qui peut aider à identifier un éventuel problème de dimensionnement d'une infrastructure cloud. Le générateur de traces d'exécution trace les temps d'arrivée des requêtes utilisateur, ce qui permet de visualiser sur un graphique les caractéristiques de la charge de travail.

\begin{figure}[!ht]
    \centering
    \includegraphics[width=0.7\textwidth]{6_Chapitre6/figures/serverless-cost.png}
    \caption{Répartition du coût en latence de bout en bout induit par les décisions d'autoscaling et d'ordonnancement.}
\label{figure:herosim-cost}
\end{figure}

\section{Étude de cas}
\label{section:herosim-case-study}

% TODO: Figures de développement (tail latency, etc.) ?

Dans cette section, nous présentons deux études de cas dans le cadre desquelles HeROsim a permis de concevoir et d'évaluer différentes politiques d'orchestration serverless. Nous avons conçu des stratégies qui reposent sur la caractérisation de plateformes hétérogènes et des charges de travail à déployer (voir figure~\ref{figure:herosim-characterization}). Nous avons mesuré plusieurs métriques liées à nos applications sur diverses plateformes matérielles et proposé un modèle de coût qui intègre ces valeurs, dans le but d'estimer les performances de l'autoscaler et de l'ordonnanceur sous différentes politiques (figure~\ref{figure:herosim-cost}).

\subsection{Stratégie d'orchestration de fonctions sans état sur ressources hétérogènes}

Dans cette première étude de cas, HeROfake (pour \textbf{He}terogeneous \textbf{R}esources \textbf{O}rchestration for deep\textbf{fake} detection~\cite{herofake}) (voir chapitre~\ref{chapter:herofake}), nous avons étudié le déploiement d'une application de détection de deepfake sur une plateforme serverless dans un cloud privé. En particulier, nous nous sommes intéressés à l'exploitation de ressources hétérogènes pour l'orchestration serverless lors de l'optimisation de la plateforme pour la \gls{QoS} et la consommation d'énergie.

L'application exécute des tâches d'inférence pour détecter des deepfakes dans des images soumises par des utilisateurs avec diverses exigences de niveau de \gls{QoS}. Son objectif est de détecter les images potentiellement falsifiées, c'est-à-dire les images susceptibles d'avoir été manipulées par ordinateur pour tromper des spectateurs ou destinataires. Elle se compose de trois fonctions, basées sur l'utilisation de réseaux de neurones, indépendantes et sans état, qui ont été mises en œuvre sur du matériel hétérogène : \gls{CPU}, \gls{GPU} et \gls{FPGA}. Ces implantations ont été utilisées pour les mesures, mais il aurait été difficile d'exécuter des scénarios réalistes sur une plateforme serverless prenant en charge ces architectures matérielles. Ainsi, nous avons eu recours à la simulation pour estimer les performances de notre stratégie d'orchestration.

Dans HeROfake, chaque exécution de tâche a un coût associé mesuré en \textbf{latence}. L'allocation de nouvelles répliques sur des ressources matérielles inactives introduit un \textbf{délai de démarrage à froid} ; le traitement des requêtes utilisateur sur différentes architectures matérielles donne lieu à des variations en matière de \textbf{temps d'exécution de la fonction}. La latence de chaque requête utilisateur est finalement comparée à l'\textbf{exigence de qualité de service} de la requête : si l'échéance de la requête n'a pas été respectée, la tâche est en \textbf{pénalité}. Ces éléments constituent la base de notre modèle de coût, comme illustré dans la figure~\ref{figure:herosim-cost}.

Les différentes architectures matérielles ont également des coûts monétaires et énergétiques différents qui ont été pris en compte dans le modèle de coût (utilisation des ressources et consommation d'énergie, les deux dimensions du coût de l'infrastructure montré dans la figure~\ref{figure:herosim-cost}). Nous avons mis en œuvre un autoscaler et un ordonnanceur qui cherchent à estimer et à minimiser ce coût composite, à la granularité de chaque requête utilisateur. L'autoscaler cherche à allouer des plateformes qui minimisent le coût global, y compris l'utilisation des ressources, la consommation d'énergie et le coût total de possession. L'ordonnanceur cherche à sélectionner les répliques qui traitent les requêtes avec le moins de pénalités et la plus faible consommation d'énergie.

HeROfake a été évalué pour un scénario synthétique avec une distribution uniforme des temps d'arrivée des requêtes, des types de fonction appelées et des niveaux de qualité de service associés aux requêtes. HeROfake s'est avéré performant (en termes de pénalités sur \gls{QoS}, de consolidation des tâches et d'énergie consommée) par rapport à Knative (l'orchestrateur serverless de Google) et \textit{Bin-Packing First Fit} (\gls{BPFF}, la politique d'\gls{AWS} pour Lambda)~\cite{herofake}.

Notre objectif principal était d'étudier la pertinence de la prise en compte de l'hétérogénéité matérielle lors de l'allocation des ressources et de l'ordonnancement des requêtes utilisateur au regard de différentes mesures de qualité de service, \textit{i.e.} la latence de bout en bout subie par les requêtes utilisateur, et la consommation d'énergie de l'infrastructure. Les résultats expérimentaux ont montré que si le temps total d'exécution des tâches dans HeROfake est similaire à celui de Knative, nous obtenons une réduction de plus de 60\% des pénalités sur qualité de service ; les tâches sont consolidées sur moins de 40\% des nœuds de l'infrastructure, 77\% des plateformes d'exécution restant inutilisées ; et la consommation d'énergie dynamique est réduite de 35\% par rapport à Knative.

HeROsim nous a également permis d'évaluer l'impact des différents composants de notre orchestrateur. Les résultats ont montré que, même en allouant uniquement des \gls{CPU}, l'ordonnancement des requêtes avec notre politique consciente de la qualité de service pouvait maintenir les pénalités en dessous de 50\%.

\subsection{Stratégie d'orchestration d'applications avec prise en compte du cache et des communications}

Dans une seconde étude, nous avons conçu un orchestrateur pour les applications serverless, composées de plusieurs fonctions interdépendantes. Nous avons proposé HeROcache (pour \textbf{He}terogeneous \textbf{R}esources \textbf{O}rchestration with a \textbf{cache} strategy~\cite{herocache}) (voir chapitre~\ref{chapter:herocache}), une stratégie d'orchestration consciente des contraintes temporelles et de données. Nous avons exploré le déploiement d'un système de détection d'intrusion (\gls{IDS}, pour \textit{Intrusion Detection System}) sur une plateforme serverless. En particulier, nous avons étudié les avantages de l'orchestration consciente des données lors de l'exploitation de matériel hétérogène pour déployer des applications sensibles au temps sur des dispositifs \textit{edge} à ressources limitées, du point de vue du fournisseur de services.

L'objectif de l'application est de détecter le trafic réseau potentiellement malveillant dans des journaux soumis par les utilisateurs. L'application se compose de deux couches pour traiter les requêtes utilisateur : deux fonctions de prétraitement, qui opèrent sur les journaux de trafic réseau pour exclure les évènements que l'on sait légitimes, et quatre fonctions d'inférence, basées sur des réseaux de neurones entraînés à détecter des motifs typiques d'activité malveillante dans le reste des journaux. Ces fonctions ont été mises en œuvre sur différentes architectures matérielles (\gls{CPU}, \gls{FPGA}, \gls{GPU}).

Il existe des dépendances de données entre les deux couches de l'application. Nous les avons modélisées sous forme de graphes acycliques dirigés (\gls{DAG}) d'appels de fonctions. Nous avons utilisé le module Python \texttt{TopologicalSorter} du module \texttt{graphlib} pour lire et écrire la représentation \gls{JSON} des graphes et pour les parcourir pendant l'exécution.

Au niveau de l'orchestrateur, nous avons dû modifier la granularité des requêtes utilisateur : dans HeROcache, une requête concerne une application, qui peut être composée d'une ou plusieurs fonctions, appelées dans l'ordre défini par le \gls{DAG} de l'application. Chaque fonction peut prendre des données en entrée et produire des données en sortie. Ces données sont définies par leur taille en octets.

Alors que HeROfake ne prend pas en compte les coûts associés opérations de stockage, HeROcache considère la \textbf{récupération des images de fonctions}, la \textbf{mise en cache des images de fonctions} et les \textbf{communications entre fonctions} dans les décisions de l'autoscaler et de l'ordonnanceur. Nous avons étendu HeROsim avec ces concepts pendant la conception de HeROcache. Ces opérations impactent la latence des requêtes utilisateur. Des mesures hors-ligne pour caractériser les fonctions permettent à HeROsim, comme montré dans la figure~\ref{figure:herosim-cost} d'estimer le temps de récupération de l'image de la fonction sur la base du débit du lien réseau spécifié par l'utilisateur dans la description de l'infrastructure, la taille de l'image de la fonction spécifiée par le type de charge de travail, et la performance du stockage local au nœud spécifiée par le type de stockage.

L'introduction d'opérations de stockage à la granularité d'un nœud nous a permis d'évaluer une stratégie de préchargement des images de fonctions au sein d'une application, qui peut accélérer les démarrages à froid pour des invocations futures.

Dans HeROcache, l'autoscaler cherche à accroître la consolidation des fonctions entre les applications, à réduire la durée de réalisation globale (ou \textit{makespan}), la consommation d'énergie et le coût total de possession. L'ordonnanceur cherche à éviter la violation des échéances des requêtes, à consommer moins d'énergie et à assurer une utilisation élevée des ressources mobilisées.

HeROfake est conçue comme une politique sans conscience du stockage, mais exploite déjà l'hétérogénéité matérielle, ce qui en fait une solution comparable. Cette comparaison nous a permis de montrer l'importance de la prise en compte des coûts de stockage dans la capacité de la plateforme à respecter les exigences de qualité de service : en prenant compte de ces coûts, HeROcache consolide les applications et parvient à réduire les délais d'initialisation moyens de 17,6\% et les délais de communication de 88,4\%. Cela permet de réduire la consommation d'énergie statique de la plateforme de 80\% tout en maintenant moins de 28\% de violations de qualité de service.

HeROcache (mis en œuvre dans HeROsim) a été soumis pour évaluation des artefacts et a reçu les trois badges de reproductibilité de l'IEEE~\footnote{\href{https://www.niso.org/standards-committees/reproducibility-badging}{https://www.niso.org/standards-committees/reproducibility-badging}} lors de sa publication : \textit{Open Research Objects} (ORO), \textit{Reusable/Research Objects Reviewed} (ROR), et \textit{Results Reproduced} (ROR-R).

\section{Conclusion et perspectives}
\label{section:herosim-conclusion}

Dans cet article, nous avons présenté HeROsim, un outil de simulation qui vise à permettre aux chercheurs de modéliser des infrastructures cloud hétérogènes, de décrire des charges de travail à une granularité fine, de mettre en œuvre diverses politiques de gestion des ressources et d'ordonnancement des tâches, et d'évaluer leur efficacité au regard de métriques telles que l'utilisation des ressources, la consommation d'énergie, les violations de la \gls{QoS} par requête, ou la latence des files d'attente. HeROsim peut générer des graphiques qui aident à visualiser ces résultats pendant la phase de mise en œuvre et peuvent être utilisés dans des publications.

Des travaux sont en cours pour étendre HeROsim et proposer une interface web permettant de visualiser l'état de la simulation en temps réel. Nous travaillons également sur un agent d'apprentissage par renforcement qui pourrait être inclus dans une prochaine version.


\part{Conclusion et perspectives}
\label{part:three}

\clearemptydoublepage
\backmatter
\chapter{Conclusion et perspectives}
\label{chapter:conclusion}
%\addcontentsline{toc}{chapter}{Conclusion et perspectives}
%\chaptermark{Conclusion et perspectives}

\section{Résumé des contributions}
%\addcontentsline{toc}{section}{Résumé des contributions}

\section{Limites des contributions}
%\addcontentsline{toc}{section}{Limites des contributions}

Contention : \cite{vanbeekCPUContentionPredictor2019} \cite{jacquetSweetspotVMOversubscribingCPU}

Interférences : \cite{kohAnalysisPerformanceInterference2007} \cite{vardasImprovedParallelApplication}

Pannes : \cite{javadiFailureTraceArchive2013, galletModelSpaceCorrelatedFailures2010}

Modèle de coût étendu : exemple, coût pour la société (carbone, eau) \cite{rickeCountrylevelSocialCost2018}

Complexité, passage à l'échelle : les noeuds remontent leur niveau de charge et leurs décisions de scaling à l'orchestrateur \cite{straesserPowerApplicationsVision2023} + job pulling ?

Traces d'exécution : Poisson vs bursty

\section{Travaux futurs}
%\addcontentsline{toc}{section}{Travaux futurs}

% Our next contribution will leverage Q-Learning to explore the design of an autonomous agent which efficiently rightsizes resources allocations on the serverless platform. It will make use of time series prediction to allow timely, proactive autoscaling. This agent will be evaluated in the simulator, showcasing the variety of policies that can be implemented with our tool.

Prédiction sur séries temporelles : \cite{bauerTimeSeriesForecasting2020}

Allocation proactive : \cite{parkGraphNeuralNetworkBased2024}

\section{Remarques finales}
%\addcontentsline{toc}{section}{Remarques finales}


% Chapitre pour la bibliographie
% Bibliography chapter
\clearemptydoublepage
\phantomsection % To have a correct link in the table of contents
\addcontentsline{toc}{chapter}{Bibliographie}

% nocite: Pour citer la totalit\'{e} des r\'{e}f\'{e}rences contenues dans le fichier biblio
% nocite: In order to cite all the references included biblio
\nocite{*}
\printbibliography
% \newpage
% \nocite{*}
% \printbibliography[heading=secondary,keyword=secondary]

\clearemptydoublepage
% Pour avoir la quatrième de couverture sur une page paire
% To have the back cover on an even page
\cleartoevenpage[\thispagestyle{empty}]
\markboth{}{}
% Plus petite marge du bas pour la quatrième de couverture
% Shorter bottom margin for the back cover
\newgeometry{inner=30mm,outer=20mm,top=40mm,bottom=20mm}

%insertion de l'image de fond du dos (resume)
%background image for resume (back)
\backcoverheader

% Switch font style to back cover style
\selectfontbackcover{ % Font style change is limited to this page using braces, just in case
    \titleFR{Allocation et placement dynamiques sur ressources hétérogènes pour le serverless}
    \keywordsFR{cloud, serverless, orchestration, hétérogénéité, optimisation, simulation}
    \abstractFR{
        Le modèle serverless est un paradigme pour le cloud dans lequel les ressources matérielles ne sont pas réservées, mais allouées à la demande, lors de la réception de requêtes utilisateur. La facturation des clients s'effectue sur la base de l'utilisation réelle des ressources. En contrepartie, la responsabilité de l'allocation des ressources et du placement des tâches incombe au fournisseur. Ce mécanisme d'orchestration peut induire un surcoût en latence et dégrader la qualité de service pour les utilisateurs.
        Nous nous sommes intéressés à la capacité pour le fournisseur à garantir la qualité de service étant donnée une infrastructure dotée de matériel hétérogène (CPU, GPU, FPGA, DLA), dans le cadre de deux déploiements : une application de détection de deepfake, basée sur des fonctions sans état, et une application de détection d'intrusions, reposant sur des chaînes de fonctions communiquant des résultats intermédiaires.
        Nous avons proposé une politique d’orchestration qui optimise le déploiement d'applications composées d'une unique fonction, en minimisant les pénalités sur qualité de service. Nous avons également modélisé des applications complexes, en prenant en compte les dépendances entre fonctions qui les composent, pour chercher à réduire la consommation énergétique et la désagrégation des tâches sur les nœuds de calcul.
        Enfin, nous avons développé un simulateur permettant de représenter de tels environnements, et d'évaluer et de comparer différentes stratégies d'orchestration pour le cloud.
    }

    \titleEN{Dynamic allocation and scheduling on heterogeneous resources in a serverless cloud}
    \keywordsEN{cloud, serverless, orchestration, heterogeneity, optimization, simulation}
    \abstractEN{
        The serverless model is a cloud paradigm in which hardware resources are not reserved but allocated on demand, triggered by incoming user requests. Clients are billed based on their actual resource usage. In return, the responsibility for resource allocation and task placement lies with the provider. This orchestration mechanism can lead to latency overhead and degrade the quality of service for users.
        We focused on the provider’s ability to guarantee quality of service given an infrastructure comprising heterogeneous hardware resources (CPU, GPU, FPGA, DLA), in the context of two deployments: a deepfake detection application, based on stateless functions, and an intrusion detection application, relying on chains of functions that communicate intermediate results.
        We proposed an orchestration policy that optimizes the deployment of applications composed of a single function, minimizing penalties on quality of service. We also modeled complex applications, taking into account the dependencies between the functions they comprise, aiming to reduce energy consumption and task disaggregation across compute nodes.
        Finally, we developed a simulator to represent such environments and to evaluate and compare different cloud orchestration strategies.
    }
}

% Rétablit les marges d'origines
% Restore original margin settings
\restoregeometry


\end{document}
