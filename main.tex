% Ce fichier main.tex est le fichier principal \`{a} partir duquel tout est g\'{e}n\'{e}r\'{e}
% This file is the main file where the final document is generated
\documentclass{these-dbl}

% Remplir les metadonnees du pdf
% Fill the pdf metadata
\hypersetup{
%    pdfauthor   = {XYZ},
%    pdftitle    = {Th\`{e}se de doctorat de XYZ},
%    pdfsubject  = {Th\`{e}se de doctorat de XYZ},
%    pdfkeywords = {mots-cl\'{e}s},
}

\geometry{vmargin=4.0cm}

% Spécifier vos fichiers de bibliographie
% Specify you bibliography files here
\addbibresource{./biblio/biblio.bib}
\addbibresource{./biblio/herofake.bib}
\addbibresource{./biblio/herocache.bib}
\addbibresource{./biblio/herosim.bib}

% Que du plaisir
\usepackage{amsmath,amssymb,amsfonts}

\usepackage{circledsteps}
\pgfkeys{/csteps/inner color=white}
\pgfkeys{/csteps/fill color=black}

\usepackage{pifont}% http://ctan.org/pkg/pifont
\newcommand{\cmark}{\color{YellowGreen}\ding{51}}%
\newcommand{\xmark}{\color{BrickRed}\ding{55}}%

\usepackage{float}
\usepackage[caption=false]{subfig}

% \usepackage[breaklinks]{hyperref}

% Tableaux
\usepackage{array}
\usepackage{multirow}
\usepackage{tabularx}
\usepackage{booktabs}
\newcolumntype{Y}{>{\centering\arraybackslash}X}
% Centered tabular column with text wrapping (L)
\newcolumntype{L}{>{\centering\arraybackslash}m{0.6\linewidth}}
\newcolumntype{S}{>{\centering\arraybackslash}m{0.1\linewidth}}
% for vertical centering text in X colum
\renewcommand\tabularxcolumn[1]{m{#1}}

% Tableaux multi-pages
% \usepackage{tabularray,adjustbox}
% \usepackage[allow-quantity-breaks]{siunitx}
% \UseTblrLibrary{booktabs}
% % make tabularray use \caption instead of own caption style
% \makeatletter\def\captionofnoinc#1{\def\@captype{#1}\addtocounter{#1}{-1}\caption}\makeatother\DefTblrTemplate{firsthead}{default}{\captionofnoinc{table}{\InsertTblrText{caption}}}\DefTblrTemplate{middlehead}{default}{\captionofnoinc{table}{\InsertTblrText{caption} \InsertTblrText{middlehead-text} (Suite)}}
% \DefTblrTemplate{lasthead}{default}{\captionofnoinc{table}{\InsertTblrText{caption} \InsertTblrText{lasthead-text} (Suite)}}

%%%%%%%%%%%%%%%%%%%%%%%%%%%%%%%%%%%%%%%%%%%%%%%%%%%%%%%%%%%%%%%%%%%%%%%%%%%%%%%%%%%%%
\newboolean{showcomments}
\setboolean{showcomments}{true}
\ifthenelse{\boolean{showcomments}}
{ \newcommand{\mynote}[3]{
    \fbox{\bfseries\sffamily\scriptsize#1}
    {\small$\blacktriangleright$\textsf{\emph{\color{#3}{#2}}}$\blacktriangleleft$}}}
{ \newcommand{\mynote}[3]{}}
\newcommand{\shrink}[1]{}
\newcommand{\jb}[1]{\mynote{Jalil}{#1}{blue}}
\newcommand{\vl}[1]{\mynote{Vincent}{#1}{red}}
%%%%%%%%%%%%%%%%%%%%%%%%%%%%%%%%%%%%%%%%%%%%%%%%%%%%%%%%%%%%%%%%%%%%%%%%%%%%%%%%%%%%%

\begin{document}

% Page de garde avec commande \maketitle
% Front cover calling \maketitle
% La page de garde est en français
% The front cover is in French
\selectlanguage{french}

% Inclure les infos de chaque établissement
% Include each institution data
\input{./Couverture/liste-ecoles-etablissements.tex}

% Inclure infos de l'école doctorale
% Include doctoral school data
\ecoledoctorale{SPIN}

% Inclure infos de l'établissement
% Include institution data
\etablissement{ENSTA}

%Inscrivez ici votre sp\'{e}cialit\'{e} (voir liste des sp\'{e}cialit\'{e}s sur le site de votre \'{e}cole doctorale)
%Indicate the domain (see list of domains in your ecole doctorale)
\spec{STIC -- Informatique}

%Attention : le pr\'{e}nom doit être en minuscules (Jean) et le NOM en majuscules (BRITTEF) 
%Attention : the first name in small letters and the name in Capital letters 
\author{Vincent LANNURIEN}

% Donner le titre complet de la th\`{e}se, \'{e}ventuellement le sous titre, si n\'{e}cessaire sur plusieurs lignes 
%Give the complete title of the thesis, if necessary on several lines
\title{Allocation et placement dynamiques sur ressources hétérogènes pour le cloud serverless}
% \lesoustitre{Optimisation de l'allocation dynamique des ressources matérielles et du placement des requêtes utilisateur dans le modèle de service serverless}

%Indiquer la date et le lieu de soutenance de la th\`{e}se 
%indicates the date and the place of the defense 
\date{20 novembre 2024}
\lieu{Brest}

%Indiquer le nom du (ou des) laboratoire (s) dans le(s)quel(s) le travail de th\`{e}se a \'{e}t\'{e} effectu\'{e}, indiquer aussi si souhait\'{e} le nom de la (les) facult\'{e}(s) (UFR, \'{e}cole(s), Institut(s), Centre(s)...), son (leurs) adresse(s)... 
%Indicates the name (or names) of research laboratories where the work has been done as well as (if desired) the names of faculties (UFR, Schools, institution...
\uniterecherche{Lab-STICC, CNRS, UMR 6285, équipe SHAKER}

%Indiquer le Numero de th\`{e}se, si cela est opportun, ou laisser vide pour faire disparaitre cet ligne de la couverture
%Indicate the number of the thesis if there is one. otherwise leave empty so the line disappeurs on the cover
% \numthese{« si pertinent »} % \numthese{}

%Indiquer le Pr\'{e}nom en minuscules et le Nom en majuscules, le titre de la personne et l’\'{e}tablissement dans lequel il effectue sa recherche  
%Indicates the first name on small letters and the Names on capital letters, the person's title and the institution where he/she belongs to.
%Exemples :  Examples :
%%%- Professeur, Universit\'{e} d’Angers 
%%%- Chercheur, CNRS, \'{e}cole Centrale de Nantes 
%%%-  Professeur d’universit\'{e} – Praticien Hospitalier, Universit\'{e} Paris V  
%%%-  Maitre de conf\'{e}rences, Oniris 
%%%- Charg\'{e} de recherche, INSERM, HDR, Universit\'{e} de Tours  
 %S’il n’y a pas de co-direction, faire disparaitre cet item de la couverture  
 %In there is no co-director, remove the item from the cover
\jury{
{\normalTwelve \textbf{Rapporteurs avant soutenance :}}\\ \newline
\footnotesizeTwelve
\begin{tabular}{@{}ll}
Laurent LEFÈVRE     & Professeur, ENS Lyon \\
Thomas LE DOUX      & Professeur, IMT Atlantique \\
\end{tabular}

\vspace{\baselineskip}
{\normalTwelve \textbf{Composition du Jury :}}\\
% {\fontsize{9.5}{11}\selectfont {\textcolor{red}{\textit{Attention, en cas d’absence d’un des membres du Jury le jour de la soutenance, la composition du jury doit être revue pour s’assurer qu’elle est conforme et devra être répercutée sur la couverture de thèse}}}}\\ \newline
\footnotesizeTwelve
\begin{tabular}{@{}lll}
Pr\'{e}sident :        & Pr\'{e}nom NOM      & Fonction et \'{e}tablissement d'exercice \\
Examinateurs :         & Laurent LEFÈVRE     & Professeur, ENS Lyon \\
                       & Thomas LE DOUX      & Professeur, IMT Atlantique \\
                       & Anne-Cécile ORGERIE & Directrice de recherche, IRISA \\
Dir. de th\`{e}se :    & Jalil BOUKHOBZA     & Professeur, ENSTA Bretagne \\
Co-dir. de th\`{e}se : & Laurent D'ORAZIO    & Professeur, Université de Rennes \\
                       & Olivier BARAIS      & Professeur, Université de Rennes \\
\end{tabular}

\vspace{\baselineskip}
{\normalTwelve \textbf{Invit\'{e}(s) :}}\\ \newline
\footnotesizeTwelve
\begin{tabular}{@{}ll}
Stéphane PAQUELET & Direction Adjointe Performance de l'Innovation, IRT b{\textless\textgreater}com \\
\end{tabular}
}


\maketitle


% Sélectionner la langue du contenu suivant cette ligne
% Select the content language following this line
\selectlanguage{french}

% Inclusion du chapitre remerciement
% Input acknowledgement chapter
\clearemptydoublepage
\input{./1_Acknowledgement/acknowledgement}

% Ne pas oublier cette commande qui g\'{e}n\`{e}re la page de couverture avant
% This command will generate the front cover
\frontmatter
\clearemptydoublepage
\renewcommand{\contentsname}{Table des Matières}
\tableofcontents %sommaire %table of content
%\shorttableofcontents{Sommaire}{0}

\clearemptydoublepage
\chapter{Introduction}
%\addcontentsline{toc}{chapter}{Introduction}
%\chaptermark{Introduction}

\section{Contexte scientifique}
%\addcontentsline{toc}{section}{Motivation}

En 1961, John McCarthy donne un discours pour célébrer les cent ans du MIT~\cite{greenberger1962management}. Il imagine alors que le partage du temps de calcul des ordinateurs pourrait permettre de vendre leur puissance d'exécution comme un service, à l'image de l'eau ou de l'électricité. Le matériel serait organisé de manière à rendre possible sa location à des clients qui paieraient ce service en fonction du volume, ou du temps d'utilisation.

Grâce à la démocratisation de l'accès à Internet à haut débit, au milieu des années 2000, l'idée de McCarthy se trouve implémentée dans ce que l'on appelle "cloud computing" : entreprises et particuliers peuvent dorénavant réduire drastiquement les coûts associés à l'achat et à l'entretien du matériel nécessaire au fonctionnement de leurs applications, en déléguant la responsabilité de l'infrastructure à des fournisseurs de services, qui bénéficient d'effets d'économie d'échelle en concentrant ces ressources dans d'immenses centres de données.

Ce modèle est appelé "Infrastructure as a Service" : les clients louent et réservent une sous-partie des ressources du fournisseur (calcul, stockage, réseau) dont ils deviennent responsables du fonctionnement~\cite{mellNISTDefinitionCloud}. De nouvelles tendances apparaissent au fil des années avec pour objectif de diminuer la surface des responsabilités du client. Par exemple, dans le modèle "Platform as a Service", les clients n'ont pas directement accès aux machines qui supportent leurs applications, et effectuent la plupart des tâches de gestion via des interfaces spécialisées.

Dans ces modèles, le client paie pour des ressources qui sont parfois dormantes. En effet, il faut la plupart du temps surprovisionner les ressources réservées, de manière à être capable d'absorber la montée en charge lors de pics d'activité.

TODO: nouveau modèle, serverless, quelques caractéristiques principales

TODO: cloud public, FaaS, cloud privé ?

\section{Problématiques de recherche}
%\addcontentsline{toc}{section}{Problématiques de recherche}

% \textbf{Problème 1} : Dans le cloud public, les ressources louées aux clients sont vues comme homogènes. Cela limite les gains du fournisseur de services en matière d'usage des ressources et de consommation d'énergie. Dans un contexte de cloud privé, le fournisseur peut ...

\textbf{Problème 1} : Tâches simples, usages variés, pas prédictibles $\Rightarrow$ orchestration dynamique sur ressources finies $\Rightarrow$ problème de performances (QoS) $\Rightarrow$ usage de ressources hétérogènes $\Rightarrow$ coûts d'opportunité $\Rightarrow$ ...

\boitemagique{Question 1 (\textbf{QR1})}{
    Comment dimensionner les allocations de ressources pour une application interactive, et comment ordonnancer efficacement les requêtes utilisateur, lorsque ces derniers ont des besoins hétérogènes en matière de qualité de service ?
}

% \textbf{Problème 2} : À l'edge, lorsque les ressources sont fortement contraintes (en capacité, en énergie), ... Ces problèmes s'accumulent dans un effet boule de neige et dégradent drastiquement la qualité de service lorsque l'on déploie des applications à plusieurs étages...

\textbf{Problème 2} : Applications complexes, dépendances temporelles et de données, charge variable $\Rightarrow$ orchestration dynamique sur ressources contraintes en performances et en énergie $\Rightarrow$ problème de contention (stockage) $\Rightarrow$ ...

\boitemagique{Question 2 (\textbf{QR2})}{
    Comment prendre en compte les délais d'initialisation et de communications lorsque l'on déploie à l'edge des chaînes de fonctions de courte durée, et comment tirer parti de l'hétérogénéité des nœuds à disposition, pour respecter la qualité de service requise par les utilisateurs et contenir la consommation d'énergie de l'infrastructure ?
}

\textbf{Problème 3} : Évaluer les politiques d'orchestration...

\boitemagique{Question 3 (\textbf{QR3})}{
    ...
}

\section{Contributions}
%\addcontentsline{toc}{section}{Contributions}

\subsection{...}

Pour répondre à la \textbf{QR1} (\textit{...}), ...

\subsection{...}

Pour répondre à la \textbf{QR2} (\textit{...}), ...

\subsection{...}

Pour répondre à la \textbf{QR3} (\textit{...}), ...

\section{Organisation de la thèse}
%\addcontentsline{toc}{section}{Organisation du mémoire}

\section{Publications}
%\addcontentsline{toc}{section}{Publications}


\part{Contexte et état de l'art}

\clearemptydoublepage
\mainmatter
\clearemptydoublepage
\chapter{Contexte}

\section{Cloud computing}

\subsection{Caractéristiques}

\begin{figure}[htbp]
    \centering
	\includegraphics[width=\textwidth]{3_Chapitre1/figures/service-models.png}
	\caption[Comparaison entre différents modèles de service pour le cloud en termes de responsabilités pour le client et le fournisseur de service.]{Comparaison entre différents modèles de service pour le cloud en termes de responsabilités pour le client et le fournisseur de service (inspiré de la documentation Red Hat \protect \footnotemark).}
	\label{fig:service-model}
\end{figure}

\footnotetext{\url{https://www.redhat.com/en/topics/cloud-computing/iaas-vs-paas-vs-saas}}

Le NIST donne une définition formelle du cloud~\cite{mellNISTDefinitionCloud} en listant les caractéristiques essentielles d'une telle plateforme :

\begin{itemize}
    \item \textbf{Service à la demande} -- Les clients réservent des ressources matérielles de manière autonome, par exemple au travers d'une interface web, sans interagir avec un opérateur. En retour, ils n'ont généralement pas de contrôle fin sur la localité précise des ressources réservées ;
    \item \textbf{Accessible par le réseau} -- Ces ressources sont immédiatement mises à disposition des clients et accessibles par Internet ;
    \item \textbf{Partage des ressources} -- La puissance de calcul, les capacités de stockage et la bande passante sont partagées entre les clients du fournisseur de services. Des techniques de virtualisation sont mises en œuvre pour isoler les tâches déployées ;
    \item \textbf{Élasticité rapide} -- Les clients peuvent à tout moment décider d'augmenter ou de diminuer la quantité et les caractéristiques des ressources qu'ils réservent, de manière à garantir les performances de leurs applications ou maîtriser leurs coûts ;
    \item \textbf{Service mesuré} -- Les infrastructures cloud sont instrumentées de manière à fournir aux clients une information précise sur leur consommation de ressources, et les coûts monétaires associés.
\end{itemize}

\subsection{Modèles de service}

Ces caractéristiques sont déclinées dans trois modèles de service :

\begin{itemize}
    \item \textbf{Software as a Service} (SaaS) -- Cible l'utilisateur final en offrant l'accès à une application entièrement administrée par le fournisseur de services ;
    \item \textbf{Platform as a Service} (PaaS) -- Cible les clients qui souhaitent déployer leurs applications sans avoir la responsabilité d'administration des serveurs ;
    \item \textbf{Infrastructure as a Service} (IaaS) -- Cible les clients qui souhaitent un contrôle à grain fin sur leurs infrastructures. Les clients d'une offre IaaS sont responsables de l'administration de leurs serveurs, souvent virtuels. 
\end{itemize}

\subsection{Modèles de déploiement}

Enfin, ...

\begin{itemize}
    \item \textbf{Cloud privé} -- L'infrastructure est dédiée à une organisation qui regroupe plusieurs utilisateurs finaux. Ce modèle de déploiement est privilégié pour la sécurité et la confidentialité des données ;
    \item \textbf{Cloud public} -- L'infrastructure est partagée entre de nombreux clients hétérogènes, professionnels comme particuliers ;
    \item \textbf{Cloud communautaire} -- L'infrastructure est partagée entre différents acteurs ayant souvent des problématiques métier similaires (secteurs bancaire ou hospitalier par exemple) ;
    \item \textbf{Cloud hybride} -- Solution de répartition des tâches entre cloud privé et cloud public, en fonction de leur niveau de criticité.
\end{itemize}

\section{Ordonnancement dans le cloud}

\subsection{Virtualisation dans un cadre multi-tenant}

\subsection{Dimensionnement et passage à l'échelle}

\subsection{Qualité de service et métriques de performances}

Problème : overbooking, overcommitting... conso d'énergie...

\section{Applications dans le cloud}

\subsection{Du monolithe aux micro-services}

\subsection{Développement cloud-native}

\subsection{Vers un nouveau modèle de service}coucou

\section{Conclusion}


\clearemptydoublepage
\chapter{Serverless, allocation et placement dynamiques dans le cloud : État de l'art}

\section{Serverless}

Le modèle serverless constitue un changement de paradigme dans le cloud public : par opposition aux modèles traditionnels, les clients serverless ne réservent pas de ressources matérielles. L'exécution de leur code est dirigée par des événements (requêtes HTTP, tâches programmées, etc.) et la facturation s'effectue sur la base de l'usage réel des ressources. En contrepartie, la responsabilité de l'allocation des ressources et du placement des tâches incombe au fournisseur.

\subsection{Allocation des ressources}

\subsection{Placement des requêtes}

\section{Conclusion}

En libérant les utilisateurs de la contrainte du dimensionnement de leur infrastructure, le modèle de service serverless pour le cloud promet de faciliter le passage à l'échelle des applications. Grâce au mécanisme d'allocation à la demande, les clients peuvent bénéficier d'économies considérables, en ne payant plus pour des ressources qui seraient essentiellement dormantes, en attente d'une requête.

Toutefois, les solutions serverless actuelles présentent des inconvénients non négligeables qui limitent l'utilisation du serverless à des cas d'usage spécifiques. Ce paradigme se réalise aujourd'hui sous la forme d'un contrat sur le modèle de programmation : les utilisateurs des offres serverless doivent concevoir leurs applications comme un ensemble de fonctions pures -- idempotentes, leur exécution n'entraîne pas d'effets de bord -- ce qui constitue un lourd effort d'ingénierie.

Le fonctionnement de cette architecture logicielle, qui présente des similitudes avec l'architecture en micro-services, repose sur la communication par passage de messages entre fonctions. Les fonctions n'étant pas directement adressables sur le réseau dans les solutions commerciales actuelles, cette communication s'effectue par le biais d'un stockage lent : cela induit un surcoût important sur les performances de l'application lors des phases de composition et de synchronisation, jusqu'à parfois contrebalancer les gains offerts par le parallélisme massif inhérent au paradigme serverless.

Par ailleurs, le passage à l'échelle depuis zéro est associé à un fréquent risque de latence lors du réveil de l'application, puisque le fournisseur de services doit alors dynamiquement allouer des ressources matérielles et instancier l'environnement d'exécution des fonctions pour répondre à l'événement déclencheur. Les fournisseurs de services ont tendance à pré-allouer des ressources de manière à éviter ces démarrages à froid, ce qui contraint leurs gains potentiels en rendant ces ressources indisponibles pour d'autres clients.

Enfin, les accélérateurs matériels sont les grands absents de l'offre serverless commerciale. À l'heure où la demande en GPU et FPGA est croissante pour répondre aux besoins en calcul massivement parallèle, notamment dans le cadre de l'apprentissage machine ou de l'analyse de données "big data", les clients doivent se tourner vers une offre cloud plus conventionnelle s'ils souhaitent bénéficier de plateformes d'exécution hétérogènes.


\part{Contributions}

\clearemptydoublepage
\chapter{HeROfake}

\section{Introduction}

\begin{table*}[t]
    \centering
        \caption{State of the Art work on autoscaling platforms}
        \begin{tabular}{lccccccc}
            \toprule
            & Serverless & Target cloud platform     & SLA & Hardware heterogeneity & Resources usage & Energy consumption & Cost-aware \\
            \cmidrule(lr){2-2}\cmidrule(lr){3-3}\cmidrule(lr){4-4}\cmidrule(lr){5-5}\cmidrule(lr){6-6}\cmidrule(lr){7-7}\cmidrule(lr){8-8}
            Swayam~\cite{gujaratiSwayamDistributedAutoscaling2017}        & \xmark         & Private (Azure, in-house) & \cmark& \xmark                     & \cmark            & \xmark                 & \xmark         \\
            Pigeon~\cite{lingPigeonDynamicEfficient2019}                  & \cmark       & Private                   & \xmark  & \cmark                   & \cmark            & \xmark                 & \xmark         \\
            MArk~\cite{zhangMArkExploitingCloud}                          & \xmark         & Public (AWS)              & \cmark& \cmark                   & \cmark            & \xmark                 & \cmark       \\
            ENSURE~\cite{sureshENSUREEfficientScheduling2020}             & \cmark       & Private                   & \xmark  & \xmark                     & \cmark            & \xmark                 & \cmark       \\
            Mampage et al.~\cite{mampageDeadlineawareDynamicResource2021} & \cmark       & Private                   & \cmark& \xmark                     & \cmark            & \xmark                 & \cmark       \\
            Atoll~\cite{singhviAtollScalableLowLatency2021}               & \cmark       & Private                   & \cmark& \xmark                     & \xmark              & \xmark                 & \xmark         \\
            INFless~\cite{yangINFlessNativeServerless2022}                & \cmark       & Private                   & \cmark& \xmark                     & \cmark            & \xmark                 & \cmark       \\
            SMIF~\cite{choSLADrivenMLInference}                           & \cmark       & Private                   & \cmark& \cmark                   & \cmark            & \xmark                 & \xmark         \\
            Target solution                                                & \cmark       & Private                   & \cmark& \cmark                   & \cmark            & \cmark               & \cmark       \\ \bottomrule
        \end{tabular}%
    \label{table:herofake-sota}
\end{table*}

%\subsection{The serverless service model}

\textbf{Serverless model}. Serverless can be understood as both a programming model, called Function as a Service (FaaS), and a deployment model for the cloud. In such a model, developers design their applications as a composition of stateless functions which execution is event-driven~\cite{SchleierSmith2021WhatSC}. 
Serverless services free tenants from complex resource reservation as they are designed to handle on-demand scaling requirements. % and address fluctuations in demand, therefore freeing tenants from the burden of having to define explicit scaling strategies.

In the FaaS model, providers only bill customers according to their actual resources usage~\cite{jonasCloudProgrammingSimplified2019}. They are fully responsible for deploying intelligent resource management and multiplexing at a finer granularity to optimize Quality of Service (QoS) metrics such as response time, energy consumption, etc. %while enabling providers to share physical resources at a higher degree of multiplexing, thus achieving better efficiency.

%\subsection{Deepfake and serverless}

\textbf{Deepfake detection and serverless}. The work presented in this paper was part of a project (at the b{\textless\textgreater}com research institute \footnote{\href{https://b-com.com/en}{https://b-com.com/en}}) aimed at deploying an energy efficient deepfake detection service in a heterogeneous cloud. Deepfakes are synthetic images, videos or speeches, digitally created to mimic an existing person so as to mislead viewers. Deepfake detection consists in training a Convolutional Neural Network (CNN) to detect patterns of inconsistencies that are introduced in the creation process.

The functions used by our deepfake application satisfy three main characteristics for suitable serverless workloads~\cite{cncf2018whitepaper}: their execution can be made parallel (several independent images and videos), they are stateless (pure transformation on input data) and event-driven (launched after data upload).

%On the one hand, deepfake detection using convolutional neural networks (CNNs) is a task that can take advantage of concurrency by running multiple convolutions in parallel and/or by handling different images on multiple threads. These tasks are essentially stateless, as they apply a pure transformation on input data -- taking an image as input and returning a boolean value as output. Such an application is event-driven, with computation starting following the upload of an input image. These are three main characteristics of suitable serverless workloads~\cite{cncf2018whitepaper}.

%On the other hand, the hardware resources needed to run that application at scale would be numerous and costly: being able to dynamically scale the resources in and out following demand would provide important savings to the client, and allow the provider to accept more clients on the same number of nodes.

%Therefore, we argue that the serverless service model is a perfect fit for cost-effective, on-demand inference using CNNs.

%\subsection{Hardware heterogeneity in the cloud}

\textbf{Hardware heterogeneity in the cloud}. Cloud infrastructures are more and more heterogeneous to fit the needs of data-intensive applications such as machine learning model training or big data analytics~\cite{reissHeterogeneityDynamicityClouds}. However, specialized processors and GPUs are yet to be made available to customers in serverless offerings~\cite{khandelwalTaureauDeconstructingServerless2020}. Hardware acceleration should be decided by the provider on a per-application or per-request basis.

State-of-the-art work shows that using such hardware in a cloud setting provides substantial gains in execution time and energy consumption~\cite{10.1145/3369583.3392679, 9195730}. However, reference orchestrators such as Kubernetes with Knative or OpenWhisk lack the support for dynamic allocation of such hardware. %capability to automatically and efficiently benefit from heterogeneous clouds. %deploy applications to nodes that provide relevant hardware to accelerate execution.

%with demand in GPUs and specialized processing cores driven by increasingly data-intensive computations with tasks such as machine learning model training or big data analytics. These heterogeneous computing units are yet to be made available to customers in serverless offerings~\cite{khandelwalTaureauDeconstructingServerless2020}. Hardware acceleration should be decided by the provider on a per-application or per-request basis.

%\subsection{Performance challenges}

\textbf{Performance challenge for serverless deployment}. Due to the transient nature of unreserved FaaS resources, latency, throughput and continuity of service are hard to guarantee~\cite{vaneykSPECRGCloud2018, dartoisCuckooOpportunisticMapReduce2019}. When applications do not receive incoming requests, function sandboxes are destroyed instead of being kept in an idle state. Then, when a new request arrives, the provider has to (re)allocate resources and initialize functions to deploy new sandboxes: this is called a cold start. Cold start times are very penalizing for the application performance, they may even dominate the total execution times~\cite{mullerLambadaInteractiveData2020}.

Furthermore, in current commercial serverless offerings, Service-Level Agreements (SLAs) are usually limited to automated retries (restarts) on failure, and FaaS providers generally limit the execution time of serverless functions to a few minutes. The absence of QoS guarantees in commercial serverless offerings prevents them from being more widely used~\cite{buyyaSLAorientedResourceProvisioning2011}.

%\subsection{Problem statement}

\textbf{Problem statement -- putting it all together}. The problem we try to solve in this paper is to determine how to automatically and reactively scale heterogeneous hardware resources in a cloud in adequacy with the application's load and the users' QoS requirements, while keeping the cost in resources and energy as low as possible for the provider. We consider a deepfake detection application as a case study in our work.

%Rather than reserving nodes in their infrastructure for an interactive application which load is unpredictable~\cite{shahradServerlessWildCharacterizing}, the provider can rely on a serverless platform for cost efficiency. The platform will automatically and reactively scale hardware resources in adequacy with the application's load.

%To be efficient in terms of cost and energy consumption, an autoscaling platform should be able to rightsize the allocated resources in a serverless cloud infrastructure, while being responsive enough to accommodate workload changes without impacting end users with spikes in latency. In order to meet per-user QoS requirements, it should take into account the characteristics of heterogeneous hardware resources, and SLAs should be negotiated on a per-request basis rather than on a per-function basis.

%\subsection{State of the Art}
\textbf{State-of-the-art}. Previous studies have explored the need for an autoscaling platform that supports short-running tasks comprised in applications such as Machine Learning as a Service. Table~\ref{table:herofake-sota} summarizes how these solutions differ from the target platform we are trying to achieve, and Section~\ref{section:herofake-sota} provides further details. While many have established the need for on-demand acceleration as a solution to guaranteeing function response time, none have measured the impact of leveraging heterogeneous resources on dynamic energy consumption. Furthermore, previous studies consider task consolidation as a means to free resources for further computations -- we argue that such techniques open possibilities for the service provider to apply power saving policies in private clouds. Finally, as serverless platforms are general purpose and designed to be highly configurable, our target solution should be cost-aware to allow the provider to make configuration choices pertaining to their own objectives.

%\subsection{Our contribution}
%\jb{à revoir ... j'ai l'impression que ça ne décrit pas correctement la contrib}

\textbf{Our contribution}. We argue that opportunistically taking advantage of hardware accelerators (GPUs and FPGAs) to schedule deepfake detection tasks may allow cloud providers to guarantee serverless tasks response time and achieve SLA while reducing resource usage and energy consumption.

In this paper, we propose a full framework to deploy a deepfake detection application on a serverless cloud. This framework comprises an offline and an online phase. The \textbf{offline phase} is used to characterize the performance and energy behavior of the deployed heterogeneous hardware platforms. The \textbf{online phase} consists of an autoscaling platform and a scheduling strategy that make efficient use of (characterized) heterogeneous hardware resources to achieve per-request SLAs while reducing the energy consumption of the platform. 

%\begin{itemize}
%    \item platform-level sandbox caching to minimize cold start delays \jb{tu fais des choses explicite sur le sujet ?};
%    \item QoS requirements at the granularity of a task to avoid penalties.
%\end{itemize}

%This is made possible by implementing these tasks for different hardware architectures, and measuring their performance and energy consumption ahead of scheduling (see Section~\ref{characterization}). \jb{tu insistes sur les inputs, plutôt que sur les techniques }

For this case study, we devised a simulation environment\footnote{The simulator repository will be made publicly available.} that models the infrastructure for a deepfake detection application, run by the provider as a Software as a Service using a serverless infrastructure.   %\vl{oui, il faut que je fasse la demande de divulgation de propriété intellectuelle à bcom - montre moi le mail d'abord, il faut insister sur l'utilité de la mise en ligne}

%By factoring in workload characterization (see Section~\ref{characterization}) in order to predict tasks performances on different execution platforms, we are able to avoid pre-allocating resources while keeping the frequency of cold starts under 0.02\% of function startups.

\textbf{Some performance figures}. With our allocation and scheduling policy, we were able to handle 50000 tasks in the same makespan as Knative with less than 36\% QoS penalties. Our framework reduces energy consumption for the execution of tasks by almost 35\%, and provides the opportunity for the provider to further reduce static power consumption by consolidating tasks on less than 29\% of the available nodes.

The paper is organized as follows: in a first section, we describe the overall platform model for the project. Then, we describe the execution platform and workload characterization phase. In Section III we describe the challenges of serverless resource orchestration, our task model and the orchestrator's allocation and scheduling policies. Section IV presents our evaluation methodology and a discussion of the experimental results. Section V gives details regarding state-of-the-art work on autoscaling platforms. Finally, we conclude with some perspectives for future work.

\section{Deploying Deepfake Detection Tasks in Serverless Cloud}

\begin{figure*}[t]
\centering
\includegraphics[width=0.8\textwidth]{5_Chapitre3/figures/placement.png}
\caption{Serverless deepfake detection platform, system overview}
\label{figure:herofake-placement}
\end{figure*}

This section introduces the used serverless platform model and the overall project. % and the lifecycle of a request. 

%Figure~\ref{figure:herofake-placement} highlights (with a red background) the two main parts of our contribution \jb{focale sur 2 point alors que le reste n'est pas encore exposé, bizarre} \vl{déplacé à la fin, mais peut-être qu'on peut juste s'en débarasser ou reformuler ?}:
%\jb{ça perturbe la lecture --> à enlever/commenter on verra après si on remet}
%\begin{enumerate}
%    \item a \textbf{measurement node}, that generates task metadata used to guide the allocation and scheduling decisions.
%    \item an \textbf{orchestrator}, that provides a gateway as an entry point for user requests, an autoscaler responsible for resources allocation, and a scheduler that places jobs on previously allocated resources;
%\end{enumerate}

\subsection{Platform model}

We consider a deepfake detection system that is deployed as a serverless application consisting of three stateless functions that achieve inference tasks on input images. These images are all RGB and $224 * 224$ pixels in size~\footnote{Note that videos are not yet considered in our project.}.

Figure~\ref{figure:herofake-placement} introduces the used platform, we differentiate between an \textit{offline} (blue blox in the figure) and an \textit{online} (green box in the figure) phase. During the offline phase, we collect metadata relative to tasks execution on the heterogeneous accelerators; during the online phase, we allocate resources and schedule tasks.

Function invocation requests from the users are received on the provider's gateway and handled by the orchestrator. In our model, a function invocation corresponds to a \textit{task}. The user selects one of the three provided models (ResNet50, VGG16 and VGG19, see Section~\ref{offline:workload}) and uses it to detect a possible deepfake on a picture.

The cloud provider's infrastructure is modeled as a set of heterogeneous \textit{nodes} (Section~\ref{model:nodes}) comprising various combinations of \textit{platforms} (Section~\ref{model:platforms}) that can execute incoming \textit{tasks} (Section~\ref{model:tasks}). 

\subsubsection{Nodes} \label{model:nodes}
A node is a server available in the service provider's infrastructure. In this work, we do not consider storage and data locality. Input data are always provided \textit{via} file upload by the user at the time of their request.
As such, the only characteristic that defines a node in our infrastructure model is the size of dedicated memory. A node consists of a set of execution platforms defined hereafter.

\subsubsection{Execution platforms} \label{model:platforms}

An execution platform is a hardware processing unit available on a node. Each platform consumes a quantity of energy in the "idle" state expressed in kilowatt-hour (kWh). When it starts executing a task, it consumes additional energy characterized by the task's properties/type: it is then in an "active" state. We differentiate "idle" and "active" time for each platform, so as to measure resources usage.
Platforms are characterized by a \textit{platform type} that encompasses the following parameters:

\begin{itemize}
    \item \textit{Hardware type} -- CPU, GPU or FPGA;
    \item \textit{Price} -- the cost of acquisition of such a platform by the cloud provider;
    \item \textit{Idle energy} -- the baseline energy consumption for the platform when it is not running any task.
\end{itemize}

\textbf{Task caching and cold start model}. We consider a simple task caching mechanism at platform-level, akin to a keep-alive mechanism~\cite{7279063}. In our system, if a platform was previously executing a task of type $t$, and a new task of the same type $t$ is scheduled on that same platform, then the cold start delay will not be incurred. However, if that same platform were to execute a task of different type $tt$, then the task will go through a cold start before entering its execution phase. Finally, if the platform was not previously allocated, the task will also experience a cold start delay.

\subsection{Overall project description}

The b{\textless\textgreater}com research institute works on a project that aims to deploy an application of deepfake detection on a private cloud. Users submit a picture to the system and when their request is fulfilled, they obtain a boolean value as a response. The application targets different classes of users -- some of them can be media or authorities with high QoS requirements, while others can be casual users tolerating a higher latency.

To differentiate between these classes of users, we propose different levels of per-request SLA. Users with higher requirements will agree to pay a higher per-request price, however if we fail to fulfill their request in the allotted response time, we will consent to a discount -- the higher the QoS level, the higher the discount. Hence, there is a strong monetary incentive for the provider to achieve QoS.

\textbf{Offline phase}. In our platform, the lifecycle of the application starts during an offline phase with the developer providing the code of their functions for different hardware architectures \Circled{1}. That code is stored in a function repository. Functions are then deployed on a measurement node \Circled{2} where they are run in order to generate metadata relative to the functions: memory requirements, execution time, cold start time and energy consumption for each function are written to a metadata store \Circled{3}. The offline phase is required to run once for a given function on a given platform, it is described in Section~\ref{offline}.

\textbf{Online phase}. When a user sends a request to the application \Circled{4}, they provide an input picture and specify their desired QoS level. The request is appended to a request queue \Circled{5} at the orchestrator level. When the scheduler pops the request from the queue, the metadata store is queried to retrieve the appropriate function metadata \Circled{6}.

The scheduler then proceeds to try to schedule a task (i.e. a function's invocation) to fulfill the request. Tasks are placed on already deployed function \textit{replicas} \Circled{7}. Such replicas can either be containers or virtual machines, i.e. sandboxed execution environments for the given function. 
Concurrently, the autoscaler monitors the request queues in all the function replicas \Circled{8}. The role of the autoscaler is to rightsize the resources allocation with regard to the fluctuations in load on each function. 
The designed scheduler and the autoscaler are described in Section~\ref{online}.

%gives further details regarding the scheduling strategy.
%Section~\ref{section:herofake-autoscaling-strategy} gives further details regarding the autoscaling strategy.
\section{Offline phase: measurement and metadata extraction}
\label{offline}
\subsection{Execution platform benchmarks}

\begin{table}[t]
\caption{Execution platform characterization}
\begin{center}
\resizebox{\columnwidth}{!}{%
\begin{tabular}{|c|c|c|c|c|}
\hline
                             \textbf{Platform} & \textbf{Hardware type}& \textbf{Price (MSRP)} & \textbf{Idle energy} \\ \hline
Intel Xeon ES-1620 v4         & CPU           & 294          & 0.067       \\ \hline
Nvidia GeForce RTX 2070 Super & GPU           & 499          & 0.010       \\ \hline
Xilinx Alveo U250             & FPGA          & 7695         & 0.030       \\ \hline
\end{tabular}%
}
\end{center}
\label{table:herofake-platforms}
\end{table}

As the usage of deep learning inference and energetic impacts grow simultaneously in computing, the power efficiency of the target devices becomes a major concern. The FPGA-based acceleration boards are described as a relevant competitor against the dominant GPU approach. Our study proposes a benchmark, using convolutional neural network (CNN)-based approaches for deepfake detection on CPU, GPU, and FPGA technologies regarding power efficiency during inference time. Our comparison includes power usage, inference speed, and accuracy using traditional CPU and GPU processing against FPGA. Those metrics are crucial for an efficient orchestration on top of heterogeneous platforms.

%For our experiments, the performance comparison has been made between a CPU, a GPU, and a FPGA. 
The used CPU was an Intel Xeon CPU ES-1620 v4 (3.5 GHz) while the GPU was an Nvidia GeForce RTX 2070 Super which can be used with the new versions of AI frameworks. Therefore, both were compatible with TensorFlow, i.e the platform used for inference. with regard to the FPGA, we used the Alveo U250, a cloud computing card from Xilinx, which is compatible with Vitis-AI~\cite{vitis-ai}. The silicon processes used for both devices are similar (12 nm for the GPU and 16 nm for the FPGA), but the GPU may get a slight advantage in this benchmark for its more advanced silicon technology.

In order to carry out the inference on the FPGA, we used Vitis-AI. At the time of this study, the latest available version (v. 2.0) has been used. Vitis-AI proposes two methods for the optimization of the models. The first one is the pruning, which consists in a reduction of the complexity of the model by a compression while removing some non-critical sections of the tree. The second one is the quantization, where we convert the 32 bits floating weights into 8 bits integer. The latter method, which is freely available, is the method we used to optimize our model before the compilation, which converts our model into DPU (Deep Learning Processing Unit) instructions.

\subsection{Workload characterization}
\label{offline:workload}

\begin{table}[t]
\caption{Workload characterization }
\centering
\resizebox{\columnwidth}{!}{%
\begin{tabular}{|c|cc|ccc|ccc|ccc|}
\hline
Task     & \multicolumn{2}{c|}{Memory (GB)} & \multicolumn{3}{c|}{Cold start (s)}                              & \multicolumn{3}{c|}{Execution time (s)}                         & \multicolumn{3}{c|}{Energy (mWh)}                            \\ \hline
         & \multicolumn{1}{c|}{CPU}  & GPU  & \multicolumn{1}{c|}{CPU}   & \multicolumn{1}{c|}{GPU}   & FPGA   & \multicolumn{1}{c|}{CPU}   & \multicolumn{1}{c|}{GPU}   & FPGA  & \multicolumn{1}{c|}{CPU}  & \multicolumn{1}{c|}{GPU}  & FPGA \\ \hline
ResNet50 & \multicolumn{1}{c|}{1.3}  & 3.3  & \multicolumn{1}{c|}{1.232} & \multicolumn{1}{c|}{2.340} & 9.952  & \multicolumn{1}{c|}{0.124} & \multicolumn{1}{c|}{0.024} & 0.009 & \multicolumn{1}{c|}{3.11} & \multicolumn{1}{c|}{1.7}  & 0.5  \\ \hline
VGG16    & \multicolumn{1}{c|}{1.8}  & 3.3  & \multicolumn{1}{c|}{2.514} & \multicolumn{1}{c|}{4.641} & 14.528 & \multicolumn{1}{c|}{0.143} & \multicolumn{1}{c|}{0.046} & 0.010 & \multicolumn{1}{c|}{4.34} & \multicolumn{1}{c|}{3.43} & 0.55 \\ \hline
VGG19    & \multicolumn{1}{c|}{1.9}  & 3.4  & \multicolumn{1}{c|}{2.559} & \multicolumn{1}{c|}{4.641} & 14.758 & \multicolumn{1}{c|}{0.167} & \multicolumn{1}{c|}{0.048} & 0.012 & \multicolumn{1}{c|}{5.16} & \multicolumn{1}{c|}{3.58} & 0.65 \\ \hline
\end{tabular}
}%
\label{table:herofake-tasks}
\end{table}

For the purpose of this study, three popular models have been trained. The first one is based on residual networks (ResNet50), which uses residual blocks and can be efficiently trained~\cite{NEURIPS2019_7716d0fc}. The second one is VGG16 (VGG for Visual Geometry Group), which uses only convolutions as blocks~\cite{DBLP:journals/corr/SimonyanZ14a} and the third one is VGG19, a variant of VGG16 with three additional layers~\cite{biom10070984}. Those networks are trained on a GPU, as training is not the subject of this study.

\subsection{Performance measurements}

\begin{figure}[t]
\centering
\includegraphics[width=\columnwidth]{5_Chapitre3/figures/characterization/time_of_inference_1_image.png}
\caption{Inference time for one image with ResNet50, VGG16 and VGG19.}
\label{figure:herofake-time-inference}
\end{figure}

%The first consideration in this benchmark is to compare the performance in terms of inference time with the different technologies. 
As the FPGA acceleration card is intended to be more efficient than a CPU or a GPU~\cite{5272532}, making a comparison of the inference time with these three technologies is a first requirement to enable the comparison of the energy cost per image. The performance evaluation in terms of execution time was realized with the same 10,000 images for the three different models. We built a two classes deepfake dataset, the real ones from the CelebA dataset~\cite{https://doi.org/10.48550/arxiv.1411.7766}, and the fake ones generated using a Generative Adversarial Network (GAN)~\cite{jimaging7080128}. The quantization and compilation of the graph was performed with Vitis-AI in order to run it on the FPGA. Only considering the inference time, it turned out that on the three tested models (ResNet50, VGG16 and VGG19), the FPGA is 13.08 to 13.79 times faster than the CPU but also 2.52 to 4.48 times faster than the GPU (see Figure~\ref{figure:herofake-time-inference}).

\subsection{Energy consumption measurements}

\begin{figure}[t]
\centering
\includegraphics[width=\columnwidth]{5_Chapitre3/figures/characterization/consumption_per_image.png}
\caption{Energy consumption of inference per image (mWh).}
\label{figure:herofake-consumption-per-image}
\end{figure}

The instant power consumption measured during inference is  the overall consumption of the machine (including CPU, memory, mainboard, and power supply) while performing the inference. 
Measurements have been done using a power distribution unit (PDU) (Raritan PX3-5190R) which is able to monitor instant power and energy consumption of the server (Dell Precision T5810). The results show that inference on CPU yields the lowest instant power consumption. This result is quite expected as the inference on GPU or FPGA also includes power consumption from the CPU. % the power usage of the GPU and the FPGA adds-up to the CPU power.


However, the sole instant power consumption does not reflect the total cost advantage of each platform properly. The execution time needed to process all the images must be considered. The relevant measurement is energy cost per image. The energy consumption has been measured in kilowatt-hour (kWh) for the 10,000 images, then converted into milliwatt-hour (mWh) per image. From that point of view, it is clear that the FPGA is the most energy efficient with regard to the execution time, consuming 6.2 to 6.9 times less than the CPU and 3.3 times to 6.2 times less than the GPU (see Figure~\ref{figure:herofake-consumption-per-image}).

\subsection{Discussion}

\begin{figure}[t]
\centering
\includegraphics[scale=0.2]{5_Chapitre3/figures/characterization/cost_devices_time.png}
\caption{Total cost of inference on selected devices over time.}
\label{figure:herofake-cost-over-time}
\end{figure}

The results of this benchmark show a clear advantage of the inference on FPGA regarding performance and energy efficiency. The gains in performance are significant, especially with deep learning networks with higher complexity~\cite{8782524}. Computing resources based on servers equipped with FPGA acceleration boards, instead of GPU acceleration boards, would benefit from these advantages.

The raw energy consumption of the inference device does not reflect the total cost of the solution. Indeed, one must also include the cost of the equipment itself. This is a major point in the comparison between GPU and FPGA, because there is a price gap between the two technologies: the GPU (RTX 2070 Super) being used for this benchmark was introduced around 600€, while the FPGA (Alveo U250) is sold around 6000€. The cost of the electric energy to perform the inference is very low (we used the European average of 0.1833€ per kWh as proposed in~\cite{energy-price}), compared to the initial cost of the device: the runtime needed to benefit from the cost advantage of the FPGA is in the order of several months of continuous operation. Figure~\ref{figure:herofake-cost-over-time} depicts cumulative cost (in euros) of the usage of a server with either GPU or FPGA acceleration versus time (in years). Our cost estimate includes the number of GPU needed with their cost to equalize the FPGA performances and uses a 2x factor~\cite{shehabiUnitedStatesData2016}, to account for the total power consumption of the infrastructure (mainly cooling and networking). The FPGA can become a cost effective solution after a few months for complex CNNs. For networks with lower complexity, the cost advantage of the FPGA is reached after more than two years.

\begin{figure}[t]
\centering
\includegraphics[width=\columnwidth]{5_Chapitre3/figures/characterization/power_usage.png}
\caption{Power usage breakdown for FPGA and GPU.}
\label{figure:herofake-power-usage}
\end{figure}

The previous analysis is valid in the situation where the inference is always performed at full load. Indeed when breaking down the power consumption of the GPU between idle power and inference power, it is clear that the GPU is able to dynamically scale its power usage with the intensity of the processing. The FPGA on the other side seems to have very limited power management. Once the DPU design is loaded into the device, its power usage at idle remains very high (see Figure. \ref{figure:herofake-power-usage}). Adding to the 38W of the FPGA board power, there is indeed a residual 60W power consumption when the DPU is idle. Even though further evolution of the DPU implementation on the FPGA may fix this issue (like reducing clock tree activity when idle), this has an impact on the total cost and must be considered if the device is not always used at full load. With only 12W of idle power, the GPU is a better candidate when full-load device usage cannot be guaranteed.

As the trend towards CNNs with more complexity continues~\cite{8807741}, using the most efficient devices will become a major challenge. The FPGA solution offers a new option to perform inference. However, FPGAs are not a drop-in replacement for GPUs yet: the compilation flow remains complex and time-consuming. A trade-off between the flexibility of the GPUs and the efficiency of the FPGAs will have to be made. The next section discusses a first orchestrator that considers the above-mentionned characterization for allocating and scheduling heterogeneous resources.

\section{Online phase: Autoscaling and scheduling}
\label{online}
In this section, we formulate the problem that our contribution addresses, and give a detailed description of our model. Finally, we present a formal description of our strategy for the autoscaling of resources and scheduling of tasks. 

%In order to optimize the placement of tasks following incoming requests, we propose an online scheduler that takes into account the level of QoS requirements associated with each task. To meet these requirements, the autoscaler has to be able to opportunistically allocate hardware accelerators according to the task's deadline, given its characteristics on each execution platform. These characteristics are made available in the task's metadata.

\subsection{Serverless resource orchestration challenges}

Scheduling workloads in the serverless paradigm is a two-fold problem: providers have to dynamically handle resource allocation (i.e. managing resources pools when scaling the number of replicas for an application) and job placement (i.e. mapping tasks to existing replicas).

Increasing the replica count introduces a performance challenge: when a new replica is spun up, be it as a container or a virtual machine, the execution sandbox has to go through its initialization phase. This is called a "cold start".

Commercial solutions such as AWS Lambda often avoid the cold start problem by maintaining pools of pre-warmed sandboxes~\cite{vahidiniaColdStartServerless2020}. These virtual machines (VMs) or containers are started in anticipation and paused in a post-initialization state. When activity resumes, incoming requests can be served without suffering a cold start delay, at the expense of resources multiplexing on the provider side. While this solution allows reducing, or even eliminating cold start delays, it takes a toll on the provider's resources multiplexing capacity~\cite{hellersteinServerlessComputingOne2019} and raises the total cost of ownership (TCO).

% Load is unpredicTable~\cite{shahradServerlessWildCharacterizing}
%\begin{itemize}
%    \item stochastic barrier;
%    \item need for an online solution.
%\end{itemize}

%Furthermore, Machine Learning as a Service (MLaaS) applications exhibit highly fluctuating load~\cite{gujaratiSwayamDistributedAutoscaling2017}, thus reinforcing the argument that overprovisioning resources for such a service is not a viable solution.

Furthermore, Machine Learning as a Service (MLaaS) applications exhibit highly fluctuating load~\cite{gujaratiSwayamDistributedAutoscaling2017}, thus reinforcing the argument that a reactive resource allocation strategy is necessary to rightsize the infrastructure. However, as the execution time of inference tasks is in the ballpark of hundredths to tenths of a second, while the initialization time of sandboxes can range between hundredths of a second to seconds~\cite{mancoMyVMLighter2017}, we need a mechanism to avoid incurring huge latency costs to the execution of functions.

% Service level is not guaranteed
%\begin{itemize}
%    \item resources availability over time is not the most relevant QoS metric;
%    \item risks of overcommitting resources.
%\end{itemize}

Mission critical tasks require service level guarantees from the provider. SLAs in cloud computing typically consist in agreeing on a resource availability rate over time; if the provider fails to meet this agreement, a discount is offered to the customer. While this may work for reserved resources, we can see that it does not make sense in the serverless paradigm. The ability to guarantee task response time would allow a serverless provider to achieve per-invocation SLAs~\cite{zhangMArkExploitingCloud}.

% Cloud resources are heterogeneous
%\begin{itemize}
%    \item various levels of performance;
%    \item various levels of cost.
%\end{itemize}

A possibility to improve performance-cost ratios is to make use of hardware accelerators. Despite being a costly investment (see Figure~\ref{figure:herofake-cost-over-time}), these devices can achieve important speedups for parallel tasks (see Figure~\ref{figure:herofake-time-inference}), thus improving function response time, with a decreased cost in energy (see Figure~\ref{figure:herofake-consumption-per-image}).

\subsection{Task model} \label{model:tasks}

\begin{table}[t]
    \caption{Notation dictionary}
    \begin{center}
    \begin{tabular}{|c|L|}
    \hline
    \textbf{Notation} & \textbf{Description} \\ \hline
    $f_{N, P}$ & A function $f$ scheduled to run on a platform $P$ available on node $N$ \\ \hline
    $QP$ & QoS penalty \\ \hline
    $QD$ & QoS deviation \\ \hline
    $WET$ & Worst execution time \\ \hline
    $TT$ & Task total time \\ \hline
    $WT$ & Wait time \\ \hline
    $CST$ & Cold start time \\ \hline
    $ET$ & Execution time \\ \hline
    $EC$ & Energy consumption \\ \hline
    $HP$ & Hardware price \\ \hline
    $TC$ & Task consolidation \\ \hline
    $Q$ & Task queue on a replica \\ \hline
    $replicaCount_{f}$ & Size of the replica pool in the system for a function $f$ \\ \hline
    $concurrency_{f}$ & Average number of in-flight requests for a function $f$ \\ \hline
    $threshold$ & Concurrency threshold for function replicas in vanilla Knative \\ \hline
    $replicaCount_{f, h}$ & Size of the replica pool for a function $f$ on hardware type $h$ \\ \hline
    $concurrency_{f, h}$ & Average number of in-flight requests for a function $f$ on replicas of hardware type $h$ \\ \hline
    $x_{f, h}$ & Concurrency threshold for a function $f$ on a replica of hardware type $h$ \\ \hline
    $scaleCost_{{f}_{N, P}}$ & Cost of creating a new replica for function $f$ on a platform $P$ available on node $N$ \\ \hline
    $schedCost_{{f}_{N, P}}$ & Cost of scheduling an execution of function $f$ on a platform $P$ available on node $N$ \\ \hline
    \end{tabular}
    \label{table:herofake-notation}
    \end{center}
\end{table}

Applications are composed of functions. A function execution is called a \textit{task}. In this work, there are no dependencies between these tasks: the application is made up of pure, stateless functions. The events that trigger the execution of a task arrive in the system at a random, bounded interval. We formulate the hypothesis that a request always succeeds and leads to the execution of a \textit{task} (an instance of a \textit{function}). When a task has started its execution on its allocated platform, it runs for the totality of its execution time. We do not consider preemption or failures in this contribution: a task always finishes its execution successfully, albeit its response time can exceed its QoS requirements. We do not consider possible interference between workloads on the same node~\cite{dartoisInvestigatingMachineLearning2021}. 

We consider tasks that can indiscriminately be executed on heterogeneous execution platforms. In the context of our specific case study, implementation of the different functions has been done by hand for each platform; however, work exists to allow automatic cross-compilation to heterogeneous architectures~\cite{hortaXartrekRuntimeExecution2021, 10.1145/3445814.3446699}. The following metadata have been measured for each function, on each execution platform: 

%In our task model, a \textit{vector} \jb{pas clair cettte histoire de vecteur}expresses a value that can be different according to the execution platform (CPU, GPU or FPGA) allocated for the task's execution. Each function is associated with its metadata:

\begin{itemize}
    \item \textit{Memory requirements} -- the quantity of memory (in GB) allocated for the task;
    \item \textit{Cold start duration} -- the duration of sandbox initialization when running the task on a platform that does not have the function in cache;
    \item \textit{Execution time} -- the expected duration of effective execution of the task, excluding its initialization phase;
    \item \textit{Energy consumption} -- the difference between idle and active energy incurred by the execution platform when it runs the task.
\end{itemize}

Equation~\ref{eq:herofake-HRO-total-time} breaks down the expected response time for the execution of a function $f$ on a platform $P$ on node $N$.
\begin{equation}
\begin{split}
%\resizebox{0.80\columnwidth}{!}{%
    %\begin{math}
    {TT}_{{f}_{N, P}} = {WT}_{{f}_{N, P}} + {CST}_{{f}_{N, P}} + {ET}_{{f}_{N, P}}
    %\end{math}%
%}
\end{split}
\label{eq:herofake-HRO-total-time}
\end{equation}

Where:
\begin{itemize}
    \item ${WT}_{{f}_{N, P}}$ is the duration of the scheduling decision, including the time spent by the user request in the queue;
    \item ${CST}_{{f}_{N, P}}$ is the duration of initialization for the function invocation, including its potential cold start time; 
    \item ${ET}_{{f}_{N, P}}$ is the execution time of the function on the platform.
\end{itemize}

We propose different levels of QoS depending on users' needs in terms of guarantees on execution time. Each level of QoS presents a different \textit{duration deviation} (noted $QD$ in Equation. ~\ref{eq:herofake-task-penalty}) -- a factor by which the worst execution time for a function is multiplied to give an upper bound on the execution time of this function for this QoS level.

Predicted function execution time is always based on the worst execution time (noted $WET_{f}$), e.g. a task's execution time when scheduled on the execution platform showing the least level of performance for said function:

\begin{equation}
\begin{split}
    \forall \, (N, P), \, WET_{f} = \max ET_{N, P}
\end{split}
\label{eq:herofake-task-wet}
\end{equation}

After a task is scheduled on an execution platform, it will go through its total time of execution described in equation \ref{eq:herofake-HRO-total-time}. The task deadline is computed by multiplying the function's worst response time (as expressed in Equation~\ref{eq:herofake-task-wet}) by the QoS duration deviation associated to the user's request's QoS level. Equation~\ref{eq:herofake-task-penalty} shows that we set a boolean value $QP_{f_{N, P}}$ for each function invocation if the tasks misses its deadline. 

\begin{equation}
\begin{split}
    QP_{f_{N, P}} = TT_{f_{N, P}} \cdot QD_{f_{N, P}} > WET_{f}
\end{split}
\label{eq:herofake-task-penalty}
\end{equation}

\subsection{Autoscaling strategy} \label{section:herofake-autoscaling-strategy}

In a serverless platform, the autoscaler has the responsibility to allocate hardware resources for function executions. For any function, an autoscaler can allocate $n$ \textit{replicas}. The number of replicas for a given function at any moment determines the concurrency level.

In Knative, the number of replicas for a given function (Equation~\ref{eq:herofake-kn-replica-count}) depends on the moving average load for a function, i.e. the average number of in-flight requests for the function on a 60 second window (in-system concurrency per function). It is bounded by a concurrency threshold per replica, i.e. the maximum number of requests queued in a function's replica at any moment. The default value in Knative is 100 in-flight requests in each replica~\cite{knative-autoscaling}.

\begin{equation}
\begin{split}
    replicaCount_{f} = \frac{concurrency_{f}}{threshold}
\end{split}
\label{eq:herofake-kn-replica-count}
\end{equation}

This sizing mechanism allows allocating CPUs under Knative, in reaction to changes in the current state of concurrency in the system. The main contribution of the autoscaler we propose is to upgrade Knative in order to take into account the heterogeneity of the execution platforms.

Simple Knative mechanism does not hold when the infrastructure consists of a variety of execution platforms. Indeed, such platforms exhibit various levels of performance, energy consumption and cost. This has a consequence on the number of replicas the provider has to deploy on these platforms: for a given level of application load, heterogeneous replicas will be able to handle different numbers of tasks in the same makespan. For our platform to handle heterogeneity in the underlying infrastructure, we propose a per-function and \textbf{per-hardware type} replica count as in Equation~\ref{eq:herofake-HRO-replica-count}.

\begin{equation}
\begin{split}
    replicaCount_{f, h} = \frac{concurrency_{f, h}}{x_{f, h}}
\end{split}
\label{eq:herofake-HRO-replica-count}
\end{equation}

An autoscaling decision can introduce opportunity costs in the system: hardware accelerators are scarcely available as compared to CPUs, and allocating them for a given function at a given time will make them unavailable for further computations. In order for the autoscaler to decide when it is relevant to allocate such accelerators, it has to be \textbf{cost-aware}. 

In order to determine the concurrency threshold per replica $x_{f, h}$ for a function $f$ on hardware type $h$ (e.g. GPU and FPGA), we fixed the concurrency threshold per replica on CPUs to $x_{f, c} = 100$ as it is the default value in Knative~\cite{knative-concurrency}. Then, we used the measurements from the offline phase (Table~\ref{table:herofake-tasks}) to establish a composite ratio (including performance, energy, platform price) as described in Equation~\ref{eq:herofake-HRO-concurrency-target}. In our policy, we chose to favor response time by setting $k_{ET} = \frac{2}{3}$, $k_{EC} = \frac{1.5}{6}$ and $k_{HP} = \frac{0.5}{6}$. For example, for the function ResNet50 (described in Table~\ref{model:tasks}), task queues in replicas are sized to 100 for CPUs, 489 for GPUs and 1292 for FPGAs.

\begin{equation}
\begin{split}
\resizebox{0.90\columnwidth}{!}{%
    \begin{math}
    x_{f, h} = x_{f, c} \cdot (k_{ET} \cdot \frac{ET_{{f}_{c}}}{ET_{{f}_{h}}} + k_{EC} \cdot \frac{EC_{{f}_{c}}}{EC_{{f}_{h}}} + k_{HP} \cdot \frac{HP_{{f}_{c}}}{HP_{{f}_{h}}})
    \end{math}%
}
\end{split}
\label{eq:herofake-HRO-concurrency-target}
\end{equation}

When the concurrency threshold for a function is exceeded in the queues of replicas on a given hardware type, the autoscaler will proceed to \textit{scale out} the function: a new replica will be spun up to handle further user requests.

Allocation starts with the complete list of nodes available in the infrastructure. First, we build a subset of the available nodes, called \textit{suitable nodes}. Given the memory requirements we measured for each function, we eliminate nodes that currently do not have enough memory available to run a replica for the function. Each replica deployed on a node's execution platform consumes the total quantity of memory required by the function type. If the node is out of memory, its execution platforms cannot be used to deploy any more replica.

In order to select the type of resource to allocate for this replica, the autoscaler minimizes the cost function given in Equation~\ref{eq:herofake-HRO-allocation-cost-function}. 
In our policy, as for the autoscaling, we chose to favor total task execution time by setting $k_{TT} = \frac{2}{3}$, $k_{EC} = \frac{1.5}{6}$ and $k_{HP} = \frac{0.5}{6}$. 
Depending on which hardware is available in the pool at scale out time, the autoscaler will favor creating a new function replica on the platform that will execute the task in the lowest total time, including cold start, with the lowest energy consumption and the lowest price.

\begin{equation}
\begin{split}
    scaleCost_{{f}_{N, P}} = \, &k_{TT} \cdot {TT}_{{f}_{N, P}} \\
    + &k_{EC} \cdot {EC}_{{f}_{N, P}} \\
    + &k_{HP} \cdot {HP}_{{f}_{N, P}}
\end{split}
\label{eq:herofake-HRO-allocation-cost-function}
\end{equation}

On the contrary, when concurrency for a function falls beneath the threshold on a given hardware type, the autoscaler will employ a best effort policy and try to deallocate any replica with an empty task queue on said hardware type. If a replica does have an empty task queue, it will be released into the available platforms pool, and the memory it had been allocated on the node will be freed.

The different weights ($k$) used in Equations~\ref{eq:herofake-HRO-concurrency-target} and~\ref{eq:herofake-HRO-allocation-cost-function} can be modified by the provider to customize the allocation policy according to different priorities.

\subsection{Scheduling strategy} \label{section:herofake-scheduling-strategy}

Workload characterization is instrumental to performance prediction, as it can guide scheduling decisions that lead to the fulfillment of QoS requirements~\cite{mampageHolisticViewResource2022}. Our scheduling strategy relies on tasks metadata as described in Section~\ref{model:tasks}. Building knowledge about serverless tasks is achieved during an offline phase on our platform, as code is pushed to the provider's registries ahead of actual execution~\cite{shahradServerlessWildCharacterizing}.

In Knative, the scheduler handles incoming tasks in a FIFO fashion. To manage the different levels of QoS requirements, we propose that our scheduler pops tasks from the gateway queue by \textbf{earliest deadline first}. We compute the task deadline by using its worst execution time on the platform using Equation~\ref{eq:herofake-task-wet}, and multiplying it by the allowed duration deviation set by the QoS level. After the task execution, we will check if we missed its deadline and set the associated penalty accordingly, as described in Equation~\ref{eq:herofake-task-penalty}. 

We iterate on the function's replicas to fetch and predict the following metrics based on task metadata:

\begin{itemize}
    \item \textbf{potential penalty}: we compute the length of the platform's queue and check whether the task's deadline will be missed, as described in Equation~\ref{eq:herofake-task-penalty};
    \item \textbf{energy consumption}: we retrieve the offline measurements to establish the dynamic energy consumption for this task on the platform;
    \item \textbf{task consolidation}: we compute the length of the platform's task queue $Q$ by summing the total times of all queued tasks, as described in Equations~\ref{eq:herofake-HRO-scheduling-platform-queue} (queue length) and~\ref{eq:herofake-HRO-total-time} (task total time). 
\end{itemize}

\begin{equation}
\begin{split}
    len \, Q_{N, P} = \sum TT_{f_{N, P}}
\end{split}
\label{eq:herofake-HRO-scheduling-platform-queue}
\end{equation}

These values are normalized to fit in a weighted cost function described in Equation~\ref{eq:herofake-HRO-scheduling-cost-function}. We used $k_{QP} = \frac{2}{3}$, $k_{EC} = \frac{0.5}{6}$ and $k_{TC} = \frac{1.5}{6}$ (same as for the autoscaler). The scheduler then minimizes that cost function for all replicas $(N, P)$ (i.e. node and execution platform).

\begin{equation}
\begin{split}
    schedCost_{{f}_{N, P}} = \, &k_{QP} \cdot QP_{{f}_{N, P}} \\
    + &k_{EC} \cdot {EC}_{{f}_{N, P}} \\
    + &k_{TC} \cdot TC_{{f}_{N, P}}
\end{split}
\label{eq:herofake-HRO-scheduling-cost-function}
\end{equation}

If the scheduler cannot find an available replica to execute the task, it will be pushed back to the gateway's task queue. This will increase in-system concurrency for the function, nudging the autoscaler into allocating another replica on relevant hardware.

\section{Evaluation}

\begin{figure*}[t]
    \subfloat[Task consolidation (based on the unused node count)\label{figure:herofake-evaluation-full-nodes}]{
        \includegraphics[width=0.3\linewidth]{5_Chapitre3/figures/evaluation/z-nodes-20221212-232143-169224.png}
    }\qquad
    \subfloat[QoS violations (based on tasks with missed deadline)\label{figure:herofake-evaluation-full-penalty}]{
        \includegraphics[width=0.3\linewidth]{5_Chapitre3/figures/evaluation/z-penalty-20221212-232143-169224.png}
    }\qquad
    \subfloat[Dynamic energy consumption (in kWh)\label{figure:herofake-evaluation-full-energy}]{
        \includegraphics[width=0.3\linewidth]{5_Chapitre3/figures/evaluation/z-energy-20221212-232143-169224.png}
    }
    \caption{Evaluation 1 -- Comparison against baselines}
    \label{figure:herofake-evaluation-hro-full}
\end{figure*}

\begin{figure*}[t]
    \subfloat[Task consolidation (based on the unused node count)\label{figure:herofake-evaluation-mixed-nodes}]{
        \includegraphics[width=0.3\linewidth]{5_Chapitre3/figures/evaluation/x-nodes-20221212-185844-053283.png}
    }\qquad
    \subfloat[QoS violations (based on tasks with missed deadline)\label{figure:herofake-evaluation-mixed-penalty}]{
        \includegraphics[width=0.3\linewidth]{5_Chapitre3/figures/evaluation/x-penalty-20221212-185844-053283.png}
    }\qquad
    \subfloat[Dynamic energy consumption (in kWh)\label{figure:herofake-evaluation-mixed-energy}]{
        \includegraphics[width=0.3\linewidth]{5_Chapitre3/figures/evaluation/x-energy-20221212-185844-053283.png}
    }
    \caption{Evaluation 2 -- Impact of HeROfake components on the overall performance}
    \label{figure:herofake-evaluation-hro-mixed}
\end{figure*}

\subsection{Experimental setup}

We used measurements from the evaluation of three different machine learning models (see Table~\ref{table:herofake-tasks}). These models have been implemented on three different execution platforms (see Table~\ref{table:herofake-platforms}) as explained in Section~\ref{offline}.

These data served as input for a simulator we built using SimPy~\cite{simpy}. The simulator follows the system model described in sections~\ref{model:nodes}, \ref{model:platforms}, \ref{model:tasks}.

We measured cold start delays for our case study applications, see Table~\ref{table:herofake-tasks}. It appears that execution times are dominated by cold start delays, making adequate resource allocation a stringent requirement to comply with SLAs.

In the performance evaluation part, we compare two autoscalers:
\begin{itemize}
    \item HeROfake (HRO) -- Our heterogeneity-aware, metadata-based resource allocator;
    \item Knative (KN) -- We modeled the Knative autoscaler behavior to the best of our knowledge.
\end{itemize}

Our evaluation extends to four schedulers:

\begin{itemize}
    \item HeROfake (HRO) -- Our cost-aware scheduler that minimizes SLA violations, energy consumption and resource usage;
    \item Knative (KN) -- Knative selects a platform on the most available node~\cite{sureshENSUREEfficientScheduling2020}. Execution platforms are sorted by in-flight requests count. The platform with the shortest queue is selected;
    \item Random Placement (RP) -- Tasks are assigned a random execution platform on a random node;
    %\item Round Robin (RR) -- Load is balanced across nodes by assigning execution platforms to tasks in a cyclic fashion;
    \item Bin Packing First-Fit (BPFF) -- Tasks are consolidated on the minimum number of execution platforms. While a node still has enough memory available for new replicas, it is systematically chosen until it runs out of memory; then, a new node is selected. BPFF is likely to be the scheduling policy for AWS Lambda~\cite{wangPeekingCurtainsServerlessb}.
\end{itemize}

%Note that for each strategy, we still apply some pre-filtering on execution platforms: a task is never assigned a node that could not achieve its execution (i.e. a node that does not meet the task's memory requirements, or that does not provide a suitable execution platform).

We designed a two-step performance evaluation based on simulations:

\begin{itemize}
    \item \textbf{Comparison against baselines} (Figure~\ref{figure:herofake-evaluation-hro-full}): in this part, we compared our HeROfake combination of autoscaler and scheduler (HRO-HRO) to: (1) full-featured Knative autoscaler and scheduler (KN-KN), (2) Knative autoscaler with BPFF scheduler (KN-BPFF), (3) Knative autoscaler with RP scheduler (KN-RP); 
    \item \textbf{Impact of HeROfake components on the overall performance} (Figure~\ref{figure:herofake-evaluation-hro-mixed}): here we discuss the individual impact of each of the autoscaler and the scheduler. To do so, we devised different strategies: (1) using HeROfake autoscaler with Knative scheduler, and (2) using Knative autoscaler with HeROfake scheduler, and we compared those strategies with full featured HeROfake and Knative.
\end{itemize}

The naming of each scenario in these figures consists of two parts divided by a dash symbol. The first part corresponds to the allocation policy; the second part corresponds to the scheduling policy (for example, HRO-KN means we used the HeROfake autoscaler in conjunction with the Knative scheduler). 

%In order to measure the relevance of both our autoscaler and scheduler policies, we proceeded to evaluate them in two configurations. Figure~\ref{figure:herofake-evaluation-hro-full} shows our solution (HRO-HRO) evaluated against vanilla Knative (KN-KN), and Knative autoscaler with baseline schedulers (KN-BPFF, KN-RP). Figure~\ref{figure:herofake-evaluation-hro-mixed} shows our autoscaler and scheduler mixed with Knative's autoscaler and scheduler to evaluate the impact of each component on the overall performances. 

For each of the combinations of autoscaler and scheduler policies, we ran the experiment on a synthetic workload scenario consisting of 50000 tasks (user requests). Tasks are assigned a random type (ResNet50, VGG16 or VGG19) and a random QoS level (high, medium, low) following a uniform distribution, with QoS duration deviations respectively set to 2, 3 and 4. The infrastructure for the scenario consists of 10 nodes (10 CPUs, 6 GPUs, 2 FPGAs).

Weights for the concurrency level (Equation~\ref{eq:herofake-HRO-concurrency-target}) have been set to $k_{ET} = \frac{2}{3}$, $k_{EC} = \frac{1.5}{6}$ and $k_{HP} = \frac{0.5}{6}$. Weights for the scale out decision (Equation~\ref{eq:herofake-HRO-allocation-cost-function}) have been set to $k_{TT} = \frac{2}{3}$, $k_{EC} = \frac{1.5}{6}$ and $k_{HP} = \frac{0.5}{6}$. Weights for the scheduling decision (Equation~\ref{eq:herofake-HRO-scheduling-cost-function}) have been set to $k_{QP} = \frac{2}{3}$, $k_{EC} = \frac{0.5}{6}$ and $k_{TC} = \frac{1.5}{6}$. 

\subsection{Experimental results}

\subsubsection{Comparison against baselines}

\textbf{Tasks consolidation}. Figure~\ref{figure:herofake-evaluation-full-nodes} shows that our combination of autoscaler and scheduler achieves the highest number of unused nodes. Under Knative's autoscaler, the BPFF scheduler ensures the best consolidation, but that policy still needs more than three times the nodes we need with our policy.% We can see that our scheduler comes second best at task consolidation, with almost 70\% of nodes left unused -- a negligible degradation compared to HRO-BPFF.

\textbf{Service Level Agreements}. Figure~\ref{figure:herofake-evaluation-full-penalty} shows that HRO-HRO performs the best in terms of QoS violations, with 35\% of tasks missing their deadlines. This is a huge improvement with regard to the Knative results, where tasks miss their deadlines more than 99\% of the time: the delay introduced by the reactive allocation of resources cannot be compensated in time using only CPUs.

\textbf{Energy consumption}. Figure~\ref{figure:herofake-evaluation-full-energy} shows that our policy, with the HRO autoscaler and scheduler working in conjunction, consistently performs the best in terms of dynamic energy consumption. This is obviously because we allocate hardware accelerators; however, during our evaluation, the makespan for our scenario is similar under Knative and HRO policies (around 13.5 minutes). The BPFF scheduling policy also performs the worst in terms of execution time, as it maximizes the task queues in execution platforms, thus yielding the worst results in terms of energy consumption.

%This is obviously because we allocate hardware accelerators; however, under a BPFF scheduler, our autoscaler shows results barely better than Knative's autoscaler: seeking task consolidation on platforms and nodes seems to yield the worst energy efficiency.

\subsubsection{Impact of each component}

\textbf{Tasks consolidation}. Figure~\ref{figure:herofake-evaluation-mixed-nodes} shows that HRO-HRO performs the best at task consolidation, leaving just under 70\% of the available nodes unused, while Knative's scheduler under our autoscaling policy only achieves 40\% of unused nodes. This result is expected, as Knative's scheduler employs a Least Connected policy. We see mediocre consolidation results for KN-HRO, but for a different reason: this is because our scheduler tries to minimize QoS violations and spreads the task across all the allocated CPUs.

\textbf{Service Level Agreements}. Figure~\ref{figure:herofake-evaluation-mixed-penalty} shows that our scheduler does not perform well in conjunction with Knative's autoscaler. This is because our scheduler tries to minimize penalties: when given only CPUs to work with, it will behave similarly to Knative's scheduler and spread tasks across theses CPUs to limit QoS violations. However, our scheduler under the Knative autoscaler still manages to keep QoS violations at around 50\% of tasks, showing that there is room for improvement even when deploying inference tasks on CPUs only. Note that during our evaluation, the Knative autoscaler gave the worst results regarding cold starts frequency (6.5 more under KN-HRO than under HRO-KN).

\textbf{Energy consumption}. Figure~\ref{figure:herofake-evaluation-mixed-energy} shows that energy consumption is always lower when using our autoscaler, which can allocate hardware accelerators. However, our scheduler used with Knative's autoscaler yields the worst results in terms of energy consumption. This is again the result of the scheduler trying to minimize penalties and spreading task across a maximum number of CPUs.

\section{State of the Art} \label{section:herofake-sota}

Previous work focused on autoscaling platforms for the deployment of short-lived tasks, comprised in applications exhibiting unpredictable load patterns. Table~\ref{table:herofake-sota} summarises how these contributions differ from our target platform.

%Gujarati et al.~\cite{gujaratiSwayamDistributedAutoscaling2017} propose a distributed framework for allocating and scaling resources to support on-demand inference workloads, while meeting SLA on incoming requests and decreasing resource utilization. Ling et al.~\cite{lingPigeonDynamicEfficient2019} propose a function-level scheduler for Kubernetes that allows running short-lived functions using a pool of pre-warmed containers, mapping each FaaS service to suitable containers according to its hardware resources requirements. Zhang et al.~\cite{zhangMArkExploitingCloud} propose a hybrid autoscaling mechanism that opportunistically uses serverless instances to cover sudden spikes in load, while relying on a pool of reserved GPU-enabled AWS VMs for the majority of inference tasks, in order to achieve reduced operation costs and achieve SLO. Suresh et al.~\cite{sureshENSUREEfficientScheduling2020} propose a function-level scheduler that caterogizes tasks between interactive, short-lived, event-triggered tasks, and massively parallel, batch, back-end tasks, with the objective to decrease resource usage and provide acceptable latency guarantees. Mampage et al.~\cite{mampageDeadlineawareDynamicResource2021} propose a function placement and resource management policy that targets reduced resource consumption while meeting users' applications deadlines. Singhvi et al.~\cite{singhviAtollScalableLowLatency2021} propose parting with reactive sandbox allocation and instead proactively provisioning nodes so as to meet user-defined function deadlines. Yang et al.~\cite{yangINFlessNativeServerless2022} propose a serverless platform aimed at deploying and scaling inference tasks under user-defined SLOs, with low-latency, high-throughput constraints. Cho et al.~\cite{choSLADrivenMLInference} propose a serverless scheduler that takes into account heterogeneity in the infrastructure and in the users' requests to achieve SLAs (target latency, inference per second) and optimize for resource usage.

Some of these contributions attempted to achieve SLA with unreserved resources~\cite{gujaratiSwayamDistributedAutoscaling2017, zhangMArkExploitingCloud, mampageDeadlineawareDynamicResource2021, singhviAtollScalableLowLatency2021, handaoui2020releaser, handaoui2020salamander, yalles2022riscless}.
Among these contributions, some focus on the use of additional heterogeneous hardware resources to accelerate workload execution~\cite{zhangMArkExploitingCloud, lingPigeonDynamicEfficient2019, yangINFlessNativeServerless2022}.
They often require overprovisioning resources to make use of hardware acceleration, e.g. by relying on reserved AWS instances that provide access to GPUs~\cite{zhangMArkExploitingCloud}, using a pool of pre-warmed containers~\cite{lingPigeonDynamicEfficient2019}, or even proactively provisioning nodes to meet user-defined function deadlines~\cite{singhviAtollScalableLowLatency2021}. These interesting solutions however may fall short in terms of resource usage and would incur additional energy consumption in a private cloud.

Furthermore, some authors focus on homogeneous infrastructures \cite{gujaratiSwayamDistributedAutoscaling2017, sureshENSUREEfficientScheduling2020, mampageDeadlineawareDynamicResource2021, singhviAtollScalableLowLatency2021, yangINFlessNativeServerless2022}. Those studies could hardly fit the private cloud setting we target, where resources are usually transient and heterogeneous. Also, some of these contributions propose task models that do not cover user-defined, per-request SLA~\cite{sureshENSUREEfficientScheduling2020, lingPigeonDynamicEfficient2019}. Finally, some of these contributions are performance-oriented rather than cost-oriented which is crucial in our cloud context~\cite{gujaratiSwayamDistributedAutoscaling2017, lingPigeonDynamicEfficient2019, singhviAtollScalableLowLatency2021, choSLADrivenMLInference}.

In spite of power being one of the topmost part of the total cost of ownership (TCO) in a datacenter -- sometimes exceeding the cost of buying hardware~\cite{7279063} -- to the best of our knowledge, none of these contributions cover the impact of dynamic allocation and dynamic placement on energy consumption, nor do they consider energy consumption as a QoS metric. This is a serious limitation, as optimizing for task consolidation opens possibilities for throttling and powering-off policies that can have a major impact on a datacenter's energy efficiency~\cite{chaurasiaComprehensiveSurveyEnergyaware2021}.

\section{Conclusion and Future Work}

In this paper, we introduced HeROfake, our framework for the deployment of short-lived, interactive deepfake detection tasks on a private, heterogeneous serverless cloud.

We presented the two phases that make up this framework: an offline phase during which we characterize execution platform performances and task requirements; and an online phase during which we dynamically allocate resources and schedule tasks to run on those platforms.

Experimental results show that while total task execution time in HeROfake is similar to vanilla Knative, we achieve more than 60\% reduction in QoS penalties; tasks are consolidated on less than 40\% of the infrastructure's nodes, with 77\% execution platforms left unused; and dynamic energy consumption is reduced by 35\% as compared to Knative.

The inclusion of video handling in the framework is an interesting challenge, as it would introduce dependencies between tasks. Function executions would not be \textit{stateless} anymore, resulting in the necessity to tackle the problem of intermediate data storage in a serverless infrastructure.%M since function state has to be persisted in disaggregated storage, applications can suffer from the shipping time of data to compute nodes \cite{mullerLambadaInteractiveData2020}.

%This perspective can be linked to the introduction of heterogeneity in user requests. Input data size and locality could vary a lot between pictures and videos, and open possibilities of batching requests to study the impact on the performances of our platform.

We also intend to extend the simulator with a parser so as to be able to use real datacenter traces as input scenarios, instead of using synthetic workloads only. %This could give us the opportunity to learn from these traces and apply machine learning algorithms in the workload characterization phase, allowing us to generalize our framework to arbitrary applications without generating \textit{a priori} metadata on tasks.


\clearemptydoublepage
\chapter{HeROcache : Applications serverless et coûts associés aux systèmes de stockage}

\section{Introduction}
\label{section:herocache-introduction}

\begin{table*}[t]
    \centering
        \caption{State of the Art work on data-aware autoscaling platforms}
        \resizebox{\textwidth}{!}{
            \begin{tabular}{lSSSSSSS}
                \toprule
                & Function chains & QoS-aware & Hardware heterogeneity & Programming constraint & Energy consumption & Function cache & Function communications \\
                \cmidrule(lr){2-2}\cmidrule(lr){3-3}\cmidrule(lr){4-4}\cmidrule(lr){5-5}\cmidrule(lr){6-6}\cmidrule(lr){7-7}\cmidrule(lr){8-8}
                Cypress~\cite{bhasiCypressInputSizesensitive2022} & \cmark & \cmark & \xmark & \cmark & \cmark & \xmark & \cmark \\
                FaDO~\cite{smithFaDOFaaSFunctions2022} & \xmark & \xmark & \xmark & \cmark & \xmark & \xmark & \cmark \\
                FaasFlow~\cite{zijunFassflowEfficient2022} & \cmark & \xmark& \xmark & \xmark & \xmark & \xmark & \xmark \\
                FIRST~\cite{zhangFIRSTExploitingMultiDimensional2023} & \xmark & \xmark & \xmark & \cmark & \cmark & \xmark & \xmark \\
                HeROfake~\cite{herofake} & \xmark & \cmark & \cmark & \cmark & \cmark & \xmark & \xmark \\
                Netherite~\cite{burckhardtNetheriteEfficientExecution} & \cmark & \xmark & \xmark & \cmark & \xmark & \xmark & \cmark \\
                Palette~\cite{abdiPaletteLoadBalancing2023} & \cmark & \xmark & \xmark & \xmark & \xmark & \cmark & \cmark \\
                Target solution & \cmark & \cmark & \cmark & \cmark & \cmark & \cmark & \cmark \\
                \bottomrule
            \end{tabular}
        }
    \label{table:herocache-sota}
\end{table*}

\textbf{IDS, des applications critiques et sensibles au temps :} 
Un large éventail de systèmes embarqués fonctionnant dans des environnements statiques et contrôlés (capteurs dans une usine) ou dynamiques et non contrôlés (essaims de drones en mouvement) peuvent être temporairement ou constamment exposés à des attaques critiques par l'intermédiaire de liaisons réseau. Comme ces attaques peuvent compromettre leur exécution et endommager gravement les infrastructures connexes, il est essentiel de les prendre en compte. Pour atténuer ces menaces, les systèmes de détection d'intrusion (IDS) sont utilisés pour analyser le trafic réseau et détecter des modèles d'activités potentiellement malveillantes. Les modèles d'apprentissage automatique sont particulièrement utiles pour une classification opportune du trafic, mais ils sont très gourmands en ressources informatiques. Par conséquent, les exécuter directement sur la plateforme embarquée n'est pas une solution sûre, car cela peut affecter leur durée de vie s'ils fonctionnent sur une batterie~\cite{slimani:hal-04159551}, interférer avec d'autres tâches critiques, ou même être carrément impossible à exécuter en raison d'un manque de ressources.

\textbf{IDS à l'edge :} Une solution pour décharger les systèmes embarqués déployés de ces algorithmes gourmands en ressources tout en maintenant le système réactif aux attaques consiste à exécuter les IDS dans le nuage, et en particulier sur les dispositifs edge~\cite{eskandari2020}. Les IDS doivent répondre à des exigences variables en matière de qualité de service (QoS) et peuvent n'être nécessaires que pendant des périodes critiques, identifiées à l'avance. Par conséquent, il pourrait être inefficace, du point de vue des coûts, de faire fonctionner les systèmes de détection d'intrusion sur des dispositifs edge réservés. En fait, différents types d'attaques peuvent avoir des impacts différents sur l'infrastructure sous-jacente. En outre, le risque d'attaque peut varier dans le temps et dans l'espace (en fonction du domaine d'application). Nous soutenons que le déploiement d'IDS sur des ressources non réservées à faible consommation d'énergie à l'edge pourrait offrir l'avantage d'une solution rentable pour l'exécution de telles applications, tout en maintenant la latence plus faible que lorsqu'on s'appuie sur le nuage.

\textbf{Serverless et IDS à l'dge :} L'un des principaux paradigmes du cloud qui permet d'exécuter des applications événementielles sur des ressources non réservées avec une granularité fine d'allocation des ressources est le serverless~\cite{Lannurien2023}. Le déploiement serverless à l'edge pour l'IDS, et plus généralement pour les applications critiques et sensibles au temps, est rentable car il ouvre des possibilités d'optimisation pour les fournisseurs de services : mise à l'échelle dynamique des ressources suite à des pics de charge dans les applications interactives, ainsi qu'une granularité d'allocation fine et mesurée pour les ressources limitées à l'edge.

\textbf{Défis pour le déploiement serverless d'applications critiques à l'edge :} Pour déployer des applications sensibles au temps composées de fonctions à courte durée de vie dans un contexte edge, hétérogène et serverless, trois défis doivent être relevés : (1) réduire les délais d'initialisation, (2) éviter les délais de communication élevés et (3) exploiter les ressources hétérogènes pour satisfaire une QoS variable.
\textbf{Les délais d'initialisation.} Les fonctions IDS sont de courte durée, et le serverless s'appuyant sur des ressources non réservées implique un taux plus élevé d'initialisations de fonctions, chacune nécessitant de tirer l'image de la fonction d'un nœud de stockage d'images dédié pour le déploiement sur les nœuds edge~\cite{yanHermesEfficientCache2020}. Les dispositifs edge exposent des dispositifs de stockage de faible capacité et de faible performance derrière des liaisons réseau limitées en termes de fiabilité et de vitesse, et cette question doit donc être examinée de près pour satisfaire la qualité de service des utilisateurs.

\textbf{Les délais de communication.} Dans une infrastructure distribuée telle que le serverless edge, les fonctions d'une même application peuvent être déployées sur plusieurs nœuds éloignés les uns des autres, ce qui implique l'utilisation du réseau lorsque ces fonctions ont besoin de communiquer des résultats intermédiaires. Cela entraîne des retards qui peuvent conduire à des violations de la qualité de service~\cite{wawrzoniakBoxerDataAnalytics2021a}.

\cite{wawrzoniakBoxerDataAnalytics2021a}. La plateforme serverless ne peut pas considérer tous les placements comme égaux, car ils produiront divers niveaux de performance. Cependant, l'affinité d'une fonction avec une plateforme d'exécution spécifique ne peut pas guider à elle seule les décisions d'ordonnanceur, car les fonctions peuvent appartenir à différentes chaînes en fonction de l'application demandée.

\textbf{Problem statement:} Le problème que nous abordons est de savoir comment prendre en compte \textbf{les délais d'initialisation et de communication} lors du déploiement de \textbf{chaînes de fonctions serverless à courte durée de vie} à \textbf{l'edge}, en tirant parti de \textbf{le matériel hétérogène} pour optimiser les applications sensibles au temps qui nécessitent \textbf{une qualité de service variable}, tout en limitant le nombre de nœuds edge utilisés.

\textbf{État de l'art:} Des études antérieures ont exploré le besoin de plateformes d'orchestration qui prennent en charge l'ordonnancement de chaînes de fonctions sur des ressources non réservées. Le tableau~\ref{table:herocache-sota} résume dans quelle mesure ces solutions ne sont pas applicables à notre étude de cas, et la section~\ref{section:herocache-sota} donne plus de détails. Ces contributions visent généralement les déploiements dans le nuage, où il s'agit d'intégrer autant de tâches que possible dans une infrastructure homogène de nœuds toujours en service, afin de maximiser l'efficacité des ressources. La portée de notre étude est de montrer qu'avec des politiques d'allocation et d'ordonnanceur adéquates, nous pouvons adapter des applications bien définies sur un nombre limité de nœuds de bordure hétérogènes et réduire la consommation d'énergie globale du cluster par la consolidation.

\textbf{Contribution : HeROcache, une plateforme d'orchestration des ressources hétérogènes, optimisée pour la qualité de service, pour le serverless à l'edge et basée sur la mise en cache et la consolidation :}. 
Dans cet article, nous présentons une solution qui répond aux trois défis mentionnés ci-dessus. HeROcache : (1) exploite un mécanisme de mise en cache sur les nœuds edge qui réduit \textbf{les délais d'initialisation} sans saturer leur capacité de stockage ;
(2) consolide les tâches sur la base d'une application afin de limiter le nombre de \textbf{délais de communication} lents entre les nœuds ;
(3) gère le respect des exigences de qualité de service pour les tâches critiques en utilisant les métadonnées collectées auprès des applications et des plateformes hétérogènes utilisées pour le déploiement. Ces données comprennent des mesures de performance et d'énergie qui guident l'orchestrateur dans la prise de décisions éclairées lors de l'ordonnancement des tâches sur \textbf{les ressources hétérogènes}.

\textbf{Résultats :} Nous avons évalué HeROcache dans le contexte d'une application IDS réelle, caractérisée sur différentes plateformes d'exécution. Cette évaluation a été réalisée à l'aide d'un simulateur ad hoc. Nous avons également mis en œuvre le comportement d'un orchestrateur Knative~\cite{knative} vanille. HeROcache parvient à surpasser Knative, en maintenant les violations de la qualité de service à moins de 28\% tout en consolidant les tâches sur 80\% de nœuds edge en moins dans l'infrastructure. La mise hors tension de ces nœuds entraînerait une réduction drastique de la consommation d'énergie statique.

Le document est organisé comme suit : Section~\ref{section:herocache-background} donne quelques informations de base ; Section~\ref{section:herocache-before-contrib} explique le projet global ; Section~\ref{section:herocache-workload} détaille notre approche de collecte de métadonnées hors-ligne ; Section~\ref{section:herocache-contribution} décrit notre stratégie d'orchestration en ligne ; Section~\ref{section:herocache-evaluation} examine les résultats de l'évaluation ; Section~\ref{section:herocache-sota} examine l'état de l'art ; Section~\ref{section:herocache-conclusion} conclut le document.

\section{Contexte et motivation}
\label{section:herocache-background}

\begin{figure}[t]
\centering
\includegraphics[width=0.8\columnwidth]{6_Chapitre4/figures/function-cache.png}
\caption{Lifecycle of a user request in a serverless platform.}
\label{figure:herocache-function-cache}
\end{figure}

\begin{figure}[t]
\centering
\includegraphics[width=0.8\columnwidth]{6_Chapitre4/figures/function-communications.png}
\caption{Serverless functions communicate intermediate results through persistent storage that can be local to edge nodes or remotely accessible.}
\label{figure:herocache-function-communications}
\end{figure}

\begin{figure*}[t]
\centering
\includegraphics[width=0.65\textwidth]{6_Chapitre4/figures/serverless-platform-storage.png}
\caption{Serverless IDS platform, system overview}
\label{figure:herocache-serverless-platform}
\end{figure*}

\subsection{Défis de l'orchestration dynamique}

Le serverless est un modèle de service tendance pour le cloud~\cite{Lannurien2023} : en transférant la responsabilité de l'allocation des ressources des clients aux fournisseurs de services, il allège une partie importante de la complexité des développeurs d'applications et ouvre de nouvelles possibilités d'optimisation et de contrôle des coûts pour le gestionnaire d'infrastructure.
Dans une architecture serverless, les développeurs conçoivent leurs applications comme une composition de fonctions sans état. Sans état (ou "pur", sans effet secondaire) signifie que le résultat du calcul dépend exclusivement des entrées \cite{burckhardtNetheriteEfficientExecution}. Ces fonctions prennent en entrée une charge utile et un contexte d'invocation, et produisent un résultat qui est stocké dans un niveau de stockage persistant accessible par le réseau. Cela signifie que les dépendances de données entre les fonctions d'une chaîne doivent être gérées par la plateforme.

Lorsqu'un événement déclenche leur exécution, les fonctions sont déployées sur des nœuds de l'infrastructure, dans des environnements d'exécution appelés \textbf{répliques}. Comme les fonctions sont sans état, les demandes peuvent être attribuées à n'importe quelle réplique disponible. La mise à l'échelle d'une application serverless, \textit{i.e.} pour maintenir un niveau de performance constant, consiste à faire croître ou décroître le pool de répliques des fonctions en suivant les pics de charge. Les plateformes serverless basées sur Kubernetes, telles que Knative \cite{knative} ou OpenWhisk \cite{openwhisk}, ont proposé un modèle basé sur le seuil pour le rightsizing du pool de réplicas. Pour toute fonction, un \textit{autoscaler} peut déployer plusieurs \textit{répliques} pour absorber la charge. Chaque réplique est allouée à une plateforme d'exécution (\textit{e.} un cœur de CPU, un GPU, etc.) et dispose d'une file d'attente de longueur fixe pour les demandes entrantes. Le nombre de répliques pour une fonction donnée à un moment donné détermine son niveau de concurrence. Un ordonnanceur place les demandes des utilisateurs dans la file d'attente des répliques de la fonction. Lorsqu'une réplique n'a plus de demandes, elle est désattribuée. Lorsqu'une fonction est demandée alors qu'aucune réplique n'existe, elle passe par un \textbf{démarrage à froid} qui entraîne un délai d'initialisation du temps de réponse de la fonction.

Dans le serverless, la fréquence des allocations de ressources augmente considérablement par rapport aux environnements à ressources réservées toujours actives, tels que les offres IaaS. La capacité des plateformes serverless à mettre à l'échelle une fonction jusqu'à zéro réplique afin d'éviter de facturer les clients pour des ressources inactives est une différence essentielle par rapport aux modèles de services cloud traditionnels.

Ce démarrage à froid présente un risque d'augmentation de la latence, car le fournisseur doit allouer des ressources matérielles et instancier l'application avant de répondre à la demande. Plus l'application est complexe, plus le risque de retards importants est élevé~\cite{mohanAgileColdStartsa}. Les fournisseurs pré-affectent généralement certaines ressources pour éviter les démarrages à froid, ce qui a un coût en termes de provisionnement des ressources. Les acteurs commerciaux tels qu'AWS, Google et Microsoft réutilisent tous, dans une certaine mesure, des instances de fonction, en les laissant fonctionner pendant une période de temporisation afin de contourner les coûts de latence induits par les démarrages à froid~\cite{vahidiniaColdStartServerless2020}.

Une étude récente a montré que 50\% des applications serverless déployées sur Microsoft Azure Durable Functions \footnote{\href{https://learn.microsoft.com/en-US/azure/azure-functions/durable/durable-functions-overview}{https://learn.microsoft.com/en-US/azure/azure-functions/durable/durable-functions-overview}} sont constituées de 3 fonctions ou moins, 65\% des applications présentant un simple DAG de fonctions agencées sous forme de chaînes linéaires \cite{mahgoubORIONThreeRights}. Notre application IDS se compose de différentes chaînes de deux fonctions, comme décrit dans la section~\ref{section:herocache-characterization-workloads}. Les travaux de caractérisation des charges de travail ont montré que 25\% des fonctions déployées sur Microsoft Azure Functions \footnote{\href{https://azure.microsoft.com/en-us/products/functions/}{https://azure.microsoft.com/en-us/products/functions/}} s'exécutent en 100 ms ou moins \cite{shahradServerlessWildCharacterizing}. Les fonctions qui composent notre application IDS s'exécutent pendant des centièmes ou des dixièmes de seconde, ce qui les rend particulièrement sujettes à des ralentissements critiques dans le contexte de ressources allouées de manière dynamique.

\subsection{Mise en cache des images de fonctions}
\label{section:herocache-background-cache}

En ligne, l'allocation dynamique des ressources et les politiques de placement des tâches peuvent aider à répondre aux exigences de qualité de service (QoS) par demande en attribuant les requêtes aux répliques appropriées dans l'infrastructure. Cependant, la plupart des articles sur l'orchestration des ressources et le placement des tâches dans le serverless ne considèrent que les meilleurs scénarios, dans lesquels les images de fonctions sont déjà disponibles sur les nœuds edge. Cela ne reflète pas la réalité, où les images de fonctions sont stockées dans des registres sur des nœuds dédiés et téléchargées sur les nœuds de calcul lors du déploiement des fonctions. En fonction de la taille de l'image, cela peut avoir des conséquences néfastes sur la latence des requêtes avec des déploiements où les démarrages à froid dominent le temps de réponse total d'une fonction \cite{yanHermesEfficientCache2020}.

Afin de répondre aux demandes des utilisateurs sans dégrader les performances, l'autoscaler ajuste périodiquement le nombre de répliques pour chaque fonction déployée : le pool de répliques croît et décroît en fonction des variations de la charge de l'application. Lorsque la charge d'une fonction augmente au-delà du seuil de concurrence de la plateforme, l'autoscaler crée une nouvelle réplique qui traitera les demandes supplémentaires des utilisateurs. Lorsque la charge diminue, les répliques inactives sont supprimées. S'il n'y a plus de demandes pour une fonction donnée, celle-ci peut être \textit{scaled to zero}, ce qui permet d'éviter le gaspillage des ressources.

Les répliques de fonctions sont initialisées à partir d'\textbf{images de fonctions} (\textit{e.g.} une image Docker). Celles-ci sont stockées dans un registre d'images. Ces registres peuvent être accessibles à distance via l'internet et sont généralement déployés dans l'infrastructure du fournisseur dans le contexte d'un nuage privé. Toutefois, de nombreuses études antérieures \cite{bhasiCypressInputSizesensitive2022, zijunFassflowEfficient2022, smithFaDOFaaSFunctions2022, zhangFIRSTExploitingMultiDimensional2023} n'envisagent que des scénarios optimaux dans lesquels les images de fonctions sont déjà disponibles sur les nœuds edge. Cela ne reflète pas la réalité où les images de fonctions sont stockées dans des registres sur des nœuds dédiés et tirées par les nœuds de bordure où et quand les fonctions sont déployées. En fonction de la taille de l'image, cela peut avoir des conséquences néfastes sur la latence des requêtes, en particulier dans les cas où les démarrages à froid dominent le temps de réponse total d'une fonction.

En fait, l'extraction d'images sur les nœuds edge peut représenter plus de 80\% du temps de réponse d'une fonction \cite{yanHermesEfficientCache2020} puisque la latence du démarrage à froid domine le temps de réponse total de la fonction. Cette situation n'est pas acceptable lorsque la plateforme doit répondre à des exigences strictes en matière de qualité de service, comme c'est le cas pour les tâches critiques telles que l'IDS.

\subsection{Communications entre les fonctions}
\label{section:herocache-background-communications}

En outre, comme ces fonctions sont parfois mises à l'échelle à partir de zéro, elles ne sont pas adressables par le réseau : les communications entre les fonctions sont réalisées par l'utilisation d'un stockage lent et accessible par le réseau (\textit{e.g.} Amazon S3). Cela induit des retards qui peuvent faire boule de neige tout au long de l'exécution de l'application et provoquer des violations de l'objectif de niveau de service (SLO) en augmentant les latences de queue de requête \cite{wawrzoniakBoxerDataAnalytics2021a}. Les offres FaaS sont un exemple typique d'architecture d'expédition de données : des gigaoctets de données sont déplacés vers des mégaoctets de code, ce qui entraîne des inefficacités qui augmentent la consommation d'énergie et la sous-utilisation des ressources.

Comme il est nécessaire de prendre en charge la mise à l'échelle dynamique des fonctions, chaque invocation d'une fonction serverless est autonome et ne porte pas les informations ou le contexte des invocations précédentes. Cela permet aux répliques de mettre en file d'attente les demandes des utilisateurs et de les traiter de manière séquentielle sans avoir besoin de procéder à un démarrage à froid entre les demandes. Cela introduit une contrainte sur la plateforme serverless : si une application est composée de plusieurs fonctions qui forment un pipeline de traitement, la sortie de chaque fonction doit être stockée dans un stockage persistant pour être alimentée en entrée de la fonction suivante dans la chaîne~\cite{mullerLambadaInteractiveData2020}.

Les travaux de l'état de l'art ont montré que les fonctions serverless qui communiquent par le biais d'un stockage distant peuvent subir un ralentissement jusqu'à 11x par rapport aux fonctions utilisant des communications directes \cite{wawrzoniakBoxerDataAnalytics2021a}. Les fonctions de notre application IDS doivent communiquer des résultats intermédiaires à chaque étape du DAG de l'application. Lorsque les fonctions sont déployées sur différents nœuds edge, les communications inter-fonctions devront être réalisées par l'utilisation d'un stockage à distance. Cela introduit des ralentissements qui peuvent faire boule de neige tout au long de l'exécution des fonctions et détériorer la qualité de service de l'ensemble de l'application.

\section{Détection d'intrusion à l'edge dans le modèle serverless}
\label{section:herocache-before-contrib}

Orchestrer des applications serverless tout en atteignant le SLA nécessite de modéliser soigneusement les caractéristiques de l'application et d'en tenir compte lors de l'allocation des ressources et de l'ordonnancement des demandes des utilisateurs sur la plateforme serverless. La figure~\ref{figure:herocache-serverless-platform} donne un aperçu du cycle de vie global d'une requête sur notre plateforme serverless. Il est divisé en deux phases ; une \textbf{phase hors-ligne} qui consiste à caractériser les applications déployées par les utilisateurs sur les plateformes edge, et une \textbf{phase en ligne} où les requêtes vers ces applications sont ordonnancées sur la plateforme.

\textbf{phase hors-ligne}. Dans notre plateforme, le cycle de vie de l'application commence par une phase hors-ligne, au cours de laquelle le développeur fournit le code de ses fonctions pour différentes architectures matérielles (GPU, CPU, DLA, etc.)~\Circled{1}. Ce code est stocké par le fournisseur de services dans un registre de fonctions. Les fonctions sont ensuite déployées sur un nœud de mesure~\Circled{2} où elles sont exécutées pour générer des métadonnées relatives à l'exécution des fonctions sur des nœuds hétérogènes. Les besoins en mémoire, le temps d'exécution, le temps de démarrage à froid, la consommation d'énergie, la taille de la fonction et la taille de la communication pour chaque fonction sont enregistrés dans un magasin de métadonnées~\Circled{3}. L'exécution de la phase hors-ligne est nécessaire une fois pour une fonction donnée sur une plateforme donnée, comme décrit dans la section~\ref{section:herocache-workload}.

\textbf{Phase en ligne}. Les demandes sont envoyées aux applications IDS avec une charge utile de trafic TCP (paquets sérialisés) à analyser~\Circled{4}, et un niveau de qualité de service souhaité associé pour le temps de réponse de la demande. La demande est ajoutée à une file d'attente~\Circled{5} au niveau de l'orchestrateur. Lorsque l'ordonnanceur extrait la demande de la file d'attente, le magasin de métadonnées est interrogé pour récupérer les métadonnées de fonction appropriées~\Circled{6}.

L'ordonnanceur tente alors de trouver une réplique disponible de la première fonction de l'application pour traiter la demande~\Circled{7}. Si une telle réplique n'existe pas encore, il sera demandé à l'autoscaler d'initialiser une nouvelle instance de la fonction~\Circled{8}. Au cours du cycle de vie de l'application, l'autoscaler vérifie périodiquement la charge moyenne de chaque fonction pour ajuster le nombre de répliques déployées sur la plateforme, en fonction du seuil de concurrence fixé par le fournisseur de services.

Lorsque l'application est terminée, elle renvoie à l'utilisateur un vecteur de classification qui indique les probabilités que le trafic soit malveillant, c'est-à-dire qu'il présente les caractéristiques d'une attaque potentielle.

\section{Phase hors-ligne : caractérisation} 
\label{section:herocache-workload}

\begin{figure}[t]
\centering
\includegraphics[width=0.8\columnwidth]{6_Chapitre4/figures/ids-application.png}
\caption{Architecture of an IDS application that can make use of different preprocessing functions, and different inference functions to provide the user with a classification of TCP traffic.}
\label{figure:herocache-ids-application}
\end{figure}

Une étape préliminaire de caractérisation de la plateforme et de la charge de travail est nécessaire pour parvenir à une allocation des ressources et à un placement des tâches adéquats pour l'exécution des modèles IDS. À cette fin, nous avons évalué plusieurs modèles IDS en termes de performance et d'énergie sur des plates-formes hétérogènes représentatives des dispositifs edge~\cite{kljucaric2020}. Ces mesures sont cruciales pour une orchestration efficace sur des plates-formes hétérogènes à l'edge. Cette section décrit notre méthodologie et nos résultats.

\subsection{Caractérisation des plateformes d'exécution} \label{section:herocache-characterization-platforms}

Nous avons utilisé des plateformes représentatives de ce que l'on peut trouver en bordure~\cite{slimani:hal-04159551,kljucaric2020} : 
 Il fonctionne sous Linux Raspbian 5.4.
\textbf{(2) Nvidia Jetson Xavier AGX} composé de trois éléments de traitement : un CPU NVIDIA ARM Carmel à 8 cœurs, un GPU NVIDIA Volta avec 512 cœurs CUDA et un accélérateur d'apprentissage profond (DLA), qui est un accélérateur matériel à fonction fixe conçu pour les réseaux neuronaux convolutifs (CNN). Il est supposé être plus économe en énergie que le GPU. Le NVIDIA Xavier AGX est équipé de 16 Go de mémoire LPDDR4 et de 32 Go de mémoire flash eMMC 5.1. Il fonctionne sous Linux Tegra 4.9.10. Le mode d'alimentation 15 Watts Desktop a été utilisé. 
\textbf{(3) PYNQ-Z2 Development Board}, une carte basée sur le système sur puce Xilinx Zynq XC7Z020. Elle est équipée d'un FPGA Artix-7, d'une mémoire DDR3 de 512 Mo et d'une carte SD de 16 Go.

\subsection{Caractérisation des applications}
\label{section:herocache-characterization-workloads}

Notre application se compose de différents préprocesseurs et classificateurs. Le préprocesseur sélectionne un sous-ensemble de caractéristiques pertinentes des paquets TCP. 3 approches de prétraitement différentes ont été utilisées : (1) utilisation de toutes les caractéristiques des paquets sans aucune sélection (NoFS : No Feature Selection) ; (2) utilisation d'un auto-encodeur DNN pour projeter les caractéristiques dans un espace latent plus petit (AE : Auto-Encoder) ; et (3) sélection experte d'un sous-ensemble de caractéristiques (ES : Expert Selection). Pour la partie classification, nous avons utilisé Random Forest (RF), deux architectures différentes de réseaux neuronaux denses (DNN) et un CNN.

\begin{table}[t]
\caption{IDS models architectures and sizes}
\resizebox{\textwidth}{!}{
\begin{tabular}{|c|c|cc|}
\hline
Model     & Architecture                                                                                                                                         & \multicolumn{1}{c|}{Model Size on CPUs (MB)} & Model Size on GPU (MB) \\ \hline
NoFS-RF   & \begin{tabular}[c]{@{}c@{}}5 trees of 100\\ maximum depth\end{tabular}                                                                               & \multicolumn{1}{c|}{28}                      & 15.4                   \\ \hline
AE-RF     & \begin{tabular}[c]{@{}c@{}}5 trees of 50 \\ maximum depth\end{tabular}                                                                               & \multicolumn{1}{c|}{-}                       & 32.9                   \\ \hline
ES-RF     & \begin{tabular}[c]{@{}c@{}}10 trees of 10 \\ maximum depth\end{tabular}                                                                              & \multicolumn{1}{c|}{9.1}                     & 5.5                    \\ \hline
NoFS-DNN1 & \multirow{3}{*}{\begin{tabular}[c]{@{}c@{}}4 Dense Layers \\ (128x64x32x10)\end{tabular}}                                                            & \multicolumn{2}{c|}{0.144}                                            \\ \cline{1-1} \cline{3-4} 
AE-DNN1   &                                                                                                                                                      & \multicolumn{2}{c|}{0.321}                                            \\ \cline{1-1} \cline{3-4} 
ES-DNN1   &                                                                                                                                                      & \multicolumn{2}{c|}{0.053}                                            \\ \hline
NoFS-DNN2 & \multirow{3}{*}{\begin{tabular}[c]{@{}c@{}}5 Dense Layers \\ (7024x704x288x64x10)\end{tabular}}                                                      & \multicolumn{2}{c|}{3.33}                                             \\ \cline{1-1} \cline{3-4} 
AE-DNN2   &                                                                                                                                                      & \multicolumn{2}{c|}{2.96}                                             \\ \cline{1-1} \cline{3-4} 
ES-DNN2   &                                                                                                                                                      & \multicolumn{2}{c|}{2.61}                                             \\ \hline
NoFS-CNN  & \multirow{3}{*}{\begin{tabular}[c]{@{}c@{}}2 Conv1D (x64) - MaxPool \\ 3 Conv1D (x256) - MaxPool\\ 3 Dense Layers (100x20x10)\end{tabular}} & \multicolumn{2}{c|}{4.77}                                             \\ \cline{1-1} \cline{3-4} 
AE-CNN    &                                                                                                                                                      & \multicolumn{2}{c|}{2.9}                                              \\ \cline{1-1} \cline{3-4} 
ES-CNN    &                                                                                                                                                      & \multicolumn{2}{c|}{2.6}                                              \\ \hline
\end{tabular}
}
\label{table:herocache-workload}
\end{table}

Le tableau~\ref{table:herocache-workload} présente les modèles IDS pris en compte dans cette étude et certaines de leurs caractéristiques. Ces modèles ont été formés et caractérisés sur l'ensemble de données de référence sur les intrusions dans le réseau UNSW-NB15\footnote{\href{https://research.unsw.edu.au/projects/unsw-nb15-dataset}{https://research.unsw.edu.au/projects/unsw-nb15-dataset}}
où chaque observation représente des caractéristiques statistiques, de contenu et de temps sur le trafic de données au cours d'une fenêtre temporelle, et est étiquetée comme "normale" ou "attaque". L'ensemble de données comprend 9 catégories d'attaques. Les modèles de réseaux neuronaux ont été exportés et optimisés à l'aide de TensorFlow Lite et TensorRT lorsqu'ils étaient destinés à des plateformes CPU et GPU/DLA, respectivement. En ce qui concerne Random Forest, les modèles ont été exportés en utilisant les frameworks Emlearn et HummingBird.ml lorsqu'ils étaient destinés aux plateformes CPU et GPU, respectivement. hls4ml a été utilisé pour exporter les modèles de réseaux neuronaux pour la cible FPGA.

\subsection{Mesures de performances}

Chacun des modèles IDS a été déployé sur les plateformes cibles et les inférences ont été exécutées avec un ensemble de 80 000 paquets provenant de l'ensemble de données UNSW-NB15 afin de caractériser la latence d'inférence. Les résultats sont présentés dans la figure~\ref{figure:herocache-performance}. Seul un modèle (ES-DNN1) a été caractérisé sur la plateforme FPGA car les autres modèles HLS n'ont pas pu être pris en compte sur la cible. La conclusion qui a été tirée de ces résultats est que pour les réseaux neuronaux, le CPU Xavier atteint la meilleure performance dans la majorité des cas, à l'exception de NoFS-CNN qui profite des capacités du GPU en raison de son nombre élevé de paramètres et de l'efficacité du GPU pour les opérations de convolution. Pour les modèles Random Forest, l'élément de traitement le plus rapide est le GPU. En termes de coût et de disponibilité, la Xavier AGX est respectivement environ 20x et 10x plus chère que les plateformes RBPI4 et Pynq-Z2. Nous dimensionnons notre infrastructure en conséquence en fournissant plus de plateformes RBPI4 que de Xavier AGX pour être représentatifs des déploiements réels.

\begin{figure}
    \centering
    \includegraphics[width=0.9\columnwidth]{6_Chapitre4/figures/latency_bar.pdf}
    \caption{Latency characterization of IDS models.}
    \label{figure:herocache-performance}
\end{figure}

\subsection{Mesures de consommation d'énergie}

Nous avons exécuté des inférences sur les modèles IDS sur chaque élément de traitement et mesuré la consommation d'énergie de la plateforme à l'aide de l'analyseur de puissance N6705A DC. Les résultats sont présentés dans la figure~\ref{figure:herocache-energy}. Pour les mêmes raisons que celles mentionnées ci-dessus, seul ES-DNN1 a été caractérisé sur FPGA. Nous observons que les éléments de traitement du CPU affichent une consommation d'énergie inférieure à celle du GPU dans la majorité des cas. Le seul cas où le GPU obtient de meilleurs résultats est lorsque la vitesse qu'il atteint par rapport aux CPU est élevée. C'est par exemple le cas pour NoFS-CNN, où le CPU RBPI4 est plus de 30 fois plus lent que le GPU. Même si Pynq-Z2 présente la meilleure efficacité énergétique avec le modèle ES-DNN1, étant donné qu'il est plus cher et présente une générosité de conception limitée, nous supposons qu'il est moins disponible que le RBPI4.

\begin{figure}
    \centering
    \includegraphics[width=0.9\columnwidth]{6_Chapitre4/figures/energy_bar.pdf}
    \caption{Energy characterization of IDS models.}
    \label{figure:herocache-energy}
\end{figure}

\section{Phase en ligne : orchestration avec HeROcache} \label{section:herocache-contribution}

\subsection{Présentation générale}

L'orchestrateur HeROcache est principalement composé de deux modules, le \textbf{autoscaler} et le \textbf{scheduler} (voir figure~\ref{figure:herocache-serverless-platform}). L'autoscaler est chargé de l'allocation dynamique des ressources : il affecte les plateformes d'exécution aux répliques de fonctions. L'ordonnanceur s'occupe du placement des demandes des utilisateurs sur les répliques.

HeROcache relève les trois défis susmentionnés en concevant des stratégies complémentaires de minimisation des coûts au niveau de l'autoscaler et de l'ordonnanceur. HeROcache minimise \textbf{les délais d'initialisation} en tenant compte des temps de latence de l'extraction d'images au niveau de l'autoscaler. Des stratégies d'extraction préalable sont également mises en œuvre pour la mise en cache des images de fonctions. Les coûts de \textbf{communication interfonction} sont pris en compte principalement dans la partie de l'ordonnanceur, qui tend naturellement à consolider les fonctions d'une même application. L'autoscaler participe indirectement à cette consolidation en préemptant les fonctions suivantes du DAG de l'application sur le même nœud de bordure. Enfin, \textbf{plateformes hétérogènes} sont prises en compte car les différents coûts d'exécution extraits pendant la phase hors-ligne (voir Section~\ref{section:herocache-workload}) sont pris en compte tout au long du processus d'autoscaling et d'ordonnanceur. Les sections suivantes décrivent les stratégies de mise à l'échelle automatique et d'ordonnanceur.

\begin{table}[t]
    \caption{Notation dictionary}
    \begin{center}
    \scalebox{0.85}{\begin{tabularx}{\linewidth}{|c|Y|}
    \hline
    \textbf{Notation} & \textbf{Description} \\ \hline
    $x_a$ & Allocation of resource for application $a$ \\ \hline
    $y_a$ & Invocation of application $a$ \\ \hline
    $z_i$ & Placement of task for function $i$ \\ \hline
    $f_{N, P}$ & A function $f$ scheduled to run on a platform $P$ available on node $N$ \\ \hline
    $f_{a}$ & A function $f$ that belongs to application $a$ \\ \hline
    $A$ & Total number of applications to be scheduled on the platform \\ \hline
    $F_{a}$ & Total number of functions that belong to an application $a$ \\ \hline
    $RT_{{f}_{N, P}}$ & Time to retrieve function image for $f$ to run on a platform $P$ available on node $N$ \\ \hline
    $NB_{N}$ & Network bandwidth between node $N$ and the infrastructure \\ \hline
    $SMT_{N}$ & Storage medium throughput on node $N$ \\ \hline
    $SML_{N}$ & Storage medium latency on node $N$ \\ \hline
    $QP$ & QoS penalty \\ \hline
    $QD$ & QoS deviation \\ \hline
    $WET$ & Worst execution time \\ \hline
    $TT$ & Task total time \\ \hline
    $CST$ & Cold start time \\ \hline
    $ST$ & Storage time \\ \hline
    $ET$ & Execution time \\ \hline
    $EC$ & Energy consumption \\ \hline
    $IS$ & Image size \\ \hline
    $HP$ & Hardware price \\ \hline
    $TC$ & Task consolidation \\ \hline
    $Q$ & Task queue on a replica \\ \hline
    $CP$ & Cache proportion \\ \hline
    $SIS^{f}_{a}$, $SOS^{f}_{a}$ & Size of resp. input, output state of function $f$ that belongs to application $a$ \\ \hline
    $threshold_{f, h}$ & Concurrency threshold for a function $f$ on a replica of hardware type $h$ \\ \hline
    $scaleCost^{{f}_{{i}_{N, P}}}_a$ & Cost of creating a new replica for function $f_i$ from application $a$ on a platform $P$ available on node $N$ \\ \hline
    $schedCost^{{f}_{{i}_{N, P}}}_a$ & Cost of scheduling an execution of function $f$ from application $a$ on a platform $P$ available on node $N$ \\ \hline
    \end{tabularx}}
    \label{table:herocache-notation}
    \end{center}
\end{table}

\subsection{Stratégie de minimisation des coûts d'allocation des ressources}

Nous formulons l'allocation des ressources comme un problème d'optimisation et nous le résolvons à l'aide d'un algorithme gourmand simple. L'objectif de l'autoscaler est de minimiser le coût de la somme des allocations $scaleCost_{a}$ pour $y_a$ invocations de l'application $a$ (équation~\ref{eq:herocache-objective-allocation}) pour toutes les applications dans $A$, sous la contrainte d'une infrastructure finie avec $x_a$ étant l'allocation de ressources pour l'application $a$ (équation~\ref{eq:herocache-constraint-allocation}). 

\begin{equation}
    \forall A, \, \min \sum_{a = 0}^{A} y_a \cdot scaleCost_{a}
\label{eq:herocache-objective-allocation}
\end{equation}

\begin{equation}
    \text{s. t.} \, \sum_{a = 0}^{A} x_a \leq Total Resources
\label{eq:herocache-constraint-allocation}
\end{equation}

Le coût de l'allocation des ressources pour une application $a$ est la somme des coûts d'allocation de ses fonctions (équation~\ref{eq:herocache-scale-cost-app}). Une réplique est allouée à une plateforme d'exécution.

\begin{equation}
    scaleCost_{a} = \, \sum_{i = 0}^{F_{a}} scaleCost^{{f}_{{i}_{N, P}}}_a
\label{eq:herocache-scale-cost-app}
\end{equation}

Chaque réplique de fonction a un coût d'allocation associé, car l'allocation dynamique des ressources matérielles introduit un temps de latence lors du traitement des demandes des utilisateurs. 

Nous avons conçu un modèle de coût (équation~\ref{eq:herocache-scale-cost-function}) pour l'allocation des ressources nécessaires au déploiement d'une fonction d'une application donnée. Il est composé de quatre éléments, dont nous devons minimiser la somme :

\begin{itemize}
    \item La \textit{proportion de cache} $CP$ traduit la dispersion des fonctions sur les différents nœuds de bordure. Plus le score est élevé, plus les fonctions sont dispersées sur les nœuds. La minimisation de ce terme permet de consolider les fonctions ;
    \item Le \textit{temps total} $TT$ représente le temps d'exécution total de la fonction. Il tient compte de la qualité de service de l'application, de l'hétérogénéité de la plateforme et du coût de déploiement (si l'image est mise en cache ou distante). Plus ce coût est élevé, plus la qualité de service est faible ;
    \item La \textit{consommation d'énergie} $EC$ traduit la consommation d'énergie du déploiement de la fonction. Plus $EC$ est élevé, plus le coût est important ;
    \item Le \textit{prix du matériel} $HP$ décrit le coût total de possession (TCO) supporté par les fournisseurs de services en fonction du temps d'exécution. Il traduit le coût de déploiement sur une plateforme matérielle donnée. Plus $HP$ est élevé, plus le coût de la solution est important.
\end{itemize}

L'objectif global du modèle de coût est de déployer une fonction au coût le plus bas possible, c'est-à-dire une consolidation accrue, une réduction du makespan, une réduction de la consommation d'énergie et une réduction du coût de possession. Nous détaillerons chaque partie de l'équation~\ref{eq:herocache-scale-cost-function} dans les paragraphes suivants. Chaque composante de l'équation est pondérée pour permettre un réglage souple ; les valeurs que nous avons choisies pour le déploiement de l'application IDS sont spécifiées dans la partie consacrée à l'évaluation (Section~\ref{section:herocache-evaluation}).

\begin{equation}
\begin{split}
 \forall N, \forall P \in N, scaleCost^{{f}_{{i}_{N, P}}}_{a} = \,   &k_{CP} \cdot {CP}_{{a}_{N}}    \\
    + &k_{TT} \cdot {TT}_{{f}_{N, P}} \\
    + &k_{EC} \cdot {EC}_{{f}_{N, P}} \\
    + &k_{HP} \cdot {HP}_{{f}_{N, P}}
\end{split}
\label{eq:herocache-scale-cost-function}
\end{equation}

\textbf{Proportion de cache}. Comme nous l'avons vu précédemment, l'application de la consolidation des tâches (exécution d'une fonction) entre les applications devrait permettre de minimiser la communication et les retards dans les chaînes de fonctions. HeROcache favorise le déploiement de répliques d'une fonction sur des nœuds où d'autres fonctions appartenant à la même application sont déjà déployées.

Pour ce faire, HeROcache suit $CF_{a}^{{f}_{i_{N, P}}}$ le nombre d'images de fonctions ${f}_{i}$ de l'application $a$ déployée sur le nœud $N$ sur une plateforme d'exécution donnée $P$ (par exemple, GPU) disponible en cache sur le stockage local du nœud. La proportion de fonctions mises en cache est calculée pour chaque application (équation~\ref{eq:herocache-cached-functions}), puis la moyenne est calculée sur toutes les applications s'exécutant sur un nœud donné et inversée pour donner une valeur élevée aux fonctions non consolidées (l'objectif étant de minimiser cette proportion), voir équation~\ref{eq:herocache-cache-proportion-app}.

\begin{equation}
    \forall a \in A, \, \forall f \in a, \, CF_{a}^{{f}_{i_{N, P}}} = \frac{\sum_{i = 0}^{Fa} isCached(f_{i}, N, P)}{F_{a}}
\label{eq:herocache-cached-functions}
\end{equation}

\begin{equation}
    \forall N, \forall P \in N, \, {CP}_{{a}_{N}} = \, \frac{A}{\sum_{i = 0}^{F_{a}} CF_{a}^{{f}_{i_{N, P}}}}
\label{eq:herocache-cache-proportion-app}
\end{equation}

En plus de la minimisation des coûts, afin de réduire les délais de déploiement, l'autoscaler "pré-fetche" les images des chaînes de fonctions lors du déploiement d'une nouvelle réplique sur un nœud. Il inspecte les chaînes de fonctions et extrait séquentiellement les images de fonctions manquantes du registre distant vers le stockage local du nœud de manière asynchrone.

TODO: Prefetch intelligent \cite{leeSPESOptimizingPerformanceResource2024a}

\textbf{Temps total}. La deuxième composante du coût de mise à l'échelle est le \textit{temps total}. La minimisation du temps total devrait empêcher les retards d'initialisation de faire boule de neige dans les chaînes de fonctions, évitant ainsi les violations des accords de niveau de service (SLA).

Grâce aux métadonnées collectées sur chaque fonction pendant la phase hors-ligne, l'autoscaler est en mesure de prédire le temps total ${TT}_{{f}_{N, P}}$ de la première requête qui sera ordonnancée sur une nouvelle réplique de fonction (équation~\ref{eq:herocache-total-time-function}).

\begin{equation}
    {TT}_{{f}_{N, P}} = \, {RT}_{{f}_{N, P}} + {WT}_{{f}_{N, P}} + {CST}_{{f}_{N, P}} + {ET}_{{f}_{N, P}}
\label{eq:herocache-total-time-function}
\end{equation}

\begin{itemize}
    \item ${RT}_{{f}_{N, P}}$ est la durée du temps de récupération de l'image de la fonction. Si l'image de la fonction est déjà mise en cache sur le nœud de calcul, cette durée est nulle ; sinon, elle dépend de la taille de l'image $IS$ et est influencée par la largeur de bande de la liaison réseau $NB$, car l'image sera lue à partir d'un registre d'images distant, et par le débit du support de stockage du nœud $SMT$ et la latence $SML$, car l'image sera écrite et stockée localement en vue d'une utilisation ultérieure (équation~\ref{eq:herocache-retrieval-time}) ;

    \begin{equation}
        {RT}_{{f}_{N, P}} = \, \frac{IS_{{f}_{N, P}}}{\min (NB_{N}, SMT_{N})} + SML_{N}
        \label{eq:herocache-retrieval-time}
    \end{equation}

    \item ${WT}_{{f}_{N, P}}$ est le temps que la tâche passera à attendre dans la file d'attente de la plateforme. Au moment de la création de la réplique, ce temps sera égal à zéro car nous ne prévoyons que la latence de la première requête sur la réplique ;
    \item ${CST}_{{f}_{N, P}}$ est le temps de démarrage à froid nécessaire pour initialiser l'instance de la fonction (\textit{i.e.} décompresser l'image, préparer le conteneur, initialiser les bibliothèques, etc. Il est mesuré en fonction des métadonnées extraites ;
    \item ${ET}_{{f}_{N, P}}$ est la durée d'exécution de la fonction, y compris le temps de communication avec ses prédécesseurs et successeurs potentiels dans le DAG. Cette durée tient compte de l'extraction des métadonnées de la plateforme (équation~\ref{eq:herocache-execution-time}).
\end{itemize}

\begin{equation}
    {ET}_{{f}_{N, P}} = \, {CT}_{{f}_{N, P}} + {ST}_{{f}_{N, P}}
\label{eq:herocache-execution-time}
\end{equation}

${CT}_{{f}_{N, P}}$ est le \textit{temps de calcul} de la fonction -- le temps attendu pour que la fonction termine son exécution une fois entièrement initialisée. La valeur dépend des performances et de la disponibilité de la plateforme d'exécution. ${ST}_{{f}_{N, P}}$ est le \textit{temps de stockage} de la fonction -- le temps attendu pour que la fonction récupère ses données d'entrée et stocke ses données de sortie. La valeur dépend des performances de la liaison réseau et des dispositifs de stockage.

Le temps de stockage de la fonction ${ST}_{{f}_{N, P}}$ dépend de la taille de son état, \textit{c.-à-d.} de ses données d'entrée et de sortie. La récupération de l'entrée et le stockage de la sortie de chaque fonction de la chaîne dépendent des performances de la liaison réseau et du débit et de la latence du support de stockage sélectionné, comme le montre l'équation~\ref{eq:herocache-storage-time}.

\begin{equation}
    {ST}_{{f}_{N, P}} = \, \frac{SIS_{a}^{f_{i_{N, P}}} + SOS_{a}^{f_{i_{N, P}}}}{\min (NB_{N}, SMT_{N})} + SML_{N}
\label{eq:herocache-storage-time}
\end{equation}

\textbf{Consommation d'énergie et prix du matériel}. Enfin, la prise en compte de la consommation d'énergie et du prix du matériel devrait permettre de départager les candidats lorsque plusieurs allocations possibles semblent produire le même coût (en fournissant le même niveau de qualité de service).

${EC}_{{f}_{N, P}}$ et ${HP}_{{f}_{N, P}}$ correspondent respectivement (a) à la consommation d'énergie dynamique générée par cette allocation obtenue grâce à la phase de caractérisation hors-ligne de la charge de travail et de la plateforme et (b) au prix de détail suggéré par le fabricant (MSRP) du matériel mobilisé $Hardware Price_{P}$ pour déployer la fonction sur ledit nœud et ladite plateforme au regard du temps d'exécution de la fonction $ET_{{f}_{N, P}}$ (équation~\ref{eq:herocache-hardware-price}).

\begin{equation}
    {HP}_{{f}_{N, P}} = \frac{Hardware Price_{P}}{ET_{{f}_{N, P}}}
\label{eq:herocache-hardware-price}
\end{equation}

\subsection{Stratégie de minimisation des coûts d'ordonnancement et de placement des données}

Comme pour l'autoscaling, nous formulons un problème d'optimisation pour trouver la configuration d'ordonnancement optimale pour chaque demande d'utilisateur (puisque la qualité de service doit être garantie sur la base d'une demande d'utilisateur) et nous le résolvons à l'aide d'un simple algorithme d'avidité. L'objectif de l'ordonnanceur est de minimiser le coût du placement de $z_i$ tâches sur $R_i$ répliques de la fonction $i$ pour $y_a$ invocations de l'application $a$ (équation~\ref{eq:herocache-objective-scheduling}), sous la contrainte d'un ensemble fini de répliques de fonctions $R_{i}$ (équation~\ref{eq:herocache-constraint-scheduling}) précédemment déployées par l'autoscaler. Nous supposons que les applications sont toujours exécutées jusqu'au bout et que les nœuds ne tombent pas en panne ; il n'y a donc pas de coût associé aux migrations de tâches ou aux nouvelles tentatives.

\begin{equation}
    \min \sum_{a = 0}^{A} y_a \cdot schedCost_{a}
\label{eq:herocache-objective-scheduling}
\end{equation}

\begin{equation}
    \text{s. t.} \, \forall a \sum_{i = 0}^{F_a} z_i \leq \sum_{i = 0}^{F_a} R_{i}
\label{eq:herocache-constraint-scheduling}
\end{equation}

Comme la plateforme fonctionne à la granularité des fonctions, le coût d'ordonnancement d'une application $a$ est la somme du coût d'ordonnancement de sa chaîne de fonctions (équation~\ref{eq:herocache-scheduling-cost-app}).

\begin{equation}
    schedCost_{a} = \, \sum_{i = 0}^{A} schedCost^{{{f}_{i}}}_{a}
\label{eq:herocache-scheduling-cost-app}
\end{equation}

Chaque fonction ordonnancée dans la chaîne a un coût associé calculé pour chaque placement possible sur une réplique existante.
Nous avons conçu un modèle de coût (équation~\ref{eq:herocache-scheduling-cost-function}) pour le placement des tâches nécessaires à l'exécution d'une demande d'utilisateur pour une application. 

\begin{equation}
    schedCost_{{f}_{{i}_{N, P}}} = \, k_{QP} \cdot QP_{{f}_{N, P}} + k_{EC} \cdot {EC}_{{f}_{N, P}} + k_{TC} \cdot TC_{{f}_{N, P}}
\label{eq:herocache-scheduling-cost-function}
\end{equation}

Elle est composée de trois éléments, dont nous devons minimiser la somme :

\begin{itemize}
    \item La \textit{pénalité de qualité de service} $QP$ est encourue par le fournisseur de services lorsqu'une demande d'utilisateur n'est pas traitée en temps voulu. Il s'agit d'une valeur booléenne qui détermine si, en raison d'un placement donné, l'application ne respectera pas son échéance ;
    \item La \textit{consommation d'énergie} $EC$ traduit la consommation d'énergie dynamique induite par l'exécution de la fonction. Plus $EC$ est élevée, plus le coût est important ;
    \item La \textit{consolidation des tâches} $TC$ décrit l'utilisation des ressources pour un placement donné. Plus $TC$ est faible, plus la file d'attente de la réplique est proche de son seuil de concurrence, ce qui maximise l'utilisation du matériel.
\end{itemize}

L'objectif global du modèle de coût est de placer les tâches dans les répliques de fonctions au coût le plus bas possible, c'est-à-dire en évitant les pénalités subies par le fournisseur de services en cas de dépassement du délai de l'application fixé par la demande de l'utilisateur, en utilisant les plates-formes d'exécution les moins gourmandes en énergie possible et en appliquant un ratio d'utilisation élevé pour les ressources allouées à chaque fonction. Nous décrivons chaque partie de l'équation~\ref{eq:herocache-scale-cost-function} dans les paragraphes suivants. Chaque composante de l'équation est pondérée pour permettre un réglage flexible ; les valeurs que nous avons choisies pour le déploiement de l'application IDS sont spécifiées dans la partie consacrée à l'évaluation (Section~\ref{section:herocache-evaluation}).

\textbf{Pénalité de qualité de service}. L'ordonnanceur sélectionne les tâches entrantes en fonction de \textbf{l'échéance la plus proche d'abord}, en s'appuyant sur les métadonnées de la fonction pour calculer un temps d'exécution dans le pire des cas noté $WET$ (équation~\ref{eq:herocache-task-wet}). La demande de l'utilisateur est associée à un niveau de qualité de service qui définit un écart de qualité de service variable $QD$ appliqué au temps d'exécution de l'application. Il s'agit du délai d'exécution de la demande.

\begin{equation}
    \forall \, (N, P), \, WET_{f} = \, \max ET_{f_{N, P}}
\label{eq:herocache-task-wet}
\end{equation}

Nous pouvons prédire la pénalité de l'application en additionnant le temps total prévu pour ses tâches et en le comparant à l'échéance de l'application (somme des échéances des fonctions), voir équation~\ref{eq:herocache-scheduling-penalty}. Nous réutilisons l'équation~\ref{eq:herocache-total-time-function} pour calculer le temps total d'exécution d'une fonction ; cependant, ici, $RT$ et $CST$ seront nuls car la réplique a déjà été initialisée par l'autoscaler lors de l'allocation. $WT$ sera égal à la somme des temps d'exécution des tâches de priorité supérieure actuellement en file d'attente sur la réplique.

\begin{equation}
   QP_{a} = \, \sum_{i = 0}^{F_a} TT_{{f}_{{i}_{N, P}}} > \sum_{i = 0}^{F_a} WET_{f_{i}} \cdot QD_{a}
\label{eq:herocache-scheduling-penalty}
\end{equation}

En prenant en compte le temps de stockage dans le coût d'ordonnancement, nous cherchons à inciter l'ordonnanceur à placer les tâches aussi près que possible des données sur lesquelles elles opèrent. Pour éviter de saturer le stockage local des nœuds, la plateforme procède au nettoyage des données intermédiaires dès que l'application a terminé son exécution, \textit{i.e.} lorsque la dernière fonction de la chaîne renvoie sa valeur.

\textbf{Consommation d'énergie}. ${EC}_{{f}_{N, P}}$ correspond à la consommation d'énergie dynamique générée par cette configuration d'ordonnancement. Elle est liée au temps d'exécution de la fonction. Les résultats des mesures hors-ligne sont utilisés pour ce terme.

\textbf{Consolidation des tâches}. Nous voulons que les files d'attente des répliques de fonctions atteignent leur longueur maximale : le pire cas est d'avoir une file d'attente vide, ce qui signifie que des ressources matérielles auraient été allouées pour rien. Nous voulons également empêcher les files d'attente de répliques de croître trop rapidement au-delà de ce seuil, car cela pourrait entraîner des violations de la qualité de service en raison de longs temps d'attente.

Nous commençons par établir le ratio d'\textit{utilisation de la plateforme} $PU$ de chaque réplique pour la fonction que nous essayons d'ordonnancer (équation~\ref{eq:herocache-platform-usage}) : plus la longueur de la file d'attente de la réplique $Q$ est proche du seuil de concurrence ($threshold$ dans l'équation), plus le score est faible. 

\begin{equation}
    PU_{f_{N, P}} = \frac{Q_{N, P}}{threshold_{f, P}}
\label{eq:herocache-platform-usage}
\end{equation}

Ensuite, nous calculons un score de consolidation des tâches $TC$ en appliquant une fonction exponentielle à $PU$ (équation~\ref{eq:herocache-task-consolidation}). Ainsi, $TC$ est le plus faible pour les placements dans les répliques inactives, et ce coût augmente fortement au fur et à mesure que les files d'attente se remplissent, ce qui conduit l'ordonnanceur à donner la priorité aux placements sur les répliques vides et à ne pas tenir compte des répliques où la file d'attente des requêtes est saturée.

\begin{equation}
    TC_{{f}_{N, P}} = \, exp(PU_{f_{N, P}})
\label{eq:herocache-task-consolidation}
\end{equation}

\section{Évaluation}
\label{section:herocache-evaluation}

Cette section présente notre méthodologie d'évaluation et les résultats obtenus dans un scénario de déploiement d'IDS sur des dispositifs edge. L'évaluation se fait en deux phases : nous comparons HeROcache à plusieurs lignes de base, puis nous évaluons l'impact de chacun de ses composants (autoscaler et ordonnanceur) pris séparément.

\begin{figure*}[t]
    \center
    \subfloat[Consolidation\label{figure:herocache-evaluation-full-unused-nodes}]{
        \includegraphics[width=0.155\linewidth]{6_Chapitre4/figures/eval/2-unused-nodes.png}
    }
    \subfloat[QoS\label{figure:herocache-evaluation-full-penalty}]{
        \includegraphics[width=0.155\linewidth]{6_Chapitre4/figures/eval/3-penalty-proportions.png}
    }
    \subfloat[Energy\label{figure:herocache-evaluation-full-energy-consumption}]{
        \includegraphics[width=0.155\linewidth]{6_Chapitre4/figures/eval/6-energy-consumption.png}
    }
    \subfloat[Consolidation\label{figure:herocache-evaluation-components-unused-nodes}]{
        \includegraphics[width=0.155\linewidth]{6_Chapitre4/figures/eval-components/2-unused-nodes.png}
    }
    \subfloat[QoS\label{figure:herocache-evaluation-components-penalty}]{
        \includegraphics[width=0.155\linewidth]{6_Chapitre4/figures/eval-components/3-penalty-proportions.png}
    }
    \subfloat[Energy\label{figure:herocache-evaluation-components-energy-consumption}]{
        \includegraphics[width=0.155\linewidth]{6_Chapitre4/figures/eval-components/6-energy-consumption.png}
    }
    \caption{Evaluation -- Comparison against baselines (a-c) and impact of individual components (d-f)}
    \label{figure:herocache-evaluation}
\end{figure*}

\subsection{Protocole expérimental}

\textbf{Métadonnées de caractérisation hors-ligne}. Pour évaluer notre contribution, nous avons effectué des mesures pour trois applications IDS (voir Section~\ref{section:herocache-characterization-workloads}). Ces applications consistent en différentes fonctions de prétraitement et d'inférence qui ont été mises en œuvre sur du matériel hétérogène (voir Section~\ref{section:herocache-characterization-platforms}). Ces métadonnées ont servi d'entrée à un simulateur~\footnote{\href{https://github.com/b-com/HeROsim}{https://github.com/b-com/HeROsim}} que nous avons construit en utilisant SimPy~\cite{simpy}.

\textbf{L'ordonnanceur en ligne}. Nous avons généré des scénarios synthétiques en modélisant les demandes des utilisateurs comme un processus de Poisson, suivant une distribution uniforme entre les invocations d'applications, comme indiqué dans~\cite{9928755}. En modifiant le paramètre $\lambda$ du processus de Poisson, nous pouvons générer diverses traces avec différents taux de requêtes par seconde (RPS). Nous avons envisagé un scénario avec 10 nœuds de bordure communiquant via une connectivité 4G (LTE). La largeur de bande pour la 4G LTE dépend de divers facteurs allant de la couverture de l'antenne à la qualité de service du fournisseur de services de communication, en passant par la qualité du récepteur. Nous avons choisi d'utiliser des valeurs générales de 100 Mbps (12,5~MB/s). Les paquets TCP à analyser ont une taille de 1,5~KB et sont envoyés par lots de 100 unités aux applications IDS. Cela donne un taux de 83~RPS dans notre scénario, pour 10 minutes de requêtes d'utilisateurs. 

Les pondérations pour les décisions de mise à l'échelle automatique (équation~\ref{eq:herocache-scale-cost-function}) ont été fixées à $k_{CP} = \frac{3}{8}$, $k_{TT} = \frac{3}{8}$, $k_{EC} = \frac{1}{8}$ et $k_{HP} = \frac{1}{8}$. Les pondérations pour les décisions d'ordonnanceur (équation~\ref{eq:herocache-scheduling-cost-function}) ont été fixées à $k_{QP} = \frac{2}{3}$, $k_{EC} = \frac{0.5}{6}$ et $k_{TC} = \frac{1.5}{6}$. Nous utilisons des valeurs inspirées de \cite{herofake} afin d'être comparables.

Pour éviter une forme d'"emballement" (\textit{thrashing}) où les répliques sont créées et détruites en boucle lorsque la concurrence dans le système est très proche du seuil de concurrence, l'autoscaler applique un temps de maintien en vie faible qui empêche le retrait d'une réplique qui a été récemment allouée. Nous avons fixé ce temps de maintien en vie à 30 secondes, ce qui est la valeur par défaut dans Knative.

Dans nos expériences, nous permettons d'évaluer l'autoscaler et l'ordonnanceur séparément afin de mieux comprendre leur comportement. Nous avons évalué différentes combinaisons pour montrer quelle partie de chaque politique est pertinente pour relever les différents défis de notre problème. Nous avons mis en œuvre trois autoscalers dans notre simulateur :

\begin{itemize}
    \item HeROcache (HRC) -- Notre politique de mise à l'échelle automatique repose sur la mise en cache d'images de fonctions sur les nœuds edge et tente de prélever des images de fonctions pour satisfaire les dépendances avant le déploiement dans le DAG d'applications ;
    \item HeROfake (HRO)~\cite{herofake} -- Applique une politique similaire à HRC, mais ne tient pas compte des coûts de stockage : il n'utilise pas le cache d'image local du nœud lors de l'instanciation des répliques de fonctions, et n'effectue pas non plus la recherche préalable des images de fonctions en fonction du DAG de leur application ;
    \item Knative (KN)~\cite{sureshENSUREEfficientScheduling2020} -- Nous avons modélisé le comportement de l'autoscaler Knative au mieux de nos connaissances. Il déploie les répliques de fonctions sur le nœud le plus disponible, \textit{i.e.} il applique l'équilibrage de la charge.
\end{itemize}

En plus de ces autoscalers, nous avons utilisé quatre ordonnanceurs :

\begin{itemize}
    \item HeROcache (HRC) -- Notre politique d'ordonnancement sélectionne les demandes des utilisateurs entrants en fonction de l'échéance la plus proche afin de maximiser la qualité de service. Elle tient compte de la latence de communication prévue et sélectionne une réplique en fonction des exigences de temps de réponse dictées par la demande de l'utilisateur ;
    \item HeROfake (HRO)~\cite{herofake} -- Applique une politique similaire à HRC, mais ne tient pas compte des coûts de stockage : il ne prédit pas la latence des communications dans le DAG de l'application ;
    \item Knative (KN)~\cite{knative} -- Knative considère les plateformes d'exécution comme homogènes et n'applique pas la QoS. Les répliques sont triées en fonction du nombre de demandes en cours d'exécution ; la réplique ayant la file d'attente la plus courte est sélectionnée ;
    \item Bin-Packing First Fit (BPFF)~\cite{wangPeekingCurtainsServerlessb} -- Les tâches sont consolidées sur le nombre minimum de nœuds et de plateformes d'exécution. Les nœuds sont triés en fonction de la mémoire disponible ; la première réplique de fonction sur un nœud ayant de la mémoire disponible sera sélectionnée pour la demande de l'utilisateur. BPFF est susceptible d'être la politique d'ordonnanceur pour AWS Lambda ;
    \item Random Placement (RP) -- Les tâches sont ordonnancées sur une réplique sélectionnée de manière aléatoire.
\end{itemize}

Le nom de chaque scénario se compose de deux parties divisées par un symbole de tiret. La première partie correspond à la politique d'autoscaler ; la deuxième partie correspond à la politique d'ordonnanceur (par exemple, HRC-KN signifie que nous avons utilisé l'autoscaler HeROcache avec l'ordonnanceur Knative).

Nous avons conçu une évaluation des performances en deux étapes : \\
(1) \textbf{Comparaison avec les politiques de base} : nous comparons HeROcache complet (HRC-HRC) à : (1) Knative complet (KN-KN), (2) HeROfake complet (HRO-HRO), (3) l'autoscaler Knative avec l'ordonnanceur BPFF (KN-BPFF), (4) l'autoscaler Knative avec l'ordonnanceur RP (KN-RP). \\
(2) \textbf{Impact des composants de HeROcache sur les performances globales} : nous discutons de l'impact individuel de l'autoscaler et de l'ordonnanceur dans différentes stratégies : (1) l'autoscaler HeROcache avec l'ordonnanceur HeROfake (HRC-HRO), et (2) l'autoscaler HeROfake avec le scheduler HeROcache (HRO-HRC), en les comparant aux versions complètes de HeROcache et HeROfake.

Nous évaluons HeROcache sur la base de trois mesures : (1) le nombre de nœuds inutilisés dans l'infrastructure, qui mesure le niveau de consolidation atteint ; (2) les pénalités de qualité de service, qui expriment la capacité de notre stratégie à répondre aux exigences des utilisateurs ; (3) la consommation d'énergie, qui est un défi important à l'edge avec des ressources limitées.

\subsection{Analyse des résultats}

\subsubsection{Comparaison aux politiques de base}

\textbf{Consolidation des tâches}. La figure~\ref{figure:herocache-evaluation-full-unused-nodes} montre que notre combinaison d'autoscaler et d'ordonnanceur réalise la meilleure consolidation des tâches, en utilisant seulement 20\% de l'infrastructure edge pour l'exécution du scénario. Knative se comporte comme prévu, en répartissant la charge sur l'ensemble de l'infrastructure. Notez que BPFF sous Knative produit des résultats légèrement différents : comme les files d'attente des tâches sont maximisées, l'autoscaler n'a pas besoin d'allouer autant de répliques. Dans ce scénario, si les nœuds de bordure inutilisés étaient mis hors tension au lieu de rester inactifs, notre stratégie permettrait au fournisseur de services d'économiser près de 100 Wh (soit 80\% de l'énergie statique et plus de 83\% de l'énergie totale) en mettant hors tension 80\% de l'infrastructure, tout en garantissant le temps de réponse de l'application pour 72\% des demandes des utilisateurs.

\textbf{Qualité de service}. La figure~\ref{figure:herocache-evaluation-full-penalty} illustre la pertinence de la prise en compte de l'hétérogénéité des ressources. En effet, notre politique parvient à maintenir les violations de la qualité de service à 27,5\% tout en laissant 80\% de l'infrastructure inutilisée. Knative viole un peu plus de 30\% de la QoS des demandes des utilisateurs tout en répartissant la charge sur tous les nœuds de bordure disponibles, ce qui est contre-intuitif. Cela s'explique par les dépendances entre les tâches que Knative ne prend pas en compte. En conséquence, les tâches communiquent sur un réseau de stockage lent. Bien que les tâches dans Knative passent moins de temps dans la file d'attente, elles présentent toujours une latence plus élevée que dans HeROcache. Lors de l'utilisation de la politique BPFF, les violations atteignent presque 70\% : dans cette situation, les files d'attente des répliques sont trop longues pour que les tâches puissent être achevées dans les délais impartis. À titre de comparaison, Knative utilisant l'ordonnanceur RP maintient les violations de la QoS autour de 50\%. HeROfake génère 39\% de violations de la qualité de service.

Notre politique maintient la proportion de démarrages à froid en dessous de 0,011\% des demandes des utilisateurs, alors que Knative souffre de 4 fois plus de démarrages à froid. Dans HeROcache, le cache d'images local au nœud est utilisé dans 33\% des initialisations de fonctions, ce qui réduit les délais d'initialisation de 17,6\%.
Avec HeROcache, 30\% des tâches parviennent à communiquer par le biais du stockage local au niveau du nœud, ce qui accélère l'exécution de l'application en réduisant la latence des communications de 88,4\%.

\textbf{Consommation d'énergie}. La figure~\ref{figure:herocache-evaluation-full-energy-consumption} montre que HeROcache parvient à réduire la consommation d'énergie dynamique d'un tiers : avec un makespan de 1505 secondes pour le scénario, l'infrastructure consomme 0,0088 kWh, contre 0,0266 kWh pour 2193 secondes de temps d'exécution sous Knative. Non seulement la stratégie de consolidation de HeROcache permet d'appliquer des politiques de mise hors tension susceptibles de réduire considérablement les besoins en énergie statique pour l'exécution d'applications IDS en périphérie, mais en sélectionnant des plates-formes d'exécution adéquates, elle réduit également la consommation globale de la grappe en périphérie. HeROfake consomme le plus d'énergie (0,31 kWh) en raison d'un temps d'exécution beaucoup plus long pour le scénario.

\subsubsection{Impact des composants individuels}

\textbf{Consolidation des tâches}. La figure~\ref{figure:herocache-evaluation-components-unused-nodes} montre que les stratégies qui ne tiennent pas compte des coûts de stockage ne parviennent pas à consolider les tâches aussi bien que HeROcache : HRO-HRC et HRO-HRO utilisent respectivement 80\% et 70\% de l'infrastructure. Nous expliquons ces résultats de la manière suivante : les dépendances n'étant pas satisfaites à temps, la charge continue d'augmenter pour les différentes fonctions, ce qui conduit l'autoscaler à augmenter le nombre de répliques, enrôlant ainsi plus de nœuds pour la durée du scénario.

\textbf{Qualité de service}. La figure~\ref{figure:herocache-evaluation-components-penalty} illustre la conséquence du point précédent : Les pénalités de qualité de service sont plus élevées avec un autoscaler qui ne tient pas compte des délais introduits par l'extraction des images des fonctions et les communications des fonctions. Bien que HRO-HRC soit conscient de l'hétérogénéité du matériel et des requêtes, il termine tout de même avec 37,9\% des applications qui ne respectent pas leur délai.

\textbf{Consommation d'énergie}. La figure~\ref{figure:herocache-evaluation-components-energy-consumption} montre que, bien que HRO-HRC alloue 70\% de l'infrastructure, il parvient à maintenir une consommation d'énergie presque aussi faible que HRC-HRC. Cela s'explique par le fait qu'il a choisi les nœuds les moins gourmands en énergie, au prix de pénalités qu'il ne pouvait pas prévoir puisqu'il n'est pas conscient du stockage.

\textbf{Note sur la complexité} : HeROcache utilise une technique d'optimisation gourmande comparable à HeROfake. Dans HeROcache, la complexité est limitée par le nombre d'applications $A$, leur taille $f_{a}$ et la taille de l'infrastructure $N$ (équation~\ref{eq:herocache-complexity-autoscaler}) : dans le pire des cas où toutes les ressources sont disponibles, l'autoscaler parcourt l'ensemble de l'infrastructure $N$ pour évaluer chaque nœud en vue de la création de répliques.

\begin{equation}
    \mathcal{O}_{autoscaling}(A \cdot f_{a} \cdot N)
\label{eq:herocache-complexity-autoscaler}
\end{equation}

Comme l'ordonnanceur travaille avec des répliques déjà créées $R_{f}$ de fonctions, sa complexité est moindre (équation~\ref{eq:herocache-complexity-scheduler}).

\begin{equation}
    \mathcal{O}_{scheduling}(A \cdot f_{a} \cdot R_{f})
\label{eq:herocache-complexity-scheduler}
\end{equation}

Comme notre étude de cas actuelle implique un sous-ensemble limité de fonctions IDS avec un nombre raisonnable de nœuds edge, l'extensibilité n'a pas posé de problème. Toutefois, ces frais généraux devraient être pris en compte pour des déploiements plus larges de différentes études de cas. Ces frais généraux n'ont pas été mesurés en simulation. Toutefois, comme l'autoscaler fonctionne de manière périodique, la fréquence de la période pourrait être réglée pour ajuster l'algorithme en fonction des besoins du fournisseur de services. L'ordonnanceur est appelé au moment des demandes des utilisateurs et pourrait être réparti sur les nœuds si la charge est trop lourde à gérer.

\section{Travaux connexes}
\label{section:herocache-sota}

Des travaux antérieurs se sont concentrés sur les plates-formes de mise à l'échelle automatique pour le déploiement de tâches de courte durée, comprises dans des applications présentant des modèles de charge imprévisibles (voir Tableau~\ref{table:herocache-sota}).

Parmi ces travaux, \cite{smithFaDOFaaSFunctions2022} propose un orchestrateur conscient des données, mais ne tient pas compte de l'effet boule de neige des retards dans les chaînes de fonctions. \cite{zhangFIRSTExploitingMultiDimensional2023} ne prend pas en charge l'ordonnanceur de ces chaînes de fonctions.
Toutes ces contributions considèrent une infrastructure homogène \cite{bhasiCypressInputSizesensitive2022, zijunFassflowEfficient2022, smithFaDOFaaSFunctions2022, zhangFIRSTExploitingMultiDimensional2023, abdiPaletteLoadBalancing2023}. Cela n'est pas représentatif de notre cas d'utilisation, où les appareils edge sont très hétérogènes. HeROfake~\cite{herofake} exploite l'hétérogénéité matérielle dans sa politique d'orchestration, mais n'intègre pas les dépendances inter-fonctions ni la mise en cache d'images dans son modèle de coût. Elle a été choisie à des fins d'évaluation pour souligner la nécessité de prendre en compte ces coûts.
Certaines de ces contributions optimisent la consommation d'énergie au niveau de l'autoscaler \cite{bhasiCypressInputSizesensitive2022, zhangFIRSTExploitingMultiDimensional2023}. Toutefois, ils se concentrent sur la partie dynamique de la consommation d'énergie : ils ne tiennent pas compte de l'impact possible de la consolidation sur la consommation d'énergie statique. Nous soutenons que les fournisseurs de services devraient chercher à consolider les tâches afin de mettre hors tension le plus grand nombre de nœuds possible, ce qui réduirait considérablement les besoins énergétiques globaux de l'infrastructure.
Dans \cite{fuerstIluvatarFastControl2023}, les auteurs ont étudié les différents frais généraux infligés par l'orchestration serverless. Cet élément n'a pas été pris en compte dans notre étude, car nous visons une infrastructure edge de taille limitée pour le déploiement d'une seule application.

\section{Conclusion et perspectives}
\label{section:herocache-conclusion}

Dans ce travail, nous avons présenté une politique d'allocation et d'ordonnanceur pour le serverless à l'edge. Cette politique cherche à optimiser le déploiement d'applications sensibles au temps pour la qualité de service sur des dispositifs à énergie limitée. En tirant parti de la caractérisation de la charge de travail, de l'hétérogénéité du matériel et des dispositifs de stockage locaux sur les nœuds edge, HeROcache consolide les applications et parvient à réduire les délais d'initialisation moyens de 17,6\% et les délais de communication de 88,4\%. Cela permet de réduire la consommation d'énergie statique de la plateforme de 80\% tout en maintenant moins de 28\% de violations de la qualité de service. Nous prévoyons de généraliser l'approche HeROcache pour des études de cas incluant plusieurs types d'applications sur des infrastructures à l'edge ou en nuage plus importantes. Dans ce cas, des stratégies d'apprentissage automatique ou des métaheuristiques pourraient être utilisées à des fins de mise à l'échelle.


\clearemptydoublepage
\chapter{HeROsim : Élaborer et évaluer des politiques d'orchestration serverless pour le cloud privé}

TODO: nouvelles références à intégrer~\cite{bambrikSurveyCloudComputing2020, byrneReviewCloudComputing2017}

TODO: màj bibtex (références raccourcies pour IEEE IC)

\section{Introduction}
\label{section:herosim-introduction}

Le modèle de service serverless permet une élasticité rapide et un service mesuré à la granularité la plus fine dans le cloud. Il libère le potentiel d'optimisation fine de l'utilisation des ressources. Dans un tel modèle, les développeurs conçoivent leurs applications comme une composition de fonctions sans état dont l'exécution est dirigée par des événements~\cite{SchleierSmith2021WhatSC}. 
Les clients sont facturés en fonction de leur utilisation réelle des ressources, et les fournisseurs sont chargés de déployer une gestion intelligente des ressources afin d'optimiser les mesures de qualité de service (QoS) telles que le temps de réponse, la consommation d'énergie et l'utilisation des ressources.

Le serverless s'accompagne d'un ensemble de défis qui doivent être relevés par les fournisseurs de services. Par exemple, la surcharge de latence causée par l'allocation dynamique du matériel pour les nouvelles instances de fonction, appelée délai de démarrage à froid, est un problème clé~\cite{Lannurien2023}. Un autre problème récurrent est l'architecture dite "data-shipping", dans laquelle des gigaoctets de données stockées à distance sont téléchargés vers des kilooctets de code sur les nœuds de calcul~\cite{yuFollowingDataNot}, ce qui entraîne une dégradation considérable des performances.

Pour surmonter de tels défis, il faut concevoir des politiques d'orchestration qui guident les décisions d'autoscaling et d'ordonnancement.
L'orchestration est un processus dynamique : chaque décision crée un nouvel état pour le système, ce qui conduit à une explosion combinatoire. De plus, les plateformes serverless sont des logiciels génériques qui exposent de nombreux boutons que le fournisseur peut régler. Ces paramètres peuvent avoir un impact sur la latence des requêtes des utilisateurs, le débit de la plateforme, l'utilisation des ressources du centre de données et la consommation d'énergie. Comme il est coûteux d'expérimenter en production, nous soutenons que les outils de simulation sont essentiels pour les fournisseurs de cloud serverless privés qui cherchent à optimiser ces mesures de qualité de service en fonction de leurs propres objectifs.

Ces politiques doivent être évaluées en fonction de leur pertinence dans différents cas d'utilisation. Différentes mesures peuvent être utiles pour cette évaluation, par exemple la durée de vie pour un scénario donné, le temps de réponse des fonctions, la consommation d'énergie dynamique et statique, etc.

Il y a deux façons possibles d'évaluer si les politiques que nous concevons ont un impact positif sur ces mesures : la simulation (estimation) ou la mise en œuvre (mesure). Il y a un compromis entre le temps et la précision dans les deux approches ; choisir la simulation plutôt que la mise en œuvre permet une exploration rapide de l'espace du problème et l'itération sur les solutions possibles. En particulier, la modélisation à événements discrets nous permet d'explorer des solutions en représentant les différents composants de la plateforme avec des processus qui modélisent l'état du système et son évolution dans le temps.

Les outils de simulation des politiques d'orchestration dans le cloud serverless privé doivent permettre de tracer les événements d'allocation des ressources et de placement des tâches à la \textbf{granularité la plus fine}, afin de comprendre l'impact des politiques sur les métriques de performance. Pour représenter des cas d'utilisation réalistes, le logiciel doit offrir la possibilité de modéliser des applications qui présentent des dépendances \textbf{données et temporelles entre les exécutions de tâches}. En outre, le simulateur doit prendre en charge l'hétérogénéité globale dans le cloud : d'une part, les centres de données sont constitués de divers matériels présentant différents niveaux de coût et de performance ; d'autre part, il existe un large éventail de clients ayant des exigences différentes en matière de qualité de service. Enfin, le simulateur devrait pouvoir \textbf{rejouer les traces d'exécution} pour différentes politiques d'orchestration et \textbf{comparer les résultats} en termes de métriques de qualité de service pour chaque politique évaluée.

Des travaux antérieurs ont proposé différents outils de simulation intéressants pour le nuage, voir le tableau~\ref{table:herosim-sota}. Certains d'entre eux ne ciblent pas le paradigme serverless~\cite{calheiros_cloudsim_2011, wickremasinghe_cloudanalyst_2010, cai_elasticsim_2017, buyyaGridSimToolkitModeling2002, nunez_icancloud_2012, mahmudIFogSim2ExtendedIFogSim2021}. Parmi les simulateurs serverless, certains se concentrent sur les offres de cloud public~\cite{nunez_icancloud_2012, mahmoudiSimFaaSPerformanceSimulator2021}, principalement pour permettre de concevoir des stratégies hybrides où une partie des charges de travail est déchargée sur un cloud public moins cher. Pour les nuages privés, certains simulateurs ne prennent pas en compte la consommation d'énergie~\cite{jeonCloudSimExtensionSimulatingDistributed2019, cai_elasticsim_2017, buyyaGridSimToolkitModeling2002, nunez_icancloud_2012} ; d'autres ne prennent pas en compte l'hétérogénéité du matériel~\cite{jeonCloudSimExtensionSimulatingDistributed2019, nunez_icancloud_2012, mahmoudiSimFaaSPerformanceSimulator2021}. La plupart des simulateurs ne permettent pas de modéliser des applications composées de fonctions multiples avec des dépendances de données~\cite{calheiros_cloudsim_2011, mampage_cloudsimsc_2023, wickremasinghe_cloudanalyst_2010, jeonCloudSimExtensionSimulatingDistributed2019, buyyaGridSimToolkitModeling2002, nunez_icancloud_2012, mahmudIFogSim2ExtendedIFogSim2021}. Dans certaines études, la qualité de service ne peut pas être appliquée à la granularité d'une requête utilisateur~\cite{calheiros_cloudsim_2011, mampage_cloudsimsc_2023, wickremasinghe_cloudanalyst_2010, cai_elasticsim_2017, nunez_icancloud_2012, mahmudIFogSim2ExtendedIFogSim2021, mastenbroekOpenDCConvenientModeling2021, mahmoudiSimFaaSPerformanceSimulator2021}.

Pour remédier à ces limitations, nous proposons \textbf{HeROsim}~\footnote{\href{https://github.com/b-com/HeROsim}{https://github.com/b-com/HeROsim}}, un \textbf{He}stéréogène \textbf{R}ressources \textbf{O}rchestration \textbf{sim}ulateur libre et à source ouverte présentant les caractéristiques suivantes :

\begin{itemize}
    \item Description fine des applications serverless sur du matériel hétérogène : performances, flux de données, besoins en mémoire, consommation d'énergie ;
    \item Allocation dynamique des ressources matérielles et placement des requêtes utilisateur compatibles avec la qualité de service : HeROsim offre des points d'entrée aux utilisateurs pour mettre en œuvre leurs propres politiques de sélection des ressources ;
    \item Évaluation des politiques d'orchestration en fonction de paramètres de qualité de service tels que la latence et l'énergie : HeROsim permet de comparer les résultats de différentes stratégies par rapport à certains paramètres de qualité de service courants.
\end{itemize}

La conception du logiciel suit l'architecture de référence des orchestrateurs de l'état de l'art tels que Google Knative~\footnote{\href{https://knative.dev}{https://knative.dev}} ou Apache OpenWhisk~\footnote{\href{https://openwhisk.apache.org}{https://openwhisk.apache.org}}. Ces orchestrateurs se composent de deux modules principaux :

\begin{itemize}
    \item Le \textbf{autoscaler} détermine comment mettre à l'échelle automatiquement et de manière réactive les ressources matérielles dans un nuage en adéquation avec la charge des applications.
    \item Le \textbf{scheduler} détermine sur quelle réplique mettre en file d'attente les requêtes utilisateur pour une fonction donnée en fonction des exigences de qualité de service.
\end{itemize}

L'article est organisé comme suit : la première section donne un aperçu du fonctionnement d'une plateforme serverless ; la section suivante détaille la conception de HeROsim ; la section " Étude de cas " montre comment HeROsim peut être exploité à travers deux cas d'utilisation tirés de nos publications précédentes ; nous discutons ensuite des travaux de l'état de l'art ; et enfin, nous concluons l'article.

\section{Présentation générale de la plateforme simulée}
\label{section:herosim-overview}

\begin{figure*}[t]
    \centering
    \includegraphics[width=0.9\textwidth]{7_Chapitre5/figures/platform.png}
    \caption{Vue de haut niveau d'une plateforme serverless, telle que modélisée dans HeROsim. L'autoscaler alloue des ressources matérielles pour les répliques de fonctions ; tandis que l'ordonnanceur place les requêtes utilisateur en file d'attente sur ces répliques}.
\label{figure:herosim-platform}
\end{figure*}

Dans le cloud, les clients réservent généralement des ressources. Il s'agit généralement d'un sous-ensemble virtualisé de ressources matérielles hétérogènes disponibles sur des serveurs appelés nœuds. Une fois la réservation effectuée, les fournisseurs de services offrent un accès à distance au client, qui est responsable du déploiement de ses applications et facturé en fonction de la quantité de ressources qu'il a réservées~\cite{Lannurien2023}.

Dans le paradigme serverless, les clients commencent par pousser leur code dans un référentiel du côté du fournisseur, voir la figure~\ref{figure:herosim-platform}. Le fournisseur alloue des ressources qui sont automatiquement mises à l'échelle en fonction de la charge. Pour que ce mécanisme fonctionne, les applications sont divisées en petites unités d'exécution sans état appelées \textbf{fonctions}. Ces fonctions sont sans état : si elles produisent des données de sortie, celles-ci doivent être conservées dans un stockage persistant~\cite{yuFollowingDataNot}.

Dans les modèles de service traditionnels, les ressources sont mises à l'échelle sur deux dimensions : horizontalement (nouvelles instances d'application créées sur d'autres nœuds) et verticalement (ressources supplémentaires allouées aux instances existantes). Dans les plateformes serverless, les ressources sont mises à l'échelle horizontalement : les variations de charge des applications sont absorbées par l'ajout de nouvelles instances des fonctions, appelées \textbf{replicas}, et leur suppression lorsqu'elles ne sont plus nécessaires. Une réplique peut être créée pour chaque requête utilisateur ou réutilisée pour plusieurs requêtes d'utilisateurs. Nous pouvons considérer les répliques de fonctions comme des files d'attente de requêtes ayant des capacités différentes.

Ces répliques sont créées par l'\textbf{autoscaler}. Pour gérer le nombre de répliques déployées pour chaque fonction, il existe trois stratégies principales : basée sur les requêtes, basée sur la concurrence et basée sur les métriques~\cite{mahmoudiSimFaaSPerformanceSimulator2021}. Dans le cadre de l'autoscaling basé sur les requêtes, les requêtes arrivant dans le système sont traitées par des répliques inactives. Si aucune réplique inactive n'est disponible, une nouvelle instance est créée. Dans l'autoscaling basé sur la concurrence, chaque réplique peut mettre en file d'attente plusieurs requêtes d'utilisateurs et les traiter séquentiellement en fonction d'un seuil de concurrence prédéfini~\cite{herofake}. Dans l'autoscaling basé sur les métriques, le nombre de répliques déployées dépend de divers objectifs, tels que le taux de requêtes par seconde (RPS) à atteindre. Pour ce faire, il faut surveiller les performances du système à l'usage de l'autoscaler.

Les répliques peuvent se trouver dans trois états différents : initialisation, exécution et inactivité~\cite{SchleierSmith2021WhatSC}. Lorsqu'une réplique de fonction vient d'être créée, elle est en état d'initialisation : la plateforme instancie son environnement d'exécution, extrait son code d'un registre distant et le met éventuellement en cache sur le nœud de déploiement, puis commence à exécuter la fonction. Lorsque la réplique traite les requêtes utilisateur, elle est en état d'exécution. Dans le cas contraire, la réplique est inactive et peut être supprimée. Lorsque les répliques sont supprimées, les ressources matérielles sont libérées. Toutefois, la création d'une nouvelle réplique pour traiter les requêtes utilisateur entraîne un surcoût appelé \textbf{démarrage à froid}. Les orchestrateurs adoptent diverses politiques pour atténuer ce problème, allant de l'application d'une période de maintien en vie sur les répliques de fonctions pour éviter de les détruire trop tôt, à l'allocation proactive de répliques.

Enfin, les requêtes utilisateur sont attribuées aux répliques de fonctions disponibles par l'\textbf{ordonnanceur} qui met en œuvre différentes stratégies : par exemple AWS Lambda\footnote{\href{https://aws.amazon.com/en/lambda/}{https://aws.amazon.com/en/lambda/}} utilise un algorithme de bin-packing, tandis qu'une plateforme open source telle que Knative met en œuvre une politique d'équilibrage de la charge~\cite{Lannurien2023}. Ces stratégies ont des résultats différents en ce qui concerne l'utilisation des ressources et la qualité de service, mais il peut être difficile de prédire dans quelle mesure elles auront un impact sur les charges de travail avant le déploiement.

\section{Choix de conception}
\label{section:herosim-herosim}

Cette section présente les choix de conception et les hypothèses formulées pour le développement de HeROsim.

\begin{figure}[t]
    \centering
    \includegraphics[width=\columnwidth]{7_Chapitre5/figures/software-architecture.png}
    \caption{Vue de haut niveau de l'architecture du simulateur.}
\label{figure:herosim-software-architecture}
\end{figure}

HeROsim utilise la bibliothèque SimPy~\footnote{\href{https://simpy.readthedocs.io}{https://simpy.readthedocs.io}} comme moteur de simulation à événements discrets. Il fournit trois classes de base - \texttt{Orchestrator}, \texttt{Autoscaler} et \texttt{Scheduler} - qui devraient être sous-classées par les utilisateurs désireux d'implémenter leurs propres algorithmes. Comme l'essentiel du comportement de la plateforme est hérité des classes de base, le coût de mise en œuvre d'une nouvelle politique est minime : le couple le plus simple d'autoscaler et d'ordonnanceur (politique de placement aléatoire) est mis en œuvre en moins de 20 lignes de code.

\subsection{Données d'entrée}

HeROsim expose une interface déclarative permettant aux utilisateurs de définir leur environnement cloud, leurs charges de travail et leurs contraintes de qualité de service. Le simulateur reproduit un scénario d'allocation et de placement selon différentes politiques d'orchestration. Une exécution du simulateur nécessite les entrées JSON suivantes pour définir un tel scénario :

\begin{enumerate}
    \item Une \textbf{description des applications} comportant les mesures des des \textbf{caractéristiques des fonctions} invoquées au cours du scénario, \textit{i.e.} leurs temps d'exécution et de démarrage à froid, leurs besoins mémoire, leur consommation d'énergie, la taille de l'image des fonctions, la taille des entrées/sorties lors des phases de communication. Ces métadonnées peuvent être définies en \textbf{mesurant} et \textbf{analysant} le comportement des applications sur une configuration de banc d'essai (voir figure~\ref{figure:herosim-characterization}) ;
    \item Une \textbf{description de l'infrastructure} listant les différents \textbf{nœuds disponibles} et les plateformes d'exécution qu'ils comportent, leurs supports de stockage et la bande passante disponible sur le réseau. Les plateformes d'exécution sont définies en termes de consommation d'énergie au repos et de prix de détail ; les dispositifs de stockage sont caractérisés par leur capacité, leur bande passante et leur latence. Ces données sont spécifiques à une plateforme cible et peuvent être \textbf{mesurées} ou obtenues auprès des fabricants ;
    \item Une \textbf{trace d'exécution} faisant état des temps d'arrivée pour toutes les \textbf{requêtes utilisateur} vers les applications déployées, associées au niveau de qualité de service demandé par l'utilisateur, dans l'ordre chronologique. Ces données peuvent être extraites par \textbf{observations} réelles des applications en production (voir figure~\ref{figure:herosim-characterization}), ou estimées statistiquement.
\end{enumerate}

Les données de nos études précédentes~\cite{herofake, herocache} sont disponibles dans le référentiel pour référence. Alors que (1) et (2) peuvent être rédigés à la main en fonction des exigences du cas d'utilisation, HeROsim fournit un générateur synthétique pour créer diverses traces pour (3) en utilisant des processus de Poisson, avec une durée variable et un taux requêtes par seconde, comme cela est couramment fait dans la littérature~\cite{herocache}. 

\subsection{Flot d'une simulation}

L'utilisateur peut choisir les politiques d'orchestration souhaitées et exécuter le programme principal. Le simulateur :

\begin{itemize}
    \item Initialise l'infrastructure comme décrit : le scénario commence avec tous les nœuds inactifs, en attente de nouvelles requêtes ;
    \item Initialise l'orchestrateur avec l'autoscaler et l'ordonnanceur choisis ;
    \item Suit les temps d'arrivée des événements à partir de la trace d'exécution et transmet les requêtes utilisateur à l'orchestrateur ;
    \item Laisse l'ordonnanceur essayer de placer ces requêtes sur des répliques de fonctions ;
    \item Laisse l'autoscaler allouer et désallouer les ressources matérielles qui hébergent ces répliques pour traiter les requêtes utilisateur.
\end{itemize}

La simulation progresse dans le traitement des requêtes utilisateur. Le simulateur connaît le temps de réponse des fonctions sur la base des métadonnées d'entrée mesurées au préalable. Ces métadonnées concernent le matériel et les charges de travail spécifiques que l'utilisateur souhaite ordonnancer. Elles peuvent être obtenues de différentes manières ; la figure~\ref{figure:herosim-characterization} donne un aperçu de la méthodologie que nous avons utilisée pour caractériser les différentes charges de travail et plates-formes d'exécution tout au long de nos expériences ; voir~\cite{herofake, herocache} pour plus de détails.

\begin{figure}[t]
    \centering
    \includegraphics[width=\columnwidth]{7_Chapitre5/figures/characterization.png}
    \caption{Une vue d'ensemble de notre méthodologie de caractérisation.}
\label{figure:herosim-characterization}
\end{figure}

Pendant la simulation, les journaux sont écrits sur le disque. Lorsque toutes les requêtes utilisateur ont été traitées, la simulation s'arrête et renvoie les résultats et les graphiques résumant la simulation, respectivement dans les répertoires \texttt{result} et \texttt{chart}.

Le simulateur est monotâche, ce qui signifie que chaque politique sera évaluée de manière séquentielle. Cependant, plusieurs instances de HeROsim peuvent être exécutées en parallèle avec différentes configurations sur plusieurs cœurs de CPU, ou distribuées sur plusieurs nœuds. Chaque instance de HeROsim exécutant une politique d'orchestration différente et toutes les instances utilisant un répertoire de sortie commun, cela peut accélérer la durée totale du scénario de simulation lors de l'évaluation de nombreuses politiques d'orchestration. Lorsque toutes les exécutions du simulateur sont terminées, les résultats peuvent être consolidés dans un seul tableau de sortie.

\subsection{Orchestrateur}

La classe de base \textbf{\texttt{Orchestrator}} fournit des méthodes d'initialisation abstraites qui doivent être surchargées pour gérer différentes structures de l'état du système. Par exemple, un ordonnanceur Round Robin devra savoir combien de fois chaque réplique de fonction a été sélectionnée pour le placement des tâches, tandis qu'un ordonnanceur Least Connected devra connaître la concurrence moyenne dans chaque réplique pour équilibrer la charge. Cette classe est le point d'entrée permettant aux utilisateurs de définir leurs propres structures de données qui représenteront au mieux l'état du système à gérer par l'autoscaler et l'ordonnanceur.

Les utilisateurs peuvent mettre en œuvre le processus de gestion de l'état du système dont ils ont besoin pour soutenir leurs politiques d'orchestration. L'orchestrateur de base est doté d'un système simple qui peut fonctionner tel quel ou être étendu. Dans notre implémentation, un processus de surveillance est appelé périodiquement pour garder une trace de la concurrence moyenne dans chaque réplique de fonction. Ceci est utile pour les politiques basées sur des seuils.

La méthode du point d'entrée de l'orchestrateur est appelée chaque fois qu'une requête utilisateur arrive sur la plateforme. Elle prend en entrée l'état du système et la requête utilisateur. Elle peut être surchargée dans chaque implémentation de politique d'orchestration pour permettre une mise à l'échelle automatique et une ordonnanceur à grain fin selon les besoins.

Enfin, cette classe est responsable de l'instanciation des implémentations sélectionnées pour \texttt{Autoscaler} et \texttt{Scheduler}. Toute combinaison de ces deux modules peut être instanciée.

\subsection{Allocateur}

La classe de base \textbf{\texttt{Autoscaler}} fournit le comportement commun de la plateforme d'autoscaling, y compris la création et la suppression des répliques de fonctions. Plusieurs méthodes abstraites doivent être remplacées pour mettre en œuvre une nouvelle politique : sélection des ressources pour la création de répliques, processus d'initialisation des répliques, sélection des répliques pour la suppression, etc. Ces méthodes opèrent à la granularité d'une seule fonction, en prenant l'état du système et la liste des ressources matérielles disponibles comme données d'entrée. Les utilisateurs sont libres de mettre en œuvre les algorithmes qu'ils souhaitent évaluer pour la gestion des ressources.

L'autoscaler de HeROsim a été principalement conçu pour une mise à l'échelle horizontale. Les répliques de fonctions sont créées en allouant une plateforme d'exécution et la quantité requise de mémoire de crête sur un nœud. Une plateforme d'exécution ne peut pas héberger plus d'une réplique de fonction à la fois. Pour faire face à l'augmentation de la charge d'une application sans dégradation de la qualité de service, de nouvelles répliques de fonctions doivent être allouées par l'autoscaler, à condition qu'il y ait suffisamment de ressources matérielles disponibles. Les répliques nouvellement allouées passent par une phase d'initialisation au cours de laquelle les images des fonctions doivent être récupérées via le réseau. L'autoscaler peut gérer un cache d'images dans la mémoire du nœud et sur le stockage local du nœud afin d'accélérer les démarrages à froid.

La suppression des répliques inactives se fait au mieux : l'autoscaler tente de supprimer les répliques dont les files d'attente de tâches sont vides. Par défaut, une réplique avec des tâches en attente ne peut pas être supprimée.

L'autoscaler garde une trace de chaque événement d'allocation pour calculer l'utilisation des ressources à la fin de la simulation. HeROsim permet à l'utilisateur de savoir quels nœuds et quelles plates-formes d'exécution ont été enrôlés pendant le scénario, à quel moment et pour quelle durée, et pour quel déploiement de fonction ils ont été choisis. Cela permet également à HeROsim de calculer la consommation d'énergie à différentes granularités : la puissance statique nécessaire au matériel alloué et la puissance dynamique utilisée par les applications pendant leur exécution.

HeROsim est livré avec les politiques de mise à l'échelle automatique suivantes, basées sur des seuils de concurrence et prêtes à l'emploi :

\begin{itemize}
    \item Random -- Sélectionne un nœud aléatoire et une plateforme d'exécution pour les nouvelles répliques ;
    \item Knative -- Sélectionne le nœud le moins chargé pour allouer de nouvelles répliques, \textit{i.e} équilibre les charges de travail sur un grand nombre de nœuds ;
    \item HeROfake -- Exploite l'hétérogénéité du matériel pour minimiser les pénalités de qualité de service, la consommation d'énergie et le coût total de possession. Plus de détails sont donnés dans la section "Étude de cas" ;
    \item HeROcache -- Optimise les allocations pour les chaînes de fonctions ; maximise les fonctions de consolidation de chaque application. Plus de détails sont donnés dans la section "Étude de cas".
\end{itemize}

\subsection{Ordonnanceur}

La classe de base \textbf{\texttt{Scheduler}} met en œuvre la sélection des tâches dans la file d'attente de la passerelle. Une méthode abstraite doit être surchargée pour mettre en œuvre une nouvelle politique : la sélection d'une réplique parmi le pool pour placer chaque requête utilisateur dans la file d'attente. Cette méthode opère à la granularité d'une requête utilisateur et prend en entrée l'état du système et la liste des répliques de fonctions disponibles.

Les requêtes utilisateur arrivent dans une file d'attente au niveau de l'ordonnanceur. Les utilisateurs peuvent mettre en œuvre leur propre politique de priorité pour la sélection des tâches ou choisir une politique déjà disponible dans le simulateur, \textit{e.} FIFO ou Earliest Deadline First~\cite{herofake}.

L'ordonnanceur de HeROsim a été conçu sans tenir compte des défaillances ou des migrations de tâches : le comportement par défaut considère les tâches qui s'exécutent toujours jusqu'à leur terme sur leur réplique. Cependant, les tâches seront marquées comme "en pénalité" si l'ordonnanceur manque son échéance. Les utilisateurs peuvent utiliser cette valeur booléenne pour évaluer la qualité de leurs politiques en ce qui concerne la latence des requêtes : elle indique la proportion de requêtes qui sont traitées en temps voulu. 

S'il n'y a pas de réplique disponible au moment de l'ordonnancement d'une requête utilisateur, l'ordonnanceur fera un appel à l'autoscaler pour forcer la création d'une première réplique pour la fonction. Dans l'intervalle, la requête est remise dans la file d'attente et reportée. Les tâches reportées sont signalées comme telles, de sorte que, par exemple, elles peuvent avoir une priorité plus élevée si l'utilisateur souhaite appliquer une telle politique~\cite{herocache}.

HeROsim est livré avec les politiques d'ordonnanceur suivantes implémentées et prêtes à l'emploi :

\begin{itemize}
    \item Random -- Sélectionne une réplique aléatoire pour le placement des tâches ;
    \item Knative -- Sélectionne la réplique avec la file d'attente la plus courte pour le placement des tâches ;
    \item BPFF -- Sélectionne la réplique avec la plus longue file d'attente de requêtes en vol pour le placement des tâches ;
    \item HeROfake -- Sélectionne la réplique qui minimise un score composé en fonction de l'échéance de la tâche, de la consommation d'énergie de la fonction et de la dispersion des tâches sur les nœuds. Plus de détails sont donnés dans la section "Étude de cas" ;
    \item HeROcache -- Sélectionne la réplique similaire à HeROfake, mais prend en compte les opérations de stockage et de communication pour calculer la latence de bout en bout de la requête, et prend en compte les chaînes de fonctions lors de l'évaluation des répliques en ce qui concerne la consolidation des tâches. De plus amples détails sont donnés dans la section "Étude de cas".
\end{itemize}


\subsection{Interface utilisateur}

HeROsim s'appuie sur la journalisation pour fournir un aperçu du déroulement de la simulation, ce qui aide l'utilisateur à déboguer ses politiques. À la fin de la simulation, les fichiers de résultats sont enregistrés sur le disque. Ces fichiers contiennent des informations sommaires sur les résultats de la simulation, y compris les performances de la politique en ce qui concerne les mesures d'évaluation. Ces résultats sont représentés sur différents graphiques qui peuvent être utilisés dans des publications ultérieures (voir figure~\ref{figure:herosim-evaluation}).

HeROsim peut également générer des graphiques supplémentaires qui peuvent être utiles lors du débogage du comportement d'une politique d'orchestration. Les utilisateurs peuvent visualiser la proportion de démarrages à froid et d'accès au cache parmi les invocations de fonctions. HeROsim peut tracer la latence de la queue pour toutes les requêtes des utilisateurs, aidant ainsi l'utilisateur à dimensionner son infrastructure. Le générateur de traces d'exécution trace les temps d'arrivée des requêtes sur un graphique afin de fournir des indications visuelles sur les caractéristiques de la charge de travail.

\begin{figure}[t]
    \centering
    \includegraphics[width=\columnwidth]{7_Chapitre5/figures/serverless-cost.png}
    \caption{Répartition du coût de latence de bout en bout induit par les décisions d'autoscaling et d'ordonnancement.}
\label{figure:herosim-cost}
\end{figure}

\section{Étude de cas}
\label{section:herosim-case-study}

\begin{figure*}[t]
    \center
    \subfloat[Consolidation\label{figure:herosim-evaluation-full-unused-nodes}]{
        \includegraphics[width=0.155\linewidth]{7_Chapitre5/figures/eval/2-unused-nodes.png}
    }
    \subfloat[QoS\label{figure:herosim-evaluation-full-penalty}]{
        \includegraphics[width=0.155\linewidth]{7_Chapitre5/figures/eval/3-penalty-proportions.png}
    }
    \subfloat[Energy\label{figure:herosim-evaluation-full-energy-consumption}]{
        \includegraphics[width=0.155\linewidth]{7_Chapitre5/figures/eval/6-energy-consumption.png}
    }
    \subfloat[Consolidation\label{figure:herosim-evaluation-components-unused-nodes}]{
        \includegraphics[width=0.155\linewidth]{7_Chapitre5/figures/eval-components/2-unused-nodes.png}
    }
    \subfloat[QoS\label{figure:herosim-evaluation-components-penalty}]{
        \includegraphics[width=0.155\linewidth]{7_Chapitre5/figures/eval-components/3-penalty-proportions.png}
    }
    \subfloat[Energy\label{figure:herosim-evaluation-components-energy-consumption}]{
        \includegraphics[width=0.155\linewidth]{7_Chapitre5/figures/eval-components/6-energy-consumption.png}
    }
    \caption{Charts generated by HeROsim for the evaluation of HeROcache -- Comparison against baselines (a-c) and impact of individual components (d-f). HRC-HRO means HeROcache autoscaler coupled with HeROfake scheduler.}
    \label{figure:herosim-evaluation}
\end{figure*}

Nous présentons deux études de cas qui ont utilisé HeROsim pour évaluer les politiques d'orchestration.
Nous avons conçu des stratégies qui reposent sur la caractérisation des plateformes hétérogènes et des charges de travail (voir figure~\ref{figure:herosim-characterization}). Nous avons mesuré plusieurs métriques liées à nos applications sur diverses plateformes matérielles et proposé un modèle de coût qui intègre ces valeurs pour estimer les performances de l'autoscaling et de l'ordonnanceur sous différentes politiques (figure~\ref{figure:herosim-cost}).

\subsection{Stratégie d'orchestration de fonctions sans état sur ressources hétérogènes}

Dans cette première étude de cas, \textbf{He}terogeneous \textbf{R}esources \textbf{O}rchestration for deep\textbf{fake} detection~\cite{herofake}, nous avons étudié le déploiement d'une application de détection de deepfake sur une plateforme serverless dans un cloud privé. En particulier, nous nous sommes intéressés à l'exploitation de ressources hétérogènes pour l'orchestration serverless lors de l'optimisation de la plateforme pour la QoS et la consommation d'énergie.

L'application exécute des tâches d'inférence pour détecter les deepfakes dans des images potentiellement trafiquées, soumises par des utilisateurs avec diverses exigences de niveau de QoS. Son objectif est de détecter les images potentiellement falsifiées, c'est-à-dire les images susceptibles d'avoir été manipulées par ordinateur pour tromper les spectateurs. Il se compose de trois tâches de réseau neuronal indépendantes et sans état qui ont été mises en œuvre sur du matériel hétérogène : CPU, GPU et FPGA. Ces implémentations ont été utilisées pour les mesures, mais il aurait été difficile d'exécuter réellement des scénarios du monde réel sur une plateforme serverless incluant ces architectures matérielles.

Dans HeROfake, chaque exécution de tâche a un coût associé mesuré en \textbf{latence}. L'allocation de nouvelles répliques sur des ressources matérielles inactives introduit un \textbf{délai de démarrage à froid} ; le traitement des requêtes utilisateur sur différents matériels a divers \textbf{temps d'exécution de la fonction}. La latence de chaque requête utilisateur est finalement comparée à la \textbf{exigence de qualité de service} de la requête : si l'échéance de la requête n'a pas été respectée, la tâche est en \textbf{pénalité}. Ces éléments constituent la base de notre modèle de coût, voir la figure~\ref{figure:herosim-cost}.

Les différentes architectures matérielles ont également des coûts monétaires et énergétiques différents qui ont été pris en compte dans le modèle de coût. Nous avons mis en œuvre un autoscaler et un ordonnanceur qui cherchent à estimer et à minimiser ce coût composite à la granularité de chaque requête utilisateur. L'autoscaler cherche à allouer des plateformes qui minimisent le coût global, y compris l'utilisation des ressources, la consommation d'énergie et le coût total de possession. L'ordonnanceur cherche à sélectionner les répliques qui traitent les requêtes avec le moins de pénalités et la plus faible consommation d'énergie.

HeROfake a été évalué à l'aide d'un scénario synthétique avec une distribution uniforme des temps d'arrivée des requêtes, des appels de fonction et des niveaux de qualité de service. HeROfake s'est avéré performant (en termes de pénalités de QoS, de consolidation et d'énergie) par rapport à Knative (l'orchestrateur serverless de Google) et Bin-Packing First Fit (la politique d'Amazon dans Lambda)~cite{herofake}. 

Notre objectif principal était d'étudier la pertinence de la prise en compte de l'hétérogénéité matérielle lors de l'allocation des ressources et de l'ordonnancement des requêtes utilisateur en ce qui concerne diverses mesures de qualité de service, \textit{i.e.} la latence de bout en bout et la consommation d'énergie. Les résultats expérimentaux ont montré que si le temps total d'exécution des tâches dans HeROfake est similaire à celui de la version vanille de Knative, nous obtenons une réduction de plus de 60\% des pénalités de qualité de service ; les tâches sont consolidées sur moins de 40\% des nœuds de l'infrastructure, 77\% des plates-formes d'exécution restant inutilisées ; et la consommation d'énergie dynamique est réduite de 35\% par rapport à Knative.

HeROsim nous a également permis d'évaluer l'impact des différents composants de notre cadre. Les résultats ont montré que, même en allouant uniquement des CPU, l'ordonnanceur des tâches d'inférence avec notre politique consciente de la qualité de service pouvait maintenir les violations des accords de niveau de service en dessous de 50\%. 

\subsection{Stratégie d'orchestration d'applications avec prise en compte du cache et des communications}

Ici, une stratégie d'orchestration consciente des contraintes temporelles et de données a été proposée. Nous avons exploré le déploiement d'un système de détection d'intrusion (IDS) sur une plateforme serverless. En particulier, nous avons étudié les avantages de l'orchestration consciente des données lors de l'exploitation de matériel hétérogène pour déployer des applications sensibles au temps sur des dispositifs périphériques à ressources limitées, du point de vue du fournisseur de services.

L'objectif de l'application est de détecter le trafic TCP potentiellement malveillant dans les journaux de réseau soumis par les utilisateurs. Comme cette application IDS est spécifiquement destinée aux missions de drones, son schéma d'accès est étroitement lié aux schémas saisonniers, ce qui en fait une bonne adaptation au paradigme serverless.

L'application se compose de deux couches : deux fonctions de prétraitement qui opèrent sur les journaux du trafic réseau, et quatre fonctions d'inférence qui détectent des modèles d'activité malveillante dans les journaux. Ces fonctions ont été mises en œuvre sur différentes architectures matérielles (CPU, FPGA, GPU). 

Il existe des dépendances de données entre les deux couches de l'application. Nous les avons modélisées sous forme de graphes acycliques dirigés (DAG) d'appels de fonctions. Nous avons utilisé le module Python \texttt{TopologicalSorter} du module \texttt{graphlib} pour la représentation JSON des graphes et pour les parcourir pendant l'exécution.

Au niveau de l'orchestrateur, nous avons dû modifier la granularité des requêtes utilisateur : une requête concerne une application, qui peut être composée d'une ou plusieurs fonctions, appelées dans l'ordre défini par le DAG de l'application. Chaque fonction peut prendre des données en entrée et produire des données en sortie. Ces données sont définies par leur taille en octets.

Alors que HeROfake ne prend pas en compte les opérations de stockage, HeROcache met en œuvre la \textbf{récupération des images de fonctions}, \textbf{la mise en cache des images de fonctions} et les \textbf{communications entre fonctions} dans l'orchestrateur de base. Ces opérations augmentent la latence de la requête utilisateur d'une quantité de temps calculée sur la base de la caractérisation de la plateforme et de la fonction, voir figure~\ref{figure:herosim-cost} : par exemple, HeROsim estime le temps de récupération de l'image de la fonction sur la base de la largeur de bande du réseau spécifiée par l'utilisateur dans la description de l'infrastructure, la taille de l'image de la fonction spécifiée par le type de charge de travail, et la performance du stockage local du nœud spécifiée par le type de stockage.

L'introduction d'opérations de stockage au niveau du nœud individuel nous a permis d'évaluer une stratégie de préchargement d'images de fonctions afin d'accélérer les démarrages à froid dans les invocations futures.

Dans HeROcache, l'autoscaler cherche à accroître la consolidation des fonctions entre les applications, à réduire le makespan global, la consommation d'énergie et le coût total de possession. L'ordonnanceur cherche à éviter le non-respect des délais des requêtes, à consommer moins d'énergie et à assurer une utilisation élevée des ressources.

Nous avons utilisé le générateur de traces d'exécution inclus dans HeROsim pour générer un scénario de 30 minutes de requêtes utilisateur suivant un processus de Poisson à un taux moyen de 83 requêtes par seconde. Cela correspond au taux de communication que la connectivité 4G LTE permettrait dans notre environnement périphérique, compte tenu de la taille des données d'entrée et de sortie de l'application. 

HeROfake exploite déjà l'hétérogénéité matérielle, mais est conçu comme une politique sans stockage, ce qui en fait une solution comparable. Cette comparaison nous a permis de montrer l'importance des coûts de stockage dans le respect de la qualité de service : avec une politique d'orchestration tenant compte des données, HeROcache consolide les applications et parvient à réduire les délais d'initialisation moyens de 17,6 % et les délais de communication de 88,4 %. Cela permet de réduire la consommation d'énergie statique de la plateforme de 80 % tout en maintenant moins de 28 % de violations de la qualité de service.

HeROcache (mis en œuvre au-dessus de HeROsim) a été soumis à l'évaluation des artefacts et a reçu les trois badges de reproductibilité de l'IEEE~\footnote{\href{https://www.niso.org/standards-committees/reproducibility-badging}{https://www.niso.org/standards-committees/reproducibility-badging}} : Open Research Objects (ORO), Reusable/Research Objects Reviewed (ROR), et Results Reproduced (ROR-R).

\section{Travaux connexes}
\label{section:herosim-sota}

\begin{table*}[t]
    \centering
    \caption{Overview of simulation tools for cloud computing}
    \resizebox{\linewidth}{!}{
        \begin{tabular}{lSSSSSSSSS}
        \toprule
        & Serverless & Deployment models & Functions chains & Hardware heterogeneity & Per-request QoS & Resource usage & Energy consumption & Visualization \\
        \cmidrule(lr){2-2}\cmidrule(lr){3-3}\cmidrule(lr){4-4}\cmidrule(lr){5-5}\cmidrule(lr){6-6}\cmidrule(lr){7-7}\cmidrule(lr){8-8}\cmidrule(lr){9-9}
        CloudSim~\cite{calheiros_cloudsim_2011} & \xmark & Public, private, hybrid & \xmark & \cmark & \xmark & \cmark & \cmark & \xmark \\
        CloudSimSC~\cite{mampage_cloudsimsc_2023} & \cmark & Public, private, hybrid & \xmark & \cmark & \xmark & \cmark & \cmark & \xmark \\
        CloudAnalyst~\cite{wickremasinghe_cloudanalyst_2010} & \xmark & Public, private, hybrid & \xmark & \cmark & \xmark & \cmark & \cmark & \cmark \\        DFaaSCloud~\cite{jeonCloudSimExtensionSimulatingDistributed2019} & \cmark & Multi-tier hybrid cloud & \xmark & \xmark & \cmark & \cmark & \xmark & \cmark \\
        ElasticSim~\cite{cai_elasticsim_2017} & \xmark & Public & \cmark & \xmark & \xmark & \cmark & \xmark & \cmark \\
        GridSim~\cite{buyyaGridSimToolkitModeling2002} & \xmark & Grid & \xmark & \cmark & \cmark & \cmark & \xmark & \cmark \\
        iCanCloud~\cite{nunez_icancloud_2012} & \xmark & Public & \xmark & \xmark & \xmark & \cmark & \xmark & \cmark \\
        iFogSim2~\cite{mahmudIFogSim2ExtendedIFogSim2021} & \xmark & Edge, Fog & \xmark & \cmark & \xmark & \cmark & \cmark & \xmark \\
        OpenDC 2.0~\cite{mastenbroekOpenDCConvenientModeling2021} & \cmark & Public, private, hybrid & \cmark & \cmark & \xmark & \cmark & \cmark & \cmark \\
        SimFaaS~\cite{mahmoudiSimFaaSPerformanceSimulator2021} & \cmark & Public & \xmark & \xmark & \xmark & \cmark & \cmark & \cmark \\
        \textbf{HeROsim} & \cmark & Private & \cmark & \cmark & \cmark & \cmark & \cmark & \cmark \\
        \bottomrule
        \end{tabular}
    }
    \label{table:herosim-sota}
\end{table*}

Nous présentons un aperçu des travaux de l'état de l'art et de leurs caractéristiques dans le Tableau~\ref{table:herosim-sota}.

CloudSim~\cite{calheiros_cloudsim_2011} est l'outil omniprésent pour les expériences de déploiement de nuages à grande échelle. Il cible les différents modèles de service traditionnels dans le cloud.
CloudSim, et ses extensions~\cite{calheiros_cloudsim_2011, mampage_cloudsimsc_2023, wickremasinghe_cloudanalyst_2010, jeonCloudSimExtensionSimulatingDistributed2019} ne prennent pas en compte les applications serverless, \textit{i. e.} la composition de fonctions pour obtenir un comportement complexe qui introduit des défis spécifiques (délais de démarrage à froid, frais généraux liés aux communications inter-fonctions, etc.~\cite{wawrzoniakBoxerDataAnalytics2021a}). Pour résoudre ces problèmes particuliers, il est nécessaire d'introduire la gestion du stockage, ainsi que le traitement des chaînes de fonctions, comme le fait HeROsim.

% DFaaSCloud~\cite{jeonCloudSimExtensionSimulatingDistributed2019} est un autre simulateur basé sur CloudSim pour le serverless distribué. Ce travail se concentre sur la distribution géographique des instances de fonction à travers une infrastructure cloud, edge et fog. Il permet d'estimer les retards induits par la localité des fonctions. Les utilisateurs définissent leurs fonctions en termes de contraintes de latence et DFaaSCloud fournit une politique de placement qui minimise les violations et les coûts. En tant que tel, le postulat de DFaaSCloud est spécifique au traitement du problème du placement géographique des tâches dans un environnement Function as a Service à plusieurs niveaux, et ne concerne pas les défis généraux du serverless.

% ElasticSim~\cite{cai_elasticsim_2017} étend également CloudSim pour fournir une allocation dynamique des ressources pour les flux de travail dans le cloud, \textit{i.e.} des chaînes de tâches connectées qui ne sont pas différentes des applications serverless. Cependant, il ne prend pas en compte l'hétérogénéité sous-jacente des ressources matérielles et ne permet pas non plus d'appliquer des objectifs de qualité de service par requête. OpenDC 2.0~\cite{mastenbroekOpenDCConvenientModeling2021} permet également à l'utilisateur de modéliser de telles chaînes de fonctions, et bien que cet outil permette de modéliser un centre de données hétérogène et d'estimer sa consommation d'énergie, il ne prend pas non plus en compte la variété des exigences des utilisateurs en termes de latence.

GridSim~\cite{buyyaGridSimToolkitModeling2002} présente des caractéristiques intéressantes, allant de la modélisation d'infrastructures hautement hétérogènes à l'application de contraintes de qualité de service par requête. iFogSim2~\cite{mahmudIFogSim2ExtendedIFogSim2021} considère également des allocations statiques qui ne peuvent pas caractériser fidèlement l'espace de problème serverless. HeROsim a été conçu pour l'espace de problème serverless et permet aux utilisateurs de tracer les événements d'allocation dynamique et d'ordonnanceur à la granularité d'une requête utilisateur.

De nombreuses contributions~\cite{jeonCloudSimExtensionSimulatingDistributed2019, cai_elasticsim_2017, buyyaGridSimToolkitModeling2002, nunez_icancloud_2012} ne permettent pas d'estimer la consommation d'énergie de la plateforme. La consommation d'énergie est une mesure cruciale lorsqu'il s'agit de relever le défi de l'ordonnancement de calculs gourmands en énergie tels que l'apprentissage automatique (ML), qui détiennent une proportion croissante des charges de travail cloud déployées~\cite{masanet2020recalibrating}. 
HeROsim estime à la fois la consommation d'énergie statique et la consommation d'énergie dynamique.

En outre, l'hétérogénéité du matériel est une caractéristique déterminante du cloud. Les accélérateurs tels que les GPU ou les TPU sont utilisés par les fournisseurs de services pour améliorer les performances. Nous avons soutenu que ce matériel pourrait permettre aux fournisseurs de services d'appliquer les contraintes de qualité de service tout en réduisant la consommation d'énergie d'une plateforme serverless~\cite{herofake}.
Parmi les outils de simulation de nuage disponibles, plusieurs contributions~\cite{jeonCloudSimExtensionSimulatingDistributed2019, cai_elasticsim_2017, nunez_icancloud_2012, mahmoudiSimFaaSPerformanceSimulator2021} ne prennent pas en compte l'hétérogénéité à une granularité fine. HeROsim permet aux utilisateurs de définir des infrastructures hautement hétérogènes pour le calcul et le stockage.

Enfin, certaines contributions~\cite{nunez_icancloud_2012, mahmoudiSimFaaSPerformanceSimulator2021} visent à simuler des infrastructures de cloud public qui se concentrent sur le courtage de ressources virtualisées considérées comme illimitées. Nos travaux~\cite{herofake, herocache} se sont concentrés sur la perspective des fournisseurs de services optimisant les plateformes serverless pour la qualité de service, d'où l'orientation cloud privé de HeROsim.


\section{Conclusion et perspectives}
\label{section:herosim-conclusion}

Dans cet article, nous avons présenté HeROsim, un outil de simulation qui vise à permettre aux chercheurs de modéliser des infrastructures cloud hétérogènes, de décrire les charges de travail à une granularité fine, de mettre en œuvre diverses politiques de gestion des ressources et d'ordonnancement des tâches, et d'évaluer leur efficacité au regard de métriques telles que l'utilisation des ressources, la consommation d'énergie, les violations de la QoS par requête, ou la latence de queue. HeROsim peut générer des graphiques qui aident à visualiser ces résultats pendant la phase de mise en œuvre et peuvent être utilisés dans des publications.

Des travaux sont en cours pour étendre HeROsim et proposer une interface web permettant de visualiser l'état de la simulation en temps réel. Nous travaillons également sur un agent d'apprentissage par renforcement qui pourrait être inclus dans une prochaine version.

\clearemptydoublepage
\backmatter
\chapter{Conclusion et perspectives}
%\addcontentsline{toc}{chapter}{Conclusion et perspectives}
%\chaptermark{Conclusion et perspectives}

\section{Résumé des contributions}
%\addcontentsline{toc}{section}{Résumé des contributions}

\section{Limites des contributions}
%\addcontentsline{toc}{section}{Limites des contributions}

\section{Travaux futurs}
%\addcontentsline{toc}{section}{Travaux futurs}

% Our next contribution will leverage Q-Learning to explore the design of an autonomous agent which efficiently rightsizes resources allocations on the serverless platform. It will make use of time series prediction to allow timely, proactive autoscaling. This agent will be evaluated in the simulator, showcasing the variety of policies that can be implemented with our tool.


\section{Remarques finales}
%\addcontentsline{toc}{section}{Remarques finales}


% Chapitre pour la bibliographie
% Bibliography chapter
\clearemptydoublepage
\phantomsection % To have a correct link in the table of contents
\addcontentsline{toc}{chapter}{Bibliographie}

% nocite: Pour citer la totalit\'{e} des r\'{e}f\'{e}rences contenues dans le fichier biblio
% nocite: In order to cite all the references included biblio
\nocite{*}
\printbibliography
% \newpage
% \nocite{*}
% \printbibliography[heading=secondary,keyword=secondary]

\clearemptydoublepage
% Pour avoir la quatrième de couverture sur une page paire
% To have the back cover on an even page
\cleartoevenpage[\thispagestyle{empty}]
\markboth{}{}
% Plus petite marge du bas pour la quatrième de couverture
% Shorter bottom margin for the back cover
\newgeometry{inner=30mm,outer=20mm,top=40mm,bottom=20mm}

%insertion de l'image de fond du dos (resume)
%background image for resume (back)
\backcoverheader

% Switch font style to back cover style
\selectfontbackcover{ % Font style change is limited to this page using braces, just in case
    \titleFR{Allocation et placement dynamiques sur ressources hétérogènes pour le serverless}
    \keywordsFR{cloud, serverless, orchestration, hétérogénéité, optimisation, simulation}
    \abstractFR{
        Le modèle serverless est un paradigme pour le cloud dans lequel les ressources matérielles ne sont pas réservées, mais allouées à la demande, lors de la réception de requêtes utilisateur. La facturation des clients s'effectue sur la base de l'utilisation réelle des ressources. En contrepartie, la responsabilité de l'allocation des ressources et du placement des tâches incombe au fournisseur. Ce mécanisme d'orchestration peut induire un surcoût en latence et dégrader la qualité de service pour les utilisateurs.
        Nous nous sommes intéressés à la capacité pour le fournisseur à garantir la qualité de service étant donnée une infrastructure dotée de matériel hétérogène (CPU, GPU, FPGA, DLA), dans le cadre de deux déploiements : une application de détection de deepfake, basée sur des fonctions sans état, et une application de détection d'intrusions, reposant sur des chaînes de fonctions communiquant des résultats intermédiaires.
        Nous avons proposé une politique d’orchestration qui optimise le déploiement d'applications composées d'une unique fonction, en minimisant les pénalités sur qualité de service. Nous avons également modélisé des applications complexes, en prenant en compte les dépendances entre fonctions qui les composent, pour chercher à réduire la consommation énergétique et la désagrégation des tâches sur les nœuds de calcul.
        Enfin, nous avons développé un simulateur permettant de représenter de tels environnements, et d'évaluer et de comparer différentes stratégies d'orchestration pour le cloud.
    }

    \titleEN{Dynamic allocation and scheduling on heterogeneous resources in a serverless cloud}
    \keywordsEN{cloud, serverless, orchestration, heterogeneity, optimization, simulation}
    \abstractEN{
        The serverless model is a cloud paradigm in which hardware resources are not reserved but allocated on demand, triggered by incoming user requests. Clients are billed based on their actual resource usage. In return, the responsibility for resource allocation and task placement lies with the provider. This orchestration mechanism can lead to latency overhead and degrade the quality of service for users.
        We focused on the provider’s ability to guarantee quality of service given an infrastructure comprising heterogeneous hardware resources (CPU, GPU, FPGA, DLA), in the context of two deployments: a deepfake detection application, based on stateless functions, and an intrusion detection application, relying on chains of functions that communicate intermediate results.
        We proposed an orchestration policy that optimizes the deployment of applications composed of a single function, minimizing penalties on quality of service. We also modeled complex applications, taking into account the dependencies between the functions they comprise, aiming to reduce energy consumption and task disaggregation across compute nodes.
        Finally, we developed a simulator to represent such environments and to evaluate and compare different cloud orchestration strategies.
    }
}

% Rétablit les marges d'origines
% Restore original margin settings
\restoregeometry


\end{document}
