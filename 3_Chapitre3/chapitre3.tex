\chapter{État de l'art -- Orchestration dans le modèle serverless}
\label{chapter:sota}

TODO: Dans ce troisième chapitre, nous commençons par livrer un aperçu des offres serverless commerciales dans le cloud public, ainsi que des plateformes open source permettant de déployer des fonctions serverless dans le cloud privé.
Nous discutons ensuite de travaux de l'état de l'art concernant l'orchestration de fonctions serverless sur des ressources matérielles hétérogènes, d'une part pour des applications simples, reposant sur des fonctions sans dépendances ; puis pour des applications complexes, exhibant des relations de dépendances entre les fonctions qui les composent. Enfin, nous présentons des travaux de l'état de l'art dans le domaine de la simulation pour les plateformes cloud.

\section{Description de l'offre serverless}
\label{section:sota-offerings}

Dans cette première section, nous proposons un aperçu des différentes offres serverless chez les fournisseurs de cloud public, ainsi que des solutions open source disponibles pour le cloud privé.

Le tableau~\ref{table:sota-commercial-faas} présente un sommaire des offres commerciales FaaS dans le cloud public, leurs modèles de tarifications ainsi que certaines de leurs caractéristiques. Ce sommaire inclut Alibaba Function Compute~\footnote{\label{footnote:alibaba-function}\href{https://www.alibabacloud.com/product/function-compute}{https://www.alibabacloud.com/product/function-compute}}, Amazon Web Services Lambda~\footnote{\label{footnote:aws-lambda}\href{https://aws.amazon.com/lambda/}{https://aws.amazon.com/lambda/}}, Microsoft Azure Functions~\footnote{\label{footnote:azure-functions}\href{https://azure.microsoft.com/products/functions/}{https://azure.microsoft.com/products/functions/}}, Google Cloud Functions~\footnote{\label{footnote:google-functions}\href{https://cloud.google.com/functions}{https://cloud.google.com/functions}}, IBM Cloud Functions~\footnote{\label{footnote:ibm-functions}\href{https://cloud.ibm.com/functions/}{https://cloud.ibm.com/functions/}} et Oracle Cloud Functions~\footnote{\label{footnote:oracle-functions}\href{https://www.oracle.com/cloud/cloud-native/functions/}{https://www.oracle.com/cloud/cloud-native/functions/}}.

Le tableau~\ref{table:sota-foss-faas} présente ensuite quelques solutions majeures disponibles dans la communauté open source, et donne un aperçu de l'état du projet en fonction de son adoption par les utilisateurs et la quantité de contributions, ainsi que les partenaires industriels qui accompagnent le projet. On y trouve Apache OpenWhisk~\footnote{\label{footnote:openwhisk}\href{https://github.com/apache/openwhisk}{https://github.com/apache/openwhisk}}, Fission~\footnote{\label{footnote:fission}\href{https://github.com/fission/fission}{https://github.com/fission/fission}}, Fn~\footnote{\label{footnote:fn}\href{https://github.com/fnproject/fn}{https://github.com/fnproject/fn}}, Knative Serving~\footnote{\label{footnote:knative}\href{https://github.com/knative/serving}{https://github.com/knative/serving}} et OpenFaaS~\footnote{\label{footnote:openfaas}\href{https://github.com/openfaas/faas}{https://github.com/openfaas/faas}}.

\subsection{Solutions commerciales dans le cloud public}

Pour positionner chacune des offres commerciales, nous avons choisi de les comparer sur la base des critères suivants :

\begin{itemize}
    \item Modèle de tarification : les modalités selon lesquelles les clients peuvent anticiper le coût de leurs déploiements dans le cadre d'une offre serverless dans le cloud public ;
    \item Caractéristiques : les limites imposées par le fournisseur de services en matière d'utilisation et de disponibilité des ressources.
\end{itemize}

TODO: Les solutions \gls{FaaS} dans le cloud public adoptent une tarification dite \textit{pay as you go} (ou \textit{à la demande}). Ce sont des offres sans engagement, facturées à une granularité fine. Cette granularité peut se situer au niveau du temps d'utilisation des ressources matérielles, ou s'exprimer en termes de requêtes utilisateur.

TODO: mettre à jour le tableau !

\begin{longtblr}[
    caption = {Offres \gls{FaaS} commerciales dans le cloud public, proposées par les acteurs majeurs de l'industrie.},
    label = {table:sota-commercial-faas},
    note{a} = {billed per \num{10000} requests (for USD 0.02)}
]{
    rowhead = 3,
    colspec = { X[l,1.8] X[2,l]X[2,l] X[l]X[l]X[l]X[1.8,l] },
    row{1-2} = { l, font = {\bfseries} },
    row{3} = { b, font = \footnotesize%, cmd={\rotatebox{60}}
                },
    row{4-Z} = { m, font = \footnotesize, rowsep = 1ex },
    column{1} = { font = {\bfseries}}
}
    \toprule
    \SetCell[r=3]{m} Service & \SetCell[c=2]{c} Modèle de tarification && \SetCell[c=4]{c} Propriétés \\
    \cmidrule[lr=-0.4]{2-3}
    \cmidrule[lr=-0.4]{4-7}
    &
    quota gratuit &
    coût à la demande &
    \\
    &
    nombre d'invocations / ressources de calcul (Go/s) &
    million de requêtes / 1 Go/s de ressources de calcul (€) &
    taille du code &
    pic mémoire &
    temps d'exécution &
    taille du contexte d'invocation \\
    \midrule

    Alibaba Function Compute~\footref{footnote:alibaba-function} &
    \num{1000000} / \num{400000} &
    \num{20}\TblrNote{a} / \num{0.000016384} &
    \qty{500}{\mega\byte} &
    \qty{3}{\giga\byte}&
    \qty{24}{\hour} &
    \qty{128}{\kilo\byte} (request), \qty{6}{\mega\byte} (response)
    \\

    AWS Lambda~\footref{footnote:aws-lambda} &
    \num{1000000} / \num{400000} &
    \num{0.2} / \num{0.0000166667} &
    \qty{10}{\giga\byte}&
    \qty{10240}{\mega\byte} &
    \qty{15}{\minute} &
    \qty{6}{\mega\byte} (synchronous), \qty{256}{\kilo\byte} (asynchronous) for requests and responses
    \\

    Azure Functions~\footref{footnote:azure-functions} &
    \num{1000000} / \num{400000} &
    \num{0.2} / \num{0.000016} &
    N/A &
    \qty{1.5}{\giga\byte} &
    \qty{10}{\minute}&
    \qty{100}{\mega\byte} (request)
    \\

    Google Cloud Functions~\footref{footnote:google-functions}   &
    \num{2000000} / \num{400000} &
    \num{0.4} / \num{0.0000025} &
    \qty{500}{\mega\byte}&
    \qty{8}{\giga\byte}&
    \qty{9}{\minute}&
    \qty{10}{\mega\byte} for requests and responses
    \\

    IBM Cloud Functions~\footref{footnote:ibm-functions}  &
    \num{5000000} / \num{400000} &
    N/A / \num{0.000017} &
    \qty{48}{\mega\byte} &
    \qty{2048}{\mega\byte} &
    \qty{60}{\second} &
    \qty{5}{\mega\byte} for requests and responses
    \\

    Oracle Cloud Functions~\footref{footnote:oracle-functions} &
    \num{2000000} / \num{400000} &
    \num{0.2} / \num{0.00001417} &
    N/A &
    \qty{2048}{\mega\byte} &
    \qty{5}{\minute}&
    \qty{6}{\mega\byte}
    \\

    \bottomrule
\end{longtblr}

\subsection{Solution open source pour le cloud privé}

Comme mesures de l'état d'un projet ainsi que de son adoption, nous choisissons deux métriques disponibles publiquement sur leurs dépôts GitHub~\footnote{\href{https://github.com/}{https://github.com/}} :

\begin{itemize}
    \item Les "\textit{GitHub stars}" (ou étoiles) indiquent combien d'utilisateurs de GitHub ont choisi de s'abonner aux mises à jour d'un projet ;
    \item Le décompte des contributeurs indique combien de personnes se sont impliquées dans le développement du logiciel, en ayant apporté des modifications (\textit{commits}) au dépôt hébergé sur GitHub.
\end{itemize}

Pour chacun de ces projets, nous relevons également les entités industrielles qui les supportent.

\begin{table}[H]
    \caption{Les plateformes serverless open source permettent aux fournisseurs de services cloud de construire leurs propres offres \gls{FaaS}.}
    \centering
    \begin{tabularx}{\textwidth} { 
      | >{\centering\arraybackslash}X 
      | >{\centering\arraybackslash}X 
      | >{\centering\arraybackslash}X 
      | >{\centering\arraybackslash}X  | }
    \hline
        & \textbf{\textit{GitHub stars}} & \textbf{Contributeurs} & \textbf{Support industriel} \\ \hline
        Apache OpenWhisk~\footref{footnote:openwhisk} & 6500 étoiles & 206 contributeurs & IBM (Apache Foundation) \\ \hline
        Fission~\footref{footnote:fission} & 8400 étoiles & 164 contributeurs & Platform9 \\ \hline
        Fn~\footref{footnote:fn} & 5700 étoiles & 85 contributeurs & Oracle \\ \hline
        Knative Serving~\footref{footnote:knative} & 5500 étoiles & 282 contributeurs & Google \\ \hline
        OpenFaaS~\footref{footnote:openfaas} & 25000 étoiles & 162 contributors & VMware \\ \hline
    \end{tabularx}
    \label{table:sota-foss-faas}
\end{table}

\section{Orchestration de fonctions serverless sur ressources hétérogènes}
\label{section:sota-herofake}

TODO: Dans cette section, nous listons et discutons des contributions de l'état de l'art qui répondent à notre première question de recherche...

\boitemagique{Question 1 (\textbf{QR1})}{
    Comment dimensionner les allocations dynamiques de ressources hétérogènes pour une application simple, constituée de fonctions de courte durée, et comment ordonnancer efficacement les requêtes utilisateur, lorsque ces derniers ont des besoins variés en matière de qualité de service ?
}

Plusieurs contributions antérieures se sont concentrées sur les plateformes de mise à l'échelle automatique pour le déploiement de tâches de courte durée, comprises dans des applications présentant des motifs d'utilisation imprévisibles. Le tableau~\ref{table:herofake-sota} résume les différences entre ces contributions et notre plateforme cible.

% SLA
Certaines de ces contributions ont tenté d'atteindre des \gls{SLA} avec des ressources non réservées~\cite{gujaratiSwayamDistributedAutoscaling2017, zhangMArkExploitingCloud, mampageDeadlineawareDynamicResource2021, singhviAtollScalableLowLatency2021, handaoui2020releaser, handaoui2020salamander, yalles2022riscless}. Parmi ces contributions, certaines se concentrent sur l'utilisation de ressources matérielles hétérogènes supplémentaires pour accélérer l'exécution de la charge de travail~\cite{zhangMArkExploitingCloud, lingPigeonDynamicEfficient2019, yangINFlessNativeServerless2022}. Ces méthodes nécessitent souvent un surprovisionnement de ressources stables pour utiliser l'accélération matérielle, par exemple en s'appuyant sur des instances \gls{AWS} dans le cloud public qui donnent accès à des \gls{GPU}~\cite{zhangMArkExploitingCloud}, en utilisant une réserve de conteneurs pré-initialisés~\cite{lingPigeonDynamicEfficient2019}, ou même en provisionnant de manière proactive des nœuds pour respecter des échéances définies par l'utilisateur~\cite{singhviAtollScalableLowLatency2021}. Ces solutions, bien qu'intéressantes en matière de performances, peuvent toutefois s'avérer insuffisantes en termes d'utilisation des ressources et entraîneraient une consommation d'énergie supplémentaire dans un cloud privé.

% Hétérogénéité
En outre, certaines contributions se concentrent sur des infrastructures homogènes~\cite{gujaratiSwayamDistributedAutoscaling2017, sureshENSUREEfficientScheduling2020, mampageDeadlineawareDynamicResource2021, singhviAtollScalableLowLatency2021, yangINFlessNativeServerless2022}. Ces études pourraient difficilement s'adapter au contexte du cloud privé que nous visons, où les ressources sont généralement éphémères et hétérogènes. En outre, certaines de ces contributions proposent des modèles de tâches qui ne couvrent pas la possibilité d'accords de niveau de service définis par l'utilisateur et à la granularité d'une requête~\cite{sureshENSUREEfficientScheduling2020, lingPigeonDynamicEfficient2019}. Enfin, certaines de ces contributions sont axées sur les performances plutôt que sur les coûts, ce qui est crucial dans notre contexte de cloud privé~\cite{gujaratiSwayamDistributedAutoscaling2017, lingPigeonDynamicEfficient2019, singhviAtollScalableLowLatency2021, choSLADrivenMLInference}.

% Conso d'énergie
Bien que la consommation d'énergie soit l'un des éléments les plus importants du coût total d'exploitation (\gls{OPEX}, pour \textit{Operational Expenditure}) dans un centre de données -- dépassant parfois le coût d'achat du matériel~\cite{7279063}, à notre connaissance, aucune de ces contributions ne couvre l'impact de l'allocation et du placement dynamiques sur la consommation d'énergie, ni ne considère la consommation d'énergie comme une métrique de qualité de service. Il s'agit d'une limite sérieuse, car l'optimisation de la consolidation des tâches ouvre la voie à des politiques de ralentissement et de mise hors tension qui peuvent avoir un impact positif majeur sur l'efficacité énergétique d'un centre de données~\cite{chaurasiaComprehensiveSurveyEnergyaware2021}.

\section{Orchestration d'applications complexes et coûts induits par le stockage}
\label{section:sota-herocache}

TODO: Dans cette section, nous listons et discutons des contributions de l'état de l'art qui répondent à notre seconde question de recherche...

\boitemagique{Question 2 (\textbf{QR2})}{
    Comment déployer des applications complexes, composées de chaînes de fonctions de courte durée, et comment tirer parti de l'hétérogénéité des nœuds disponibles à l'edge, pour respecter la qualité de service requise par les utilisateurs tout en contenant la consommation d'énergie de l'infrastructure ?
}

Des travaux antérieurs se sont concentrés sur les plateformes de mise à l'échelle automatique pour le déploiement de tâches de courte durée et comprises dans des applications présentant des modèles de charge imprévisibles (voir tableau~\ref{table:herocache-sota}).

TODO: Travaux cloud public...

% Conscience des données
Parmi ces travaux, \cite{smithFaDOFaaSFunctions2022} propose un orchestrateur conscient des dépendances de données, mais ne tenant pas compte de l'effet boule de neige des retards dans les chaînes de fonctions. \cite{zhangFIRSTExploitingMultiDimensional2023} ne prend pas en charge l'ordonnancement de ces chaînes de fonctions.

TODO: Certains de ces travaux imposent une contrainte de programmation aux développeurs souhaitant déployer leurs applications sur une telle plateforme serverless.

% Hétérogénéité
Toutes ces contributions considèrent une infrastructure homogène~\cite{bhasiCypressInputSizesensitive2022, zijunFassflowEfficient2022, smithFaDOFaaSFunctions2022, zhangFIRSTExploitingMultiDimensional2023, abdiPaletteLoadBalancing2023}. Cela n'est pas représentatif de notre cas d'utilisation, dans lequel les dispositifs edge sont très hétérogènes. HeROfake~\cite{herofake} exploite l'hétérogénéité matérielle dans sa politique d'orchestration, mais n'intègre pas les dépendances inter-fonctions ni la mise en cache des images dans son modèle de coût. Elle a été choisie à des fins d'évaluation, pour souligner la nécessité de prendre en compte ces coûts.

% Conso d'énergie
Certaines de ces contributions optimisent la consommation d'énergie au niveau de l'autoscaler~\cite{bhasiCypressInputSizesensitive2022, zhangFIRSTExploitingMultiDimensional2023}. Toutefois, elles se concentrent sur la partie dynamique de la consommation d'énergie : elles ne tiennent pas compte de l'impact possible de la consolidation sur la consommation d'énergie statique. Nous soutenons que les fournisseurs de services devraient chercher à consolider les tâches afin de mettre hors tension le plus grand nombre de nœuds possible, ce qui réduirait considérablement les besoins énergétiques globaux de leurs infrastructures.

% Complexité algorithmique
Dans~\cite{fuerstIluvatarFastControl2023}, les auteurs ont étudié les différents coûts induits par la complexité de l'orchestration serverless. Cet élément n'a pas été pris en compte dans notre étude, car nous visons une infrastructure edge de taille limitée pour le déploiement d'applications d'IDS bien identifiées.

\section{Évaluer et comparer différentes politiques d'orchestration serverless}
\label{section:sota-herosim}

TODO: Dans cette section, nous listons et discutons des contributions de l'état de l'art qui répondent à notre troisième question de recherche...

\boitemagique{Question 3 (\textbf{QR3})}{
    Du point de vue d'un fournisseur de services pour le cloud, comment évaluer et comparer l'impact sur la qualité de service de différentes politiques d'allocation de ressources et d'ordonnancement de tâches dans le modèle serverless ?
}

Nous présentons un aperçu des travaux de l'état de l'art en matière de simulation pour le cloud, et proposons une comparaison d'un ensemble de leurs caractéristiques dans le tableau~\ref{table:herosim-sota}.

CloudSim~\cite{calheiros_cloudsim_2011} est l'outil de référence pour les expériences de déploiement cloud à grande échelle. Il cible les différents modèles de service traditionnels dans le cloud (\gls{IaaS}, \gls{PaaS}, \gls{SaaS} ; voir chapitre~\ref{chapter:context}).
CloudSim et ses extensions~\cite{calheiros_cloudsim_2011, mampage_cloudsimsc_2023, wickremasinghe_cloudanalyst_2010, jeonCloudSimExtensionSimulatingDistributed2019} ne prennent pas en compte les applications serverless, \textit{i.e.} la composition de fonctions pour décrire un comportement complexe qui introduit des défis spécifiques (délais de démarrage à froid, coûts liés aux communications inter-fonctions, etc.~\cite{wawrzoniakBoxerDataAnalytics2021a}).
Pour relever ces défis en simulation, il est nécessaire d'introduire la gestion du stockage, ainsi que le traitement des chaînes de fonctions, comme le fait HeROsim.

DFaaSCloud~\cite{jeonCloudSimExtensionSimulatingDistributed2019} est un autre simulateur basé sur CloudSim pour le serverless distribué. Ce travail se concentre sur la distribution géographique des instances de fonction à travers une infrastructure cloud, edge et fog. Il permet d'estimer les retards induits par la localité des fonctions. Les utilisateurs définissent la qualité de service pour leurs fonctions en termes de contraintes de latence et DFaaSCloud fournit une politique de placement qui minimise les violations et les coûts. Nous n'avons pas abordé la dimension géographique du problème de placement dans les travaux de cette thèse.

ElasticSim~\cite{cai_elasticsim_2017} étend également CloudSim pour fournir une allocation dynamique des ressources pour \textit{workflows} dans le cloud, \textit{i.e.} des chaînes de tâches interdépendantes, qui présentent des similitudes avec les applications serverless. Cependant, il ne prend pas en compte l'hétérogénéité des ressources matérielles et ne permet pas non plus d'appliquer des objectifs de qualité de service par requête. OpenDC 2.0~\cite{mastenbroekOpenDCConvenientModeling2021} est un outil généraliste et très complet qui autorise également l'utilisateur à modéliser de telles chaînes de fonctions. Bien que cet outil permette de représenter un centre de données hétérogène et d'estimer sa consommation d'énergie, il ne prend pas non plus en compte la variété des exigences des utilisateurs en termes de latence.

GridSim~\cite{buyyaGridSimToolkitModeling2002} présente des caractéristiques intéressantes, allant de la modélisation d'infrastructures hautement hétérogènes à la prise en charge de contraintes de qualité de service par requête. Cependant, il se concentre sur des infrastructures dites "en grille", que l'on trouve généralement dans le monde du calcul haute performance (\gls{HPC}, pour \textit{High Performance Computing}), et ne permet pas d'explorer des problèmes liés à l'allocation dynamique de ressources. iFogSim2~\cite{mahmudIFogSim2ExtendedIFogSim2021} considère également des allocations statiques qui ne peuvent pas caractériser fidèlement l'espace de problème serverless.
HeROsim a été conçu pour le serverless et permet aux utilisateurs de tracer les événements d'allocation et d'ordonnancement à la granularité d'une requête utilisateur.

De nombreuses contributions~\cite{jeonCloudSimExtensionSimulatingDistributed2019, cai_elasticsim_2017, buyyaGridSimToolkitModeling2002, nunez_icancloud_2012} ne permettent pas d'estimer la consommation d'énergie de la plateforme. La consommation d'énergie est une mesure cruciale lorsqu'il s'agit de relever le défi de l'ordonnancement de calculs gourmands en énergie tels que l'apprentissage automatique (\gls{ML}), qui représentent une proportion croissante des charges de travail déployées dans le cloud~\cite{masanetRecalibratingGlobalData2020}.
HeROsim estime à la fois la consommation d'énergie statique et la consommation d'énergie dynamique.

En outre, l'hétérogénéité du matériel est une caractéristique déterminante du cloud. Les accélérateurs tels que les \gls{GPU} ou les \gls{TPU} sont utilisés par les fournisseurs de services pour améliorer la performance de charges de travail adaptées, bénéficiant largement de ces architectures matérielles. Nous avons soutenu qu'exploiter ce matériel de manière opportuniste pourrait permettre aux fournisseurs de services de proposer des accords de niveau de service pour le cloud serverless, tout en réduisant la consommation d'énergie de leur plateforme~\cite{herofake}.
Parmi les outils de simulation disponibles pour le cloud, plusieurs contributions~\cite{jeonCloudSimExtensionSimulatingDistributed2019, cai_elasticsim_2017, nunez_icancloud_2012, mahmoudiSimFaaSPerformanceSimulator2021} ne prennent pas en compte l'hétérogénéité à une granularité fine. HeROsim permet aux utilisateurs de définir des infrastructures hautement hétérogènes, à la fois pour le calcul et pour le stockage.

Enfin, certaines contributions~\cite{nunez_icancloud_2012, mahmoudiSimFaaSPerformanceSimulator2021} visent à simuler des infrastructures de cloud public et se concentrent sur la réservation de ressources virtualisées considérées comme illimitées, utilisées par exemple dans le cadre d'un panachage, pour répartir les charges de travail entre un cloud privé et un cloud public dans le but de diminuer les coûts.
Nos travaux~\cite{herofake, herocache} se sont concentrés sur la perspective de fournisseurs de services optimisant leurs plateformes serverless pour la qualité de service, d'où l'orientation cloud privé de HeROsim.

\section{Conclusion}

En libérant les utilisateurs de la contrainte du dimensionnement de leur infrastructure, le modèle de service serverless pour le cloud promet de faciliter le passage à l'échelle des applications. Grâce au mécanisme d'allocation à la demande, les clients peuvent bénéficier d'économies considérables, en ne payant plus pour des ressources qui seraient essentiellement dormantes, en attente d'une requête.

TODO: SLA

La mise à l'échelle depuis zéro est associée à un risque de latence lors du réveil de l'application, puisque le fournisseur de services doit alors dynamiquement allouer des ressources matérielles et instancier l'environnement d'exécution des fonctions en réponse à un évènement. Les fournisseurs de services ont tendance à pré-allouer des ressources de manière à éviter ces démarrages à froid, ce qui contraint leurs gains potentiels en rendant ces ressources indisponibles pour d'autres clients.

TODO: Data-aware

Le fonctionnement de cette architecture logicielle, qui présente des similitudes avec l'architecture en microservices, repose sur la communication par passage de messages entre fonctions. Les fonctions n'étant pas directement adressables sur le réseau dans les solutions commerciales actuelles, cette communication s'effectue par le biais d'un stockage lent : cela induit un surcoût important sur les performances de l'application lors des phases de composition et de synchronisation, jusqu'à parfois contrebalancer les gains offerts par le parallélisme massif inhérent au paradigme serverless.

TODO: Hétérogénéité

Enfin, les accélérateurs matériels sont les grands absents de l'offre serverless commerciale. À l'heure où la demande en \gls{GPU} et \gls{FPGA} est croissante pour répondre aux besoins en calcul massivement parallèle, notamment dans le cadre de l'apprentissage automatique ou de l'analyse de données en masse, les clients doivent se tourner vers une offre cloud plus conventionnelle s'ils souhaitent bénéficier de plateformes d'exécution hétérogènes.
